%%%%%%%%%%%%%%%%%%%%%%%%%%%%%%%%%%%%%%%%%%%%%%%%%%%%%%%%%%%%%%%%%%%%%%%%%%%%%%%%%%%%%%%%%%%%
% Заметки размещаются в окружении слайд класса beamer. 
% Удалять эти окружения не нужно -- при вёртке книги это осуществляется автоматически.
%%%%%%%%%%%%%%%%%%%%%%%%%%%%%%%%%%%%%%%%%%%%%%%%%%%%%%%%%%%%%%%%%%%%%%%%%%%%%%%%%%%%%%%%%%%%
\subsection{2020-06-07 -- Запуск программы для сборки документации по проектной деятельности}
%%%%%%%%%%%%%%%%%%%%%%%%%%%%%%%%%%%%%%%%%%%%%%%%%%%%%%%%%%%%%%%%%%%%%%%%%%%%%%%%%%%%%%%%%%%%

В этот день Маратом Идрисовым (студент группы РК6-82Б) была разработана подсистема автоматичеческой систематизации и вёрстки отчетов о 
научно-образовательной деятельности. 

Задача заключалась в разработке программной инфраструктуры для автоматизации процесса обработки и сбора постоянно формируемой отчетной документации с последующим их объединением в единые документы. Предполагается, что формируемые таким образом документы, позволят наглядно, и, что самое главное, в сжатой форме, демонстрировать во времени процессы проведения исследований по разным научным направлениям, развиваемым в некотором подразделении.

%\begin{figure}[!htbp]%
%	\centering
%	\begin{tabular}{cc}
%		\includegraphics[width=0.44\textwidth]{Notes/hme@sln@_@tml@_@year@_@month@_@day@/XXXXXXX.jpg} &
%		\includegraphics[width=0.44\textwidth]{Notes/hme@sln@_@tml@_@year@_@month@_@day@/XXXXXXX.jpg} \\
%		(а) & (б) \\
%	\end{tabular}
%	\label{fig:complex.sidsolution.sid.fig.id}
%	\caption{Подпись к рисунку: а) подпись к блоку (а); б) подпись к блоку (б)}
%\end{figure}
		
%\end{frame}
%%%%%%%%%%%%%%%%%%%%%%%%%%%%%%%%%%%%%%%%%%%%%%%%%%%%%%%%%%%%%%%%%%%%%%%%%%%%%%%%%%%%%%%%%%%%
