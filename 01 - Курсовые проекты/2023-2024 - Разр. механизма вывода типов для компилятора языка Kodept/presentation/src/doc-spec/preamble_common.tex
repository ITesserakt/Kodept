%----------------------------------------------------------
\usepackage{mathtext}
%\usepackage[warn]{mathtext} 
\usepackage[T2A]{fontenc}
\usepackage[utf8]{inputenc}
\usepackage[english,russian]{babel} %% это необходимо для включения переносов
\usepackage{float}
\usepackage{algorithm,algpseudocode}

\usepackage{listings}
\usepackage{longtable}
\usepackage{rotating}
\usepackage{multirow}
\usepackage{pdflscape}
\usepackage{cmap} % необходимо для возможности копирования и поиска в готовом PDF
\usepackage{array}
\usepackage{multicol}
\usepackage{mathtools}

\usepackage{datetime}
\usepackage{tikz}
\usepackage{relsize}


\usepackage{etex} % этот пакет нужен для того, чтобы была возможность создавать свои окружения \newtheorem без ошибок 'no room for a new \count'

\usepackage{hyperref}
\hypersetup{colorlinks=true, linkcolor=red}

\usepackage[style=long4colheader, translate=babel, section=section]{glossaries}
\usepackage[abbreviations, toc=true, xindy, automake]{glossaries-extra}
\setglossarystyle{treenoname}%+
\makeglossaries
% выключает разворачивание терминов и аббревиатур при первом использовании в том числе, - всегда термины и аббревиатуры будут выводиться кратко 
\glsunsetall

\usepackage{pdfpages}
\usepackage{xspace}
\usepackage{amssymb,amsfonts,amsmath,cite,enumerate,float} %подключаем нужные пакеты расширений

% включаем кириллицу и добавляем некоторые опции
% подгружаемые языки — подробнее в документации listings
\lstloadlanguages{[LaTeX]TeX, bash, MetaPost, Perl, XML, [Visual]C++, [ISO]C++, make}
% включаем кириллицу и добавляем кое?какие опции
%\lstset{language = [Visual]C++, % выбираем язык по умолчанию
%extendedchars=true, % включаем не латиницу
%escapechar=|, % |«выпадаем» в LATEX|
%frame=tb, % рамка сверху и снизу
%commentstyle=\itshape, % шрифт для комментариев
%stringstyle=\bfseries} % шрифт для строк


\lstset{language=[ISO]C++,
breaklines=true,                 % sets automatic line breaking
frame=tb, % рамка сверху и снизу
                basicstyle=\ttfamily,
                keywordstyle=\color{blue}\ttfamily,
                stringstyle=\color{red}\ttfamily,
                commentstyle=\color{green}\ttfamily,
                morecomment=[l][\color{magenta}]{\#}
}

\lstset{language=bash,
breaklines=true,                 % sets automatic line breaking
frame=tb, % рамка сверху и снизу
                basicstyle=\ttfamily,
                keywordstyle=\color{black}\ttfamily,
                stringstyle=\color{red}\ttfamily,
                commentstyle=\color{green}\ttfamily,
                morecomment=[l][\color{magenta}]{\#}
}

%\frametitle{Inserting source code without setting typewriter}
%\lstset{language=C++,
%                keywordstyle=\color{blue},
%                stringstyle=\color{red},
%                commentstyle=\color{green},
%                morecomment=[l][\color{magenta}]{\#}
%}
\renewcommand{\lstlistingname}{Исходный код}

%\setbeamercovered{transparent}

\lstdefinelanguage{Ini}
{
    basicstyle=\ttfamily\small,
		frame=single,
    columns=fullflexible,
		numbers=left,                    % where to put the line-numbers; possible values are (none, left, right)
    numbersep=5pt,                   % how far the line-numbers are from the code
    numberstyle=\tiny\color{gray}, % the style that is used for the line-numbers
%		breakatwhitespace=false,         % sets if automatic breaks should only happen at whitespace
    morecomment=[s][\color{black}\bfseries]{[}{]},
    morecomment=[s][\color{purple}\bfseries]{=[}{]},
    morecomment=[s][\color{pink}\bfseries]{[[}{]]},
    morecomment=[s][\color{blue}\bfseries]{}{},
    morecomment=[l]{//},
    commentstyle=\color{gray}\itshape,
    otherkeywords={=,$},
    keywordstyle={\color{brown}\bfseries}
}

\makeatletter
\renewcommand{\ALG@name}{Алгоритм}
\makeatother

%-----------------------------------------------------------
\usepackage{pgffor}
\usepackage{filecontents}% Used so that the external files can be placed in this file

\usepackage{keyval}
\usepackage{totcount}
% суперсчетчики всего ! :-)
\regtotcounter{page}

\newtotcounter{ffigure}
\def\oldfigure{} \let\oldfigure=\figure
\def\figure{\stepcounter{ffigure}\oldfigure}

\newtotcounter{ttable}
\def\oldtable{} \let\oldtable=\table
\def\table{\stepcounter{ttable}\oldtable}

\newtotcounter{cchapter}
\def\oldchapter{} \let\oldchapter=\chapter
\def\chapter{\stepcounter{cchapter}\oldchapter}

\newtotcounter{eequation}
\def\oldequation{} \let\oldequation=\equation
\def\equation{\stepcounter{eequation}\oldequation}

\newtotcounter{bibcnt}
\def\oldbibitem{} \let\oldbibitem=\bibitem
\def\bibitem{\stepcounter{bibcnt}\oldbibitem}


\usepackage[strings]{underscore}

% Произвольная нумерация списков.
\usepackage{enumerate}

% 3. Изменение начертания шрифта --- после чего выглядит таймсоподобно.
% apt-get install scalable-cyrfonts-tex

\IfFileExists{cyrtimes.sty}
    {
        \usepackage{cyrtimespatched}
    }
    {
        % А если Times нету, то будет CM...
    }

\usepackage{datetime}	

%----------------------------------------------------------
% команда для вывода ссылки на литературу в виде сноски
\usepackage[numbers]{natbib}
\usepackage{bibentry}
%----------------------------------------------------------
%***natbib, bibentry***%
% Следующий код необходим для того, чтобы исправить конфликт между пакетами natbib+bibentry и стилем оформления ссылок согласно российскому ГОСТу cp1251gost705u
\ifx\undefined\selectlanguageifdefined
\def\selectlanguageifdefined#1{}\else\fi
\ifx\undefined\BibEmph
\def\BibEmph#1{\emph{#1}}\else\fi
%----------------------------------------------------------


