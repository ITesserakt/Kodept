%----------------------------------------------------------
\chapter{Анализ результатов}\label{chap6_results_analysis}
%----------------------------------------------------------

В примерах было продемонстрировано, что разработанный редактор удовлетворяет всем поставленным требованиям, однако некоторый функционал требует дополнительной отладки с целью выявления необработанных пользовательских сценариев и других неточностей в реализации. Также в будущем планируется расширить функционал приложения, чтобы сделать его более удобным для конечного пользователя. Далее будет описан реализованный на 2021.12.31 функционал.

Для каждой из реализованных функций будет составлен следующий список:
\begin{enumerate}
	\item Реализованный функциконал
	\item Сложности, возникшие при реализации
	\item Возможное улучшение функционала
\end{enumerate}

\subsection{Добавление веришны}
Для доблавения вершины необходимо выбрать соответствующий пункт в меню, а затем кликнуть на холст.
После этого потребуется уточнить название вершины, после ввода названия вершины построение будет полностью завершено.

~\\Сложности, возникшие при реализации:
\begin{enumerate}
	\item Валидация всех необходимых параметров в процессе создания вершины: положение вершины, название вершины;
	\item Корректное сохранение всех данных о вершине для дальнейшего использования
\end{enumerate}

~\\Возможное улучшение функционала:
\begin{enumerate}
	\item Возможность редактировать вершину после ее создания. На текущий момент в случае ошибки при создании вершины ее надо удалить, а затем построить еще раз.
	\item Использование набора горячих клавиш для ускорения процесса создания вершины.
	\item При вводе метки вершины предлагать пользователю автозаполненное название, в том случае если прошлые названия имеют инкрементирующийся для каждой вершины постфикс, например, уже созданы вершины $s_1, s_2$, при создании новой вершины в поле ввода названия система должна предложить пользователю название $s_3$.
\end{enumerate}

\subsection{Добавление ребра}
Для добавления ребра необходимо поочередно выбрать две вершины, которые будут соединены ребром. Затем система потребует ввода предиката и функции, в том случае если введенный предикат или функция являются уникальнальными в рамках графа, то система потребует ввода информации о module и entry func. После этого построение ребра будет завершен.

~\\Сложности, возникшие при реализации:
\begin{enumerate}
	\item Использование множества вспомогательных функций: подсчет количества ребер между вершинами, для определения типа построения ребра (прямое или кривые Безье), подсчет количества ребер выходящих из вершины для уточнения типа параллелизма, проверка полей ввода предиката и функции (предикат может быть неопределен, а функция должна быть определена), проверка введенных предиката и функции на уникальность в рамках графа.
	\item Выбор типа построения ребра: прямое, кривые Безье. Обработка ситуации, когда пользователь создает цикл между этими вершинами.
\end{enumerate}

~\\Возможное улучшение функционала:
\begin{enumerate}
	\item Возможность перевыбора вершин для построения ребра между ними. На данный момент нельзя допустить ошибку при выборе вершин, необходимо закончить процесс построения ребра, удалить его и создать новое, выбрав другие вершины.
	\item Алгоритм для построения ребра с использованием кривых Безье. На данный момент ребро строится с помощью одной кривой Безье, что не позволяет строить ребра более сложного вида, тем самым разрешая коллизии, когда ребро проходит через другие вершины.
\end{enumerate}

\subsection{Удаление вершины}
Для удаления вершины необходимо выбрать соответствующий пункт в меню, а затем кликнуть на вершину для удаления - вершина удаляется с холста вместе со всеми связанными с ней ребрами.

~\\Сложности, возникшие при реализации:
\begin{enumerate}
	\item Корректно удалить все связанные с вершиной ребра, а затем удалить информацию об этих ребрах и связанных вершин из объектов.
\end{enumerate}

~\\Возможное улучшение функционала:
\begin{enumerate}
	\item Возможность одновременного выбора нескольких вершин для удаления.
	\item Возможность откатить действия в том случае если была удалена лишняя вершина. В целом это касается всего приложения, на данный момент любое действие неотвратимо.
\end{enumerate}

\subsection{Удаление ребра}
Для удаления ребра необходимо выбрать соответствующий пункт в меню, а затем клкинуть на ребро для удаления - ребро удалится с холста.

~\\Сложности, возникшие при реализации:
\begin{enumerate}
	\item Удалить информацию о ребре из связанных этим ребром вершин.
\end{enumerate}

~\\Возможное улучшение функционала:
\begin{enumerate}
	\item Более удобный выбор ребра, на данный момент надо кликать ровно на ребро поскольку для перехвата события клика используется event.target.closest.
	\item Возможность одновременного выбора нескольких ребер для удаления.
\end{enumerate}

\subsection{Экспорт графа в формате aDOT}
Для экспорта в формат aDOT необходимо выбрать соответствующий пункт в меню, а затем поочередно выбрать две вершины, которые будут являться стартовой и конечной вершиной соответственно. Затем сформируется текстовый файл с описание графа в формате aDOT и автоматически начнется его загрузка.

~\\Сложности, возникшие при реализации:
\begin{enumerate}
	\item Корректное формирование файла в формате aDOT.
\end{enumerate}

\subsection{Импорт графа из формата aDOT}
Для импорта из формата aDOT необходимо выбрать соответствующий пункт в меню, а затем выбрать файл для импорта. Далее система сама построит граф, сохранив всю необходимую информацию.

~\\Сложности, возникшие при реализации:
\begin{enumerate}
	\item Разработка алгоритма визуализации графа.
\end{enumerate}

~\\Возможное улучшение функционала:
\begin{enumerate}
	\item Оптимизация алгоритма, уменьшение асимптотической сложности.
\end{enumerate}

Перейдем к функционалу, который планируется реализовать. В первую очередь стоит обратить внимание, что при построении графа, невозможно уменьшить масштаб или наоборот приблизить, также невозможно и перемещаться по холсту, что не позволяет строить графы, которые из-за своих размеров выходят за пределы холста. Еще одной полезной функциональной возможностью является сохранение текущего графа в базе данных, это позволит одновременно работать с несколькими графами путем переключения между ними, а также работать над одним графом с разных устройств 

