%----------------------------------------------------------
\chapter{Постановка задачи}
%----------------------------------------------------------
\section{Концептуальная постановка задачи}

В разработанном А.П.Соколовым и А.Ю.Першиным графоориентированном программном каркасе, для описания графовых моделей был разработан формат aDOT, который является расширением формата описания графов DOT. Для визуализации графов, описанных с использоваением формата DOT, используются специальные программы визуализации. Самой популярной из них является пакет утилит Graphviz, пакет позволяет визуализировать графы описанные в формате DOT и получать изображение графа в разных форматах: PNG, SVG и т.д. Формат aDOT расширяет формат DOT с помощью дополнительных атрибутов и определений, которые описывают функции-предикаты, функции-обработчики и функции-перехода в целом. Таким образом, становится очевидно, что для построения графовых моделей с использованием формата aDOT необходим графический редактор. 

Разрабатываемый графический редактор должен удовлетворять следующим требованиям:
\begin{itemize}
	\item Разрабатываемый редактор должен представлять из себя web-приложение - это позволяет сделать редактор независимым от операционной системы.
	\item Редактор должен предоставлять возможность создавать ориентированный граф с нуля
	\item Возможность экспотировать созданный граф в формат aDOT
	\item Возможность загрузить граф из формата aDOT
\end{itemize}
