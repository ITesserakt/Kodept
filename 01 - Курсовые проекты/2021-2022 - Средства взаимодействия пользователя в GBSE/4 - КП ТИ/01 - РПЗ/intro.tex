%----------------------------------------------------------
\chapter*{ВВЕДЕНИЕ}\label{chap.introduction}
\addcontentsline{toc}{chapter}{ВВЕДЕНИЕ}

% --------------------------------------------------------
При разработке программных систем, оперирующих большими объёмами данных, как, например, системы автоматизированного проектирования (CAD) и инженерного анализа (CAE) необходимо предусмотреть возможность предоставления результатов моделирования пользователю в доступном для него виде. Главным инструментом, реализующим такую возможность, можно считать средство визуализации. Основной её идеей является предоставление большого объёма информации исследователю в форме, удобной для восприятия и анализа~\cite{Berch2017}. Примером такого анализа может быть поиск закономерностей влияния одних переменных на другие. Таким образом, визуализация поддерживает принятие решений~\cite{RomVisualization2016}.

Кроме того, существует потребность помимо самих данных отображать некоторую связанную с ними мета-информацию, которая служит для их документирования. Действительно, для удобства восприятия и последующего анализа исследователю необходимо понимать, что представляет тот или иной элемент визуализируемых данных. Таким образом, необходимо некоторое средство создания и связывания мета-информации с данными.

В ходе программной реализации сложной системы (например, системы инженерного анализа), инструмент визуализации является отдельной подсистемой, имеющей некоторый программный интерфейс взаимодействия со всей остальной системой. Логика работы системы с средством визуализации представлена на рисунке~\ref{fig:visSystemStructure}.

\begin{figure}[!ht]
    \centering
    \includegraphics[width=0.7\textwidth]{figures/structure.visualizationSystem.png}
    \caption{Схема работы средства визуализации}
    \label{fig:visSystemStructure}
\end{figure}

Примерами реализации таких инструментов может служить отдельное веб-приложение, получающее данные от сервера, где проводятся основные вычисления\cite{Xu2022}, или отдельный компонент, отвечающий за формирование и вывод графического представления данных, в подсистеме взаимодействия с пользователем. На рисунке~\ref{fig:visSystemStructure} показано, что данные в средство визуализации передаются в некотором формате. Это может быть как их двоичное внутреннее представление в оперативной памяти, все компоненты системы имеют общее адресное пространство, или некоторый файл, где данные структурированы особым образом.

В распределённой вычислительной системе GCD средство визуализации данных является отдельным программным модулем. От вычислительного ядра системы данные должны передаваться через файлы. Таким образом, одной из задач исследования было подобрать (или при необходимости разработать) наиболее подходящий формат представления данных для их передачи в средство визуализации.

Одними из первых рассмотренных были форматы для хранения научных данных HDF4 и HDF5~\cite{HDFOffCite}. Данные бинарные форматы позволяют хранить большие объёмы гетерогенной информации и поддерживают иерархическое представление данных. В нём используется понятие набора данных (англ. dataset), которые объединяются в группы (англ. group). Кроме того, формат HDF5 считается <<самодокументирующимся>>, поскольку каждый его элемент -- набор данных или их группа -- имеет возможность хранить метаданные, служащие для описания содержимого элемента. Существует официальный API данного формата для языка С++ с открытым исходным кодом. Одним из гланвых недостатков HDF5 является необходимость дополнительного ПО для просмотра и редактирования данных в этом формате, поскольку он является бинарным.

Альтернативой бинарным форматам описания данных являются текстовые. Среди них были рассмотрены форматы XML (Extensible Markup Language) и JSON (Javascript Object Notation). Главным преимуществом формата XML является его ориентированность на древовидные структуры данных и лёгкость лексико-синтаксического разбора файлов этого формата. Среди недостатков стоит выделить потребность в сравнительно большом количестве вспомогательных синтаксических конструкций, необходимых для структурирования (тегов, атрибутов). Они затрудняют восприятие чистых данных и увеличивают итоговый объём файла.

Формат JSON, так же, как и XML рассчитан на иерархические структуры данных, но является не столь синтаксически нагруженным, что облегчает восприятие информации человеком~\cite{JSONvsXML}. Кроме того, крайне важным преимуществом JSON является его поддержка по-умолчанию средствами языкы программирования Javascript, который используется при разработке веб-приложений. При этом JSON также обладает рядом недостатков. Среди них сниженная, по сравнению с XML надёжность, отсутствие встроенных средств валидации и отсутствие поддержки пространств имён, что снижает его расширяемость.

На основании проведённого анализа преимуществ и недостатков выбор был сделан в пользу формата JSON. Ключевыми факторами для этого стали лёгкость восприятия информации в этом формате и нативная поддержка этого формата языком Javascript.

%----------------------------------------------------------
