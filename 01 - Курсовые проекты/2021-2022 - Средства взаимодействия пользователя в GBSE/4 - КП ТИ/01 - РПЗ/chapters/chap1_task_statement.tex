%----------------------------------------------------------
\chapter{Постановка задачи}
%----------------------------------------------------------
\section{Концептуальная постановка задачи}
В ходе выполнения работы необходимо было реализовать программные инструменты описания состояний данных согласно их определению, данному в~\cite{SokolovPershin2018}. Должна также быть реализована возможность сохранять описания состояний данных в файлы с использованием некоторого формата. Должен быть проведён аналитический обзор различных форматов для хранения информации.

Т.н. <<состояние данных>>\cite{SokolovPershin2018} представляет собой множество именованных переменных фиксированного типа, характерное для конкретного этапа вычислительного метода или алгоритма. Данные в соответствующем состоянии, как правило, удобно хранить в виде ассоциативного массива.

%Достоиством использования таких ассоциативных массивов является возможность группировки данных.

%Помимо этого, поскольку каждый элемент ассоциативного массива обладает ключом, типом и значением, что повторяет общую структуру элемента состояния данных, возникает возможность организовать состояния данных в виде иерархических структур.

% This here is literally a middle of a paragraph. I need to get it out of my head.
Элемент состояния данных описывается парой <<имя параметра -- множество допустимых значений параметра>>\cite{SokolovPershin2018}. При этом множество допустимых значений параметра ограничено типом этого параметра. Таким образом, при реализации структуры данных для внутренного представления состояний данных необходимо для каждого элемента состояния хранить его имя и тип. Помимо этого в целях повышения удобства восприятия данных должна быть включена возможность добавить к каждому элементу состояния данных краткое описание для пояснения роли конкретного входного параметра или промежуточной переменной в реализации вычислительного метода.

Тип отдельной переменной может быть как скалярным (целое, логическое, вещественное с плавающей запятой и пр.), так и сложным <<векторным>> (структурой, классом, массивом и пр.). Примером сложного <<векторного>> типа является, в свою очередь, ассоциативный массив со строковыми ключами, при этом конкретная переменная этого типа будет хранить, как правило, адрес этого массива. В общем случае элементы данного массива могут иметь разные типы. В рассматриваемом случае возникает возможность организации хранения состояния данных в виде иерархических структур.

Таким образом, для описания состояний данных требуется формат, который бы поддерживал гетерогенные (т.е. разнотипные) иерархические структуры данных.

%----------------------------------------------------------

