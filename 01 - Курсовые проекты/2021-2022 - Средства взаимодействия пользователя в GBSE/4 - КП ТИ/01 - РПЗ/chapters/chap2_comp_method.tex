%----------------------------------------------------------
\chapter{Аналитический обзор}\label{chap2_review}
%----------------------------------------------------------
Одними из первых рассмотренных были форматы для хранения научных данных HDF4 и HDF5~\cite{HDFOffCite}. Данные бинарные форматы позволяют хранить большие объёмы гетерогенной информации и поддерживают иерархическое представление данных. В нём используется понятие набора данных (англ. dataset), которые объединяются в группы (англ. group). Кроме того, формат HDF5 считается <<самодокументирующимся>>, поскольку каждый его элемент -- набор данных или их группа -- имеет возможность хранить метаданные, служащие для описания содержимого элемента. Существует официальный API данного формата для языка С++ с открытым исходным кодом. Одним из гланвых недостатков HDF5 является необходимость дополнительного ПО для просмотра и редактирования данных в этом формате, поскольку он является бинарным.

Альтернативой бинарным форматам описания данных являются текстовые. Среди них были рассмотрены форматы XML (Extensible Markup Language) и JSON (Javascript Object Notation). Главным преимуществом формата XML является его ориентированность на древовидные структуры данных и лёгкость лексико-синтаксического разбора файлов этого формата. Среди недостатков стоит выделить потребность в сравнительно большом количестве вспомогательных синтаксических конструкций, необходимых для структурирования (тегов, атрибутов). Они затрудняют восприятие чистых данных и увеличивают итоговый объём файла.

Формат JSON, так же, как и XML рассчитан на иерархические структуры данных, но является не столь синтаксически нагруженным, что облегчает восприятие информации человеком~\cite{JSONvsXML}. Кроме того, крайне важным преимуществом JSON является его поддержка по-умолчанию средствами языкы программирования Javascript, который используется при разработке веб-приложений. При этом JSON также обладает рядом недостатков. Среди них сниженная, по сравнению с XML надёжность, отсутствие встроенных средств валидации и отсутствие поддержки пространств имён, что снижает его расширяемость.

На основании проведённого анализа преимуществ и недостатков выбор был сделан в пользу формата JSON. Ключевыми факторами для этого стали лёгкость восприятия информации в этом формате и нативная поддержка этого формата языком Javascript.

%----------------------------------------------------------

