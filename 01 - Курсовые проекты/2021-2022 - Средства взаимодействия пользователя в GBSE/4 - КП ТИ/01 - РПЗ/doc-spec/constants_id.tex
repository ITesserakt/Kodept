%----------------------------------------%
% общие определения
\newcommand{\UpperFullOrganisationName}{Министерство науки и высшего образования Российской Федерации}
\newcommand{\ShortOrganisationName}{МГТУ~им.~Н.Э.~Баумана}
\newcommand{\FullOrganisationName}{федеральное государственное бюджетное образовательное\newline учреждение высшего профессионального образования\newline <<Московский государственный технический университет имени Н.Э.~Баумана\newline (национальный исследовательский университет)>> (\ShortOrganisationName)}
\newcommand{\OrganisationAddress}{105005, Россия, Москва, ул.~2-ая Бауманская, д.~5, стр.~1}
%----------------------------------------%
\newcommand{\gitlabdomain}{sa2systems.ru:88}
%----------------------------------------------------------
\newcommand{\doctypesid}{kp} % vkr (выпускная квалификационная работа) / kp (курсовой проект) / kr (курсовая работа) / nirs (научно-исследовательская работа студента) / nkr (научно-квалификационная работа)
\newcommand{\Title}{Разработка программных средств описания состояний данных в рамках реализации графоориeнтированного подхода}
\newcommand{\TitleSource}{кафедра} % кафедра, предприятие, НИР, НИР кафедры, заказ организации

\newcommand{\SubTitle}{по дисциплине <<Технологии интернет>>} % Методы оптимизации
\newcommand{\faculty}{<<Робототехники и комплексной автоматизации>>}
\newcommand{\facultyShort}{РК}
\newcommand{\department}{<<Системы автоматизированного проектирования (РК-6)>>}
\newcommand{\departmentShort}{РК-6}
\newcommand{\Author}{Тришин~И.В.}
\newcommand{\AuthorFull}{Тришин~Илья~Вадимович}
\newcommand{\ScientificAdviserPosition}{канд.~физ.-мат.~наук}	% Должность научного руководителя
\newcommand{\ScientificAdviser}{Соколов А.П.}	% Научный руководитель
\newcommand{\ConsultantA}{Першин А.Ю.}				% Консультант 1
\newcommand{\ConsultantB}{@Фамилия~И.О.@}				% Консультант 2
\newcommand{\Normocontroller}{Грошев~С.В.}		% Нормоконтролёр
\newcommand{\group}{РК6-81Б}
\newcommand{\Semestr}{весенний семестр} % Например: осенний семестр или весенний семестр
\newcommand{\BeginYear}{2022}
\newcommand{\Year}{2022}
\newcommand{\Country}{Россия}
\newcommand{\City}{Москва}
\newcommand{\TaskStatementDate}{<<\underline{\textit{31}}>> \underline{марта} \Year~г.} %Дата выдачи задания 

\newcommand{\depHeadPosition}{Заведующий кафедрой}		% Должность руководителя подразделения
\newcommand{\depHeadName}{А.П.~Карпенко}		% Должность руководителя подразделения

% Цель выполнения 
\newcommand{\GoalOfResearch}{предложить программные архитектурные решения для создания программных средств визуализации состояний данных в контексте применения графоориeнтированного подхода} % с маленькой буквы и без точки на конце

% Объектом исследования называют то, что исследуется в работе. Напри-мер, для указанной выше темы объектом может быть популяция медуз, но никак ни модель SIS, ни Южно-Китайское море, ни метод моделирования популяции медуз. 
\newcommand{\ObjectOfResearch}{методы и технологии программного описания многомерных ассоциативных массивов}

% Предмет исследований (уже чем объект, определяется, исходя из задач: формулируется как существительное, как правило, во множественном числе, определяющее "конкретный объект исследований" среди прочих в рамках более общего)
\newcommand{\SubjectOfResearch}{}

% Основная задача, на решение которой направлена работа
\newcommand{\MainProblemOfResearch}{разработка информационных компонентов средств визуализации многомерных ассоциативных массивов}

% Выполненные задачи
\newcommand{\SubtasksPerformed}{%
	В результате выполнения работы:
	\begin{inparaenum}[1)]
		\item изучены форматы представления разнотипной научной информации;
		\item изучены теоретические основы ``графоориeнтированного подхода'', введено понятие состояния данных;
		\item разработаны структуры данных, поддерживающее хранение информации о состояниях данных;
	\end{inparaenum}}

% Ключевые слова (представляются для обеспечения потенциальной возможности индексации документа)
\newcommand{\keywordsru}{%
	визуализация ассоциативных массивов, структуры данных, метаданные, предметно-ориентированные языки пограммирования} % 5-15 слов или выражений на русском языке, для разделения СЛЕДУЕТ ИСПОЛЬЗОВАТЬ ЗАПЯТЫЕ
\newcommand{\keywordsen}{%
	@keywordsen@} % 5-15 слов или выражений на английском языке, для разделения СЛЕДУЕТ ИСПОЛЬЗОВАТЬ ЗАПЯТЫЕ

% Краткая аннотация
\newcommand{\Preface}{В современных системах, осуществляющих многоэтапные вычисления, существует потребность в средстве визуализации обрабатываемых данных. Данная работа направлена на разработку информационных программных структур, описывающих обрабатываемые данных с их типами и возможными пояснениями, и подбор некоторого формата, в котором описания данных могли бы храниться на постоянном запоминающем устройстве.} % с большой буквы с точкой в конце

%----------------------------------------%
% выходные данные по документу
\newcommand{\DocOutReference}{\Author. \Title\xspace\SubTitle. [Электронный ресурс] --- \City: \Year. --- \total{page} с. URL:~\url{https://\gitlabdomain} (система контроля версий кафедры РК6)}

%----------------------------------------------------------

