%-------------------------
\newpage
%-------------------------
\officialheaderfull[]{ЗАДАНИЕ}{на выполнение \doctypec}
%-------------------------

\noindent Источник тематики (кафедра, предприятие, НИР): \underline{\TitleSource}

\myconditionaltext{\doctypesid}{vkr}{%
	\noindent Тема \doctypec\xspace утверждена распоряжением по факультету \facultyShort~№~\underline{\textcolor{white}{XXXX}} от \datetofill
}

\myconditionaltext{\doctypesid}{kp}{%
	\noindent Тема \doctypec\xspace утверждена на заседании кафедры \department, Протокол~№~\underline{\textcolor{white}{XXXX}} от \datetofill
}

\noindent \textbf{Техническое задание}

\noindent \textbf{Часть 1.} \textit{Аналитический обзор литературы.\\
	\uline{В рамках аналитического обзора литературы должны быть освещены современные подходы к визуализации научной информации.}}

\noindent \textbf{Часть 2.} \textit{Разработка архитектуры программной реализации\\
	\uline{Предложить конкретные архитектурые решения проектируемого программного обеспечения, включающего в себя UML диаграммы планируемых к созданию классов, блок-схемы и т.п.}}

\noindent \textbf{Оформление \doctypec:}

\noindent Расчетно-пояснительная записка на \total{page} листах формата А4.

\noindent Перечень графического (иллюстративного) материала (чертежи, плакаты, слайды и т.п.):

\noindent\begin{tabular}{|p{0.95\textwidth}|}
	\hline
	\textit{количество: \total{ffigure}~рис., \total{ttable}~табл., \total{bibcnt}~источн.} \\
	\hline
	\\
	\hline
\end{tabular}

\noindent Дата выдачи задания \TaskStatementDate\\

\noindent \begin{tabular}{p{0.55\textwidth}>{\raggedleft}p{0.2\textwidth}P{0.2\textwidth}}
	\signerline{\textbf{Студент}}{\Author}                           \\[5pt]
	\signerline{\textbf{Руководитель \doctypec}}{\ScientificAdviser} \\
\end{tabular}

%\vspace{10pt}
\noindent {\smaller[1] Примечание: Задание оформляется в двух экземплярах: один выдается студенту, второй хранится на кафедре.}
