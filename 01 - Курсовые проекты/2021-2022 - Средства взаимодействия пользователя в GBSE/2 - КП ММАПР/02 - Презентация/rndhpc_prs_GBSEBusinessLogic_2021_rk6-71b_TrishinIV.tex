\documentclass[9pt]{beamer}
%----------------------------------------------------------
% Стиль презентации (выбрать понравившийся)
\usetheme{bmstu}
%------------------------------------------
% определения значений стандартных параметров
%----------------------------------------------------------
\newcommand{\doctype}{presentation} % о научно-исследовательской работе / об опытно-конструкторской работе / об опытно-технологической работе / о патентных исследованиях
\newcommand{\doctypesid}{prs} % prs (Презентация) / vkr (выпускная квалификационная работа) / kp (курсовой проект) / kr (курсовая работа) / nirs (научно-исследовательская работа студента) / nkr (научно-квалификационная работа)
\def\titlepagestyle{detail} % brief (краткая) / detail (подробная)
%----------------------------------------%
\newcommand{\Title}{Разработка механизма вывода типов с использованием системы типов\\Хиндли-Милнера}%Научные основы автоматизированного проектирования композиционных материалов
\newcommand{\SubTitle}{Системы типов в языках программирования} % Методы оптимизации
%----------------------------------------------------------
\newcommand{\eventplace}{факультет \facultyShort, кафедра \department} % Место проведения мероприятия
\newcommand{\eventtype}{название события} % Тип мероприятия (лекция|лабораторная работа|семинар|мозговой штурм)
%\newcommand{\eventplace}{@место проведения@} % Место проведения мероприятия
\newcommand{\eventduration}{5 минут} % Продолжительность мероприятия
%----------------------------------------------------------
\newcommand{\Author}{Никитин В.Л.}
\newcommand{\AuthorFull}{Никитин Владимир Леонидович, студент группы РК6-85Б}
\newcommand{\AuthorEmail}{nikitinvl@student.bmstu.ru}
\newcommand{\ScientificAdviserPosition}{кандидат физико-математических наук}	% Должность научного руководителя
\newcommand{\ScientificAdviser}{Соколов А.П.}	% Научный руководитель
\newcommand{\group}{РК6-85Б}
\newcommand{\Semestr}{весенний семестр} % Например: осенний семестр или весенний семестр
\newcommand{\begdate}{08 февраля 2024} % Дата начала разработки
\newcommand{\Year}{2024}
\newcommand{\country}{Россия}	% Страна, в которой проводится конференция
\newcommand{\city}{Москва}		% Город, в котором проводится конференция
%----------------------------------------%
\newcommand{\depHeadPosition}{Заведующий кафедрой}		% Должность руководителя подразделения
\newcommand{\depHeadName}{А.П.~Карпенко}		% Должность руководителя подразделения
%----------------------------------------%
\newcommand{\presentationtitle}{\Title}
\newcommand{\conferenceperiod}{\country, \city, \begdate\ -- \today}
%----------------------------------------%

% Цель исследования/доклада
\newcommand{\GoalOfResearch}{реализация системы вывода и проверки типов} % с маленькой буквы и без точки на конце

% Объектом исследования называют то, что исследуется в работе. Например, для указанной выше темы объектом может быть популяция медуз, но никак ни модель SIS, ни Южно-Китайское море, ни метод моделирования популяции медуз.
\newcommand{\ObjectOfResearch}{система типов}

% Предмет исследований (уже чем объект, определяется, исходя из задач: формулируется как существительное, как правило, во множественном числе, определяющее "конкретный объект исследований" среди прочих в рамках более общего)
\newcommand{\SubjectOfResearch}{система типов Хиндли-Милнера}

% Основная задача, на решение которой направлена работа
\newcommand{\MainProblemOfResearch}{реализация алгоритма вывода типов на основе выбранной системы типов}

% Выполненные задачи
\newcommand{\SubtasksPerformed}{%
	В результате выполнения работы:
	\begin{inparaenum}[1)]
	\item спроектировано представление абстрактного синтаксического дерева в компиляторе;
	\item реализован семантический анализатор;
	\item показано, что компилятор успешно может вывести тип функции
	\end{inparaenum}}

% Ключевые слова (представляются для обеспечения потенциальной возможности индексации документа)
\newcommand{\keywordsru}{%
	теория типов, языки программирования, компиляторы, фукнциональное программирование, система типов Хиндли-Милнера} % 5-15 слов или выражений на русском языке, для разделения СЛЕДУЕТ ИСПОЛЬЗОВАТЬ ЗАПЯТЫЕ
\newcommand{\keywordsen}{%
	type theory, programming languages, compilers, functional programming, Hindley-Milner type system} % 5-15 слов или выражений на английском языке, для разделения СЛЕДУЕТ ИСПОЛЬЗОВАТЬ ЗАПЯТЫЕ

% Краткая аннотация
\newcommand{\Preface}{
	Работа посвящена реализации механизма вывода типов для языка программирования Kodept.
	Программирование выстроено вокруг глубокой математической теории.
	Благодаря этому появляются возможности для оптимизации, развития и улучшения языков посредством применения математики.
	Одним из важных применений является теория типов, которая помогает программисту в написании кода.
	В последнее время все больше и больше языков почерпывают что-то из этой области.
	Применение мощной системы типов позволяет зачастую снизить количество ошибок, возникающих при разработке.
} % с большой буквы с точкой в конце

%----------------------------------------%
% выходные данные по документу
\newcommand{\DocOutReference}{\Author. \Title\xspace\SubTitle. [Электронный ресурс] --- \City: \Year. --- \total{page} с. URL:~\url{https://\gitlabdomain} (система контроля версий кафедры РК6)}

%----------------------------------------------------------

%------------------------------------------
% определения значений стандартных общих параметров для различных документов, в т.ч. различных типов
%----------------------------------------------------------
% Информация о лицензии
\newcommand{\doclicense}{\copyright\xspace}
%\newcommand{\doclicense}{\includegraphics[width=0.1\textwidth]{../common_doc_spec/by.eps}\xspace}%\ccShareAlike
%----------------------------------------------------------
\newcommand{\doctextlicense}{\copyright\xspace} % Copyright ©
%\newcommand{\doctextlicense}{CC BY 4.0}% \ccAttribution
%----------------------------------------%
% общие определения
\newcommand{\UpperFullOrganisationName}{Министерство науки и высшего образования Российской Федерации}
\newcommand{\ShortOrganisationName}{МГТУ~им.~Н.Э.~Баумана}
\newcommand{\FullOrganisationName}{Московский Государственный Технический Университет имени Н.Э.~Баумана\newline (национальный исследовательский университет)>> (\ShortOrganisationName)}
\newcommand{\OrganisationAddress}{105005, Россия, Москва, ул.~2-ая Бауманская, д.~5, стр.~1}
%----------------------------------------%
\newcommand{\gitlabdomain}{gitlab.sa2systems.ru}
%----------------------------------------%
\newcommand{\faculty}{<<Робототехники и комплексной автоматизации>>}
\newcommand{\facultyShort}{РК}
\newcommand{\department}{<<Системы автоматизированного проектирования (РК-6)>>}
\newcommand{\departmentShort}{РК-6}
\newcommand{\departmentcloud}{\href{https://archrk6.bmstu.ru}{https://archrk6.bmstu.ru}}
%----------------------------------------------------------


%------------------------------------------
% общая кастомизируемая преамбула для презентаций
%----------------------------------------------------------
\usepackage[T2A]{fontenc}
\usepackage[utf8]{inputenc}
\usepackage[russian]{babel} %% это необходимо для включения переносов
\usepackage{bm}
\usepackage{float}
%\usepackage{algpseudocode}      % для окружения algorithmic
%\usepackage{algorithm}          % для нумерации алгоритмов
\usepackage{cmap} % необходимо для возможности копирования и поиска в готовом PDF
%-------------------------
% Сохранение метаданных в PDF об авторе документа
\hypersetup{%
    pdftitle={\Title},    	% title
    pdfauthor={\Author},    % author
		pdfcopyright={\doctextlicense \begdate -- \today, \Author. Все права защищены.},
    pdfsubject={\Title},   % subject of the document
    pdfcreator={\pdftexbanner},   % creator of the document
		pdfpublisher={\department, \ShortOrganisationName},
		pdfcaptionwriter={\Author},
    pdfproducer={\Author, \begdate -- \today, \ShortOrganisationName}, % producer of the document
    pdfkeywords={\keywordsru, \keywordsen}, % producer of the document
}
%----------------------------------------------------------
% final - удаляет все всплывающие комментарии
\usepackage[author={\Author},opacity=0.1]{pdfcomment}
%----------------------------------------------------------
\usepackage{tikz}
\usetikzlibrary{tikzmark}
\usetikzlibrary{matrix,automata,positioning,arrows,arrows.meta,graphs}
\usetikzlibrary{trees,topaths}
\usetikzlibrary{calc, circuits.ee.IEC}
\usetikzlibrary{patterns,decorations.pathmorphing,decorations.markings}
%----------------------------------------------------------
\renewcommand{\lstlistingname}{Исходный код}

%------------------------------------------
% дополнительные параметры, имеющие отношение к данному документу
%----------------------------------------------------------
\newcommand\ddfrac[2]{\frac{\displaystyle #1}{\displaystyle #2}}
\newcommand\T{\textup{T}}
\def\argmin{\operatornamewithlimits{argmin}}
\def\argmax{\operatornamewithlimits{argmax}}
\def\diag{\operatornamewithlimits{diag}}
\def\dep{\operatornamewithlimits{dep}}
%----------------------------------------------------------
\newcommand{\bvec}[1]{\mbox{\mathversion{bold}${#1}$}}
\newcommand{\periodichnost}[1]{[\hspace{-2pt}[#1]\hspace{-2pt}]}
\newcommand{\volna}[1]{\mathop{#1}\limits\_{\sim}}
\newcommand{\crujok}[1]{#1}
\newcommand{\driv}[2]{\frac{d#1}{d#2}}
\newcommand{\pdriv}[2]{\frac{\partial #1}{\partial #2}}
\newcommand{\uparr}[1]{\overrightarrow{#1}}
\newcommand{\intThreeD}[1]{\hspace{6pt}\mathop{\int\hspace{-17pt}\int\hspace{-17pt}\int}\limits\_{#1}\hspace{6pt}}
\newcommand{\intTwoD}[1]{\hspace{6pt}\mathop{\int\hspace{-17pt}\int}\limits\_{\hspace{-6pt}#1}}
%----------------------------------------------------------
\newcommand{\multirowbox}[2]
{
   \parbox{#1}{\smallskip #2 \smallskip}
}
%----------------------------------------------------------
\usepackage{amsthm}
\usepackage{thmtools}

\declaretheoremstyle[
  headfont=\normalfont\bfseries,
	numberwithin=section,
  bodyfont=\normalfont,
  spaceabove=1em plus 0.75em minus 0.25em,
  spacebelow=0em plus 0.75em minus 0.25em,
%  qed={$\blacksquare$},
%  qed={$\natural$},%\sharp},%$\square$},
	headpunct={\newline},%Punctuation after theorem head
]{thrmstyle}
\declaretheorem[
  style=thrmstyle,
  title=Замечание,
  refname={замечание,замечания},
  Refname={Замечание,Замечания}
]{remark}

%\declaretheorem[
  %style=thrmstyle,
  %title=Определение,
  %refname={определение,определения},
  %Refname={Определение,Определения}
%]{definition}

\theoremstyle{thrmstyle}

%\newtheorem{question}{Вопрос}
%\newtheorem{approval}{Утверждение}
%\newtheorem{demand}{Требование}
\newtheorem{statement}{Утверждение}
%\newtheorem{syntax}{Синтаксис}
%renewtheorem{definition}{Определение}
\newtheorem{designation}{Обозначение}
%\newtheorem{property}{Свойство}
%\newtheorem{task}{Задача}
%\newtheorem{mathmod}{Математическая модель}
%\newtheorem{mytheorem}{Теорема}

%----------------------------------------------------------
\hyphenation{супер-ком-пью-тер-ный супер-ком-пью-тер-ное вы-со-ко-произ-во-ди-тель-ный}
%----------------------------------------------------------
\newcommand\MyProcess[3]
{
%\begin{tabular}{p{0.1\textwidth}p{0.9\textwidth}}
%&
\begin{lstlisting}[caption={#1}, label={#2}, language=ALGO, basicstyle=\small]
#3
\end{lstlisting}			
%\end{tabular}
}
%----------------------------------------------------------
\newcommand\myquote[2]{%
\textit{``#1''}
\begin{flushright}
\textcolor{gray}{#2}
\end{flushright}
}
%----------------------------------------------------------
\newcommand{\mydescr}[1]{%

\colorbox{black!5}{\parbox{0.7\textwidth}{\smaller[1]\textit{#1}}}
}
%----------------------------------------------------------
% вёрстка в две колонки
\newcommand\cols[4]
{\smaller[1]%
		\begin{columns}
		\begin{column}{#1}
    #2
		\end{column}
		\begin{column}{#3}
		#4
		\end{column}
		\end{columns}}
%----------------------------------------------------------
% вёрстка в две колонки двух блоков с itemize
\newcommand\colsblocks[4]
{\smaller[1]%
		\begin{columns}
		\begin{column}{0.5\textwidth}
		\begin{block}{#1}
		\begin{itemize}
			#2
		\end{itemize}
		\end{block}
		\end{column}
		\begin{column}{0.5\textwidth}
		\begin{block}{#3}
		\begin{itemize}
			#4
		\end{itemize}
		\end{block}
		\end{column}
		\end{columns}}
%----------------------------------------------------------
%\usepackage{ifthen}
%%----------------------------------------------------------
%\usepackage{totcount}
%%----------------------------------------------------------
%%\newboolean{shortversion}
%% #1 - current \doctype
%% #2 - destination document
%% #3 - text
%\newcommand{\myconditionaltext}[3]%
%{%
	%\ifthenelse{\equal{#1}{avtoreferat}\AND\equal{#2}{avtoreferat}}{#3}{}% short version
	%\ifthenelse{\equal{#1}{avtoreferat}\AND\equal{#2}{thesis}}{}{}% short version
	%\ifthenelse{\equal{#1}{avtoreferat}\AND\equal{#2}{presentation}}{}{}% short version
	%\ifthenelse{\equal{#1}{thesis}\AND\equal{#2}{presentation}}{}{}% short version
	%\ifthenelse{\equal{#1}{thesis}\AND\equal{#2}{thesis}}{#3}{}% short version
	%\ifthenelse{\equal{#1}{presentation}\AND\equal{#2}{presentation}}{#3}{}% short version
	%\ifthenelse{\equal{#1}{presentation}\AND\equal{#2}{thesis}}{\pdfcomment{#3}}{}% short version
%}

%----------------------------------------------------------
% #1 - Короткое название практики
% #2 - Короткое описание
% #3 - Назначение в форме itemize
% #4 - LaTeX код
\newcommand{\bestpractice}[4]{%
\subsection{#1}
\begin{frame}%[fragile]

{\smaller[1]
#2}

\begin{columns}[t]

\begin{column}{0.5\textwidth}
\begin{block}{Назначение}
{\smaller[1]%
#3}
\end{block}
\end{column}

\begin{column}{0.5\textwidth}
\begin{lstlisting}[language={TeX}, basicstyle=\scriptsize]
#4
\end{lstlisting}
\end{column}
\end{columns}

\end{frame}}
%----------------------------------------------------------
\newenvironment{arreqn}[2]
    {\smaller[1]
		\begin{block}{\smaller[1] #1}
    \begin{equation}\label{#2}
    \begin{array}{l}
    }
		{
    \end{array}
    \end{equation}
		\end{block}
    }
%----------------------------------------------------------





%------------------------------------------
% база терминов! 
%----------------------------------------------------------
%Термины и определения по тексту в большинстве случаев выделяются курсивом.
%В настоящем отчете о НИР применяют следующие термины с соответствующими определениями, также используются представленные обозначения и сокращения.


%\newabbreviation[category=inline]{html}{HTML}{hypertext markup language}
%\newabbreviation[category=footer]{shtml}{SHTML}{server-parsed HTML}

%\newglossaryentry{sample2}{name={sample2},
%	symbol={\ensuremath{\mathcal{S}_2}},
%	category=symbol,
%	description={the second sample entry}}

%\newabbreviation
%[prefix={an\space},
%prefixfirst={a~}]
%{svm}{SVM}{support vector machine}

%\newabbreviation
%[category=initialism,description={for example}]
%{eg}{eg}{exempli gratia}
% define the entries:

%\newabbreviation{html}{html}{hypertext markup language}

%\newabbreviation[category=initialism]{eg}{eg}{for example}
%\newabbreviation[category=initialism]{si}{SI}{sample initials}
%\newabbreviation{xml}{XML}{extensible markup language}
%\newabbreviation{css}{CSS}{cascading style sheet}
%\newacronym[description={a device that emits a narrow intense 
%	beam of light}]{laser}{laser}{light amplification by stimulated 
%	emission of radiation}

%\newacronym[description={a form of \gls{laser} generating a beam of
%	microwaves}]{maser}{maser}{microwave amplification by stimulated 
%	emission of radiation}

%\newacronym[description={a system for detecting the location and
%	speed of ships, aircraft, etc, through the use of radio waves}]{radar}{radar}{radio detection and ranging}

%\newacronym[description={portable breathing apparatus for divers}]{scuba}{scuba}{self-contained underwater breathing apparatus}

%%%%% для обычных newglossaryentry по умолчанию category==general.
%%%%% для обычных newabbreviation по умолчанию category==abbreviation.
%%%%% команда для создания своей категории \glscategory{<label>}
%----------------------------------------------------------
\newglossaryentry{slver}{name={Solver}, description={Решатель системы \gls{dcs-gcd}. Регистрируется в таблице \textbf{com.slvrs} БД \gls{gcddb} \gls{dcs-gcd}.}}
\newabbreviation[category=initialism]{ПО}{ПО}{-- программное обеспечение}
\newabbreviation[category=initialism]{API}{API}{-- прикладной программный интерфейс (Application Programming Interface)}
\newabbreviation[category=initialism]{DFD}{DFD}{-- диаграмма потоков данных (Data Flow Diagram)}
\newabbreviation[category=initialism]{GBSE}{GBSE}{-- графоориентированный подход к разработке программного обеспечения (graph based software engineering)}
\newabbreviation[category=initialism]{LCPD}{LCPD}{-- платформы малокодовой разработки (low-code development platforms)}
\newabbreviation[category=initialism]{CASE}{CASE}{-- втоматизированная разработка программного обеспечения (Computer aided software engineering)}
\newabbreviation[category=initialism]{DOT}{DOT}{-- язык описания графов}
\newabbreviation[category=initialism]{JSON}{JSON}{-- файловый формат для хранения структур данных (Javascrtipt Object Notation)}
\newabbreviation[category=initialism]{TO}{ТО}{технический объект, в т.ч. сложный процесс, система}
\newabbreviation[category=initialism]{aINI}{aINI}{-- расширенный формат INI (\href{https://archrk6.bmstu.ru/index.php/f/846701}{описание представлено в \cite{SokAINI}})}
\newabbreviation[category=initialism]{aDOT}{aDOT}{-- расширенный формат DOT (\href{https://archrk6.bmstu.ru/index.php/f/777612}{описание представлено в \cite{SokolovADOT2020}})}


\GlsXtrEnableEntryCounting
{abbreviation}% list of categories to use entry counting
{2}% trigger value

\GlsXtrEnableEntryCounting
{symbol}% list of categories to use entry counting
{2}% trigger value


%----------------------------------------------------------



%---------------------------------------------
\includeonly{%
,cpxsln_prs_prj_YYYY_Group_SurnameNS_intro
,cpxsln_prs_prj_YYYY_Group_SurnameNS_taskstatement
,cpxsln_prs_prj_YYYY_Group_SurnameNS_method
,cpxsln_prs_prj_YYYY_Group_SurnameNS_software
,cpxsln_prs_prj_YYYY_Group_SurnameNS_calculation
,cpxsln_prs_prj_YYYY_Group_SurnameNS_conclusion
}
%---------------------------------------------
% выключает разворачивание терминов и аббревиатур при первом использовании в том числе, - всегда термины и аббревиатуры будут выводиться кратко 
\glsunsetall
%----------------------------------------------------------
% Это следует использовать и визуализировать с помощью Dual-side PDF Viewer - dspdfviewer-1.15.1.2)
%\setbeameroption{show notes on second screen=right} % Both
%----------------------------------------------------------
%\setbeameroption{hide notes} % Only slides
%\setbeameroption{show only notes} % Only notes
%\setbeameroption{show notes} % Both
%\setbeamertemplate{note page}{\pagecolor{yellow!5}\insertnote}
%---------------------------------------------
\includeonly{
    sections/intro,
    sections/chap2_task_statement,
    sections/chap3_software,
    sections/conclusion
}

\begin{document}
%---------------------------------------------
% создание стандартной титульной страницы
%----------------------------------------------------------
% Определение значений параметров для формирования титульной страницы
\title{\presentationtitle} 
\subtitle{\small
\SubTitle \hfill\break
\raggedright\noindent  \textit{\eventplace} \hfill\break
\raggedright\noindent  \textit{\eventduration}
} 
\date[\city\enskip\year]{\scriptsize\conferenceperiod} 
\author[\Author]{%
\raggedright \AuthorFull \\
\href{mailto:\AuthorEmail}{\AuthorEmail}
}
\institute{\ShortOrganisationName}

%----------------------------------------------------------
% Создание заглавной страницы
\frame{%
\nobibliography{bibliography}
%\nocite{*}
\titlepage} 
\setcounter{framenumber}{0}
%----------------------------------------------------------

%---------------------------------------------
\section*{Содержание доклада}
%---------------------------------------------
\begin{frame}

{\smaller[1]
\setcounter{tocdepth}{1}
\tableofcontents}
\setcounter{tocdepth}{2}

\end{frame}
%---------------------------------------------
\section{Введение}
%----------------------------------------------------------
\chapter*{ВВЕДЕНИЕ}\label{chap.introduction}
\addcontentsline{toc}{chapter}{ВВЕДЕНИЕ}

% --------------------------------------------------------
% Определения
\newglossaryentry{GraphOrientedApproach}{
    name={Графоориентированный подход},
    description={подход к организации процесса решения сложной вычислительной задачи, состоящего из нескольких процессов обработки данных, подразумевающий представление данного решения в виде графа}
    }

%----------------------------------------------------------

Процесс работы со сложными САПР и системами инженерного анализа для пользователя обычно сопряжен с требованием глубоких знаний предметной области решаемых задач. Обычно многие вычислительные задачи, которые требуется решать в процессе проектирования сложных технических объектов, предполагают процедуры пре- и пост- процессинга, и кроме того процедуры обработки данных, каждая из которых может включать необходимость запуска множества специальных функций, назначение и принципы работы которых пользователь вынужден предварительно изучить, пользуясь документацией по системе. В связи с разнообразностью вычислительных задач и часто изменяющимися требованиями соответствующее программное обеспечение претерпевает изменения, что также должно отражаться в документации \cite{SokolovPershin2020}.
\begin{statement}
    Одним из возможных путей к упрощению процесса применения и изучения наукоемкого ПО служит исключение необходимости его изучения за счет автоматизации самопроцедуры его использования. Такая автоматизация предполагает: а) формализацию метода организации вычислительных процессов в автоматизированной системе; б) определение классов прикладных задач, поддерживаемых в системе, и сопоставление с каждым классом формального описания метода решения; в) ограничение доступа к функциональным возможностям системы, обеспечивающим решение отдельных подзадач.\cite{SokolovGolubev2021}
\end{statement}
\begin{statement}
    Многие известные методики предполагают организацию вычислительных процессов в графовой форме. Например, применяют диаграммы потоков данных (DFD), граф-схемы, конечные автоматы, диаграммы перехода состояний. Такое описание позволяет выделять структурные единицы приложения в виде функций или подпрограмм и связывать их между собой в определенной последовательности.
\end{statement}
Более подробное описание применяемых методик можно найти в \cite{SokolovGolubev2021}. В данном разделе внимание будет сосредоточено на сравнении нескольких конкретных уже реализованных продуктов для решения различных задач проектирования, где вычислительные процессы организованы в графовой форме.
К сравнению были выбраны и рекомендованы следующие программные комплексы:
\begin{enumerate}
    \item Pradis - разработка отечественной компании "Ладуга"
    \item pSeven - разработка отечественной компаниии DATADVANCE
    \item GBSE - разработка группы преподавателей и студентов МГТУ им. Баумана
\end{enumerate}
Для сравнения были выделены следующие группы признаков:
\begin{itemize}
    \item Общие признаки;
    \item Признаки, относящиеся к топологии создаваемых графов, описывающих процессы обработки данных;
    \item Признаки, относящиеся к обходу данных графов.
\end{itemize}
К первой группе относятся такие признаки, как спектр задач, особенности работы с входными и выходными данными, файловая структура проектов. Ко второй группе относятся формат описания графов, подход к их формированию, возможность включения одного графа в состав другого, особенности передачи параметров между узлами, наличие поддержки ветвлений и циклов. К третьей группе признаков относятся поддержка параллельной обработки данных, поддержка распределённого выполнения, особенности ввода дополнительных данных Особенности ввода дополнительных данных и взаимодействия с пользователем в процессе обработки данных, особенности отбора корректных результатов расчета вручную, возможности доопределять значения входных данных в процессе обхода графа.

Программный комплекс Pradis, разработанный отечественной компанией "Ладуга", был рекомедован к обзору и сравнению, однако после проведённого обзора официальной документации\cite{PradisGeneral2007}\cite{PradisMethods2007}, не было получено достаточного представления об использовании графооринтированного подхода в данном копмлексе, поэтому было принято решение исключить его из дальнейшего рассмотрения.

Что касается программного комплекса pSeven, разработанном компанией DATADVANCE, используется методология диаграмм потоков данных, т.е. топология графа, описывающего процесс решения некоторой задачи проектирования, определяется только зависимостями между входными и выходными данными каждого отдельного процесса обработки данных, входящиего в решение. \cite{Nazarenko2015} В реализованном в pSeven подходе вводятся следующие понятия:
\begin{itemize}
    \item \emph{Расчётная схема (workflow)} - формальное описание процесса решения некоторой задачи в виде ориентированного графа;
    \item \emph{Блок} - программный контейнер для некоторого процесса обработки данных, входные и выходные данные для которого задаются через порты;
    \item \emph{Порт} - переменная определённого типа, имеющая определённое имя, привязанная к блоку;
    \item \emph{Связь} - направленное соединение типа "один к одному" между входным и выходным портами разных блоков;
\end{itemize}

С учётом данных понятий можно описать методологию диаграмм потов данных следующим образом. Расчётная схема содержит в себе набор процессов обработки данных (блоков), каждый из которых имеет (возможно, пустой) набор именованных входов и выходов (портов). Данные передаются через связи. Для избежания гонок данных множественные связи с одним и тем же входным портом не поддерживаются. Для начала выполнения каждому блоку требуются данные на всех входных портах. Все данные на выходных портах формируются по завершении исполнения блока.\cite{Nazarenko2015}

Все порты, которые не привязаны к другим блокам, автоматически становятся внешними входами и выходами для всей расчётной схемы. Для начала обхода расчётной схемы должен быть предоставлен набор входных данных и указаны внешние выходные порты, значения которых обязательно должны быть вычислены в результате обхода. Он производится в несколько этапов: сперва отслеживаются пути от необязательных выходных портов к входным, все встреченные на пути блоки помечаются, как неактуальные и не будут выполнены в дальнейшем; затем отслеживаются пути от обязательных выходных портов к входным и все встреченные на пути блоки помечаются, как обязательные к исполнению. Наконец обязательные к исполнению блоки запускаются, начиная с тех, которые подключены к внешним входам расчётной схемы, а неактуальные игнорируются. Обход прекращается, когда не остаётся необходимых для выполнения блоков. \cite{Nazarenko2015}

Результаты проведённого сравнения были оформлены в общую таблицу, приведённую ниже.
\noindent\begin{longtable}{|p{3.5cm}|p{6.5cm}|p{6.5cm}|}
    \caption{Сравнительная таблица \label{thetable}} \\
    \hline
    \textbf{Признак} & \textbf{pSeven} & \textbf{GBSE} \\
    \hline
    Cпектр задач & Задачи оптимизации, анализ данных & Задачи автоматизированного проектирования, анализ данных \\
    \hline
    Подход к формированию графа & Согласно описанному в \cite{Nazarenko2015} подходу, узлами графа являются блоки, рёбрами - связи, по которым передаются данные. & Узлами графа являются состояния данных, рёбрами - переходы между состояниями, к которым привязываются функции-обработчики. \cite{SokolovPershin2018} \\
    \hline
    Формат описания графа & Сформированное описание сохраняеся в двоичном файле закрытого формата с расширением \textsf{.p7wf} & Описание графа и функций-обработчиков сохраняется в текстовом файле специального формата \textsf{.aDOT}, являющегося расширением формата DOT\cite{SokolovPershin2018} \\
    \hline
    Файловая структура проекта & Проект состоит из непосредственно файла проекта, в котором хранятся ссылки на созданные расчётные схемы и базу данных, сами расчётные схемы, файлы с их входными данными, файлы отчётов, где сохраняются выходные данные последних расчётов и результаты их анализа. & Проект состоит из \textsf{.aDOT} файла с описанием графа, \textsf{.aINI}-файлов с описанием входных данных, библиотеки функций-обработчиков, файлов, куда записываются выходные данные. \\
    \hline
    Особенности работы с входными и выходными данными & Входные данные должны быть указаны при настройках внешних входных портов расчётной схемы. Данные с выходных портов схемы сохраняются в локальной базе данных. Для их записи в файлы для обработки/анализа вне pSeven необходимо воспользоваться специально предназначенными для этого блоками. & Входные данные хранятся в файле с расширением .aINI, откуда считываются при запуске обхода графа\cite{SokolovPershin2017}. Для записи выходных/промежуточных данных в файлы или базы данных необходимо добавить соответствующие функции-обработчики. \\
    \hline
    Особенности передачи параметров между узлами & Данные между узлами передаются через связи, которые на уровне выполнения создают пространство в памяти для ввода и вывода данных для выполняемых в раздельных процессах блоков.  Транзитная передача данных, которые не изменяются в данном блоке, на выход невозможна. & Поскольку узлами графа являются состояния данных, существует возможность задействовать в расчётах только часть данных, оставляя их другую часть без изменений \\
    \hline
    Поддержка ветвлений и циклов & Присутствует. Достигается засчёт специальных управляющих блоков, которые отслеживают выполнение условий & Присутствует по умолчанию\\
    \hline
    Поддержка параллельной обработки данных & Присутствует. Блоки, входящие в состав различных ветвлений схемы могут быть выполнены параллельно, поскольку они не зависят друг от друга по используемым данным. & Присутствует. Существует возможность обойти различные ветвления графа одновременно.\\
    \hline
    Особенности отбора корректных результатов расчёта вручную & Производится на этапе анализа результатов с помощью отчётов, где можно задать фильтрацию выходных данных по указанным параметрам. В случае, если результаты являются промежуточными, расчётную схему приходится разбивать на части. & Планируется реализовать средство визуализации данных, которое вкупе с автоматической генерацией форм ввода позволят отбирать корректные результаты промежуточных вычислений во время обхода одного цельного графа. \\
    \hline
    Возможность доопределения значений входных данных в процессе обхода графа & Отсутствует & Реализована при помощи функций-обработчиков, создающих формы ввода \\
    \hline
\end{longtable}
%Во введении должны быть представлены: введение в проблему, описание объекта исследований, обзор научно-технических источников\footnote{Следует изучать источники следующих типов в следующем порядке по убыванию приоритетности: \textbf{научные статьи}, патенты, электронные источники, книги (общее количество не менее 15)} по направлению поставленной задачи, примеры существующих аналогичных научно-технических решений.

%\textbf{Целью} обзора научно-технических источников является \textbf{обоснование актуальности} решения поставленной задачи.

%Обоснование актуальности предполагает проведение обзора литературы. Обзор литературы рекомендуется осуществлять, используя инструкцию\footnote{Инструкция о проведении обзора литературы: \url{https://archrk6.bmstu.ru/index.php/f/2597}}.

%В состав материалов проводимого обзора литературы должны включаться выводы/заключения, ставшие результатом анализа соответствующих источников, на которые при этом обязательно следует делать ссылки (например, так). Источник, при этом, следует включать в список литературы в последний раздел настоящего документа (в настоящем документе список источников формируется автоматически с помощью компилятора \textsf{BibTeX} на основании файла \textsf{bibliography.bib} и ссылок по тексту).

%В результате анализа всех источников должно стать возможным сделать вывод об обоснованности работ в направлении поставленной задачи.

%\begin{remark}
%Отметим, что в процессе подготовке текста возникает необходимость вводить аббревиатуры и использовать специальные термины, которые для документов большого объёма, выносятся в отдельные разделы ``Сокращения'' и ``Определения''. При использовании \LaTeX\xspace нет необходимости формировать этим разделы специально, -- рекомендуется использовать т.н. глоссарии. Создаётся файл \textsf{abbreviations.tex}, вносятся в него все необходимые термины и аббревиатуры и далее в любом месте текста ссылаются на них с использованием команды {\verb_\gls{SID}_} с одновременным появлением соответствующей расшифровки термина в соответствующем разделе (например, вызов команды {\verb_\gls{IND}_} приведёт к формированию \gls{IND}).
%\end{remark}


%\textbf{В последнем абзаце} введения следует указывать цель работы в целом.

%\underline{Обязательность представления:} раздел обязателен. 

% \underline{Объём:} как правило, не должен быть больше 5-7 страниц.

%----------------------------------------------------------

%---------------------------------------------
\section{Постановка задачи}
%----------------------------------------------------------
\chapter{Постановка задачи}
%----------------------------------------------------------
\section{Требования к графовому модулю comsdk}\label{section:requirements}
Концептуальная постановка задачи.

Для описания процесса решения задачи в виде графа в программном каркасе GBSE был разработан специальный текстовый формат \gls{aDOT}. Он построен на основе распространённого языка описания графов Dot (Graphviz)\cite{GraphvizDot2022}. Основным его отличием является необходимость хранить помимо узлов, рёбер и их атрибутов дополнительную информацию о вызываемых в процессе обхода графовой модели функций, а именно:
\begin{itemize}
    \item тип функции -- обработчик, предикат, селектор;
    \item путь к библиотеке, из которой можно извлечь данную функцию;
    \item идентификатор.
\end{itemize}
Подробное описание этого формата приведено в~\cite{SokolovADOT2020}.
Ниже приведён пример описания графовой модели на языке aDot (листинг \ref{lst:aDotExample})
\begin{lstlisting}[label={lst:aDotExample}, caption={Пример описания графовой модели на языке aDot}]
digraph SIMPLEST {
    FUNCA [module=libtest, entry_func=IncA]
    FUNCB [module=libtest, entry_func=IncB]
    
    CHECKA [module=libtest, entry_func=CheckAEq4]
    CHECKB [module=libtest, entry_func=CheckBEq4]
    
    SETA [module=libtest, entry_func=SetAEq1]
    SETB [module=libtest, entry_func=SetBEq1]

    PASS [module=libtest, entry_func=PassFunc]
    PRED [module=libtest, entry_func=PassPred]

    INCR_A [predicate=PRED, function=FUNCA]
    INCR_B [predicate=PRED, function=FUNCB]
    CH_A [predicate=CHECKA, function = PASS]
    SET_A [predicate=PRED, function=SETA]
    SET_B [predicate=PRED, function=SETB]
    CH_B [predicate=CHECKB, function = PASS]
    
    __BEGIN__ -> ROT [morphism=SET_A]
    ROT -> ROOT[morphism=SET_B]
    ROOT ->  BR1, BR2 [morphism=(INCR_A, INCR_B)]
    BR1 -> BR1_ST [morphism=INCR_A]
    BR2 -> BR2_ST [morphism=INCR_B]
    BR1_ST, BR2_ST -> MERGE [morphism=(INCR_A, INCR_B)]
    MERGE -> __END__, __END__ [morphism=(CH_A, CH_B)]
}
\end{lstlisting}

В данном описании объявляются функции-обработчики \textsf{PASS}, \textsf{FUNCA}, \textsf{FUNCB}, \textsf{SETA} и \textsf{SETB} и функции-предикаты \textsf{CHECKA}, \textsf{CHECKB} и \textsf{PRED}, которые можно найти в библиотеке \textsf{libtest}. Кроме того, объявляются морфизмы, содержащие предикаты и обработчики. На рисунке~\ref{fig:aDotExample} изображено визуальное представление модели, описанной выше с некоторыми пояснениями.
\begin{figure}[!ht]
    \centering
    \includegraphics[width=\textwidth]{figures/adot_example.png}
    \caption{Пример графовой модели}
    \label{fig:aDotExample}
\end{figure}

В результате анализа текущего синтаксиса языка aDOT были сформулированы следующие требования к структуре графовых моделей в новой версии графового модуля библиотеки comsdk:
\begin{enumerate}[1)]
    \item Каждое ребро графа должно иметь возможность привязать к нему до трёх морфизмов -- препроцессор, обработчик и постпроцессор;
    \item Каждый морфизм должен содержать в себе функцию-предикат и функцию-обработчик;
    \item Каждый узел графа должен хранить состояние данных (т.е. сведения о типах и именах переменных);
    \item Каждый узел графа должен иметь возможность привязать к нему функцию-селектор;
    \item Каждый узел графа должен хранить данные о стратегии выполнения рёбер, исходящих из него (поочерёдное выполнение, выполнение в отдельных потоках, выполнение в отдельных процессах, выполнение на удалённом узле через SSH-подключение);
\end{enumerate}

%----------------------------------------------------------
\section{Анализ текущей версии графового модуля comsdk}
%----------------------------------------------------------
На рисунке~\ref{fig:oldGraphStructure} представлена UML-диаграмма классов, связанных с представлением в comsdk ориентированного графа, описывающего организацию вычислительных процессов.

\begin{figure}[!ht]
    \centering
    \includegraphics[width=\textwidth]{figures/structure.png}
    \caption{Текущая структура классов, связанная с графовыми моделями в comsdk}
    \label{fig:oldGraphStructure}
\end{figure}

В существующей структуре классов можно выделить следующие недостатки:
\begin{enumerate}[1)]
    \item Отсутствует класс графа, который давал бы удобный интерфейс графовым моделям.
    \item Отсутствует контейнер, который бы инкапсулировал все узлы, относящиеся к конкретной графовой модели.
    \item Индекс узла графа задаётся пользователем при инициализации, что не гарантирует его уникальности.
    \item Отсутствует контейнер, который бы инкапсулировал все рёбра, относящиеся к конкретной графовой модели.
    \item Отсутствует объект, который бы описывал связи между узлами и рёбрами; вместо этого эти связи прописаны в самих узлах и рёбрах, что затрудняет операции с графовой моделью (преобразования и проч.).
    \item Функции-предикаты привязываются к узлам, а не к рёбрам, что не соответствует новым требованиям.
    \item В текущей версии задачей функций-предикатов фактически является отбор рёбер, которые должны быть выполнены, а не проверка соответствия данных в узле определённому формату.
\end{enumerate}

Таким образом, процесс разработки новой структуры графового модуля библиотеки comsdk должен быть направлен как на выполнение требований, описанных в разделе~\ref{section:requirements}, так и на устранение недостатков текущей версии, описанных выше.
%---------------------------------------------
\section{Архитектура программной реализации}
%%%%%%%%%%%%%%%%%%%%%%%%%%%%%%%%%%%%%%%%%%%%%%%%%%%%%%%%%%%%%%%%
\subsection{Текущая реализация описанного подхода}
%%%%%%%%%%%%%%%%%%%%%%%%%%%%%%%%%%%%%%%%%%%%%%%%%%%%%%%%%%%%%%%%
\begin{frame}
	В библиотеке comsdk, являющейся реализацией описанного выше подхода на языке C++, помимо прочих описаны следующие структуры данных:
	\begin{itemize}
		\item \textsf{Anymap} -- ассоциативный массив, позволяющий хранить в себе разнотипные данные;
		\item \textsf{ActionItem} -- функциональный объект, реализующий функцию-обработчик;
		\item \textsf{ActionItemContext} -- объект, осуществляющий запуск функций-обработчиков и хранящий данные об их выполнении;
		\item \textsf{Predicate} -- функциональный объект, являющийся обёрткой над некоторой функцией, ставящей в соответствие входному набору данных логическое значение (0 или 1);
	\end{itemize}	
\end{frame}

%%%%%%%%%%%%%%%%%%%%%%%%%%%%%%%%%%%%%%%%%%%%%%%%%%%%%%%%%%%%%%%%
\subsection{Предлагаемая структура данных узла графовой модели}
%%%%%%%%%%%%%%%%%%%%%%%%%%%%%%%%%%%%%%%%%%%%%%%%%%%%%%%%%%%%%%%%
\begin{frame}%[allowframebreaks=0.9,t]
	Структуры данных разрабатывались с учётом возможностей стандарта C++-11 и библиотеки Standart Template Library (STL). Для описания разработанных структур данных был использован язык графического описания UML (Unified Modeling Language).
	\begin{figure}
		\begin{minipage}{0.49\textwidth}
			\centering
			\includegraphics[width=\textwidth]{images/class.node.png}
			\caption{UML-диаграмма класса узла графа}
		\end{minipage}\hfill\begin{minipage}{0.49\textwidth}
			\begin{itemize}
				\item Использует умные указатели (\textsf{shared_ptr}) -- эффективное использование памяти.
				\item Защищённый конструктор -- объекты данного класса создаются только в пределах класса графовой модели.
				\item Интерфейс и информационные поля соответствуют требованиям.
			\end{itemize}
		\end{minipage}
	\end{figure}
\end{frame}
%%%%%%%%%%%%%%%%%%%%%%%%%%%%%%%%%%%%%%%%%%%%%%%%%%%%%%%%%%%%%%%%
\subsection{Предлагаемая структура данных ребра графовой модели}
%%%%%%%%%%%%%%%%%%%%%%%%%%%%%%%%%%%%%%%%%%%%%%%%%%%%%%%%%%%%%%%%
\begin{frame}%[allowframebreaks=0.9,t]
	\begin{figure}
		\begin{minipage}{0.49\textwidth}
			\begin{itemize}
				\item Использует умные указатели (\textsf{shared_ptr}) -- эффективное использование памяти.
				\item Защищённый конструктор -- объекты данного класса создаются только в пределах класса графовой модели.
				\item Интерфейс и информационные поля соответствуют требованиям.
				\item Для прохода по ребру необходимо назначить ему хотя бы одну пару предикат-обработчик (объединённых в структуру данных \textsf{Morphism}) с помощью метода set_function.
			\end{itemize}
		\end{minipage}\hfill\begin{minipage}{0.49\textwidth}
			\centering
			\includegraphics[height=0.75\textheight]{images/class.edge.png}
			\caption{UML-диаграмма класса ребра графа}
		\end{minipage}
	\end{figure}
\end{frame}

%%%%%%%%%%%%%%%%%%%%%%%%%%%%%%%%%%%%%%%%%%%%%%%%%%%%%%%%%%%%%%%%
\subsection{Предлагаемая структура данных графовой модели}
%%%%%%%%%%%%%%%%%%%%%%%%%%%%%%%%%%%%%%%%%%%%%%%%%%%%%%%%%%%%%%%%
\begin{frame}%[allowframebreaks=0.9,t]
	\begin{figure}
		\begin{minipage}{0.49\textwidth}
			\centering
			\includegraphics[width=0.8\textwidth]{images/class.graph.png}
			\caption{UML-диаграмма класса графа}
		\end{minipage}\hfill\begin{minipage}{0.49\textwidth}
			\begin{itemize}
				\item Хранит в себе узлы и рёбра графовой модели.
				\item Хранит в себе динамическую матрицу смежности.
				\item Позволяет добавлять узлы и рёбра и связывать их.
				\item Позволяет запустить обход графовой модели с заданным набором входных данных.
			\end{itemize}
		\end{minipage}
	\end{figure}
\end{frame}
%%%%%%%%%%%%%%%%%%%%%%%%%%%%%%%%%%%%%%%%%%%%%%%%%%%%%%%%%%%%%%%%
\subsection{Дополнительные структуры данных}
%%%%%%%%%%%%%%%%%%%%%%%%%%%%%%%%%%%%%%%%%%%%%%%%%%%%%%%%%%%%%%%%
\begin{frame}
	\begin{figure}
		\begin{minipage}{0.49\textwidth}
			\begin{itemize}
				\item NodeOp -- сокращение от ``Node Operation''.
				\item Дополнительная структура данных для операций с узлами.
				\item Хранит только индекс узла -- эффективное использование памяти.
				\item Позволяет получить соседние узлы для данного узла;
				\item Позволяет получить входящие и исходящие рёбра для данного узла;
				\item Позволяет при необходимости обратиться к интерфейсу самого узла.
			\end{itemize}
		\end{minipage}\hfill\begin{minipage}{0.49\textwidth}
			\centering
			\includegraphics[width=0.75\textwidth]{images/class.nodeop.png}
			\caption{UML-диаграмма класса операции с узлом}
		\end{minipage}
	\end{figure}
\end{frame}

\begin{frame}
	\begin{figure}
		\begin{minipage}{0.49\textwidth}
			\begin{itemize}
				\item EdgeOp -- сокращение от ``Edge Operation''
				\item Дополнительная структура данных для операций с рёбрами.
				\item Хранит только индекс ребра -- эффективное использование памяти.
				\item Позволяет получить начальный и конечный узлы для данного ребра.
				\item Позволяет при необходимости обратиться к интерфейсу самого ребра.
			\end{itemize}
		\end{minipage}\hfill\begin{minipage}{0.49\textwidth}
			\centering
			\includegraphics[width=0.75\textwidth]{images/class.edgeop.png}
			\caption{UML-диаграмма класса операции с ребром}
		\end{minipage}
	\end{figure}
\end{frame}
%%%%%%%%%%%%%%%%%%%%%%%%%%%%%%%%%%%%%%%%%%%%%%%%%%%%%%%%%%%%%%%%

%---------------------------------------------
\section*{Заключение}
%----------------------------------------------------------
\chapter*{ЗАКЛЮЧЕНИЕ}\label{chap_conclusion}
\addcontentsline{toc}{chapter}{ЗАКЛЮЧЕНИЕ}
%----------------------------------------------------------

В результате выполнения данной работы были собраны сведения о необходимых для реализации средствах взаимодействия пользователя в системе автматизированного решения исследовательских задач GBSE, что даёт направление для дальнейшей разработки и программной реализации данных средств. Кроме того, были рассмотрены некоторые существующие на рынке аналоги GBSE, в частности, продукт pSeven, и была проведена сравнительная характеристика данных программных комплексов с учётом возможностей взаимодействия пользователя с процессом решения задач.

%----------------------------------------------------------

%---------------------------------------------
\def\secname{}

\begin{frame}

\begin{center}
\LARGE
Спасибо за внимание!
\vspace{1cm}

Вопросы?
\end{center}

\end{frame}
%---------------------------------------------
\section*{Приложение. Основная терминология}
%---------------------------------------------
\begin{frame}[allowframebreaks=0.9,t]

\printglossary[style=index, type=\acronymtype, title=Аббревиатуры, nopostdot=false]
%\printglossary[style=index, type=main, title=Термины и определения, nopostdot=true]

\end{frame}
%---------------------------------------------
\end{document}
%---------------------------------------------



