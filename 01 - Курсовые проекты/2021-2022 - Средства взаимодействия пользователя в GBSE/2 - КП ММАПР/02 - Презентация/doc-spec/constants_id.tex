%----------------------------------------------------------
\newcommand{\doctype}{presentation} % о научно-исследовательской работе / об опытно-конструкторской работе / об опытно-технологической работе / о патентных исследованиях
\newcommand{\doctypesid}{prs} % prs (Презентация) / vkr (выпускная квалификационная работа) / kp (курсовой проект) / kr (курсовая работа) / nirs (научно-исследовательская работа студента) / nkr (научно-квалификационная работа)
%----------------------------------------%
\newcommand{\Title}{Проектирование структуры графовых моделей в программном комплексе GBSE}
\newcommand{\SubTitle}{} % Методы оптимизации
%----------------------------------------------------------
%\newcommand{\eventplace}{факультет \facultyShort, кафедра \department} % Место проведения мероприятия
\newcommand{\eventtype}{} % Тип мероприятия (лекция|лабораторная работа|семинар|мозговой штурм)
\newcommand{\eventplace}{} % Место проведения мероприятия
\newcommand{\eventduration}{7 минут} % Продолжительность мероприятия
%----------------------------------------------------------
\newcommand{\Author}{Тришин~И.В.}
\newcommand{\AuthorFull}{Тришин Илья Вадимович}
\newcommand{\AuthorEmail}{}
\newcommand{\ScientificAdviserPosition}{@должность научного руководителя@}	% Должность научного руководителя
\newcommand{\ScientificAdviser}{Соколов~А.П.}	% Научный руководитель
\newcommand{\ConsultantA}{Першин~А.Ю.}				% Консультант 1
\newcommand{\group}{РК6-71Б}
\newcommand{\Semestr}{осенний семестр} % Например: осенний семестр или весенний семестр
\newcommand{\begdate}{}
\newcommand{\Year}{2022}
\newcommand{\country}{Россия}	% Страна, в которой проводится конференция
\newcommand{\city}{Москва}		% Город, в котором проводится конференция
%----------------------------------------%
\newcommand{\depHeadPosition}{Заведующий кафедрой}		% Должность руководителя подразделения
\newcommand{\depHeadName}{А.П.~Карпенко}		% Должность руководителя подразделения
%----------------------------------------%
\newcommand{\presentationtitle}{\Title}
\newcommand{\conferenceperiod}{\country, \city, \begdate -- \today}
%----------------------------------------%

% Цель исследования/доклада 
\newcommand{\GoalOfResearch}{выявить существующие недостатки в реализации описанного метода и предложить программную архитектуру, которая бы позволила их устранить} % с маленькой буквы и без точки на конце

% Объектом исследования называют то, что исследуется в работе. Например, для указанной выше темы объектом может быть популяция медуз, но никак ни модель SIS, ни Южно-Китайское море, ни метод моделирования популяции медуз. 
\newcommand{\ObjectOfResearch}{программный каркас GBSE}

% Предмет исследований (уже чем объект, определяется, исходя из задач: формулируется как существительное, как правило, во множественном числе, определяющее "конкретный объект исследований" среди прочих в рамках более общего)
\newcommand{\SubjectOfResearch}{@Предмет исследований@}

% Основная задача, на решение которой направлена работа
\newcommand{\MainProblemOfResearch}{@Основная задача, на решение которой направлена работа@}

% Выполненные задачи
\newcommand{\SubtasksPerformed}{%
В результате выполнения работы: 
\begin{inparaenum}[1)]
	\item предложено ...;
	\item создано ...;
	\item разработано ...;
	\item проведены вычислительные эксперименты ...
\end{inparaenum}}

% Ключевые слова (представляются для обеспечения потенциальной возможности индексации документа)
\newcommand{\keywordsru}{%
	@keywordsru@} % 5-15 слов или выражений на русском языке, для разделения СЛЕДУЕТ ИСПОЛЬЗОВАТЬ ЗАПЯТЫЕ
\newcommand{\keywordsen}{%
	@keywordsen@} % 5-15 слов или выражений на английском языке, для разделения СЛЕДУЕТ ИСПОЛЬЗОВАТЬ ЗАПЯТЫЕ

% Краткая аннотация
\newcommand{\Preface}{@Начать можно так: ``Работа посвящена...''. Объём около 0.5 страницы. Здесь следует кратко рассказать о чём работа, на что направлена, что и какими методами было достигнуто. Реферат должен быть подготовлен так, чтобы после её прочтения захотелось перейти к основному тексту работы.@} % с большой буквы с точкой в конце

%----------------------------------------%
% выходные данные по документу
\newcommand{\DocOutReference}{\Author. \Title./\SubTitle. [Электронный ресурс] --- \City: \Year. --- \total{page} с. URL:~\url{https://\gitlabdomain} (система контроля версий кафедры РК6)}
%----------------------------------------------------------

