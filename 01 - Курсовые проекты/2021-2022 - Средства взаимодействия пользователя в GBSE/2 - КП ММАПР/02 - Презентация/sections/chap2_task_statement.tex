%%%%%%%%%%%%%%%%%%%%%%%%%%%%%%%%%%%%%%%%%%%
\subsection{Концептуальная постановка задачи}
%%%%%%%%%%%%%%%%%%%%%%%%%%%%%%%%%%%%%%%%%%%
\begin{frame}%[allowframebreaks=0.9,t]

\begin{block}{Объект исследований}
\textexample{\ObjectOfResearch}
\end{block}

\begin{block}{Цель исследования}
\textexample{\GoalOfResearch}
\end{block}

\begin{block}{Задачи исследования}
\begin{enumerate}
	\arrowitem{Сравнить подходы к реализации графоориентированного подхода к решению задач проектирования на примере нескольких существующих программных комплексов}
	\arrowitem{Исследовать программную структуру модуля каркаса GBSE, отвечающего за струкутуру графовых моделей}
	\arrowitem{Определить требования к структуре данного модуля}
	\arrowitem{Разработать новую структуру, которая бы отвечала сформулированным требованиям}
\end{enumerate}
\end{block}

\end{frame}

%%%%%%%%%%%%%%%%%%%%%%%%%%%%%%%%%%%%%%%%%%%
\subsection{Требования к программной реализации структуры данных <<графовая модель>>}
%%%%%%%%%%%%%%%%%%%%%%%%%%%%%%%%%%%%%%%%%%%
\begin{frame}%[allowframebreaks=0.9,t]

\begin{itemize}
	\item Должна обеспечиваться поддержка актуальной версии формата aDOT~\footcite{SokADOT}.
	\item Реализуемая структура данных должна содержать в себе все узлы и рёбра, относящиеся к данной графовой модели.
	\item Топология графовой модели должна описываться матрицей смежности.
	\item Реализуемая структура данных должна предоставлять интерфейс для доступа к данным о топологии.
\end{itemize}

\end{frame}

%%%%%%%%%%%%%%%%%%%%%%%%%%%%%%%%%%%%%%%%%%%
\subsection{Требования к программной реализации структуры данных узла графовой модели}
%%%%%%%%%%%%%%%%%%%%%%%%%%%%%%%%%%%%%%%%%%%
\begin{frame}[t]
	\begin{itemize}
		\arrowitem{Реализуемая струкутра данных должна содержать в себе:}
		\begin{itemize}
			\item описание состояния данных (в некотором внутреннем формате);
			\item стратегию параллельного исполнения\footnote{т.е. данные о том, как параллельно исполнять исходящие из данного узла рёбра};
			\item функцию-селектор (или ссылку на неё).
		\end{itemize}
		\arrowitem{Реализуемая структура данных должна предоставлять интерфейс для:}
		\begin{itemize}
			\item назначения стратегию параллельного выполнения;
			\item назначения функции-селектора;
			\item вызова функции селектора с заданным набором данных.
		\end{itemize}
	\end{itemize}
	\begin{figure}
		\includegraphics[height=0.4\textheight]{images/graph.node_data.png}
	\end{figure}
\end{frame}

%%%%%%%%%%%%%%%%%%%%%%%%%%%%%%%%%%%%%%%%%%%
\subsection{Требования к программной реализации структуры данных ребра графовой модели}
%%%%%%%%%%%%%%%%%%%%%%%%%%%%%%%%%%%%%%%%%%%
\begin{frame}%[allowframebreaks=0.9,t
	\begin{itemize}
		\arrowitem{Реализуемая структура должна содержать в себе до трёх структур данных (или ссылок на них) типа <<морфизм>> -- препроцессор, обработчик и постпроцессор.}
		\arrowitem{Функциональная структура данных типа <<морфизм>> должна содержать в себе:}
		\begin{itemize}
			\item ссылку на функцию-предикат;
			\item ссылку на функцию-обработчик;
		\end{itemize}
		\arrowitem{Реализуемая структура должна предоставлять интефейс для:}
		\begin{itemize}
			\item назначения препроцессора, обработчика и постпроцессора;
			\item вызова хранящихся в ней <<морфизмов>> с заданным входным набором данных.
		\end{itemize}
	\end{itemize}
	\begin{figure}
		\includegraphics[height=0.275\textheight]{images/graph.edge_data.png}
	\end{figure}
\end{frame}

