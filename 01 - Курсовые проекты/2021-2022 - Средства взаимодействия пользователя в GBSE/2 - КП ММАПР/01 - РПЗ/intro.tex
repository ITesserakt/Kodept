%----------------------------------------------------------
\chapter*{ВВЕДЕНИЕ}\label{chap.introduction}
\addcontentsline{toc}{chapter}{ВВЕДЕНИЕ}

%----------------------------------------------------------
При проведении современных исследований возникает необходимость автоматизировать процессы решения сложных вычислительных задач. Достижение подобной цели не представляется возможным без формально определенного метода организации процессов в автоматизированной системе. Проектирование, создание и сопровождение подобных систем является трудоемкой задачей, для решения которой применяют инструментальные средства и среды разработки автоматизированных систем (\glsxtrshort {CASE}-системы)\cite{Golubev2020} В некоторых очень узко направленных системах подобная организация напрямую зависит от поставленной задачи. В более универсальных же системах, о которых пойдёт речь в этой работе, разрабатывается особая архитектура, которая позволяет организовать различные процессы для решения различных задач по-разному. Подходы к построению данной архитектуры освещены, помимо прочего, в~\cite{SokolovCADCMConcept2020}.

В данной работе рассматривается, в первую очередь т.н. графоориентированный подход к решению задач проектирования. Данный подход подразумевает организацию различных вычислительных процессов в виде графа. Многие известные методики предполагают организацию вычислительных процессов в графовой форме. Например, применяют: диаграммы потоков данных (\glsxtrshort{DFD}), граф-схемы, конечные автоматы, диаграммы перехода состояний. Такое описание позволяет выделять структурные единицы приложения в виде функций или подпрограмм и связывать их между собой в определенной последовательности\cite{SokolovGolubev2021}.

Одной из реализаций данного подхода является разработка Соколова А.П. и Першина А.Ю. графоориентированный программный каркас для решения задач проектирования \glsxtrshort{GBSE}. Данный программный каркас направлен на упрощение и структуризацию разработки прикладного программного обеспечения для решения описанных выше задач.

\begin{definition}
    \emph{Графовой моделью} процесса решения задачи назовём некоторое формализованное описание этого решения, где отдельные вычислительные процессы организуются в виде графа.
\end{definition}

При описании графовых моделей в GBSE вводятся следующие понятия:
\begin{itemize}
    \item \textit{состояние данных} -- некоторый строго определённый набор именованных переменных фиксированного типа, характерных для решаемой задачи;
    \item \textit{морфизм} -- некоторое отображение одного состояния данных в другое;
    \item \textit{функция-предикат} -- функция, определяющая соответствие подаваемого ей на вход набора данных тому виду, который требуется для выполнения отображения;
    \item \textit{функция-обработчик} -- функция, отвечающая за преобразование данных из одного состояния в другое;
    \item \textit{функция-селектор} -- функция, отвечающая в процессе обхода графовой модели за выбор тех рёбер, которые необходимо выполнить на следующем шаге в соответствии с некоторым условием.
\end{itemize}

Более подробное теоретическое описание концепции, реализованной в GBSE представлено в \cite{SokolovPershin2018}. В данной работе внимание сосредоточено в первую очередь на программной реализации этой концепции в библиотеке comsdk для языка программирования C++.
%----------------------------------------------------------
