% ----------------------------------------------------------------
\chapter*{ВВЕДЕНИЕ}\label{chap.introduction}
\addcontentsline{toc}{chapter}{ВВЕДЕНИЕ}

% ----------------------------------------------------------------
Современные научно-технические исследования зачастую включают в себя задачи, решения которых требуют большого количества вычислений. Для упрощения процесса решения используется или разрабатывается различное программное обеспечение (\glsxtrshort{ПО}). Например, могут применяться программные средства, предоставляющие универсальный базовый математический аппарат (MathCAD, Maple, Mathematica и~т.п.). Как правило, подобные программные продукты предоставляют некоторый формальный язык описания математических выражений или редактор формул. Преимущества и недостатки множества подобных программ освещены в~\cite{WikiAlgebra2022}.

При всех преимуществах применения подобных программ при решении сложных вычислительных задач за пользователем остаётся необходимость формулировать их математические постановки (т.е. формировать математические модели, составлять системы уравнений и~т.д.). Зачастую требуется решать множество задач с схожей постановкой, но с различными входными параметрами (как, например, при анализе прочностных характеристик технических объектов). Становится целесообразным разработать автоматизированные средства решения подобных типовых задач. При этом от разработчика требуются глубокие познания в предметной области задачи. Возникает потребность в некоторой промежуточной системе, позволяющей формально поэтапно описать метод решения некоторой задачи для удобства его последующей реализации. В данной работе внимание сосредоточено на подобных системах.

С точки зрения разработки ПО, при описании некоторого вычислительного метода целесообразно выделить в нём отдельные этапы, отдельные операции обработки данных. Каждой такой операции требуются входные данные. По завершении выполнения операции получаются выходные данные. При этом выходные данные одной операции могут являться входными для одной или нескольких других операций. Между ними формируются зависимости по входным и выходным данным. Для учёта этих зависимостей возникает необходимость правильным образом организовать выполнение операций в пределах отдельно взятого метода.

В наши дни популярность приобретает применение научных систем организации рабочего процесса (англ. scientific workflow systems). Такие системы позвояют автоматизировать процессы решения научно-технических задач, предоставляя средства организации и управления вычислительными процессами~\cite{DeelmanWorkflow2009}. Процесс работы с подобными системами состоит из 4 основных этапов:
\begin{enumerate}
    \item составление описания операций обработки данных и зависимостей между ними;
    \item распределение процессов обработки данных по вычислительным ресурсам;
    \item выполнение обработки данных;
    \item сбор и анализ результатов и статистики.
\end{enumerate}

Примерами подобных систем могут служить Pegasus\cite{DeelmanPegasus2016}, Kepler\cite{AltintasKepler2004} и pSeven\cite{NazarenkoDFM2015}.

Одной из ключевых особенностей подобного подхода к реализации решений научно-технических задач является выделение операций обработки данных в отдельные программные модули (функции, подпрограммы, скрипты). При известных входных и выходных данных каждого модуля становится возможной их независимая разработка\cite{DanilovPar2011}. Это позволяет распределить их разработку между несколькими людьми. Вследствие этого уменьшается объём работы по написанию исходных кодов, приходящийся на одного исследователя. Это в свою очередь облегчает отладку и написание документации, что положительно сказывается на общем качестве реализуемого ПО.

В основном, в научных системах организации рабочего процесса для описания связей между отдельными операциями обработки данных используются ориентированные графы. Помимо описания связей между вычислительными процессами ориентированные графы также находят применение при планировании деятельности (сетевые графики, граф-схемы). В научно-технической среде большее распространение получили сети Петри, диаграммы потоков данных (DFD) и диаграммы перехода состояний. В данной работе основное внимание отводится т.н. ``графоориентированному подходу''(\glsxtrshort{GBSE})\cite{SokolovPershin2018} и его реализации.
% ----------------------------------------------------------------
