%----------------------------------------------------------
\chapter*{ВВЕДЕНИЕ}\label{chap.introduction}
\addcontentsline{toc}{chapter}{ВВЕДЕНИЕ}

% --------------------------------------------------------
% Определения
\newglossaryentry{GraphOrientedApproach}{
    name={Графоориентированный подход},
    description={подход к организации процесса решения сложной вычислительной задачи, состоящего из нескольких процессов обработки данных, подразумевающий представление данного решения в виде графа}
    }

%----------------------------------------------------------

Процесс работы со сложными САПР и системами инженерного анализа для пользователя обычно сопряжен с требованием глубоких знаний предметной области решаемых задач. Обычно многие вычислительные задачи, которые требуется решать в процессе проектирования сложных технических объектов, предполагают процедуры пре- и пост- процессинга, и кроме того процедуры обработки данных, каждая из которых может включать необходимость запуска множества специальных функций, назначение и принципы работы которых пользователь вынужден предварительно изучить, пользуясь документацией по системе. В связи с разнообразностью вычислительных задач и часто изменяющимися требованиями соответствующее программное обеспечение претерпевает изменения, что также должно отражаться в документации \cite{SokolovPershin2020}.
\begin{statement}
    Одним из возможных путей к упрощению процесса применения и изучения наукоемкого ПО служит исключение необходимости его изучения за счет автоматизации самопроцедуры его использования. Такая автоматизация предполагает: а) формализацию метода организации вычислительных процессов в автоматизированной системе; б) определение классов прикладных задач, поддерживаемых в системе, и сопоставление с каждым классом формального описания метода решения; в) ограничение доступа к функциональным возможностям системы, обеспечивающим решение отдельных подзадач.\cite{SokolovGolubev2021}
\end{statement}
\begin{statement}
    Многие известные методики предполагают организацию вычислительных процессов в графовой форме. Например, применяют диаграммы потоков данных (DFD), граф-схемы, конечные автоматы, диаграммы перехода состояний. Такое описание позволяет выделять структурные единицы приложения в виде функций или подпрограмм и связывать их между собой в определенной последовательности.
\end{statement}
Более подробное описание применяемых методик можно найти в \cite{SokolovGolubev2021}. В данном разделе внимание будет сосредоточено на сравнении нескольких конкретных уже реализованных продуктов для решения различных задач проектирования, где вычислительные процессы организованы в графовой форме.
К сравнению были выбраны и рекомендованы следующие программные комплексы:
\begin{enumerate}
    \item Pradis - разработка отечественной компании "Ладуга"
    \item pSeven - разработка отечественной компаниии DATADVANCE
    \item GBSE - разработка группы преподавателей и студентов МГТУ им. Баумана
\end{enumerate}
Для сравнения были выделены следующие группы признаков:
\begin{itemize}
    \item Общие признаки;
    \item Признаки, относящиеся к топологии создаваемых графов, описывающих процессы обработки данных;
    \item Признаки, относящиеся к обходу данных графов.
\end{itemize}
К первой группе относятся такие признаки, как спектр задач, особенности работы с входными и выходными данными, файловая структура проектов. Ко второй группе относятся формат описания графов, подход к их формированию, возможность включения одного графа в состав другого, особенности передачи параметров между узлами, наличие поддержки ветвлений и циклов. К третьей группе признаков относятся поддержка параллельной обработки данных, поддержка распределённого выполнения, особенности ввода дополнительных данных Особенности ввода дополнительных данных и взаимодействия с пользователем в процессе обработки данных, особенности отбора корректных результатов расчета вручную, возможности доопределять значения входных данных в процессе обхода графа.

Программный комплекс Pradis, разработанный отечественной компанией "Ладуга", был рекомедован к обзору и сравнению, однако после проведённого обзора официальной документации\cite{PradisGeneral2007}\cite{PradisMethods2007}, не было получено достаточного представления об использовании графооринтированного подхода в данном копмлексе, поэтому было принято решение исключить его из дальнейшего рассмотрения.

Что касается программного комплекса pSeven, разработанном компанией DATADVANCE, используется методология диаграмм потоков данных, т.е. топология графа, описывающего процесс решения некоторой задачи проектирования, определяется только зависимостями между входными и выходными данными каждого отдельного процесса обработки данных, входящиего в решение. \cite{Nazarenko2015} В реализованном в pSeven подходе вводятся следующие понятия:
\begin{itemize}
    \item \emph{Расчётная схема (workflow)} - формальное описание процесса решения некоторой задачи в виде ориентированного графа;
    \item \emph{Блок} - программный контейнер для некоторого процесса обработки данных, входные и выходные данные для которого задаются через порты;
    \item \emph{Порт} - переменная определённого типа, имеющая определённое имя, привязанная к блоку;
    \item \emph{Связь} - направленное соединение типа "один к одному" между входным и выходным портами разных блоков;
\end{itemize}

С учётом данных понятий можно описать методологию диаграмм потов данных следующим образом. Расчётная схема содержит в себе набор процессов обработки данных (блоков), каждый из которых имеет (возможно, пустой) набор именованных входов и выходов (портов). Данные передаются через связи. Для избежания гонок данных множественные связи с одним и тем же входным портом не поддерживаются. Для начала выполнения каждому блоку требуются данные на всех входных портах. Все данные на выходных портах формируются по завершении исполнения блока.\cite{Nazarenko2015}

Все порты, которые не привязаны к другим блокам, автоматически становятся внешними входами и выходами для всей расчётной схемы. Для начала обхода расчётной схемы должен быть предоставлен набор входных данных и указаны внешние выходные порты, значения которых обязательно должны быть вычислены в результате обхода. Он производится в несколько этапов: сперва отслеживаются пути от необязательных выходных портов к входным, все встреченные на пути блоки помечаются, как неактуальные и не будут выполнены в дальнейшем; затем отслеживаются пути от обязательных выходных портов к входным и все встреченные на пути блоки помечаются, как обязательные к исполнению. Наконец обязательные к исполнению блоки запускаются, начиная с тех, которые подключены к внешним входам расчётной схемы, а неактуальные игнорируются. Обход прекращается, когда не остаётся необходимых для выполнения блоков. \cite{Nazarenko2015}

Результаты проведённого сравнения были оформлены в общую таблицу, приведённую ниже.
\noindent\begin{longtable}{|p{3.5cm}|p{6.5cm}|p{6.5cm}|}
    \caption{Сравнительная таблица \label{thetable}} \\
    \hline
    \textbf{Признак} & \textbf{pSeven} & \textbf{GBSE} \\
    \hline
    Cпектр задач & Задачи оптимизации, анализ данных & Задачи автоматизированного проектирования, анализ данных \\
    \hline
    Подход к формированию графа & Согласно описанному в \cite{Nazarenko2015} подходу, узлами графа являются блоки, рёбрами - связи, по которым передаются данные. & Узлами графа являются состояния данных, рёбрами - переходы между состояниями, к которым привязываются функции-обработчики. \cite{SokolovPershin2018} \\
    \hline
    Формат описания графа & Сформированное описание сохраняеся в двоичном файле закрытого формата с расширением \textsf{.p7wf} & Описание графа и функций-обработчиков сохраняется в текстовом файле специального формата \textsf{.aDOT}, являющегося расширением формата DOT\cite{SokolovPershin2018} \\
    \hline
    Файловая структура проекта & Проект состоит из непосредственно файла проекта, в котором хранятся ссылки на созданные расчётные схемы и базу данных, сами расчётные схемы, файлы с их входными данными, файлы отчётов, где сохраняются выходные данные последних расчётов и результаты их анализа. & Проект состоит из \textsf{.aDOT} файла с описанием графа, \textsf{.aINI}-файлов с описанием входных данных, библиотеки функций-обработчиков, файлов, куда записываются выходные данные. \\
    \hline
    Особенности работы с входными и выходными данными & Входные данные должны быть указаны при настройках внешних входных портов расчётной схемы. Данные с выходных портов схемы сохраняются в локальной базе данных. Для их записи в файлы для обработки/анализа вне pSeven необходимо воспользоваться специально предназначенными для этого блоками. & Входные данные хранятся в файле с расширением .aINI, откуда считываются при запуске обхода графа\cite{SokolovPershin2017}. Для записи выходных/промежуточных данных в файлы или базы данных необходимо добавить соответствующие функции-обработчики. \\
    \hline
    Особенности передачи параметров между узлами & Данные между узлами передаются через связи, которые на уровне выполнения создают пространство в памяти для ввода и вывода данных для выполняемых в раздельных процессах блоков.  Транзитная передача данных, которые не изменяются в данном блоке, на выход невозможна. & Поскольку узлами графа являются состояния данных, существует возможность задействовать в расчётах только часть данных, оставляя их другую часть без изменений \\
    \hline
    Поддержка ветвлений и циклов & Присутствует. Достигается засчёт специальных управляющих блоков, которые отслеживают выполнение условий & Присутствует по умолчанию\\
    \hline
    Поддержка параллельной обработки данных & Присутствует. Блоки, входящие в состав различных ветвлений схемы могут быть выполнены параллельно, поскольку они не зависят друг от друга по используемым данным. & Присутствует. Существует возможность обойти различные ветвления графа одновременно.\\
    \hline
    Особенности отбора корректных результатов расчёта вручную & Производится на этапе анализа результатов с помощью отчётов, где можно задать фильтрацию выходных данных по указанным параметрам. В случае, если результаты являются промежуточными, расчётную схему приходится разбивать на части. & Планируется реализовать средство визуализации данных, которое вкупе с автоматической генерацией форм ввода позволят отбирать корректные результаты промежуточных вычислений во время обхода одного цельного графа. \\
    \hline
    Возможность доопределения значений входных данных в процессе обхода графа & Отсутствует & Реализована при помощи функций-обработчиков, создающих формы ввода \\
    \hline
\end{longtable}
%Во введении должны быть представлены: введение в проблему, описание объекта исследований, обзор научно-технических источников\footnote{Следует изучать источники следующих типов в следующем порядке по убыванию приоритетности: \textbf{научные статьи}, патенты, электронные источники, книги (общее количество не менее 15)} по направлению поставленной задачи, примеры существующих аналогичных научно-технических решений.

%\textbf{Целью} обзора научно-технических источников является \textbf{обоснование актуальности} решения поставленной задачи.

%Обоснование актуальности предполагает проведение обзора литературы. Обзор литературы рекомендуется осуществлять, используя инструкцию\footnote{Инструкция о проведении обзора литературы: \url{https://archrk6.bmstu.ru/index.php/f/2597}}.

%В состав материалов проводимого обзора литературы должны включаться выводы/заключения, ставшие результатом анализа соответствующих источников, на которые при этом обязательно следует делать ссылки (например, так). Источник, при этом, следует включать в список литературы в последний раздел настоящего документа (в настоящем документе список источников формируется автоматически с помощью компилятора \textsf{BibTeX} на основании файла \textsf{bibliography.bib} и ссылок по тексту).

%В результате анализа всех источников должно стать возможным сделать вывод об обоснованности работ в направлении поставленной задачи.

%\begin{remark}
%Отметим, что в процессе подготовке текста возникает необходимость вводить аббревиатуры и использовать специальные термины, которые для документов большого объёма, выносятся в отдельные разделы ``Сокращения'' и ``Определения''. При использовании \LaTeX\xspace нет необходимости формировать этим разделы специально, -- рекомендуется использовать т.н. глоссарии. Создаётся файл \textsf{abbreviations.tex}, вносятся в него все необходимые термины и аббревиатуры и далее в любом месте текста ссылаются на них с использованием команды {\verb_\gls{SID}_} с одновременным появлением соответствующей расшифровки термина в соответствующем разделе (например, вызов команды {\verb_\gls{IND}_} приведёт к формированию \gls{IND}).
%\end{remark}


%\textbf{В последнем абзаце} введения следует указывать цель работы в целом.

%\underline{Обязательность представления:} раздел обязателен. 

% \underline{Объём:} как правило, не должен быть больше 5-7 страниц.

%----------------------------------------------------------
