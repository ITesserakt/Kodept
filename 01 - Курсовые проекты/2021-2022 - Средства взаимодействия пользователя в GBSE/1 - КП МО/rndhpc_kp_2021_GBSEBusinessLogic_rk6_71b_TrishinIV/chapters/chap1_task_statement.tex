%----------------------------------------------------------
\chapter{Постановка задачи}
%----------------------------------------------------------
\section{Концептуальная постановка задачи}

В разделе концептуальная постановка задачи должны быть представлены: объект исследований (разработки), цель исследования (разработки), кратко задачи (по пунктам, не более 8), исходные данные (если предусмотрены), что требуется получить.

\underline{Обязательность представления:} раздел обязателен. 

\underline{Объём:} как правило, не должен быть больше 1-2 страниц.

%----------------------------------------------------------
\section{Математическая постановка задачи (представляется в зависимости от задачи)}

Раздел математическая постановка задачи обязателен для проектов, предполагающих применение методов математического моделирования и, как следствие, проведение вычислительные экспериментов.

Если проект предполагает разработку программного обеспечения и не предполагает проведение вычислений, то этот раздел не обязателен.

В разделе математическая постановка задачи подробно по подразделам следует описать планируемые к применению математические модели, вычислительные методы. Следует описывать особые ситуации их применения, которые предполагается изучить.
Модели следует описывать с использованием математически строгих формулировок, не допускающих неоднозначности прочтения.

\underline{Обязательность представления:} раздел представляется в зависимости от задачи. 

\underline{Объём:} как правило, может составлять около 10 страниц.
%----------------------------------------------------------

