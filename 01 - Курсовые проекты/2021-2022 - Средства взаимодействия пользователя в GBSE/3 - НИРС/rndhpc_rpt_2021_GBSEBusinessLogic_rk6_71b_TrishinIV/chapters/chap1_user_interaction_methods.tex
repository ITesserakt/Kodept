% ---------------------------------------------------
\chapter{Обзор средств взаимодействия пользователя в графоориентированных системах} \label{chap1.comparison}
Среди вычислительных задач, с которыми сталкиваются современные исследователи и разработчики наукоёмкого программного обеспечения, можно выделить те, в которых в результате проведения расчётов получается несколько различных результатов, из которых требуется выбрать наиболее подходящий на основе каких-то критериев или провести ту или иную операцию в зависимости от полученных результатов. В наши дни наблюдается тенденция к автоматизации подобного процесса на основе различных алгоритмов анализа и принятия решений, некоторые из которых разрабатываются специально под конкретную задачу, а некоторые, более универсальные, адаптируются под неё, как, например, описано в \cite{KatalOpt2020}. Тем не менее, остаётся широкий спектр задач, где разработка подобных алгоритмов не ведётся ввиду слишком узкой направленности или отсутствии технической возможности автоматизировать принятие решений (как правило, в исследовательских задачах). В таких случаях за него отвечает лицо, принимающее решение (\glsxtrshort{ЛПР}). При разработке универсального программного комплекса, позволяющего решать различные задачи проектирования и оптимизации было бы полезно включить возможность \glsxtrshort{ЛПР} взаимодействовать с промежуточными результатами вычислений. Помимо прочего, подобная необходимость возникает, когда:
\begin{enumerate}
    \item нет формально определённых критериев отбора, на основе которых его можно было бы автоматизировать;
    \item критериев анализа результатов слишком много для того, чтобы реализовать автоматизированную процедуру для его проведения в пределах исследовательской работы;
\end{enumerate}

Поскольку в данной работе рассматривается, в первую очередь, система, реализующая графоориентированный подход к решению сложных вычислительных задач, то целесообразно рассмотреть подходы к организации взаимодействия пользователя с процессом решения. На основании изложенного выше были выделены следующие сценарии взаимодействия с пользователем в данной системе:
\begin{itemize}
    \item Введение дополнительных данных, которые требуются на дальнейших этапах расчётов, но которые не были получены автоматически до этого;
    \item Выбор конкретных данных из некоторого однородного набора для его сужения;
    \item Выбор дальнейшей логики выполнения расчётов на основании полученных на текущем этапе результатов.
\end{itemize}
Помимо этого, для эффективной работы с подобной системой исследователю необходим графический пользовательский интерфейс, в котором модель организации вычислений может быть представлена визуально. Большие перспективы в автоматизации процесса решения сложных задач перед исследователем открывает возможность прямого взаимодействия с вычислительной моделью: остановка вычислительного процесса на определенном этапе, изучение обрабатываемых данных, просмотр истории изменения обрабатываемых данных, возврат к определенному этапу вычислений, ввод
дополнительных параметров на определенной
стадии вычислений и т. д.\cite{SokolovCADCMInteraction2021}
Кроме того, целесообразно рассмотреть основные понятия, вводимые в данной системе.
\begin{itemize}
    \item \emph{Состояние данных} - некоторый набор данных, в котором они хранятся тройками вида "тип - имя - значение". В GBSE реализован в виде специального класса с названием \textsf{Anymap}.
    \item \emph{Функция-обработчик} - функция, которая вызывается при переходе из одного состояния данных в другое. Фактически данная функция каким-то образом модифицирует объект состояния данных.
    \item \emph{Функция-предикат} - функция, связанная с тем же переходом, что и некоторая функция-обработчик, проверяющая соответствие входных данных тому формату, в котором они ожидаются на входе обработчика.
\end{itemize}
На концептуальном уровне абстракции в рассматриваемой системе получение каких-то данных или решений от пользователя может быть реализовано, как и любой другой процесс модификации данных, через соответствующие функции-обработчики и предикаты. Рассмотрим возможные подходы к реализации сценариев взаимодействия пользователя, описанных выше.

Для введения дополнительных данных в GBSE реализован специальный инструмент автоматической генерации графических программных интерфейсов (\glsxtrshort{GUIen}) с формами ввода необходимых данных\cite{SokolovPershin2017}, однако этот инструмент ещё не связан с основным графоориентированным программным каркасом. Средства, реализующие два других сценария на момент написания данной работы находятся в разработке. Для предоставления пользователю возможности сделать выбор относительно дальнейшей обработки данных необходимо разработать следующие средства:
\begin{enumerate}
    \item Средство визуализации текущего состояния данных
    \item Средство содержательной интерпретации текущего состояния данных
\end{enumerate}
Кроме того, для реализации рассматриваемых сценариев будет необходимо доработать средство генерации форм ввода и реализовать в нём дополнительную категорию форм, направленных не на внесение новых данных, а на выбор, в том числе и множественный, из представленных вариантов, отображённых, в том числе, и с помощью средства визуализации.