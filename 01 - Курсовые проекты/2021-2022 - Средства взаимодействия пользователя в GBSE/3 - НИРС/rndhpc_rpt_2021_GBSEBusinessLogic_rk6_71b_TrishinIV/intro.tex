%----------------------------------------------------------
\chapter*{ВВЕДЕНИЕ}\label{chap.introduction}
\addcontentsline{toc}{chapter}{ВВЕДЕНИЕ}

%----------------------------------------------------------
При проведении современных исследований возникает необходимость автоматизировать процессы решения сложных вычислительных задач. Достижение подобной цели не представляется возможным без формально определенного метода организации процессов в автоматизированной системе. Проектирование, создание и сопровождение подобных систем является трудоемкой задачей, для решения которой применяют инструментальные средства и среды разработки автоматизированных систем (\glsxtrshort {CASE}-системы)\cite{Golubev2020} В некоторых очень узко направленных системах подобная организация напрямую зависит от поставленной задачи. В более универсальных же системах, о которых пойдёт речь в этой работе, разрабатывается особая архитектура, которая позволяет организовать различные процессы для решения различных задач по-разному. Подходы к построению данной архитектуры освещены, помимо прочего, в \cite{SokolovCADCMConcept2020}. 

В данной работе внимание сосредоточено на примении подобных универсальных автоматизированных систем к решению задач, требующих внимания со стороны пользователя непосредственно в процессе решения, и на средствах, которыми взаимодействие с пользователем реализуется.

%----------------------------------------------------------
