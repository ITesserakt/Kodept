%----------------------------------------------------------
\chapter*{ВВЕДЕНИЕ}\label{chap.introduction}
\addcontentsline{toc}{chapter}{ВВЕДЕНИЕ}

%----------------------------------------------------------

При разработке любого программного обеспечения разработчики ставят перед собой несколько очень важных задач: эффективность и гибкость дальнейшей поддержки программного обеспечения. В современной разработке существует множество паттернов проектирования, архитектур и вспомогательных систем, которые позволяют быстрее создавать программное обеспечения и в дальнейшем упрощают его поддержку для новых разработчиков. Таким образом, даже в небольших проектах, над которыми работает небольшая команда разработчиков стараются использовать системы контроля версий, юнит-тестирование. Если приложение представляет из себя API при его разработке используют специальные системы для быстрой документации API. Такие подходы считаются стандартом при разработке программного обеспечения рассчитанное на удовлетворение потребностей конечных пользователей - разработка мобильных приложений, проектирование и реализация API и прочее.

Однако при появлении первых ЭВМ в 1930-х годах ~\cite{FirstComputers} программирование использовалось для решения научных задач. Научное программирование сильно отличается от других видов программирования, при разработке научного программного обеспечения очень важно получить корректный и стабильный конечный продукт, а также четко разделить интерфейсную и научную часть. Из этого следует, что разработчик научного программного обеспечения должен быть экспертом в предментной области, а сам разработчик обычно является конечным пользователем ~\cite{ScientificDevelopment}, в то время как в индустриальном программировании разработчик зачастую не является конечным пользователем и от него не требуется быть экспертом в предметной области разработываемого приложения.

Также научное программирование отличается от индустриального программирования тем, что стандарты проектирования научного программного обеспечения вырабатываются существенно медленне или не вырабатываются вовсе, что приводит к отсутствию каких-либо системных подходов к разработке. Это приводит к сложностям при валидации и дальнейшей поддержке кода. Так, код, написанный эффективно и корректно, может оказаться бесполезным в том  если поддержка кода оказывается затруднительной, это приводит к формированию и накоплению ``best practices'' при программировании численных методов. Вследствие чего стали появляться системы, которые позволяют минимизировать написание кода и снизить трудозатраты на его поддержку.

\section{Результаты поиска источников литературы}

В основном существующие платформы используют визуальное программирование ~\cite{CSEComputationalSteering}. Одними из самых популярных и успешных разработок в этой области являются Simulink и LabView. Simulink позволяет моделировать вычислительные методы с помощью графических блок-диаграмм и может быть интегрирован со средой MATLAB. Также Simunlink позволяет автоматически генерировать код на языке на C для реализации вычислительного метода в режиме реального времени. LabView используется для аналогичных задах - модерование технических систем и устройств ~\cite{LabViewUsage}. Среда позволяет создавать виртуальные приборы с помощью графической блок-диаграммы, в которой каждый узел соответствует выполнению какой-либо функции. Представление программного кода в виде такой диаграмы делает его интуитивно понятным инженерам и позволяет осуществлять разработку системы более гибко и быстро. В составе LabView есть множество специализированных библиотек для моделирования систем из конкретных технических областей.

Существует множество других систем и языков программирования для реализации вычислительных методов, однако, каждый из них является узкоспециализированным и решает определенную задачу. Так, например, система визуального модерования FEniCS ~\cite{FEniCSFramework} используется для решения задач с использованием метода конечных элементов. Система имеет открытый исходный код, а также предоставляет удобный интерфейс для работы с системой на языках Python или C++. FEniCS предоставляет механизмы для работы с конечно-элементными расчетными сетками и функциями решения систем нелинейных уравнений, а также позволяет вводить математические модели в исходной интегрально-дифференциальной форме.

Отдельного упоминания стоит система TensorFlow. TensorFlow представляет из себя библиотеку с открытым исходным кодом, которая используется для машинного обучения. Аналогично LabView и Simulink, TensorFlow позволяет строить программные реализации численных методов. Стоит обратить внимание, что в основе TensorFlow лежит такое понятие как граф потока данных. В самом графе ребра - тензоры, представляют из себя многомерные массивы данных, а узлы - матетиматические операции над ними.

Применение ориентированных графов очень удобно для построения архитектур процессов обработки данных (как в автоматическом, так и в автоматизированном режимах). Вместе с тем многочисленные возникающие в инженерной практике задачи предполагают проведение повторяющихся в цикле операций. Самым очевидным примером является задача автоматизированного проектирования (АП). Эта задача предполагает, как правило, постановку и решение некоторой обратной задачи, которая в свою очередь, часто, решается путём многократного решения прямых задач (простым примером являются задачи минимизации некоторого функционала, которые предполагаю варьирование параметров объекта проектирования с последующим решением прямой задачи и сравнения результата с требуемым согласно заданному критерию оптимизации). Отметим, что прямые задачи (в различных областях) решаются одними методами, тогда как обратные - другими. Эти процессы могут быть очевидным образом отделены друг от друга за счет применения единого уровня абстракции, обеспечивающего определение интерпретируемых архитектур  алгоритмом, реализующих методы решения как прямой, так и обратной задач. Очевидным способом реализации такого уровня абстракции стало использование ориентированных графов.

А.П.Соколов и А.Ю.Першин разработали графориентированный программный каркас для реализации сложных вычислительных методов, теоретические основы представлены в работе ~\cite{SokPersh2018GBSE}, а принципы применения графоориентированного подхода зафиксированы в патенте ~\cite{patentRU2681408}. Заметим, что в отличии от TensorFlow, в представленном графориентированном программном каркасе узлы определяют фиксированные состояния общих данных, а ребра определяют функции преобразования данных. Для описания графовых моделей был разработан формат aDOT, который расширяет формат описания графов DOT ~\cite{GraphsDOT}, входящий в пакет утилит визуализации графов Graphviz. В aDOT были введены дополнительные атрибуты и определения, которые описывают функции-предикаты, функции-обработчики и функции перехода в целом. Подроробное описание формата aDOT приведено в ~\cite{SokADOT}.

В процессе исследования графоориентированного программного каркаса разработанного А.П.Соколовым и А.Ю.Першиным стало очевидно, что для построения графовых моделей необходим удобный графический редактор, который будет также выступать в роли визуализатора, то есть должен иметь иметь возможность экспортировать графовую модель в формат aDOT, а также возможность и загружать граф из этого формата.

%----------------------------------------------------------
