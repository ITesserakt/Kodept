%----------------------------------------------------------
\chapter*{ВВЕДЕНИЕ}\label{chap.introduction}
\addcontentsline{toc}{chapter}{ВВЕДЕНИЕ}

%----------------------------------------------------------

Решение любой проблемы представляет из себя поиск способа получения желаемого результата из текущего набора данных. Одним из спобов решения проблемы является применение вычислительных методов. Однако для того, чтобы проблему можно было решить с помощью вычислительных методов она должна обладать следующими параметрами:
\begin{itemize}
	\item Проблема должна быть четко определена, то есть должна быть четко определена конечная цель, текущая ситуация и возможные средства и методы для достижения конечной цели
	\item Проблема должна быть вычислимой. Необходимо проаналазировать какие типы рассчетом потребуются и возможно ли их совершить в разумные сроки
\end{itemize}
Для реализации сложных вычислительных методов очень удобно визуализировать процессч с помощью графов, в особенности ориентированных. Визуализация позволяет более кратко и в более понятной форме представить метод.

Применение ориентированных графов очень удобно для построения архитектур процессов обработки данных (как в автоматическом, так и в автоматизированном режимах). Вместе с тем многочисленные возникающие в инженерной практике задачи предполагают проведение повторяющихся в цикле операций. Самым очевидным примером является задача автоматизированного проектирования (АП). Эта задача предполагает, как правило, постановку и решение некоторой обратной задачи, которая в свою очередь, часто, решается путём многократного решения прямых задач (простым примером являются задачи минимизации некоторого функционала, которые предполагаю варьирование параметров объекта проектирования с последующим решением прямой задачи и сравнения результата с требуемым согласно заданному критерию оптимизации). Отметим, что прямые задачи (в различных областях) решаются одними методами, тогда как обратные - другими. Эти процессы могут быть очевидным образом отделены друг от друга за счет применения единого уровня абстракции, обеспечивающего определение интерпретируемых архитектур  алгоритмом, реализующих методы решения как прямой, так и обратной задач. Очевидным способом реализации такого уровня абстракции стало использование ориентированных графов.

Программная реализация сложных вычислительных методов предполагает написание научного кода, который, в свою очередь, должен быть эффективным и поддерживаемым. Основным фактором поддерживаемости кода, реализующего численный метод, является то, насколько легко код усваивается новым разработчиком. Вследствие наличия таких требований к научному коду, стали появляться системы, которые позволяют минимизировать написание кода и снизить трудозатраты на его поддержку.

В основном существующие платформы используют визуальное программирование ~\cite{CSEComputationalSteering}. Одними из самых популярных и успешных разработок в этой области являются Simulink и LabView. Simulink позволяет моделировать вычислительные методы с помощью графических блок-диаграмм, также Simulink может быть интегрирован со средой MATLAB. Также Simunlink позволяет автоматически генерировать код на языке на C для реализации вычислительного метода в режиме реального времени. LabView используется для аналогичных задах - модерование технических систем и устройств ~\cite{LabViewUsage}. Среда позволяет создавать виртуальные приборы с помощью графической блок-диаграммы, в которой каждый узлел соответствует выполнению какой-либо функции. Представление программного кода в виде такой диаграмы делает его интуитивно понятным инженерам и позволяет осуществлять разработку системы более гибко и быстро. В составе LabView есть множество специализированных библиотек для моделирования систем из конкретных технических областей.

Отдельного упоминания стоит система TensorFlow. TensorFlow представляет из себя библиотеку с открытым исходным кодом, которая используется для машинного обучения. Аналогично LabView и Simulink, TensorFlow позволяет строить программные реализации численных методов. Стоит обратить внимание, что в основе TensorFlow лежит такое понятие как граф потока данных. В самом графе ребра - тензоры, представляют из себя многомерные массивы данных, а узлы - матетиматические операции над ними.

Существует множество других систем и языков программирования для реализации вычислительных методов, каждый из них является узкоспециализированным и решает определенную задачу. Так, например, система визуального модерования FEniCS ~\cite{FEniCSFramework} используется для решения задач с использованием метода конечных элементов. Система имеет открытый исходный код, а также предоставляет удобный интерфейс для работы с системой на языках Python или C++. FEniCS предоставляет механизмы для работы с конечно-элементными расчетными сетками и функциями решения систем нелинейных уравнений, а также позволяет вводить математические модели в исходной интегрально-дифференциальной форме.


%----------------------------------------------------------
