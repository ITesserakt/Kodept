%----------------------------------------%
% общие определения
\newcommand{\UpperFullOrganisationName}{Министерство науки и высшего образования Российской Федерации}
\newcommand{\ShortOrganisationName}{МГТУ~им.~Н.Э.~Баумана}
\newcommand{\FullOrganisationName}{федеральное государственное бюджетное образовательное\newline учреждение высшего профессионального образования\newline <<Московский государственный технический университет имени Н.Э.~Баумана\newline (национальный исследовательский университет)>> (\ShortOrganisationName)}
\newcommand{\OrganisationAddress}{105005, Россия, Москва, ул.~2-ая Бауманская, д.~5, стр.~1}
%----------------------------------------%
\newcommand{\gitlabdomain}{sa2systems.ru:88}
%----------------------------------------------------------
\newcommand{\doctypesid}{kp} % vkr (выпускная квалификационная работа) / kp (курсовой проект) / kr (курсовая работа) / nirs (научно-исследовательская работа студента) / nkr (научно-квалификационная работа)

% Тема должна быть сформулирована так, чтобы рассказать, о чем работа, но сделать это так, чтобы у читателя возникло желание читать аннота-цию. При формулировке темы не следует стараться рассказать о работе всё. Пример корректной темы: "Математическое моделирование процесса размножения медуз в Южно-Китайском море". Пример некорректной темы: "Применение модели SIS для моделирования процесса размножения медуз в Южно-Китайском море с использованием метода Рунге-Кутты и многопроцессорных вычислительных систем".
\newcommand{\Title}{@Тема работы@}%{}
\newcommand{\TitleSource}{кафедра} % кафедра, предприятие, НИР, НИР кафедры, заказ организации

\newcommand{\SubTitle}{по дисциплине <<Модели и методы анализа проектных решений>>} % Методы оптимизации
\newcommand{\faculty}{<<Робототехники и комплексной автоматизации>>}
\newcommand{\facultyShort}{РК}
\newcommand{\department}{<<Системы автоматизированного проектирования (РК-6)>>}
\newcommand{\departmentShort}{РК-6}

\newcommand{\Author}{@Фамилия~И.О.@}
\newcommand{\AuthorFull}{@Фамилия~Имя~Отчество@}
\newcommand{\ScientificAdviser}{@Фамилия~И.О.@}	% Научный руководитель
\newcommand{\ConsultantA}{@Фамилия~И.О.@}				% Консультант 1
\newcommand{\ConsultantB}{@Фамилия~И.О.@}				% Консультант 2
\newcommand{\Normocontroller}{@Фамилия~И.О.@}		% Нормоконтролёр
\newcommand{\group}{@РК6-5XБ@}
\newcommand{\Semestr}{осенний семестр} % Например: осенний семестр или весенний семестр
\newcommand{\BeginYear}{2021}
\newcommand{\Year}{2021}
\newcommand{\Country}{Россия}
\newcommand{\City}{Москва}
\newcommand{\TaskStatementDate}{<<\underline{\textit{DD}}>> \underline{месяц} \Year~г.} %Дата выдачи задания 

\newcommand{\depHeadPosition}{Заведующий кафедрой}		% Должность руководителя подразделения
\newcommand{\depHeadName}{А.П.~Карпенко}		% Должность руководителя подразделения

% Цель выполнения 
\newcommand{\GoalOfResearch}{@Цель выполнения работы@} % с маленькой буквы и без точки на конце

% Объектом исследования называют то, что исследуется в работе. Напри-мер, для указанной выше темы объектом может быть популяция медуз, но никак ни модель SIS, ни Южно-Китайское море, ни метод моделирования популяции медуз. 
\newcommand{\ObjectOfResearch}{@Объект исследований@}

% Предмет исследований (уже чем объект, определяется, исходя из задач: формулируется как существительное, как правило, во множественном числе, определяющее "конкретный объект исследований" среди прочих в рамках более общего)
\newcommand{\SubjectOfResearch}{@Предмет исследований@}

% Основная задача, на решение которой направлена работа
\newcommand{\MainProblemOfResearch}{@Основная задача, на решение которой направлена работа@}

% Выполненные задачи
\newcommand{\SubtasksPerformed}{%
В результате выполнения работы: 
\begin{inparaenum}[1)]
	\item предложено ...;
	\item создано ...;
	\item разработано ...;
	\item проведены вычислительные эксперименты ...
\end{inparaenum}}

% Ключевые слова (представляются для обеспечения потенциальной возможности индексации документа)
\newcommand{\keywordsru}{%
	@keywordsru@} % 5-15 слов или выражений на русском языке, для разделения СЛЕДУЕТ ИСПОЛЬЗОВАТЬ ЗАПЯТЫЕ
\newcommand{\keywordsen}{%
	@keywordsen@} % 5-15 слов или выражений на английском языке, для разделения СЛЕДУЕТ ИСПОЛЬЗОВАТЬ ЗАПЯТЫЕ

% Краткая аннотация
\newcommand{\Preface}{@Начать можно так: ``Работа посвящена...''. Объём около 0.5 страницы. Здесь следует кратко рассказать о чём работа, на что направлена, что и какими методами было достигнуто. Реферат должен быть подготовлен так, чтобы после её прочтения захотелось перейти к основному тексту работы.@} % с большой буквы с точкой в конце

%----------------------------------------%
% выходные данные по документу
\newcommand{\DocOutReference}{\Author. \Title\xspace\SubTitle. [Электронный ресурс] --- \City: \Year. --- \total{page} с. URL:~\url{https://\gitlabdomain} (система контроля версий кафедры РК6)}

%----------------------------------------------------------

