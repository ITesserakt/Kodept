%----------------------------------------------------------
\chapter{Тестирование и отладка}
\label{ch:chap4_soft_testing}
%----------------------------------------------------------

При разработке приложения не обойтись без тестирования.
Оно помогает выявить различные ошибки и исправить их.
Популярным вариантом тестирования является модульное тестирование (unit тестирование).
Unit test - функция или набор функций, который проверяет корректность работы отдельного нетривиального куска программы.

В ходе разработки компилятора были написаны модульные тесты, они покрывают большое количество кода и успешно выполняются.
Для запуска можно использовать систему сборки Cargo.
Из папки с проектом следует запустить следующую команду:~\lstinline{cargo test --all-features --all --lib --no-fail-fast}.
Cargo соберет проект и последовательно запустит все модульные тесты.

Компилятор Kodept является консольным приложением, поэтому для него был разработан интерфейс командной строки (CLI).
С помощью него можно настроить вид выходных данных, поведение работы и др.

На текущий момент поддерживается 3 команды: справка по использованию, генерация графа AST в формате DOT и анализ файла.

Ранее уже были приведены примеры работы команды по генерации графа (\ref{fig:ast_dot}) для листинга~\ref{lst:kodept}.
Продемонстрируем работу механизма вывода типов на примере этого же кода.
Для этого тоже можно воспользоваться Cargo:~\lstinline{cargo run -- -d examples/test.kd}.
В результате в консоль будет выведены тип функции~\lstinline{compose}:

\texttt{cargo run -- -d examples/test.kd}

\texttt{kodept\_interpret::type\_checker: [compose: $\forall a, b, c => ('b -> 'a) -> ('c -> 'b) -> 'c -> 'a$]}

Действительно, если преобразовать эту функцию в термы, то получится $\lambda f, g. \lambda x. f(g(x))$.
Тип этого выражения действительно совпадает с выведенным типом.

%----------------------------------------------------------

