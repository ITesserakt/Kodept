%----------------------------------------------------------
\chapter{Тестирование и отладка}
\label{ch:chap4_soft_testing}
%----------------------------------------------------------

При разработке приложения не обойтись без тестирования.
Оно помогает выявить различные ошибки и исправить их.
Популярным вариантом тестирования является модульное тестирование (unit тестирование).
Unit test - функция или набор функций, который проверяет корректность работы отдельного нетривиального куска программы.

В ходе разработки компилятора были написаны модульные тесты, они покрывают большое количество кода и успешно выполняются.
Для запуска можно использовать систему сборки Cargo~\cite{CargoBook}.
Из папки с проектом следует запустить следующую команду:~\lstinline{cargo test --all-features --all --lib --no-fail-fast}.
Cargo соберет проект и последовательно запустит все модульные тесты.

\todo{Дописать}

%----------------------------------------------------------

