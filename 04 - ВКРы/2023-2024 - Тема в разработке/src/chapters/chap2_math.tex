\chapter{Математическая постановка задачи}
\label{ch:math}

\todo{Не уверен насчет названия главы, может даже переименовать в вычислительный метод}

\section{Форма Бэкуса-Наура}
\label{sec:bnf}

\todo{Целая секция для просто определения BNF. звучит плохо...}

В работе определяются синтаксические конструкции, которые используют форму Бэкуса-Наура (англ. Backus-Naur form, BNF).
Такая форма используется для записи контекстно-свободных грамматик и состоит из двух элементов: терминалов и нетерминалов.
Терминалы являются примитивными элементами, в т.ч. строковым литералом, числовым и прочее.
Нетерминалы могут включать в себя другие нетерминалы или терминалы.

Для определения нового нетерминала $A$ используется обозначение: $A \coloneqq B$, где $B$ может быть одной из следующих конструкций:
\begin{itemize}
    \item конкатенация $x y$ - нетерминалы расположены последовательно (\lstinline{"dog"  "cat"}),
    \item выбор $x ~|~ y$ - либо $x$, либо $y$.
\end{itemize}
При этом нетерминалы в конструкции выбора будем называть вариантами.
Каждый вариант может быть рассмотрен отдельно от других.
Слева от знака $\coloneqq$ может быть записано несколько <<примеров>> определяемого нетерминала.
Они могут быть использованы внутри конструкций $B$, создавая таким образом рекурсивный нетерминал.

Будем считать, что при введении нетерминалов, любой не оговоренный заранее элемент является строковым терминалом.
Благодаря этому уменьшается многословность записи.

Каждый нетерминал формирует множество допустимых значений.
Так, например, для нетерминала $K_1, K_2 \coloneqq * ~|~ K_1 \to K_2$, множеством возможных значений является $\left\{ *, * \to *, (* \to *) \to *, \ldots \right\}$.

\section{Теория типов}
\label{sec:type_theory}

В разделе представлена информация о специальном разделе математики - теории типов~\cite{TypeTheoryBook}.
Освещены важные понятия - \textit{терм}, \textit{тип}, \textit{суждение} и \textit{система типов}.

Теория типов является альтернативой для теории множеств и теории категорий.
В отличие от остальных, она позволяет исследовать свойства объекта, учитывая его структуру, а не множество, которому он принадлежит.
Поэтому теория типов нашла свое применение в программировании, в частности, в компиляторах в фазе статического анализа программы, как для вывода, так и для проверки соответствия типов.
Более того, согласно изоморфизму Карри-Ховарда~\cite{TypeTheoryArticle}~\tabref{tab:}, программы могут быть использованы для доказательства логических высказываний.
Такие доказательства называют автоматическими, и они широко применяется среди таких языков, как Agda, Coq, Idris.

Терм $x$ - чаще всего элемент языка программирования, будь то переменная, константа, вызов функции и др.
Термы могут включать в себя другие термы.
Например, термом является конструкция $(x + 1) * (x + 1)$, построенная из других термов: $x$, $1$, $+$ и $*$.

Типом $A$ обозначается метка, например объекты на натюрмортах принадлежат к типу (классу) <<фрукты>>.
Обычно каждому терму соответствует определенный тип - $x: A$.
Типы позволяют строго говорить о возможных действиях над объектом, а также формализовать взаимоотношения между ними.

Система типов определяет правила взаимодействия между типами и термами.
В программировании это понятие равноценно понятию типизации.

Кроме того, также используются такие термины, как \textit{суждения} и \textit{предположения}.
С помощью суждений (англ. judgments) можно создавать логические конструкции: выражение $\vdash x: T$ говорит, что терм $x$ имеет тип $T$.
Слева от знака $\vdash$ записывается контекст: $x: integer \vdash (x + 1): integer$ - если терм $x$ имеет тип числа, то $x + 1$ тоже имеет тип числа.
Таким образом, суждение обозначается так:

\begin{equation}
    \Gamma \vdash P,
    \label{eq:judgment}
\end{equation}
\begin{eqrem}
    & $\Gamma$ - контекст, \\
    & $P$ - предположение. \\
\end{eqrem}

Всего существует 6 видов суждений:

\begin{table}[h]
    \centering
    \label{tab:types_of_judgments}
    \begin{tabular}{l r}
        $\Gamma \vdash$                  & $\Gamma$ - верный контекст,                  \\
        $\Gamma \vdash \tau$             & $\tau$ - тип в контексте,                    \\
        $\Gamma \vdash x: \tau$          & терм $x$ имеет тип $\tau$ в контексте,       \\
        $\vdash \Gamma = \Delta$         & контексты $\Gamma$ и $\Delta$ равны,         \\
        $\Gamma \vdash \tau_1 = \tau_2$  & типы $\tau_1$ и $\tau_2$ в контексте равны,  \\
        $\Gamma \vdash x_1 = x_2 : \tau$ & терм $x_1$ равен терму $x_2$ с типом $\tau$.
    \end{tabular}
\end{table}

\begin{table}[h]
    \centering
    \caption{Изоморфные соотвествия между логическими высказываниями и системой типов}
    \label{tab:curry-hovard-iso}
    \begin{tabular}{|c|c|}
        \hline
        \textbf{Логическое высказывание} & \textbf{Система типов} \\\hline
        Высказывание, $F$, $Q$           & Тип, $A$, $B$                  \\\hline
        Доказательство высказывания $F$  & $x: A$                         \\\hline
        Высказывание доказуемо           & Тип $A$ обитаем                \\\hline
        $F \implies Q$                   & Функция, $A \to B$             \\\hline
        $F \wedge Q$                     & Тип-произведение, $A \times B$ \\\hline
        $F \vee Q$                       & Тип-сумма, $A + B$             \\\hline
        Истина                           & Единичный тип, $\top$          \\\hline
        Ложь                             & Пустой тип, $\bot$             \\\hline
        $\neg F$                         & $A \to \bot$                   \\\hline
    \end{tabular}
\end{table}

Тип $T$ обитаем (англ. inhabitat), если выполняется следующее: $\exists t: \Gamma \vdash t: T$ - найдется терм $t$, такой, что в контексте $\Gamma$ он будет иметь тип $T$.

Из одних суждений можно получить другие суждения по определенным правилам.
Такие правила называются \textit{правилами вывода} и выглядят следующим образом: $\displaystyle \frac{J_1}{J_2}$, что означает если верно суждение $J_1$, то и верно суждение $J_2$.
Таким образом из правил вывода получаются деревья вывода, где каждое входное суждение $J_1$ заменяется правилом вывода.
Например, следующее правило определяет тип применения функции $f$ к аргументу $x$:

\begin{equation}
    \label{eq:judgement_substitution}
    \frac{\Gamma \vdash x: T_1, f: T_1 \to T_2}{\Gamma \vdash f(x): T_2}
\end{equation}

Выражение~\ref{eq:judgement_substitution} можно трактовать следующим образом: если в контексте $\Gamma$ терм $x$ имеет тип $T_1$, а терм $f$ - $T_1 \to T_2$ (функциональный тип), то можно судить, что терм $f(x)$ (применение функции) имеет тип $T_2$.

%----------------------------------------------------------


\section{Классификация систем типов}
\label{sec:classification}
%----------------------------------------------------------

Известно, что системы типов можно разделить на \textit{динамические} и \textit{статические}~\cite{Typing}.
Это влияет на то, в какой момент в программе происходит проверка соответствия типов.
В динамических системах - во время исполнения программы, а в статических - соответственно во время компиляции.
Кроме того, существуют особые языки программирования, где все данные имеют один тип.
К таким относятся многие низкоуровневые языки, например ассемблер.
Все данные в нем (адреса в памяти, числа, указатели на функции) являются всего лишь последовательностью байт.

Ниже приведены основные критерии, по которым можно классифицировать систему типов в языках программирования:

\begin{enumerate}[1)]
    \item по времени проверки соответствия типам: статическая и динамическая,
    \item по поддержке неявных конверсий: сильная (англ. strong) и слабая,
    \item по необходимости вручную типизировать выражение: явная и неявная.
\end{enumerate}

Например, типизация в языке Python является динамической, сильной и неявной с точки зрения этой классификации~\cite{PythonWiki}.
Интерпретатор знает тип переменной только во время выполнения и не может неявно изменить его.

Статические системы типов обладают несомненным преимуществом, по сравнению с динамическими - компилятор может использовать накопленную во время семантического анализа информацию для оптимизации кода.
Но необходимо учитывать, что такая типизация вносит некоторые неудобства: программисту постоянно приходится прилагать усилия по устранению ошибок, связанных с типами.
Это делает языки со статической типизацией, хоть и более сложными в использовании, но более быстрыми, а динамически типизированным языкам приходится использовать различные специфические оптимизации вроде \textit{JIT-компиляции}, чтобы добиться сопоставимой производительности.

JIT-компиляцией (just-in-time компиляцией) называется прием оптимизации при выполнении программы, когда компиляция происходит во время работы программы.
Она была создана, чтобы решить проблемы с производительностью при интерпретации кода.

Проанализируем системы типов, используемые в некоторых современных языках программирования с целью выявить их сильные и слабые стороны.

\subsection{Система типов C}
\label{subsec:c_type_system}

C - язык программирования со статической, слабой, явной типизацией, разработанный в 1970-х годах.

Типом в языке C является интерпретация набора байт, составляющих объект~\cite{CSpec}.
Все типы бывают двух видов: базовые и производные~\figref{fig:c_types}.
Также их можно разделить на две группы: скалярные и агрегатные.

В группу скалярных типов относят примитивные (базовые) типы и указатели.
Базовые типы в свою очередь делятся на целые и вещественные числа.
Указатели - тоже скалярная величина, их размер зависит от архитектуры системы.

В группу агрегатных относятся структуры и массивы.
Они позволяют определить тип, который включает в себя несколько других.

Кроме того, существуют <<специальные>> типы - объединения и указатели на функции.
С помощью объединений можно задать варианты представления данных~\lstref{lst:union}.

\begin{figure}[H]
    \centering
    \input{figures/.generated/c_types}
    \caption{Схематичное изображение типов в C}
    \label{fig:c_types}
\end{figure}

\begin{lstlisting}[label={lst:union},language=C,caption={Объявление безымянного объединения в языке C. В одной переменной твкого типа может содержаться либо целое число, либо вещественное.}]
union {
    int first_variant;
    float second_variant;
};
\end{lstlisting}

Достоинства:
\begin{itemize}
    \item C прост для понимания, он содержит только основные типы данных,
    \item язык позволяет эффективно работать с данными в том виде, как они реализованы в ЭВМ.
\end{itemize}

Недостатки:
\begin{itemize}
    \item недостаточная выразительность по сравнению с другими языками программирования,
    \item мало гарантий и проверок, осуществляемых компилятором~\lstref{lst:weak_c}.
\end{itemize}

Здесь под выразительностью стоит понимать то, насколько много идей можно реализовать и насколько лаконично они при этом будут выглядеть.
Например, хоть в C и можно выразить идею объекто-ориентированного программирования, но это будет выглядеть гораздо более громоздко, чем в C++~\cite{OOP_in_C}.

\begin{lstlisting}[label={lst:weak_c},language=C,caption={Неправильное использование \lstinline{void*} не может быть отслежено компилятором}]
// компилятор не знает исходный тип аргумента
long* foo(void* arg) { return (long*) arg; }

int main() {
    void *value = &foo;
    long result = (*foo(value)) + 1; // неопределенное поведениe
}
\end{lstlisting}

\subsection{Система типов Java}
\label{subsec:java_type_system}

Язык Java разработан компанией Sun Microsystems в 1995 году.
Благодаря использованию дополнительной абстракции в виде виртуальной машины, может выполняться на большом количестве архитектур ЭВМ.
Популярен среди разработчиков самых разных областей: от банковского сектора до приложений под ОС Android~\cite{JavaUsage}.

Систему типов, применяемую в этом языке можно охарактеризовать как статическую, сильную и явную с возможностью введения неявно типизированных выражений~\cite{JavaTypeSystem}.
Java создавалась под сильным влиянием идей объектно-ориентированного программирования, поэтому она включает классы, интерфейсы, обобщения и прочее.
Язык эволюционировал, но пытался сохранить обратную совместимость с прошлыми версиями, поэтому имеются и недостатки - вся информация об обобщенной переменной стирается во время исполнения программы.
Это иногда приводит к ошибкам при работе с коллекциями.

Все типы делятся на примитивные и объектные (пользовательские).
Их различает значение по-умолчанию: в случае пользовательских - \lstinline{null}, примитивных - в зависимости от типа.
К примитивным типам относятся различные виды представления чисел, символы и логический тип.
Для использования примитивных типов в коллекциях, используется упаковка (англ. boxing).
Суть её заключается в том, что значение такого типа <<упаковывается>> в соответствующий объектный тип, который представляет собой указатель на значение вместе с метаданными.

Достоинства:
\begin{itemize}
    \item наличие в системе типов обобщений позволяет уменьшить количество повторяемого кода,
    \item поддержка некоторых особенностей динамической типизации, при преобладании статической.
\end{itemize}

Недостатки:
\begin{itemize}
    \item из-за специфики работы обобщенных типов в виртуальной машине Java, могут возникать ошибки с приведением типов,
    \item из-за разделения на примитивные и объектные типы, а также дополнительных затрат на упаковку, ухудшается производительность,
    \item избыточность определений типов в очевидных местах; хоть это и было исправлено в последующих версиях языка с помощью локального вывода типов, синтаксис языка все ещё перегружен.
\end{itemize}

\subsection{Система типов ML-подобных языков}
\label{subsec:ml_type_system}

К семейству ML-подобных языков относят функциональные языки программирования, восходящие к ML (Meta Language).
Этот язык был разработан Робином Милнером в 1973 году как язык для системы автоматического доказательства теорем.
В основе ML лежит типизированное лямбда-исчисление - формализованная система, предложенная Алонзо Чёрчем в 1930 году~\cite{sep-church}.
Это делает языки семейства ML статически типизированными.
Кроме того, в них распространено применение алгоритмов вывода типов, поэтому они также являются неявно типизированными.

Лямбда-исчисление само по себе формально описывает некоторый набор термов, среди которых обязательно должны присутствовать 2 варианта: применение $e_1(e_2)$ и абстракция $\lambda x. e$.
Функциями в лямбда-исчислении являются некие алгоритмы, в результате выполнения которых может быть получен тот или иной терм.
Применение обозначает вычисление выражения $e_1$ с аргументом $e_2$ и сродни результату вызова функции.
Абстракция же представляет собой способ создания новых функций посредством определения входной переменной $x$ и тела $e$.
К термам лямбда-исчисления можно применять различные преобразования, в том числе $\alpha$-эквивалентность, $\beta$-редукцию и $\eta$-преобразование~\cite{LambdaCalculus}.

В отличие от обычного лямбда-исчисления, каждому терму в типизированном лямбда-исчислении сопоставляется тип.
Существует множество алгоритмов для автоматической проверки типов в типизированном лямбда-исчислении.
Поэтому оно используется в качестве модели для ML-подобных языков, таких как Haskell или Lisp.
Кроме того, в тексте программы есть возможность отдельно не указывать типы различных конструкций: параметров функций, переменных и прочих.

Существует большое количество различных видов типизированного лямбда-исчисления, однако в 1991 году была предложена их наглядная классификация~\cite{LambdaCalculusWithTypes}.
Это обобщение называется лямбда-кубом и туда входят восемь типизированных лямбда-исчислений~\figref{fig:lambda_cube}.
Также на лямбда-кубе отмечены зависимости между его элементами так, что ребро $\to$ обозначает отношение включения ($\subseteq$).

\begin{figure}[H]
    \centering
    \import{figures/.generated}{lambda_cube.pdf_tex}
    \caption{Графическое изображение лямбда-куба}
    \label{fig:lambda_cube}
\end{figure}

С помощью куба описываются четыре основных зависимости между типами и термами:
\begin{itemize}
    \item термы зависят от термов ($\lambda^{\to}$),
    \item термы зависят от типов ($\lambda 2$),
    \item типы завися от типов ($\lambda \underline{\omega}$),
    \item типы зависят от термов ($\lambda P$).
\end{itemize}

Согласно рисунку~\ref{fig:lambda_cube}, система $\lambda^{\to}$ (просто типизованное лямбда-исчисление) является самой простейшей и включена во все остальные.
Рассмотрим её подробнее.
В множество возможных типов входит переменная типа $\alpha$ и функциональный тип $\tau_1 \to \tau_2$.
Таким образом, $\tau \coloneqq \alpha ~|~ \tau_1 \to \tau_2$.
В множество термов входят переменная $x$, применение $e_1(e_2)$ и абстракция $\lambda x: \tau. e$.
Такая система типов во многом похожа на систему типов в языке C, за исключением пользовательских типов и массивов.

В систему $\lambda 2$ (полиморфное лямбда-исчисление), по сравнению с предыдущей системой, добавляется так называемый полиморфный тип $\sigma \coloneqq \forall \alpha. \tau ~|~ \tau$.
Таким образом, терм может зависеть от конкретного типа $\tau$: $f: \forall \alpha. \alpha \to \alpha \implies f(\tau): \tau \to \tau$.
Абстракции с полиморфным типом похожи на шаблонные функции из языка C++.

Зависимость типов от типов можно понимать как функции (операторы) над типами.
Простейшим примером будет функция из типа $\alpha$ в тип $\alpha$: $f: \alpha \to \alpha$.
Более формально записывают так: $f \equiv \lambda \alpha: *. \alpha \to \alpha$, где $*$ обозначает категорию типов.
Однако возникает проблема, что функция $f$ не является ни типом, ни термом.
Решением этой проблемы является добавление новой категории $K$ видов (англ. kinds):

\begin{equation}
    \label{eq:kinds}
    K \coloneqq * ~|~ * \to *,
\end{equation}
\begin{eqrem}
    & $*$ - категория типов.
\end{eqrem}

Функции, похожие на рассмотренную функцию $f$, называют конструкторами типа.
С точки зрения языка C++, такие функции можно сравнить с шаблонными классами (структурами).
Действительно, шаблонный класс определяет шаблон типа (срав. вида), конкретный экземпляр которого является типом.

Зависимость типов от термов является гораздо более сложной, по сравнению с уже рассмотренными и не имеет аналогов в привычных языках программирования.
Положим функцию $f: \tau \to *$, где $*$ - ранее рассмотренная категория типов.
Тогда $f$ является конструктором, а $(\lambda a: \tau. f(a)) : *$, где $a$ - терм с типом $\tau$.
Таким образом функция $f$ продуцирует зависимые от термов типы.
Такой подход может быть использован для явного определения контрактов функции в программировании.
Например, функция деления, оба аргумента которой являются числом, но второй отличен от нуля.
Такой контракт выносится на уровень сигнатуры функции, что позволяет строго ему следовать.

Остальные вершины куба формируются комбинацией уже рассмотренных систем типов в порядке, формируемом направлениями рёбер.
Например, система $\lambda P \omega$ является наиболее полной и в ней выполняются все четыре зависимости.
Кроме того, это можно показать с помощью следующей таблицы:

\begin{table}[h]
    \centering
    \caption{Наличие зависимостей между категорией типов ($*$) и категорией видов ($\square$) в системах типов, описанных в лямбда-кубе.}
    \label{tab:set_rules}
    \begin{tabular}{|c|c|}
        \hline
        \textbf{Система типов}        & \textbf{Используемые правила}                                  \\\hline
        $\lambda^{\to}$               & $(*, *)$                                                       \\\hline
        $\lambda 2$                   & $(*, *)$, $(\square, *)$                                       \\\hline
        $\lambda P$                   & $(*, *)$, $(*, \square)$                                       \\\hline
        $\lambda \underline{\omega}$  & $(*, *)$, $(\square, \square)$                                 \\\hline
        $\lambda \omega$              & $(*, *)$, $(\square, *)$, $(\square, \square)$                 \\\hline
        $\lambda P\underline{\omega}$ & $(*, *)$, $(*, \square)$, $(\square, \square)$                 \\\hline
        $\lambda P\omega$             & $(*, *)$, $(\square, *)$, $(*, \square)$, $(\square, \square)$ \\
        \hline
    \end{tabular}
\end{table}

Конструкция вида $(s_1, s_2)$, где $s \in \left\{ *, \square \right\}$ означает следующее:

\begin{equation}
    \label{eq:type_system_rules}
    \frac{
        \Gamma \vdash A: s_1 ~~ \Gamma, x: A \vdash B: s_2
    }{
        \Gamma \vdash (\prod x: A. B): s_2
    },
\end{equation}
\begin{eqrem}
    &$\prod x: A. B$ означает декартово произведение всех типов (видов) $B$, образованных от переменной $x$ с типом (видом) $A$.
\end{eqrem}

Достоинства:
\begin{itemize}
    \item разработчик может выразить больше инвариантов, контрактов и др. в сигнатуре функции, таким образом код становится самодокументируемым,
    \item математически строгое обоснование корректности и надёжности,
    \item компилятор, использующий эту систему типов, имеет больше возможностей по обнаружению ошибок.
\end{itemize}

Недостатки:
\begin{itemize}
    \item использование дополнительного синтаксиса при обозначении типов может ухудшить читаемость, а также усложнить написание кода,
    \item может значительно увеличить время компиляции из-за обилия проверок,
    \item неявная типизация может тоже повлечь ухудшение читаемости, так как программисту необходимо выводить типы самому; однако это исправляется использованием сред разработки.
\end{itemize}

В итоге была проведена классификация некоторых систем типов с описанием их достоинств и недостатков.
Особое внимание далее уделяется системе типов Хиндли-Милнера, являющейся применением $\lambda 2$ в языках программирования.
Различные ее модификации широко используются в языках ML-группы за счёт того, что используя эту систему типов, можно автоматически выводить тип термов, одновременно обеспечивая статическую типизацию.
Далее приводится модификация системы типов Хиндли-Милнера.

%----------------------------------------------------------


\section{Система типов Хиндли-Милнера}
\label{sec:hindley-milner}

\todo{может по-другому назвать секцию, я использую своего рода свою систему типов}

Для начала определим термы:

\begin{subequations}
    \label{eq:terms}
    \begin{align}
        e_1, e_2, e_3 \coloneqq ~ &x \label{eq:terms_1} \\
        &| ~ e_1(e_2) \label{eq:terms_2} \\
        &| ~ \lambda x : \tau. e_1 \label{eq:terms_3} \\
        &| ~ \text{let } x: \tau = e_2 \text{ in } e_2 \label{eq:terms_4} \\
        &| ~ \text{:num:} \\
        &| ~ (e_1, e_2) \label{eq:terms_6} \\
        &| ~ \text{if } e_1 \text{ then } e_2 \text{ otherwise } e_3, \label{eq:terms_7}
    \end{align}
\end{subequations}
\begin{eqrem}
    & $e_1, e_2, e_3$ - термы,              \\
    & $x$ - имя переменной (англ. binding), \\
    & :num: - числовой литерал,             \\
    & $\tau$ - тип.                         \\
\end{eqrem}

Поясним значение некоторых вариантов терма подробнее:

\begin{enumerate}
    \item[\eqref{eq:terms_1}] переменная позволяет сослаться на другое имя, определенное выше по тексту программы, например на функцию или другую переменную.
    \item[\eqref{eq:terms_2}] запись $e_1(e_2)$ обозначает применение функции ($e_1$) к ее аргументу ($e_2$); проще говоря - вызов функции.
    \item[\eqref{eq:terms_3}] $\lambda x: \tau. e_1$ означает создание безымянной, лямбда, функции.
    \item[\eqref{eq:terms_4}] объявление новых переменных происходит с помощью конструкции $\text{let } \ldots \text{ in } \ldots$.
    \item[\eqref{eq:terms_6}] конструкция необходима для создания пар из двух других термов,
\end{enumerate}

В некоторых элементах к переменной $x$ приписывается ограничение на тип $\tau$.
Это необходимо для того, чтобы явно указать необходимый тип, как это делается в других языках программирования~\lstref{lst:type_bound}.

\begin{lstlisting}[label={lst:type_bound},language=C,caption={Явное указание типа аргумента в языке C.}]
    void foo(int a) { }
\end{lstlisting}

Теперь можно ввести определение типа $\tau$.

\begin{equation}
    \label{eq:types}
    \begin{aligned}
        \tau_1, \tau_2 \coloneqq ~ &\text{Integer} ~|~ \text{Real} ~|~ \text{Boolean} \\
        &| ~ \alpha \\
        &| ~ \tau_1 \to \tau_2 \\
        &| ~ (\tau_1, \tau_2, \ldots, \tau_n) \\
        &| ~ C,
    \end{aligned}
\end{equation}
\begin{eqrem}
    & $C$ - имя типа (константа), определенное пользователем,                                               \\
    & $\alpha$ - переменная типа,                                                                           \\
    & Integer, Real, Boolean - примитивные типы для целых, вещественных и логических данных соответственно. \\
\end{eqrem}

Переменная типа необходима для тех же целей, что и обычная переменная: она может ссылаться на другой тип или быть любым типом.
Под записью $(\tau_1, \tau_2, \ldots, \tau_n)$ стоит понимать тип-кортеж, образованный несколькими другими типами.

Также к обычным типам необходимо добавить так называемые \textit{полиморфные} типы~\eqref{eq:poly_types}.
Они необходимы для введения квантора всеобщности по отношению к переменным типа.
С точки зрения обычных языков программирования, такие полиморфные типы можно оценивать как обобщенные типы~\lstref{lst:generics}.

\begin{equation}
    \label{eq:poly_types}
    \sigma \coloneqq \tau ~|~ \forall A. \sigma,
\end{equation}
\begin{eqrem}
    & $A = \left\{ \alpha \right\}$ - неупорядоченное множество переменных типа.\\
\end{eqrem}

\begin{lstlisting}[language=C++,label=lst:generics,caption={Определение обобщенной функции в C++}]
    template<typename T>
    void foo(T value) {}
\end{lstlisting}

Далее введем следующие понятия:
\textit{контекст} $\Gamma$~\eqref{eq:context},
множество \textit{свободных типов} (англ. free types)~\eqref{eq:free_types},
\textit{обобщение} (англ. generalize)~\eqref{eq:generalize},
\textit{подстановка} (англ. substitutions)~\eqref{eq:subst} и
\textit{конкретизация} (англ. instantiate)~\eqref{eq:instantiate}.
В скобках указан номер выражения, содержащего необходимое определение.

\begin{equation}
    \label{eq:context}
    \Gamma = \left\{ x: \sigma ~|~ x \in X, \sigma \in \Sigma \right\},
\end{equation}
\begin{eqrem}
    & $X$ - множество термов,                 \\
    & $\Sigma$ - множество полиморфных типов. \\
\end{eqrem}

\begin{equation}
    \label{eq:free_types}
    \begin{aligned}
        ft(\sigma) &= ft(\tau) \backslash A, \\
        ft(\tau)   &= \left\{ \alpha ~|~ \alpha \in \tau \right\}, \\
        ft(\Gamma) &= \bigcup_{x: \sigma \in \Gamma} ft(\sigma)
    \end{aligned}
\end{equation}
\begin{eqrem}
    & $\alpha \in \tau$ - переменная типа, использованная в типе $\tau$, \\
    & $X \backslash Y$ - множество $X$, исключая элементы множества $Y$. \\
\end{eqrem}

\begin{equation}
    \label{eq:generalize}
    gn(\Gamma, \tau) = \forall A. \tau,
\end{equation}
\begin{eqrem}
    & $A = ft(\tau) \backslash ft(\Gamma)$.\\
\end{eqrem}

\begin{equation}
    \label{eq:subst}
    \mathcal{S} = \left[ \alpha_1 \coloneqq \tau_1, \alpha_2 \coloneqq \tau_2, \ldots \alpha_n \coloneqq \tau_n \right],
\end{equation}
где ни одна пара $\alpha_i \coloneqq \tau_i$ не должна быть вида $\alpha_i \coloneqq \alpha_i$, иначе будет получена подстановка для бесконечно рекурсивного типа.

Композиция подстановок:

\begin{equation}
    \label{eq:subst_comp}
    \mathcal{S}_1 \circ \mathcal{S}_2 = \mathcal{S}_1 \cup \left[ \alpha_i \coloneqq \mathcal{S}_1 \tau_i \right]
\end{equation}

Применение подстановки $\mathcal{S} \tau$ - операция замены всех вхождений очередной $\alpha_i$ из подстановки $\mathcal{S}$ на $\tau_i$ в типе $\tau$.

Будем называть $\beta$ - уникальную переменную типа.

\begin{equation}
    \label{eq:instantiate}
    it(\sigma) = \mathcal{S}_{\sigma} \tau,
\end{equation}
\begin{eqrem}
    & $\mathcal{S}_\sigma = \left[ \alpha \coloneqq \beta ~|~ \alpha \in A, \beta \ne \alpha \right]$.\\
\end{eqrem}

Правила вывода, разработанные Милнером~\cite{UrbanN2009}, позволяют получить доказательство, что любой терм, заданный выражением~\eqref{eq:terms} можно типизировать определенным типом.
Алгоритм, построенный с такими правилами, называется алгоритмом $\mathcal{W}$.
Одним из его недостатков является неточное определение места ошибки при типизации выражения.
Вместо него предлагается использовать модификацию этого алгоритма с использованием ограничений (англ. constraints) и отложенной унификацией.

Унификацией называется процесс поиска такой подстановки $\mathcal{S}$ для двух типов $\tau_1$ и $\tau_2$, что $\mathcal{S} \tau_1 = \mathcal{S} \tau_2$.
Для решения этой задачи существует алгоритм $\mathcal{U}$:

\begin{equation}
    \label{eq:algo_u}
    \begin{aligned}
        \mathcal{U}(\tau_1, \tau_1) &= \left[  \right], \\
        \mathcal{U}(\alpha_1, \tau_2) &= \left[ \alpha_1 \coloneqq \tau_2 \right] \text{ если } \alpha_1 \notin \tau_2, \\
        \mathcal{U}(\tau_1, \alpha_2) &= \left[ \alpha_2 \coloneqq \tau_1 \right] \text{ если } \alpha_2 \notin \tau_1, \\
        \mathcal{U}(\tau^{in}_1 \to \tau^{out}_1, \tau^{in}_2 \to \tau^{out}_2) &= \mathcal{U}(\tau^{in}_1, \tau^{in}_2) \cup \mathcal{U}(\tau^{out}_1, \tau^{out}_2), \\
        \mathcal{U}((\tau_1^A, \ldots, \tau_n^A), (\tau_1^B, \ldots, \tau_n^B)) &= \bigcup_{i = 1}^{n} \mathcal{U}(\tau_i^A, \tau_i^B)
    \end{aligned}
\end{equation}


\section{Использование ограничений при выводе типов}
\label{sec:constratints_usage}

Алгоритм вывода типов, использующий ограничения - алгоритм $\mathcal{W}_c$ - имеет, по сравнению с алгоритмом $\mathcal{W}$, два основных преимущества:
\begin{itemize}
    \item он использует отложенную унификацию и вместо нее работает с ограничениями, а не с подстановками, что позволяет выводить тип для более узких случаев,
    \item ему не нужен глобальный контекст при выводе типа - всю требуемую информацию он сохраняет в ограничениях и \textit{множестве предположений}.
\end{itemize}

Под ограничением $c$ будем понимать следующее определение:

\begin{equation}
    \label{eq:cst}
    \begin{aligned}
        c \coloneqq ~ &\tau_1 \equiv \tau_2, \\
        &| ~ \tau_1 \leq_{\mathcal{M}} \tau_2, \\
        &| ~ \tau \preceq \sigma
    \end{aligned}
\end{equation}

\begin{itemize}
    \item Ограничение эквивалентности ($\tau_1 \equiv \tau_2$) говорит, что типы $\tau_1$ и $\tau_2$ должны быть унифицированы.
    \item Явное ограничение на экземпляр ($\tau \preceq \sigma$) указывает, что тип $\tau$ должен быть каким-то конкретным экземпляром полиморфного типа $\sigma$.
    \item Неявное ограничение на экземпляр ($\tau_1 \leq_{\mathcal{M}} \tau_2$) показывает, что тип $\tau$ должен быть конкретным экземпляром типа, полученного после обобщения типа $\tau_2$ в контексте $\mathcal{M}$.
\end{itemize}

Последние два ограничения возникают из-за полиморфных свойств объявления переменной.
А именно - тип переменной $x$ в выражении $\text{let } x = e_1 \text{ in } e_2$ должен быть конкретным экземпляром полиморфного типа для каждого места использования.
Явное ограничение на экземпляр не подходит, так как на момент обработки выражения тип $e_1$ не может быть полиморфным.
Поэтому необходимо использовать неявное ограничение, контекст $\mathcal{M}$ которого пополняется по ходу выполнения алгоритма.

Правила вывода, используемые в алгоритме $\mathcal{W}_c$, состоят из суждений вида $\mathcal{A}, \mathcal{C} \vdash e: \tau$, где $\mathcal{A}$ - множество предположений, $\mathcal{C}$ - ограничения, $e$ - терм, $\tau$ - тип.
Сами правила представлены в подразделе~\ref{subsec:inference_rules}
В отличие от контекста $\Gamma$, используемом в алгоритме $\mathcal{W}$, в множество предположений для каждой переменной сохраняется набор её возможных типов~\eqref{eq:assumption_set}.
Кроме того, при рекурсивном применении правил, множества предположений просто соединяются между собой.

\begin{equation}
    \label{eq:assumption_set}
    \mathcal{A} = \left\{ x: T ~|~ x \in X \right\},
\end{equation}
\begin{eqrem}
    & $X$ - множество переменных,                                                        \\
    & $T = \left\{ \tau_1, \tau_2, \ldots, \tau_n \right\}$ - множество возможных типов. \\
\end{eqrem}

Неявное ограничение на экземпляр использует контекст $\mathcal{M}$.
Он может быть получен для каждого терма $e$ следующим образом:

\begin{equation}
    \label{eq:monomorphic_set}
    \begin{aligned}
        \mathcal{M}(x) &= \emptyset, \\
        \mathcal{M}(e_1(e_2)) &= \mathcal{M}(e_1) \cup \mathcal{M}(e_2), \\
        \mathcal{M}(\lambda x: \tau. e_1) &= \left\{ \beta \right\} \cup \mathcal{M}(e_1), \\
        \mathcal{M}(\text{let } x: \tau = e_1 \text{ in } e_2) &= \mathcal{M}(e_1) \cup \mathcal{M}(e_2), \\
        \mathcal{M}(\text{:num:}) &= \emptyset, \\
        \mathcal{M}((e_1, e_2)) &= \mathcal{M}(e_1) \cup \mathcal{M}(e_2), \\
        \mathcal{M}(\text{if } e_1 \text{ then } e_2 \text{ otherwise } e_3) &= \mathcal{M}(e_1) \cup \mathcal{M}(e_2) \cup \mathcal{M}(e_3), \\
    \end{aligned}
\end{equation}

\subsection{Правила вывода}
\label{subsec:inference_rules}

Каждое правило записано одним выражением в том виде, который определен в разделе~\ref{sec:type_theory}.
Для каждого варианта терма определено собственное правило.
Таким образом любой терм может быть типизирован.

Правило для варианта терма $e = x$:

\begin{equation}
    \label{eq:var_infer}
    \frac{}{\left\{ x: \left\{ \beta \right\} \right\}, \emptyset \vdash e: \beta}
\end{equation}

Правило для варианта терма $e = e_1(e_2)$:

\begin{equation}
    \label{eq:app_infer}
    \frac{
        \mathcal{A}_1, \mathcal{C}_1 \vdash e_1: \tau_1 ~~ \mathcal{A}_2, \mathcal{C}_2 \vdash e_2: \tau_2
    }{
        \mathcal{A}_1 \cup \mathcal{A}_2, \mathcal{C}_1 \cup \mathcal{C}_2 \cup \left\{ \tau_1 \equiv \tau_2 \to \beta \right\} \vdash e: \beta
    }
\end{equation}

Правило для варианта терма $e = \lambda x: \tau. e_1$:

\begin{equation}
    \label{eq:abs_infer}
    \frac{
        \mathcal{A}, \mathcal{C} \vdash e_1: \tau''
    }{
        \mathcal{A} \backslash x, \mathcal{C} \cup \left\{ \tau' \equiv \beta ~|~ x: \tau' \in \mathcal{A} \right\} \cup \left\{ \beta \equiv \tau \right\} \vdash e: (\beta \to \tau'')
    }
\end{equation}

Правило для варианта терма $e = \text{let } x: \tau = e_1 \text{ in } e_2$:

\begin{equation}
    \label{eq:let_infer}
    \frac{
        \mathcal{A}_1, \mathcal{C}_1 \vdash e_1: \tau_1 ~~ \mathcal{A}_2, \mathcal{C}_2 \vdash e_2: \tau_2
    }{
        \mathcal{A}_1 \cup \mathcal{A}_2 \backslash x, \mathcal{C}_1 \cup \mathcal{C}_2 \cup \left\{ \tau' \leq_{\mathcal{M}} \tau_1 ~|~ x: \tau' \in \mathcal{A}_2 \right\} \cup \left\{ \tau \leq_{\mathcal{M}} \tau_1 \right\} \vdash e: \tau_2
    }
\end{equation}

Правило для варианта терма $e = \text{:num:}$:

\begin{equation}
    \label{eq:num_infer}
    \frac{}{\emptyset, \emptyset, \vdash e: \text{Integer}}
\end{equation}

Правило для варианта терма $e = (e_1, e_2)$:

\begin{equation}
    \label{eq:tuple_infer}
    \frac{
        \mathcal{A}_1, \mathcal{C}_1 \vdash e_1: \tau_1 ~~ \mathcal{A}_2, \mathcal{C}_2 \vdash e_2: \tau_2
    }{
        \mathcal{A}_1 \cup \mathcal{A}_2, \mathcal{C}_1 \cup \mathcal{C}_2 \vdash e: (\tau_1, \tau_2)
    }
\end{equation}

Правило для варианта терма $e = \text{if } e_1 \text{ then } e_2 \text{ otherwise } e_3$:

\begin{equation}
    \label{eq:if_infer}
    \frac{
        \mathcal{A}_1, \mathcal{C}_1 \vdash e_1: \tau_1 ~~ \mathcal{A}_2, \mathcal{C}_2 \vdash e_2: \tau_2 ~~ \mathcal{A}_3, \mathcal{C}_3 \vdash e_3: \tau_3
    }{
        \mathcal{A}_1 \cup \mathcal{A}_2 \cup \mathcal{A}_3, \mathcal{C}_1 \cup \mathcal{C}_2 \cup \mathcal{C}_3 \cup \left\{ \tau_1 \equiv \text{Boolean}, \tau_2 \equiv \tau_3 \right\} \vdash e: \tau_3
    }
\end{equation}

Продемонстрируем применение правил вывода на примере следующего выражения:

\begin{equation}
    \label{eq:expr_example}
    \begin{aligned}
        \lambda w. ~&\text{let } y = w \\
        &\text{in let } x = y(0) \\
        &\text{in } x
    \end{aligned}
\end{equation}

В этом выражении присутствует лямбда-функция, поэтому все места использования переменной $w$ должны иметь одинаковый тип.
В результате применения правил (\todo{добавить их в приложение}), получим набор ограничений~\ref{eq:consts_example} и предварительный тип терма - $\tau_5 \to \tau_4$.
В нем содержится два неявных ограничения, образованных использованием конструкции $\text{let } \ldots$, с контекстом $\mathcal{M} = \left\{ \tau_5 \right\}$ - типом переменной $w$.

\begin{equation}
    \label{eq:consts_example}
    \begin{aligned}
        \mathcal{C}_{example} &= \left\{ \tau_2 \equiv \text{Integer} \to \tau_3 \right\} \\
        &\cup \left\{ \tau_4 \leq_{\left\{ \tau_5 \right\}} \tau_3 \right\} \\
        &\cup \left\{ \tau_2 \leq_{\left\{ \tau_5 \right\}} \tau_1 \right\} \\
        &\cup \left\{ \tau_5 \equiv \tau_1 \right\}
    \end{aligned}
\end{equation}


\section{Решение ограничений}
\label{sec:constraint_solving}

После применения правил вывода к терму, получаются множества ограничений и предположений.
В то время как множество предположений не требует дополнительной обработки, множество ограничений необходимо разрешить, используя специальный алгоритм.
В результате будет получена такая подстановка $\mathcal{S}$, которая будет верно типизировать заданный терм.

К набору ограничений сама по себе может быть применена подстановка согласно следующему правилу:

\begin{equation}
    \label{eq:consts_subst}
    \begin{aligned}
        \mathcal{S} (\tau_1 \equiv \tau_2)             &= \mathcal{S} \tau_1 \equiv \mathcal{S} \tau_2, \\
        \mathcal{S} (\tau \preceq \sigma)              &= \mathcal{S} \tau \preceq \mathcal{S} \sigma,  \\
        \mathcal{S} (\tau_1 \leq_{\mathcal{M}} \tau_2) &= \mathcal{S} \tau_1 \leq_{\mathcal{S} \mathcal{M}} \mathcal{S} \tau_2,
    \end{aligned}
\end{equation}
\begin{eqrem}
    & $\mathcal{S} \mathcal{M}$ - применение подстановки к каждому типу из множества $\mathcal{M}$, \\
    & $\mathcal{S} \sigma = \forall A. \mathcal{S} \tau$.                                           \\
\end{eqrem}

Определим, какие переменные типа активны в текущем контексте используя~\ref{eq:free_types}:

\begin{equation}
    \label{eq:active_vars}
    \begin{aligned}
        active(\tau_1 \equiv \tau_2) &= ft(\tau_1) \cup ft(\tau_2), \\
        active(\tau \preceq \sigma) &= ft(\tau) \cup ft(\sigma), \\
        active(\tau_1 \leq_{\mathcal{M}} \tau_2) &= ft(\tau_1) \cup (ft(\mathcal{M}) \cap ft(\tau_2))
    \end{aligned}
\end{equation}

Наконец, определим алгоритм решения множества ограничений~\eqref{eq:solve_algo}.
На вход он принимает набор ограничений, а в результате будет получена подстановка.
Из алгоритма видно, что явное и неявное ограничения сводятся к форме ограничения эквивалентности, после чего из него получается подстановка.

\begin{equation}
    \label{eq:solve_algo}
    \begin{aligned}
        solve(\emptyset) &= \emptyset, \\
        solve(\left\{ \tau_1 \equiv \tau_2 \right\} \cup \mathcal{C}) &= solve(\mathcal{S} \mathcal{C}) \circ \mathcal{S}, \\
        solve(\left\{ \tau \preceq \sigma \right\} \cup \mathcal{C}) &= solve(\left\{ \tau \equiv it(\sigma) \right\} \cup C), \\
        solve(\left\{ \tau_1 \leq_{\mathcal{M}} \tau_2 \right\} \cup \mathcal{C}) &= solve(\left\{ \tau_1 \preceq gn(\mathcal{M}, \tau_2) \right\} \cup C) \\
        &\text{если } (ft(\tau_2) \backslash \mathcal{M}) \cap active(\mathcal{C}) = \emptyset,
    \end{aligned}
\end{equation}
\begin{eqrem}
    & $\mathcal{S} = \mathcal{U}(\tau_1, \tau_2)$.\\
\end{eqrem}

Рассмотрим решение множества ограничений~\ref{eq:consts_example}:

\begin{equation}
    \label{eq:consts_solve}
    \begin{aligned}
        &solve(\mathcal{C}_{example}) = \\
        &= solve(\left\{ \tau_2 \equiv \text{I} \to \tau_3, \tau_4 \leq_{\left\{ \tau_5 \right\}} \tau_3, \tau_2 \leq_{\left\{ \tau_5 \right\}} \tau_1, \tau_5 \equiv \tau_1 \right\}) &= \\
        &= solve(\left\{ \tau_4 \leq_{\left\{ \tau_5 \right\}} \tau_3, \text{I} \to \tau_3 \leq_{\left\{ \tau_5 \right\}} \tau_1, \tau_5 \equiv \tau_1 \right\}) \circ \left[ \tau_2 \coloneqq \text{I} \to \tau_3 \right] &= \\
        &= solve(\left\{ \tau_4 \leq_{\left\{ \tau_1 \right\}} \tau_3, \text{I} \to \tau_3 \leq_{\left\{ \tau_1 \right\}} \tau_1 \right\})                       \circ \left[ \tau_5 \coloneqq \tau_1 \right] \circ \left[ \tau_2 \coloneqq \text{I} \to \tau_3 \right] &= \\
        &= solve(\left\{ \tau_4 \leq_{\left\{ \tau_1 \right\}} \tau_3, \text{I} \to \tau_3 \preceq \tau_1 \right\})                                              \circ \left[ \tau_5 \coloneqq \tau_1 \right] \circ \left[ \tau_2 \coloneqq \text{I} \to \tau_3 \right] &= \\
        &= solve(\left\{ \tau_4 \leq_{\left\{ \tau_1 \right\}} \tau_3, \text{I} \to \tau_3 \equiv \tau_1 \right\})                                               \circ \left[ \tau_5 \coloneqq \tau_1 \right] \circ \left[ \tau_2 \coloneqq \text{I} \to \tau_3 \right] &= \\
        &= solve(\left\{ \tau_4 \leq_{\left\{ \tau_3 \right\}} \tau_3 \right\})                                                                                  \circ \left[ \tau_1 \coloneqq \text{I} \to \tau_3 \right] \circ \left[ \tau_5 \coloneqq \tau_1 \right] \circ \left[ \tau_2 \coloneqq \text{I} \to \tau_3 \right] &= \\
        &= solve(\left\{ \tau_4 \preceq \tau_3 \right\})                                                                                                         \circ \left[ \tau_1 \coloneqq \text{I} \to \tau_3 \right] \circ \left[ \tau_5 \coloneqq \tau_1 \right] \circ \left[ \tau_2 \coloneqq \text{I} \to \tau_3 \right] &= \\
        &= solve(\left\{ \tau_4 \equiv \tau_3 \right\})                                                                                                          \circ \left[ \tau_1 \coloneqq \text{I} \to \tau_3 \right] \circ \left[ \tau_5 \coloneqq \tau_1 \right] \circ \left[ \tau_2 \coloneqq \text{I} \to \tau_3 \right] &= \\
        &= solve(\emptyset)                                                                                                                                      \circ \left[ \tau_4 \coloneqq \tau_3 \right] \circ \left[ \tau_1 \coloneqq \text{I} \to \tau_3 \right] \circ \left[ \tau_5 \coloneqq \tau_1 \right] \circ \left[ \tau_2 \coloneqq \text{I} \to \tau_3 \right] &= \\
        &= \left[ \tau_4 \coloneqq \tau_3, \tau_1 \coloneqq \text{I} \to \tau_3, \tau_5 \coloneqq \text{I} \to \tau_3, \tau_2 \coloneqq \text{I} \to \tau_3 \right],
    \end{aligned}
\end{equation}
\begin{eqrem}
    & $\text{I} = \text{Integer}$.\\
\end{eqrem}

Применив полученную подстановку к предварительному типу $\tau_5 \to \tau_4$, получим наиболее общий тип выражения~\ref{eq:expr_example}: $(\text{Integer} \to \tau_3) \to \tau_3$.
Как альтернативу полученному типу, можно привести следующий пример из языка C++:

\todo{может лучше std::function<T(std::function<T(int)>)>}
\begin{lstlisting}[label={lst:type_example},language=C++,caption={Вид полученного типа с точки зрения языка C++.}]
    template<typename T>
    using ResultingType = T (std::function<T (int)>)(*);
\end{lstlisting}

Согласно определению~\ref{eq:solve_algo}, существует возможность, что два неявных ограничения на экземпляр будут зависимы друг от друга.
Например, $\mathcal{C} = \left\{ \tau_1 \leq_{\emptyset} \tau_2, \tau_2 \leq_{\emptyset} \tau_1 \right\}$.
В таком случае обе переменных типа являются активными (согласно~\ref{eq:active_vars}) и множество ограничений не может быть разрешено.
Однако такое множество не может быть создано путём работы алгоритма $\mathcal{W}_c$.


\section{Алгоритм вывода типов $\mathcal{W}_c$}
\label{sec:inference_algo}

Используя рассмотренные выше определения и алгоритмы, можно определить и сам алгоритм для вывода типов $\mathcal{W}_c$.
Как уже было сказано, он использует правила вывода из раздела~\ref{subsec:inference_rules} и алгоритм решения ограничений.
На вход он принимает контекст $\Gamma$ и терм $e$, возвращает - подстановку $\mathcal{S}$ и тип терма $\tau$.

Введём явное ограничение на экземпляр и для множеств $\mathcal{A}$ и $\Gamma$~\eqref{eq:explicit_cst}.
Благодаря этому определению, типизируемый терм может использовать переменные, определенные вне него.
Например, пусть $\mathcal{A} = \left\{ id: \tau_2, id: \tau_3, f: \tau_4 \right\}$, а $\Gamma = \left\{ id: \forall \left\{ \alpha_1 \right\}. \alpha_1 \to \alpha_1, f: \tau_1 \to \tau_1 \right\}$.
Тогда $\mathcal{A} \preceq \Gamma = \left\{ \tau_2 \preceq \forall \left\{ \alpha_1 \right\}. \alpha_1 \to \alpha_1, \tau_3 \preceq\forall \left\{ \alpha_1 \right\}. \alpha_1 \to \alpha_1, \tau_4 \preceq \tau_1 \to \tau_1 \right\}$.
Таким образом внешние переменные правильно используются в алгоритме.

\begin{equation}
    \label{eq:explicit_cst}
    \mathcal{A} \preceq \Gamma = \left\{ \tau \preceq \sigma ~|~ x: \tau \in \mathcal{A}, x: \sigma \in \Gamma \right\}
\end{equation}

На рисунке~\ref{fig:infer_algo} представлена блок-схема рассматриваемого алгоритма.
Из нее видно, что процесс применения правил вывода не зависит от использования контекста $\Gamma$, что позволяет типизировать любой терм, вне зависимости от внешних переменных.
Для определения того, что все переменные из множества предположений определены, используется условие $dom(\mathcal{A}) \nsubseteq dom(\Gamma) = \left\{ x ~|~ x: \tau \in \mathcal{A} \right\} \nsubseteq \left\{ x ~|~ x: \sigma \in \Gamma \right\}$.

\begin{figure}[H]
    \centering
    \input{figures/.generated/infer_algo}
    \caption{Блок-схема алгоритма $\mathcal{W}_c$}
    \label{fig:infer_algo}
\end{figure}


%----------------------------------------------------------
