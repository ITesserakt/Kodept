%----------------------------------------------------------
\chapter*{ЗАКЛЮЧЕНИЕ}
\label{ch:conclusion}
\addcontentsline{toc}{chapter}{ЗАКЛЮЧЕНИЕ}
%----------------------------------------------------------

В результате выполнения данной работы был реализован механизм вывода типов для языка программирования Kodept.
Для этого была сформирована система типов на основе системы типов Хиндли-Милнера, а также алгоритм для вывода типов $\mathcal{W}_c$.
Показано, что программная реализация в виде компилятора может выводить типы в предоставляемых исходных файлах на языке Kodept.

В ходе работы рассмотрена архитектура компилятора, а также методы, применяемые при работе с абстрактным синтаксическим деревом.
Рассмотрены программные модули, отвечающие как за сам вывод типов, так и за смежные с ним части компилятора.
Успешно решены все поставленные задачи, а именно:
\begin{enumerate}[1)]
    \item проанализированы некоторые системы типов в современных языках программирования,
    \item приведена модификация системы типов Хиндли-Милнера,
    \item рассмотрен алгоритм вывода типов согласно выбранной системы типов,
    \item реализован модуль в компиляторе языка Kodept для семантического анализа,
    \item выполнено тестирование компилятора на правильность вывода типов.
\end{enumerate}

На данной работе разработка как языка программирования Kodept, так и компилятора для него не завершается.
Поэтому в качестве дальнейших планов можно выделить следующие:
\begin{itemize}
    \item дальнейшая доработка и усложнение системы типов,
    \item расширение семантического анализатора,
    \item доработка компилятора: добавление поддержки многопоточного выполнения, реализация следующих стадий компиляции, таких как генерация промежуточного представления и генерация машинного кода.
\end{itemize}

%----------------------------------------------------------
