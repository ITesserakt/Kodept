\subsection{Сравнение реализуемого подхода с другими}
%%%%%%%%%%%%%%%%%%%%%%%%%%%%%%%%%%%%%%%%%%%
\begin{frame}
    \begin{block}{Общие обозначения}
        \emph{Графовой моделью} вычислительного метода назовём совокупность операций обработки данных, которые включает в себя данный метод, и ориентированный граф, определяющий очерёдность и логику выполнения обозначенных операций.
    \end{block}

\end{frame}

\begin{frame}
    \resizebox{\textwidth}{!}
    {
        \begin{tabular}{|p{0.025\paperwidth}|p{0.275\paperwidth}|p{0.35\paperwidth}|p{0.35\paperwidth}|}
            \hline
            \textbf{№} & \textbf{Признак} & \textbf{GBSE} & \textbf{DFD} \\
            \hline
            1          &                  &               &              \\
            \hline
        \end{tabular}
    }
\end{frame}
%%%%%%%%%%%%%%%%%%%%%%%%%%%%%%%%%%%%%%%%%%%