%%%%%%%%%%%%%%%%%%%%%%%%%%%%%%%%%%%%%%%%%%%
%\subsection{}
%\def\subsecname{}
%%%%%%%%%%%%%%%%%%%%%%%%%%%%%%%%%%%%%%%%%%%
% --------------------------------------------------------------------------- %
\subsection*{Анализ результатов}
% --------------------------------------------------------------------------- %
\begin{frame}%[allowframebreaks=0.9,t]

	В результате выполнения работы:
	\begin{enumerate}[1)]
		\item расширены функциональные возможности библиотеки comsdk
		\item создана новая архитектура классов, позволяющая в дальнейшем упростить процесс автоматического формирования программного представления графовых моделей;
		\item разработаны структуры данных для программного представления графовых моделей алгоритмов и их элементов;
		\item был разработан алгоритм, осуществляющий выполнение этапов алгоритма в соответствии с его графовой моделью;
	\end{enumerate}

\end{frame}
% --------------------------------------------------------------------------- %
\subsection*{Перспективы разработки}
% --------------------------------------------------------------------------- %
\begin{frame}
	Перспективы развития программного каркаса comsdk включают в себя:
	\begin{itemize}
		\item реализации алгоритма обхода графовой модели с задействованием различных вычислительных ресурсов
		\item интеграцию средства генерации форм ввода\footcite{SokolovPershin2017};
		\item разработку средства визуализации обрабатываемых данных;
		\item разработку средства автоматической документации реализуемых алгоритмов;
	\end{itemize}

\end{frame}

%%%%%%%%%%%%%%%%%%%%%%%%%%%%%%%%%%%%%%%%%%%
