%----------------------------------------------------------
\newcommand\ddfrac[2]{\frac{\displaystyle #1}{\displaystyle #2}}
\newcommand\T{\textup{T}}
\def\argmin{\operatornamewithlimits{argmin}}
\def\argmax{\operatornamewithlimits{argmax}}
\def\diag{\operatornamewithlimits{diag}}
%----------------------------------------------------------
\newcommand{\bvec}[1]{\mbox{\mathversion{bold}${#1}$}}
\newcommand{\periodichnost}[1]{[\hspace{-2pt}[#1]\hspace{-2pt}]}
\newcommand{\volna}[1]{\mathop{#1}\limits\_{\sim}}
\newcommand{\crujok}[1]{#1}
\newcommand{\driv}[2]{\frac{d#1}{d#2}}
\newcommand{\pdriv}[2]{\frac{\partial #1}{\partial #2}}
\newcommand{\uparr}[1]{\overrightarrow{#1}}
\newcommand{\intThreeD}[1]{\hspace{6pt}\mathop{\int\hspace{-17pt}\int\hspace{-17pt}\int}\limits\_{#1}\hspace{6pt}}
\newcommand{\intTwoD}[1]{\hspace{6pt}\mathop{\int\hspace{-17pt}\int}\limits\_{\hspace{-6pt}#1}}
%\newcommand{\bold}[1]{\mbox{\textbf{#1}}}
%----------------------------------------------------------
\newcommand{\multirowbox}[2]
{
   \parbox{#1}{\smallskip #2 \smallskip}
}
%----------------------------------------------------------
\newcommand{\multirowboxfixed}[1]
{
   \parbox{7cm}{\smallskip #1 \smallskip}
}
%----------------------------------------------------------
\makeatletter
\def\renewtheorem#1{%
	\expandafter\let\csname#1\endcsname\relax
	\expandafter\let\csname c@#1\endcsname\relax
	\gdef\renewtheorem@envname{#1}
	\renewtheorem@secpar
}
\def\renewtheorem@secpar{\@ifnextchar[{\renewtheorem@numberedlike}{\renewtheorem@nonumberedlike}}
\def\renewtheorem@numberedlike[#1]#2{\newtheorem{\renewtheorem@envname}[#1]{#2}}
\def\renewtheorem@nonumberedlike#1{
	\def\renewtheorem@caption{#1}
	\edef\renewtheorem@nowithin{\noexpand\newtheorem{\renewtheorem@envname}{\renewtheorem@caption}}
	\renewtheorem@thirdpar
}
\def\renewtheorem@thirdpar{\@ifnextchar[{\renewtheorem@within}{\renewtheorem@nowithin}}
\def\renewtheorem@within[#1]{\renewtheorem@nowithin[#1]}
\makeatother
%----------------------------------------------------------
\usepackage{amsthm}

\newtheoremstyle{mythr} % namei
{6pt}  % Space above
{3pt}  % Space below
{\normalfont}% Body font
{\parindent} % Indent amount
{\bfseries}  % Theorem head font
{}          % Punctuation after theorem head
{\newline}   % Space after theorem head
{}           % Theorem head spec (can be left empty, meaning ‘normal’)i

\theoremstyle{mythr}

\newtheorem{remark}{\texttt{Замечание}}
\newtheorem{syntax}{\textbf{\underline{Синтаксис}}}
%\renewtheorem{theorem}{Теорема}
%\newtheorem{subtheorem}{Теорема}
%\newtheorem{lemm}{Лемма}
%\newtheorem{sublemm}{Лемма}
%\renewtheorem{corollary}{Следствие}
\newtheorem{statement}{Задача}
\newtheorem{demand}{Требование}

\renewtheorem{definition}{Определение}
\newtheorem{designation}{Обозначение}
%\newtheorem{property}{\texttt{Свойство}}
\newtheorem{task}{\texttt{Задача}}
%\newtheorem{mathmod}{\texttt{Математическая модель}}
%\newtheorem{mytheorem}{\texttt{Теорема}}
%\newtheorem{task}{Задача}

% Необходимо для обеспечения переносов на следующую строку в длинных заголовках в окружениях newtheorem
%\makeatletter
%\renewcommand*{\@opargbegintheorem}[3]{\trivlist
%      \item[\hskip \labelsep{\bfseries #1\ #2}] \textbf{(#3)}\ \itshape}
%\makeatother
%----------------------------------------------------------
\def\pbs{\raggedright\baselineskip3.0ex}%
\def\Dfmn{Д-р~физ.-мат.~наук}
\def\dfmn{д-р~физ.-мат.~наук}
\def\dtn{д-р~тех.~наук}
\def\Kfmn{Канд.~физ.-мат.~наук}
\def\Ktn{Канд.~тех.~наук}
\def\ktn{канд.~тех.~наук}
\def\kfmn{канд.~физ.-мат.~наук}
\def\mns{мл.~науч.~сотр.}
\def\sns{ст.~науч.~сотр.}
\def\Tehn{Техник}
\def\tehn{техник\enskip}
\def\signvrule{\vrule height 0.5pt width30mm depth0pt}
\def\Inj{Инженер}
\def\inj{инженер}
\def\Dotsent{Доцент}
\def\dotsent{доцент}
\def\prof{проф.}
\def\regi{\textsuperscript{\textregistered}\enskip}
\newcommand{\datetofill}{<<\underline{~~~~~}>>~\underline{~~~~~~~~~~~~~~~~~~~}~\Year~\cyrg.}

\def\appdxSymb{А}

%----------------------------------------------------------
\graphicspath{{pictures/}}
%\usepackage{cite}
%\usepackage{changepage}
%\usepackage{pdflscape}
%\EqInChaper % формулы будут нумероваться в пределах раздела
%\TableInChaper % таблицы будут нумероваться в пределах раздела
%\PicInChaper   % рисунки будут нумероваться в пределах раздела
%\setlength\GostItemGap{2mm}% для красоты можно менять от 0мм
\bibliographystyle{unsrt} %Стиль библиографических ссылок БибТеХа
%\usepackage{RNP112_sa}
%----------------------------------------------------------
\newcommand{\pageno}[1]{\setcounter{page}{#1}}
\newcommand{\chapterno}[1]{\setcounter{chapter}{#1}\addtocounter{chapter}{-1}}
%----------------------------------------------------------
\newcommand{\PersonSign}[1]
{
    \textunderset{}{\signvrule} \quad #1\par
    \vskip2mm
    \datetofill
}
%----------------------------------------------------------
\newcommand{\PersonPosition}[1]
{
   #1\\
}
%----------------------------------------------------------
\newcommand{\PersonPositionWithSign}[2]
{
	\PersonPosition{#1}\par
  \vskip4mm
	\PersonSign{#2}
}
%----------------------------------------------------------
\newcommand{\PersonPositionWithSignRight}[2]
{
    \parbox[t]{0.5\textwidth}
    {
       \raggedright
	     \PersonPositionWithSign{#1}{#2}
       \vskip6mm
    }
}
%----------------------------------------------------------
\newcommand{\SignOKEH}[3]
{
    \parbox[t]{75.0mm}
    {
       \centering
       #1\par
       \vskip2mm
			 \PersonPositionWithSign{#2}{#3}
       \vskip2mm
    }
}
%----------------------------------------------------------
\newcommand{\Appendix}[2]
{
\stepcounter{appdxcnt}
%\newpage
\addcontentsline{toc}{chapter}{ПРИЛОЖЕНИЕ #1 -- #2}%\label{appdx#1}
%\chapter*{chapter}{#2}%\label{appdx#1}
\bgroup
\begin{center}
		ПРИЛОЖЕНИЕ #1\\
		(к отчету о НИР) \\
		\vskip6mm
		{\Large\textbf{#2}}
\end{center}
\egroup
}
%----------------------------------------------------------
\def\Chapter#1{%
	\refstepcounter{appdxcnt}
  \refstepcounter{chapter}% +1
  \setcounter{section}0%
  \setcounter{figure}0%
  \setcounter{table}0%
  \setcounter{equation}0%
\newpage
\begin{center}
		\chaptername~\thechapter\\
		(к отчету о НИР) \\
		\vskip6mm
		{\Large\textbf{#1}}
\end{center}
  \addcontentsline{toc}{chapter}{\chaptername~\thechapter.~#1}%
}
%----------------------------------------------------------
\def\ChapterEX#1{%
	\refstepcounter{appdxcnt}
  \refstepcounter{chapter}% +1
  \setcounter{section}0%
  \setcounter{figure}0%
  \setcounter{table}0%
  \setcounter{equation}0%
\newpage
\addcontentsline{toc}{chapter}{\chaptername~\thechapter.~#1}%
}
%----------------------------------------------------------
\newcommand{\AppendixEX}[2]
{
\stepcounter{appdxcnt}
\newpage
\addcontentsline{toc}{section}{ПРИЛОЖЕНИЕ #1 -- #2}
}
%----------------------------------------------------------
% Подписи к листингам на русском языке.
\renewcommand*\thelstnumber{\oldstylenums{\the\value{lstnumber}}}
\renewcommand\lstlistingname{\cyr\CYRL\cyri\cyrs\cyrt\cyri\cyrn\cyrg}
\renewcommand\lstlistlistingname{\cyr\CYRL\cyri\cyrs\cyrt\cyri\cyrn\cyrg\cyri}

\newcommand\executer[3]{\pbs #1 & \textunderset{\scriptsize{подпись, дата}}{\signvrule} & #2 #3 \\}
\newcommand\executertit[3]{
\pbs #1 & \vspace*{\fill} \textunderset{\scriptsize{подпись, дата}}{\signvrule} & \vspace*{\fill} #2 \newline #3 \\
}
%----------------------------------------------------------
%\input{doc-spec/edupmi_evn_PMSoftware_DeploymentDistribution-spec.def}

\hyphenation{супер-ком-пью-тер-ный}
\hyphenation{супер-ком-пью-тер-ное}
\hyphenation{вы-со-ко-произво-ди-тель-ный}
\makeatletter
\def\makefnmark{\hbox{\textsuperscript{\normalfont\@thefnmark)}}}
\makeatother

%\newcolumntype{x}[1]{ >{\centering\hspace{0pt}}p{#1}}%
\newcolumntype{x}[1]{ >{\centering}p{#1}}%

%----------------------------------------------------------
%% Следующие команды можно использовать для форматирования абзацев
%% в таблице при использовании пакетов типа tabularx, mdwtab:
\def\L{\leftskip=0pt \rightskip=0pt plus 1 fil}
\def\C{\leftskip=0pt plus 1 fil \rightskip=0pt plus 1 fil \parfillskip=0pt}
\def\CB{\C \baselineskip=12pt}
\def\CBF{\C \baselineskip=10pt}
\def\LB{\L \baselineskip=12pt}
\def\LBF{\L \baselineskip=10pt}
\def\TCS#1 {\tabcolsep=#1\tabcolsep}


\newcount\NN \NN=0
\def\NNN{\global\advance\NN by 1\relax\the\NN\relax}
%----------------------------------------------------------
% добавление поддержки команды вывода текста на полях \marginnote
\usepackage{marginnote}
% добавление поддержки команды \color
\usepackage{xcolor}
%Вывод восклицательного знака на полях для указания о необходимости внесения корректировки
\newcommand{\mes}{\marginnote{\color[rgb]{1,0,0}\Huge\textbf{!}}[-1.0cm]}
%----------------------------------------------------------
\newenvironment{arreqn}[2]
    {\smaller[1]
		\begin{block}{\smaller[1] #1}
    \begin{equation}\label{#2}
    \begin{array}{l}
    }
		{
    \end{array}
    \end{equation}
		\end{block}
    }
%----------------------------------------------------------









