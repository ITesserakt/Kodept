%----------------------------------------------------------
%Термины и определения по тексту в большинстве случаев выделяются курсивом.

%%%%% для обычных newglossaryentry по умолчанию category==general.
%%%%% для обычных newabbreviation по умолчанию category==abbreviation.
%%%%% команда для создания своей категории \glscategory{<label>}

%\newabbreviation[category=initialism]{aINI}{aINI}{Расширенный формат \href{https://en.wikipedia.org/wiki/INI_file}{INI} (\href{http://sa2systems.ru/svn/public/sa2pdf/comfrm_ugd_AdvancedINI.pdf}{описание в документе <<Соколов А.П., Першин А.Ю. Руководство системного программиста. Формат данных Advanced INI (aINI) // Каркас системы comfrm – 2007-2017 – 18 стр.>>})}

%\newglossaryentry{AI}{name={ActionItem}, description={Функциональная возможность в системе \gls{dcs-gcd}. Понятие, введенное с целью определить абстракции над функциями различных типов, на основе которых возможно расширение функциональных возможностей разрабатываемой программной системы.}}

%\newglossaryentry{slver}{name={Solver}, description={Решатель системы \gls{dcs-gcd}. Регистрируется в таблице \textbf{com.slvrs} БД \gls{gcddb} \gls{dcs-gcd}.}}

\newabbreviation[category=initialism]{PO}{ПО}{Программное обеспечение}
%\newabbreviation[category=initialism]{ASUTP}{АСУ ТП}{Автоматизированная система управления технологическими процессами.}
%\newabbreviation[category=initialism]{chb}{БзЧ}{базовая часть}

%\newglossaryentry{Gij}{name={$G_{ij}$},
	%	symbol={\ensuremath{\mathcal{\nu}}},
%	category=symbol,
%	description={модули сдвига ортотропного материала}}
 
%\GlsXtrEnableEntryCounting
%{abbreviation}% list of categories to use entry counting
%{2}% trigger value

%\GlsXtrEnableEntryCounting
%{symbol}% list of categories to use entry counting
%{2}% trigger value


%----------------------------------------------------------



