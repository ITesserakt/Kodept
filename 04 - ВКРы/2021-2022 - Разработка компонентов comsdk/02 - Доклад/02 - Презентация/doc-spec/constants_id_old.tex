%----------------------------------------------------------
% общие
\newcommand{\UpperFullOrganisationName}{Министерство образования и науки Российской Федерации}
\newcommand{\ShortOrganisationName}{МГТУ~им.~Н.Э.~Баумана}
\newcommand{\FullOrganisationName}{федеральное государственное бюджетное образовательное учреждение высшего профессионального образования\\
\MakeUppercase{<<Московский государственный технический}\\ 
\MakeUppercase{университет имени Н.Э.~Баумана>>}\\
(\ShortOrganisationName)}

\def\No{№}

%тема работы
\newcommand{\Title}{@Полная Тема@}
\newcommand{\ShortTitle}{@Краткая Тема@}

\newcommand{\Author}{@Фамилия@, @Имя@ @Отчество@}
\newcommand{\ShortAuthor}{@Фамилия@, @И@.@О@.}
%%%%%%%%%%%%%%%%%%%%%%%%%%%%%%%%%%%%%%%%%%%%%%%%%%%%%%%%%%%%%%%%%%%%%%%%%%%%%%%%
% Параметры для мероприятий: лабы, лекции, семинары, мозговые штурмы
\newcommand{\eventtype}{семинар (мини-конференция)} % Тип мероприятия (лекция|лабораторная работа|семинар|мозговой штурм)
\newcommand{\eventplace}{кафедра РК-6 "Системы автоматизированного проектирования"} % Место проведения мероприятия
\newcommand{\eventduration}{10 минут} % Продолжительность мероприятия
\newcommand{\eventID}{} % Идентификатор мероприятия (документа в рамках серии документов)
\newcommand{\tutorFullUSName}{@должность@ @кафедра@, @уч.степень@, @Фамилия@, @Имя@ @Отчество@}

\newcommand{\ObjectOfResearch}{ObjectOfResearch} % Объект исследования (с маленькой буквы и без точки на конце)
\newcommand{\GoalOfResearch}{GoalOfResearch} % Цель исследования (с маленькой буквы и без точки на конце)
\newcommand{\MainTaskOfResearch}{MainTaskOfResearch} % Главная задача исследования (с маленькой буквы и без точки на конце)

% Данные по документу: тип
\newcommand{\WorkTypeShort}{WorkTypeShort} % НИР / ОКР / ОТР / ПИ / ПНИ
\newcommand{\ReportType}{курсовая работа} % о научно-исследовательской работе / об опытно-конструкторской работе / об опытно-технологической работе / о патентных исследованиях
\newcommand{\DocumentType}{\ReportType} % описание применения / описание программы / УП (учебное пособие) / МУ (Методические указания)...
\newcommand{\NIRReportType}{\ReportType} % заключительный / промежуточный

\newcommand{\keywordsru}{keywordsru} % распределенные системы ...} % 5-15 слов или выражений
\newcommand{\keywordsen}{keywordsen} %distributed systems ...}
\newcommand{\CompilationDate}{\SVNDate}
\newcommand{\Year}{@Год@}
%\newcommand{\CompileAffiliation}{\textit{UNDEFINED}}
\newcommand{\Town}{Москва}
\newcommand{\Country}{Россия}
\newcommand{\City}{Москва}

\newcommand{\Preface}{@Аннотация@}
%----------------------------------------------------------

