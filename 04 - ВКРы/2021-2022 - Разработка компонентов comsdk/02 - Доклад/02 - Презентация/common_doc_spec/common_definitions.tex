%----------------------------------------------------------
\newcommand\ddfrac[2]{\frac{\displaystyle #1}{\displaystyle #2}}
\newcommand\T{\textup{T}}
\def\argmin{\operatornamewithlimits{argmin}}
\def\argmax{\operatornamewithlimits{argmax}}
\def\diag{\operatornamewithlimits{diag}}
\def\dep{\operatornamewithlimits{dep}}
%----------------------------------------------------------
\newcommand{\bvec}[1]{\mbox{\mathversion{bold}${#1}$}}
\newcommand{\periodichnost}[1]{[\hspace{-2pt}[#1]\hspace{-2pt}]}
\newcommand{\volna}[1]{\mathop{#1}\limits\_{\sim}}
\newcommand{\crujok}[1]{#1}
\newcommand{\driv}[2]{\frac{d#1}{d#2}}
\newcommand{\pdriv}[2]{\frac{\partial #1}{\partial #2}}
\newcommand{\uparr}[1]{\overrightarrow{#1}}
\newcommand{\intThreeD}[1]{\hspace{6pt}\mathop{\int\hspace{-17pt}\int\hspace{-17pt}\int}\limits\_{#1}\hspace{6pt}}
\newcommand{\intTwoD}[1]{\hspace{6pt}\mathop{\int\hspace{-17pt}\int}\limits\_{\hspace{-6pt}#1}}
%----------------------------------------------------------
\newcommand{\multirowbox}[2]
{
  \parbox{#1}{\smallskip #2 \smallskip}
}
%----------------------------------------------------------
\usepackage{amsthm}
\usepackage{thmtools}

\declaretheoremstyle[
  headfont=\normalfont\bfseries,
  numberwithin=section,
  bodyfont=\normalfont,
  spaceabove=1em plus 0.75em minus 0.25em,
  spacebelow=0em plus 0.75em minus 0.25em,
  %  qed={$\blacksquare$},
  %  qed={$\natural$},%\sharp},%$\square$},
  headpunct={\newline},%Punctuation after theorem head
]{thrmstyle}
\declaretheorem[
  style=thrmstyle,
  title=Замечание,
  refname={замечание,замечания},
  Refname={Замечание,Замечания}
]{remark}

\theoremstyle{thrmstyle}

%\newtheorem{question}{Вопрос}
%\newtheorem{approval}{Утверждение}
%\newtheorem{demand}{Требование}
%\newtheorem{statement}{Утверждение}
%\newtheorem{syntax}{Синтаксис}
%\newtheorem{definition}{Определение}
%\newtheorem{designation}{Обозначение}
%\newtheorem{property}{Свойство}
%\newtheorem{task}{Задача}
%\newtheorem{mathmod}{Математическая модель}
%\newtheorem{mytheorem}{Теорема}

%----------------------------------------------------------
\hyphenation{супер-ком-пью-тер-ный супер-ком-пью-тер-ное вы-со-ко-произ-во-ди-тель-ный}
%----------------------------------------------------------
\newcommand\MyProcess[3]
{
  %\begin{tabular}{p{0.1\textwidth}p{0.9\textwidth}}
  %&
  \begin{lstlisting}[caption={#1}, label={#2}, language=ALGO, basicstyle=\small]
#3
\end{lstlisting}
  %\end{tabular}
}
%----------------------------------------------------------
\newcommand\myquote[2]{%
  \textit{``#1''}
  \begin{flushright}
    \textcolor{gray}{#2}
  \end{flushright}
}
%----------------------------------------------------------
\newcommand{\mydescr}[1]{%

  \colorbox{black!5}{\parbox{0.7\textwidth}{\smaller[1]\textit{#1}}}
}
%----------------------------------------------------------
% вёрстка в две колонки
\newcommand\cols[4]
{\smaller[1]%
  \begin{columns}
    \begin{column}{#1}
      #2
    \end{column}
    \begin{column}{#3}
      #4
    \end{column}
  \end{columns}}
%----------------------------------------------------------
% вёрстка в две колонки двух блоков с itemize
\newcommand\colsblocks[4]
{\smaller[1]%
  \begin{columns}
    \begin{column}{0.5\textwidth}
      \begin{block}{#1}
        \begin{itemize}
          #2
        \end{itemize}
      \end{block}
    \end{column}
    \begin{column}{0.5\textwidth}
      \begin{block}{#3}
        \begin{itemize}
          #4
        \end{itemize}
      \end{block}
    \end{column}
  \end{columns}}
%----------------------------------------------------------
%\usepackage{ifthen}
%%----------------------------------------------------------
%\usepackage{totcount}
%%----------------------------------------------------------
%%\newboolean{shortversion}
%% #1 - current \doctype
%% #2 - destination document
%% #3 - text
%\newcommand{\myconditionaltext}[3]%
%{%
%\ifthenelse{\equal{#1}{avtoreferat}\AND\equal{#2}{avtoreferat}}{#3}{}% short version
%\ifthenelse{\equal{#1}{avtoreferat}\AND\equal{#2}{thesis}}{}{}% short version
%\ifthenelse{\equal{#1}{avtoreferat}\AND\equal{#2}{presentation}}{}{}% short version
%\ifthenelse{\equal{#1}{thesis}\AND\equal{#2}{presentation}}{}{}% short version
%\ifthenelse{\equal{#1}{thesis}\AND\equal{#2}{thesis}}{#3}{}% short version
%\ifthenelse{\equal{#1}{presentation}\AND\equal{#2}{presentation}}{#3}{}% short version
%\ifthenelse{\equal{#1}{presentation}\AND\equal{#2}{thesis}}{\pdfcomment{#3}}{}% short version
%}

%----------------------------------------------------------
% #1 - Короткое название практики
% #2 - Короткое описание
% #3 - Назначение в форме itemize
% #4 - LaTeX код
\newcommand{\bestpractice}[4]{%
  \subsection{#1}
  \begin{frame}%[fragile]

    {\smaller[1]
      #2}

    \begin{columns}[t]

      \begin{column}{0.5\textwidth}
        \begin{block}{Назначение}
          {\smaller[1]%
            #3}
        \end{block}
      \end{column}

      \begin{column}{0.5\textwidth}
        \begin{lstlisting}[language={TeX}, basicstyle=\scriptsize]
#4
\end{lstlisting}
      \end{column}
    \end{columns}

  \end{frame}}
%----------------------------------------------------------
\newenvironment{arreqn}[2]
{\smaller[1]
  \begin{block}{\smaller[1] #1}
    \begin{equation}\label{#2}
      \begin{array}{l}
        }
        {
      \end{array}
    \end{equation}
  \end{block}
}
%----------------------------------------------------------




