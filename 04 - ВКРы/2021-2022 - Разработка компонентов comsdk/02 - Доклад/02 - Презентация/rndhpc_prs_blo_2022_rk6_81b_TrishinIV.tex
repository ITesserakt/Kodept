\documentclass[9pt]{beamer}
%----------------------------------------------------------
% Стиль презентации (выбрать понравившийся)
\usetheme{bmstu}
%------------------------------------------
% определения значений стандартных параметров
%----------------------------------------------------------
\newcommand{\doctype}{presentation} % о научно-исследовательской работе / об опытно-конструкторской работе / об опытно-технологической работе / о патентных исследованиях
\newcommand{\doctypesid}{prs} % prs (Презентация) / vkr (выпускная квалификационная работа) / kp (курсовой проект) / kr (курсовая работа) / nirs (научно-исследовательская работа студента) / nkr (научно-квалификационная работа)
\def\titlepagestyle{detail} % brief (краткая) / detail (подробная)
%----------------------------------------%
\newcommand{\Title}{Разработка механизма вывода типов с использованием системы типов\\Хиндли-Милнера}%Научные основы автоматизированного проектирования композиционных материалов
\newcommand{\SubTitle}{Системы типов в языках программирования} % Методы оптимизации
%----------------------------------------------------------
\newcommand{\eventplace}{факультет \facultyShort, кафедра \department} % Место проведения мероприятия
\newcommand{\eventtype}{название события} % Тип мероприятия (лекция|лабораторная работа|семинар|мозговой штурм)
%\newcommand{\eventplace}{@место проведения@} % Место проведения мероприятия
\newcommand{\eventduration}{5 минут} % Продолжительность мероприятия
%----------------------------------------------------------
\newcommand{\Author}{Никитин В.Л.}
\newcommand{\AuthorFull}{Никитин Владимир Леонидович, студент группы РК6-85Б}
\newcommand{\AuthorEmail}{nikitinvl@student.bmstu.ru}
\newcommand{\ScientificAdviserPosition}{кандидат физико-математических наук}	% Должность научного руководителя
\newcommand{\ScientificAdviser}{Соколов А.П.}	% Научный руководитель
\newcommand{\group}{РК6-85Б}
\newcommand{\Semestr}{весенний семестр} % Например: осенний семестр или весенний семестр
\newcommand{\begdate}{08 февраля 2024} % Дата начала разработки
\newcommand{\Year}{2024}
\newcommand{\country}{Россия}	% Страна, в которой проводится конференция
\newcommand{\city}{Москва}		% Город, в котором проводится конференция
%----------------------------------------%
\newcommand{\depHeadPosition}{Заведующий кафедрой}		% Должность руководителя подразделения
\newcommand{\depHeadName}{А.П.~Карпенко}		% Должность руководителя подразделения
%----------------------------------------%
\newcommand{\presentationtitle}{\Title}
\newcommand{\conferenceperiod}{\country, \city, \begdate\ -- \today}
%----------------------------------------%

% Цель исследования/доклада
\newcommand{\GoalOfResearch}{реализация системы вывода и проверки типов} % с маленькой буквы и без точки на конце

% Объектом исследования называют то, что исследуется в работе. Например, для указанной выше темы объектом может быть популяция медуз, но никак ни модель SIS, ни Южно-Китайское море, ни метод моделирования популяции медуз.
\newcommand{\ObjectOfResearch}{система типов}

% Предмет исследований (уже чем объект, определяется, исходя из задач: формулируется как существительное, как правило, во множественном числе, определяющее "конкретный объект исследований" среди прочих в рамках более общего)
\newcommand{\SubjectOfResearch}{система типов Хиндли-Милнера}

% Основная задача, на решение которой направлена работа
\newcommand{\MainProblemOfResearch}{реализация алгоритма вывода типов на основе выбранной системы типов}

% Выполненные задачи
\newcommand{\SubtasksPerformed}{%
	В результате выполнения работы:
	\begin{inparaenum}[1)]
	\item спроектировано представление абстрактного синтаксического дерева в компиляторе;
	\item реализован семантический анализатор;
	\item показано, что компилятор успешно может вывести тип функции
	\end{inparaenum}}

% Ключевые слова (представляются для обеспечения потенциальной возможности индексации документа)
\newcommand{\keywordsru}{%
	теория типов, языки программирования, компиляторы, фукнциональное программирование, система типов Хиндли-Милнера} % 5-15 слов или выражений на русском языке, для разделения СЛЕДУЕТ ИСПОЛЬЗОВАТЬ ЗАПЯТЫЕ
\newcommand{\keywordsen}{%
	type theory, programming languages, compilers, functional programming, Hindley-Milner type system} % 5-15 слов или выражений на английском языке, для разделения СЛЕДУЕТ ИСПОЛЬЗОВАТЬ ЗАПЯТЫЕ

% Краткая аннотация
\newcommand{\Preface}{
	Работа посвящена реализации механизма вывода типов для языка программирования Kodept.
	Программирование выстроено вокруг глубокой математической теории.
	Благодаря этому появляются возможности для оптимизации, развития и улучшения языков посредством применения математики.
	Одним из важных применений является теория типов, которая помогает программисту в написании кода.
	В последнее время все больше и больше языков почерпывают что-то из этой области.
	Применение мощной системы типов позволяет зачастую снизить количество ошибок, возникающих при разработке.
} % с большой буквы с точкой в конце

%----------------------------------------%
% выходные данные по документу
\newcommand{\DocOutReference}{\Author. \Title\xspace\SubTitle. [Электронный ресурс] --- \City: \Year. --- \total{page} с. URL:~\url{https://\gitlabdomain} (система контроля версий кафедры РК6)}

%----------------------------------------------------------

%------------------------------------------
% определения значений стандартных общих параметров для различных документов, в т.ч. различных типов
%----------------------------------------------------------
% Информация о лицензии
\newcommand{\doclicense}{\copyright\xspace}
%\newcommand{\doclicense}{\includegraphics[width=0.1\textwidth]{../common_doc_spec/by.eps}\xspace}%\ccShareAlike
%----------------------------------------------------------
\newcommand{\doctextlicense}{\copyright\xspace} % Copyright ©
%\newcommand{\doctextlicense}{CC BY 4.0}% \ccAttribution
%----------------------------------------%
% общие определения
\newcommand{\UpperFullOrganisationName}{Министерство науки и высшего образования Российской Федерации}
\newcommand{\ShortOrganisationName}{МГТУ~им.~Н.Э.~Баумана}
\newcommand{\FullOrganisationName}{Московский Государственный Технический Университет имени Н.Э.~Баумана\newline (национальный исследовательский университет)>> (\ShortOrganisationName)}
\newcommand{\OrganisationAddress}{105005, Россия, Москва, ул.~2-ая Бауманская, д.~5, стр.~1}
%----------------------------------------%
\newcommand{\gitlabdomain}{gitlab.sa2systems.ru}
%----------------------------------------%
\newcommand{\faculty}{<<Робототехники и комплексной автоматизации>>}
\newcommand{\facultyShort}{РК}
\newcommand{\department}{<<Системы автоматизированного проектирования (РК-6)>>}
\newcommand{\departmentShort}{РК-6}
\newcommand{\departmentcloud}{\href{https://archrk6.bmstu.ru}{https://archrk6.bmstu.ru}}
%----------------------------------------------------------


%------------------------------------------
% общая кастомизируемая преамбула для презентаций
%----------------------------------------------------------
\usepackage[T2A]{fontenc}
\usepackage[utf8]{inputenc}
\usepackage[russian]{babel} %% это необходимо для включения переносов
\usepackage{bm}
\usepackage{float}
%\usepackage{algpseudocode}      % для окружения algorithmic
%\usepackage{algorithm}          % для нумерации алгоритмов
\usepackage{cmap} % необходимо для возможности копирования и поиска в готовом PDF
%-------------------------
% Сохранение метаданных в PDF об авторе документа
\hypersetup{%
    pdftitle={\Title},    	% title
    pdfauthor={\Author},    % author
		pdfcopyright={\doctextlicense \begdate -- \today, \Author. Все права защищены.},
    pdfsubject={\Title},   % subject of the document
    pdfcreator={\pdftexbanner},   % creator of the document
		pdfpublisher={\department, \ShortOrganisationName},
		pdfcaptionwriter={\Author},
    pdfproducer={\Author, \begdate -- \today, \ShortOrganisationName}, % producer of the document
    pdfkeywords={\keywordsru, \keywordsen}, % producer of the document
}
%----------------------------------------------------------
% final - удаляет все всплывающие комментарии
\usepackage[author={\Author},opacity=0.1]{pdfcomment}
%----------------------------------------------------------
\usepackage{tikz}
\usetikzlibrary{tikzmark}
\usetikzlibrary{matrix,automata,positioning,arrows,arrows.meta,graphs}
\usetikzlibrary{trees,topaths}
\usetikzlibrary{calc, circuits.ee.IEC}
\usetikzlibrary{patterns,decorations.pathmorphing,decorations.markings}
%----------------------------------------------------------
\renewcommand{\lstlistingname}{Исходный код}

%------------------------------------------
% дополнительные параметры, имеющие отношение к данному документу
%----------------------------------------------------------
\newcommand\ddfrac[2]{\frac{\displaystyle #1}{\displaystyle #2}}
\newcommand\T{\textup{T}}
\def\argmin{\operatornamewithlimits{argmin}}
\def\argmax{\operatornamewithlimits{argmax}}
\def\diag{\operatornamewithlimits{diag}}
\def\dep{\operatornamewithlimits{dep}}
%----------------------------------------------------------
\newcommand{\bvec}[1]{\mbox{\mathversion{bold}${#1}$}}
\newcommand{\periodichnost}[1]{[\hspace{-2pt}[#1]\hspace{-2pt}]}
\newcommand{\volna}[1]{\mathop{#1}\limits\_{\sim}}
\newcommand{\crujok}[1]{#1}
\newcommand{\driv}[2]{\frac{d#1}{d#2}}
\newcommand{\pdriv}[2]{\frac{\partial #1}{\partial #2}}
\newcommand{\uparr}[1]{\overrightarrow{#1}}
\newcommand{\intThreeD}[1]{\hspace{6pt}\mathop{\int\hspace{-17pt}\int\hspace{-17pt}\int}\limits\_{#1}\hspace{6pt}}
\newcommand{\intTwoD}[1]{\hspace{6pt}\mathop{\int\hspace{-17pt}\int}\limits\_{\hspace{-6pt}#1}}
%----------------------------------------------------------
\newcommand{\multirowbox}[2]
{
   \parbox{#1}{\smallskip #2 \smallskip}
}
%----------------------------------------------------------
\usepackage{amsthm}
\usepackage{thmtools}

\declaretheoremstyle[
  headfont=\normalfont\bfseries,
	numberwithin=section,
  bodyfont=\normalfont,
  spaceabove=1em plus 0.75em minus 0.25em,
  spacebelow=0em plus 0.75em minus 0.25em,
%  qed={$\blacksquare$},
%  qed={$\natural$},%\sharp},%$\square$},
	headpunct={\newline},%Punctuation after theorem head
]{thrmstyle}
\declaretheorem[
  style=thrmstyle,
  title=Замечание,
  refname={замечание,замечания},
  Refname={Замечание,Замечания}
]{remark}

%\declaretheorem[
  %style=thrmstyle,
  %title=Определение,
  %refname={определение,определения},
  %Refname={Определение,Определения}
%]{definition}

\theoremstyle{thrmstyle}

%\newtheorem{question}{Вопрос}
%\newtheorem{approval}{Утверждение}
%\newtheorem{demand}{Требование}
\newtheorem{statement}{Утверждение}
%\newtheorem{syntax}{Синтаксис}
%renewtheorem{definition}{Определение}
\newtheorem{designation}{Обозначение}
%\newtheorem{property}{Свойство}
%\newtheorem{task}{Задача}
%\newtheorem{mathmod}{Математическая модель}
%\newtheorem{mytheorem}{Теорема}

%----------------------------------------------------------
\hyphenation{супер-ком-пью-тер-ный супер-ком-пью-тер-ное вы-со-ко-произ-во-ди-тель-ный}
%----------------------------------------------------------
\newcommand\MyProcess[3]
{
%\begin{tabular}{p{0.1\textwidth}p{0.9\textwidth}}
%&
\begin{lstlisting}[caption={#1}, label={#2}, language=ALGO, basicstyle=\small]
#3
\end{lstlisting}			
%\end{tabular}
}
%----------------------------------------------------------
\newcommand\myquote[2]{%
\textit{``#1''}
\begin{flushright}
\textcolor{gray}{#2}
\end{flushright}
}
%----------------------------------------------------------
\newcommand{\mydescr}[1]{%

\colorbox{black!5}{\parbox{0.7\textwidth}{\smaller[1]\textit{#1}}}
}
%----------------------------------------------------------
% вёрстка в две колонки
\newcommand\cols[4]
{\smaller[1]%
		\begin{columns}
		\begin{column}{#1}
    #2
		\end{column}
		\begin{column}{#3}
		#4
		\end{column}
		\end{columns}}
%----------------------------------------------------------
% вёрстка в две колонки двух блоков с itemize
\newcommand\colsblocks[4]
{\smaller[1]%
		\begin{columns}
		\begin{column}{0.5\textwidth}
		\begin{block}{#1}
		\begin{itemize}
			#2
		\end{itemize}
		\end{block}
		\end{column}
		\begin{column}{0.5\textwidth}
		\begin{block}{#3}
		\begin{itemize}
			#4
		\end{itemize}
		\end{block}
		\end{column}
		\end{columns}}
%----------------------------------------------------------
%\usepackage{ifthen}
%%----------------------------------------------------------
%\usepackage{totcount}
%%----------------------------------------------------------
%%\newboolean{shortversion}
%% #1 - current \doctype
%% #2 - destination document
%% #3 - text
%\newcommand{\myconditionaltext}[3]%
%{%
	%\ifthenelse{\equal{#1}{avtoreferat}\AND\equal{#2}{avtoreferat}}{#3}{}% short version
	%\ifthenelse{\equal{#1}{avtoreferat}\AND\equal{#2}{thesis}}{}{}% short version
	%\ifthenelse{\equal{#1}{avtoreferat}\AND\equal{#2}{presentation}}{}{}% short version
	%\ifthenelse{\equal{#1}{thesis}\AND\equal{#2}{presentation}}{}{}% short version
	%\ifthenelse{\equal{#1}{thesis}\AND\equal{#2}{thesis}}{#3}{}% short version
	%\ifthenelse{\equal{#1}{presentation}\AND\equal{#2}{presentation}}{#3}{}% short version
	%\ifthenelse{\equal{#1}{presentation}\AND\equal{#2}{thesis}}{\pdfcomment{#3}}{}% short version
%}

%----------------------------------------------------------
% #1 - Короткое название практики
% #2 - Короткое описание
% #3 - Назначение в форме itemize
% #4 - LaTeX код
\newcommand{\bestpractice}[4]{%
\subsection{#1}
\begin{frame}%[fragile]

{\smaller[1]
#2}

\begin{columns}[t]

\begin{column}{0.5\textwidth}
\begin{block}{Назначение}
{\smaller[1]%
#3}
\end{block}
\end{column}

\begin{column}{0.5\textwidth}
\begin{lstlisting}[language={TeX}, basicstyle=\scriptsize]
#4
\end{lstlisting}
\end{column}
\end{columns}

\end{frame}}
%----------------------------------------------------------
\newenvironment{arreqn}[2]
    {\smaller[1]
		\begin{block}{\smaller[1] #1}
    \begin{equation}\label{#2}
    \begin{array}{l}
    }
		{
    \end{array}
    \end{equation}
		\end{block}
    }
%----------------------------------------------------------





%------------------------------------------
% база терминов! 
%----------------------------------------------------------
%Термины и определения по тексту в большинстве случаев выделяются курсивом.
%В настоящем отчете о НИР применяют следующие термины с соответствующими определениями, также используются представленные обозначения и сокращения.


%\newabbreviation[category=inline]{html}{HTML}{hypertext markup language}
%\newabbreviation[category=footer]{shtml}{SHTML}{server-parsed HTML}

%\newglossaryentry{sample2}{name={sample2},
%	symbol={\ensuremath{\mathcal{S}_2}},
%	category=symbol,
%	description={the second sample entry}}

%\newabbreviation
%[prefix={an\space},
%prefixfirst={a~}]
%{svm}{SVM}{support vector machine}

%\newabbreviation
%[category=initialism,description={for example}]
%{eg}{eg}{exempli gratia}
% define the entries:

%\newabbreviation{html}{html}{hypertext markup language}

%\newabbreviation[category=initialism]{eg}{eg}{for example}
%\newabbreviation[category=initialism]{si}{SI}{sample initials}
%\newabbreviation{xml}{XML}{extensible markup language}
%\newabbreviation{css}{CSS}{cascading style sheet}
%\newacronym[description={a device that emits a narrow intense 
%	beam of light}]{laser}{laser}{light amplification by stimulated 
%	emission of radiation}

%\newacronym[description={a form of \gls{laser} generating a beam of
%	microwaves}]{maser}{maser}{microwave amplification by stimulated 
%	emission of radiation}

%\newacronym[description={a system for detecting the location and
%	speed of ships, aircraft, etc, through the use of radio waves}]{radar}{radar}{radio detection and ranging}

%\newacronym[description={portable breathing apparatus for divers}]{scuba}{scuba}{self-contained underwater breathing apparatus}

%%%%% для обычных newglossaryentry по умолчанию category==general.
%%%%% для обычных newabbreviation по умолчанию category==abbreviation.
%%%%% команда для создания своей категории \glscategory{<label>}
%----------------------------------------------------------
\newglossaryentry{slver}{name={Solver}, description={Решатель системы \gls{dcs-gcd}. Регистрируется в таблице \textbf{com.slvrs} БД \gls{gcddb} \gls{dcs-gcd}.}}
\newabbreviation[category=initialism]{ПО}{ПО}{-- программное обеспечение}
\newabbreviation[category=initialism]{API}{API}{-- прикладной программный интерфейс (Application Programming Interface)}
\newabbreviation[category=initialism]{DFD}{DFD}{-- диаграмма потоков данных (Data Flow Diagram)}
\newabbreviation[category=initialism]{GBSE}{GBSE}{-- графоориентированный подход к разработке программного обеспечения (graph based software engineering)}
\newabbreviation[category=initialism]{LCPD}{LCPD}{-- платформы малокодовой разработки (low-code development platforms)}
\newabbreviation[category=initialism]{CASE}{CASE}{-- втоматизированная разработка программного обеспечения (Computer aided software engineering)}
\newabbreviation[category=initialism]{DOT}{DOT}{-- язык описания графов}
\newabbreviation[category=initialism]{JSON}{JSON}{-- файловый формат для хранения структур данных (Javascrtipt Object Notation)}
\newabbreviation[category=initialism]{TO}{ТО}{технический объект, в т.ч. сложный процесс, система}
\newabbreviation[category=initialism]{aINI}{aINI}{-- расширенный формат INI (\href{https://archrk6.bmstu.ru/index.php/f/846701}{описание представлено в \cite{SokAINI}})}
\newabbreviation[category=initialism]{aDOT}{aDOT}{-- расширенный формат DOT (\href{https://archrk6.bmstu.ru/index.php/f/777612}{описание представлено в \cite{SokolovADOT2020}})}


\GlsXtrEnableEntryCounting
{abbreviation}% list of categories to use entry counting
{2}% trigger value

\GlsXtrEnableEntryCounting
{symbol}% list of categories to use entry counting
{2}% trigger value


%----------------------------------------------------------



%---------------------------------------------
\includeonly{%
chapters/00_intro,
chapters/01_review,
chapters/02_taskstatement,
chapters/03_software,
chapters/XX_conclusion
}
%---------------------------------------------
% выключает разворачивание терминов и аббревиатур при первом использовании в том числе, - всегда термины и аббревиатуры будут выводиться кратко 
\glsunsetall
%----------------------------------------------------------
% Это следует использовать и визуализировать с помощью Dual-side PDF Viewer - dspdfviewer-1.15.1.2)
%\setbeameroption{show notes on second screen=right} % Both
%----------------------------------------------------------
%\setbeameroption{hide notes} % Only slides
%\setbeameroption{show only notes} % Only notes
%\setbeameroption{show notes} % Both
%\setbeamertemplate{note page}{\pagecolor{yellow!5}\insertnote}
%---------------------------------------------
\begin{document}
%---------------------------------------------
% создание стандартной титульной страницы
%----------------------------------------------------------
% Определение значений параметров для формирования титульной страницы
\title{\presentationtitle} 
\subtitle{\small
\SubTitle \hfill\break
\raggedright\noindent  \textit{\eventplace} \hfill\break
\raggedright\noindent  \textit{\eventduration}
} 
\date[\city\enskip\year]{\scriptsize\conferenceperiod} 
\author[\Author]{%
\raggedright \AuthorFull \\
\href{mailto:\AuthorEmail}{\AuthorEmail}
}
\institute{\ShortOrganisationName}

%----------------------------------------------------------
% Создание заглавной страницы
\frame{%
\nobibliography{bibliography}
%\nocite{*}
\titlepage} 
\setcounter{framenumber}{0}
%----------------------------------------------------------

%---------------------------------------------
\section{Введение}
%%%%%%%%%%%%%%%%%%%%%%%%%%%%%%%%%%%%%%%%%%%
\subsection{Пример сложного вычислительного метода}
%%%%%%%%%%%%%%%%%%%%%%%%%%%%%%%%%%%%%%%%%%%
\begin{frame}
  \begin{figure}
    \centering
    \includegraphics[width=\textwidth]{images/flowchart.fem.png}
    \caption{Применение метода конечных элементов для анализа прочности конструкции}
  \end{figure}

\end{frame}
% --------------------------------------------------------------------------- %
\subsection{Разработка трудоёмкого научно-технического ПО}
% --------------------------------------------------------------------------- %

\begin{frame}
  \begin{figure}
    \centering
    \includegraphics[width=\textwidth]{images/illustration.teamwork.png}
    \caption{Пример делегирования разработки отдельных этапов вычислительного метода}
  \end{figure}

  {\smaller[1]
  Делегирование реализации отдельных этапов реализуемого метода отдельным разработчикам положительно влияет на общее качество реализуемого ПО.
  }
\end{frame}

% --------------------------------------------------------------------------- %
\subsection{Современные средства, направленные на упрощение разработки}
% --------------------------------------------------------------------------- %
\begin{frame}



\end{frame}

% --------------------------------------------------------------------------- %
\subsection{Цели и задачи разработки}
% --------------------------------------------------------------------------- %
\begin{frame}

  \begin{block}{Цель}
    Разработать программные средства для создания и интерпретации графовых описаний вычислительных методов в программном каркасе comsdk.
  \end{block}

  \begin{block}{Задачи}
    \begin{enumerate}
      \item Провести сравнение объекта разработки с некоторым аналогичным.
      \item Сформировать требования к алгоритму, выполняющему этапы алгоритма по его описанию, составленному по методологии GBSE.
      \item Спроектировать структуры данных для описания и представления описаний алгоритмов и их элементов в программном каркасе comsdk.
      \item Разработать алгоритм обхода графовых моделей с использованием спроектированных структур данных.
      \item Представить интерфейсы или реализации разработанных алгоритмов и структур данных на языке С++.
    \end{enumerate}
  \end{block}

\end{frame}



%---------------------------------------------
\section{Аналитический обзор}
% --------------------------------------------------------------------------- %
\subsection{Диаграммы потоков данных}
% --------------------------------------------------------------------------- %
\begin{frame}
    \begin{itemize}
        \item Описание алгоритма представляет собой ориентированный граф.
        \item С его вершинами связываются процессы обработки данных.
        \item Его рёбра определяют пути данных между процессами.
    \end{itemize}

    \begin{figure}
        \centering
        \includegraphics[width=0.9\textwidth]{images/illustration.dataflow.pdf}
        \caption{Пример диаграммы потоков данных, описывающей вычисление среднего арифметического и среднего геометрического двух половин массива целых чисел с последующей записью результатов в файл}
    \end{figure}

\end{frame}


\subsection{pSeven -- реализация диаграмм потоков данных}
%%%%%%%%%%%%%%%%%%%%%%%%%%%%%%%%%%%%%%%%%%%
\begin{frame}
    \begin{figure}
        \centering
        \includegraphics[width=\textwidth]{images/workflow1.png}
        \caption{Графический пользовательский интерфейс pSeven}
    \end{figure}

\end{frame}

% --------------------------------------------------------------------------- %
\subsection{Результаты сравнения. Выявленные достоинства объекта разработки}
% --------------------------------------------------------------------------- %
\begin{frame}
    Основные достоинства comsdk по сравнению с pSeven:
    \begin{itemize}
        \item Нет необходимости указывать входные и выходные данные при описании алгоритма.
        \item По умолчанию поддерживаются алгоритмы, подразумевающие взаимодействие с пользователем
    \end{itemize}

    \begin{figure}
        \centering
        \includegraphics[height=0.33\textheight]{images/illustration.form_generation.png}
        \caption{Пример получения данных от пользователя при помощи генерируемой формы ввода}
    \end{figure}

    \begin{itemize}
        \item Результат применения -- компилируемая программа с возможностью запуска на различных платформах.
    \end{itemize}
\end{frame}

% --------------------------------------------------------------------------- %
\subsection{Результаты сравнения. Выявленные недостатки объекта разработки}
% --------------------------------------------------------------------------- %
\begin{frame}
    Основные недостатки comsdk по сравнению с pSeven:
    \begin{itemize}
        \item Нет возможности использовать матрицы.
        \item Нет возможности визуализировать результаты работы алгоритма.
        \item Нет возможности запускать выполнение алгоритма на вычислительных кластерах;
    \end{itemize}
\end{frame}

% --------------------------------------------------------------------------- %
\subsection{Цели и задачи разработки}
% --------------------------------------------------------------------------- %
\begin{frame}

    \begin{block}{Цель}
        Разработать обновлённые программные средства для создания и интерпретации графовых описаний вычислительных методов в программном каркасе comsdk.
    \end{block}

    \begin{block}{Задачи}
        \begin{enumerate}
            \item Сформировать требования к алгоритму, выполняющему этапы алгоритма по его описанию, составленному по методологии GBSE.
            \item Спроектировать структуры данных для описания и представления описаний алгоритмов и их элементов в программном каркасе comsdk.
            \item Разработать алгоритм обхода графовых моделей с использованием спроектированных структур данных.
            \item Представить интерфейсы или реализации разработанных алгоритмов и структур данных на языке С++.
        \end{enumerate}
    \end{block}

\end{frame}
%---------------------------------------------
\section{Постановка задачи}
% --------------------------------------------------------------------------- %
\subsection{Состояния данных}
% --------------------------------------------------------------------------- %
\begin{frame}
	\begin{figure}
		\centering
		\includegraphics[width=\textwidth]{images/illustration.datastate.png}
	\end{figure}
	Состояние данных определяет, какие переменные какого типа должны быть определены на данном этапе алгоритма.

	Данные алгоритма модифицируются по ходу его выполнения.
\end{frame}

% --------------------------------------------------------------------------- %
\subsection{Функции-обработчики}
% --------------------------------------------------------------------------- %
\begin{frame}
	\begin{figure}
		\centering
		\includegraphics[width=0.7\textwidth]{images/illustration.transfer.png}
	\end{figure}

	Функции-обработчики отвечают за обработку данных и их перевод из одного состояния в другое.
\end{frame}


% --------------------------------------------------------------------------- %
\subsection{Функции-предикаты и функции перехода}
% --------------------------------------------------------------------------- %
\begin{frame}
	\begin{figure}
		\begin{minipage}{0.49\textwidth}
			\centering
			\includegraphics[height=0.5\textheight]{images/illustration.predicate.png}
			\caption{Принцип работы функции-предиката}
		\end{minipage}\hfill\begin{minipage}{0.49\textwidth}
			\centering
			\includegraphics[height=0.5\textheight]{images/flowchart.Transfer.png}
			\caption{Блок-схема логики функции перехода}
		\end{minipage}\hfill
	\end{figure}

	Функции-предикаты отвечают за предварительную проверку данных перед их обработкой.

	Функция перехода -- составная функция $F=<p,f>$, содержащая в себе функцию-предикат $p$ и функцию-обработчик $f$.
\end{frame}

% --------------------------------------------------------------------------- %
\subsection{Функции-селекторы}
% --------------------------------------------------------------------------- %

\begin{frame}
	Функции-селекторы отвечают за проверку условий при ветвлении алгоритма.

	\begin{figure}
		\centering
		\includegraphics[width=0.9\textwidth]{images/illustration.selector.png}
		\caption{Принцип работы функций-селекторов. $h$ -- функция селектор. Красным показана ветвь алгоритма, которая будет выполнена.}
	\end{figure}
\end{frame}

% --------------------------------------------------------------------------- %
\subsection{Графовая модель}
% --------------------------------------------------------------------------- %
\begin{frame}
	Графовая модель сложного вычислительного метода описывает его логику в виде ориентированного графа, где узлам ставятся в соответствие состояния данных, а рёбрам -- функции перехода.

	\begin{figure}
		\centering
		\includegraphics[width=\textwidth]{images/illustration.graph.png}
		\caption{Пример графовой модели, описывающей вычисление среднего арифметического и среднего геометрического двух половин массива целых чисел с последующей записью результатов в файл}
	\end{figure}

\end{frame}

\begin{frame}
	\begin{figure}
		\begin{minipage}{0.49\textwidth}
			\centering
			\includegraphics[width=\textwidth]{images/illustration.node.png}
		\end{minipage}\hfill\begin{minipage}{0.49\textwidth}
			Атрибуты вершины:
			\begin{enumerate}
				\item имя;
				\item состояние данных;
				\item селектор;
				\item режим параллельного обхода исходяших ветвей
			\end{enumerate}
		\end{minipage}\hfill
	\end{figure}
\end{frame}

%---------------------------------------------
\section{Программная реализация}
% --------------------------------------------------------------------------- %
\subsection{Функциональные структуры данных}
% --------------------------------------------------------------------------- %
\begin{frame}
    \smaller[1]
    \begin{itemize}
        \item Были разработаны классы, представляющие функции-предикаты (Predicate), обработчики (Processor), селекторы (Selector) и функции перехода (Transfer).
        \item Для хранения данных был использован универсальный ассоциативный массив Anymap.
    \end{itemize}

    \begin{figure}
        \smaller[1]
        \centering
        \includegraphics[height=0.64\textheight]{images/UML.graph_functions.pdf}
        \caption{UML-диаграмма разработанных функциональных структур данных}
    \end{figure}

\end{frame}
% --------------------------------------------------------------------------- %
\subsection{Информационные структуры данных}
% --------------------------------------------------------------------------- %
\begin{frame}

    \begin{itemize}
        \item Вершины графовой модели реализованы в классе Node.
        \item Рёбра графовой модели реализованы в классе Edge.
    \end{itemize}

    \begin{figure}
        \centering
        \includegraphics[height=0.6\textheight]{images/UML.graphElements.png}
        \caption{Разработанные структуры данных, представляющие элементы графовой модели}
    \end{figure}
\end{frame}

\begin{frame}

    \smaller[1]

    \begin{itemize}
        \item За программное представление графовых моделей отвечает класс Graph.
        \item Для объединения возможностей взаимодействия с вершинами и рёбрами и обращения к топологии графа спроектированы вспомогательные структуры данных NodeOp (операция с вершиной) и EdgeOp (операция с ребром).
    \end{itemize}

    \begin{figure}
        \begin{minipage}{0.38\textwidth}
            \centering
            \includegraphics[width=\textwidth]{images/class.graph.png}
        \end{minipage}\hfill\begin{minipage}{0.62\textwidth}
            \centering
            \includegraphics[width=\textwidth]{images/UML.topologyOperations.png}
        \end{minipage}
    \end{figure}


\end{frame}
% --------------------------------------------------------------------------- %
\subsection{Управляющие структуры данных}
% --------------------------------------------------------------------------- %
\begin{frame}
    \begin{itemize}
        \item Интерфейс ExecutionContainer отвечает за отслеживание выполнения параллельных ветвей графовой модели.
        \item Интерфейс ExecutionBranch отвечает за обход одной ветви графа и хранение вспомогательной информации о нём.
        \item ExecutionBranch уведомляет контейнер по завершении каждой функции перехода в обходимой ветви.
    \end{itemize}

    \begin{figure}
        \centering
        \includegraphics[width=0.75\textwidth]{images/UML.all.png}
        \caption{UML-диаграмма управляющих структур данных}
    \end{figure}


\end{frame}
% --------------------------------------------------------------------------- %
\subsection{Алгоритм обхода графовой модели}
\begin{frame}
    Приведённый алгоритм реализован в методе run() класса Graph.

    \begin{figure}
        \centering
        \includegraphics[height=0.7\textheight]{images/flowchart.graphRunning2.png}
        \caption{Общий алгоритм обхода графовой модели}
    \end{figure}
\end{frame}

\begin{frame}
    Приведённые ниже алгоритмы реализованы в методах run() классов ExecutionContainer и ExecutionBranch соответственно.

    \begin{figure}
        \begin{minipage}{0.35\textwidth}
            \centering
            \includegraphics[height=0.6\textheight]{images/flowchart.executionContainer.png}
            \caption{Блок-схема алгоритма следящей структуры данных ``контейнер выполнения''}
        \end{minipage}\hfill\begin{minipage}{0.645\textwidth}
            \centering
            \includegraphics[width=\textwidth]{images/flowchart.executionBranch.png}
            \caption{Блок-схема алгоритма обхода одной ветви графовой модели}
        \end{minipage}
    \end{figure}

\end{frame}

%---------------------------------------------
\section*{Заключение}
%%%%%%%%%%%%%%%%%%%%%%%%%%%%%%%%%%%%%%%%%%%
%\subsection{}
%\def\subsecname{}
%%%%%%%%%%%%%%%%%%%%%%%%%%%%%%%%%%%%%%%%%%%
% --------------------------------------------------------------------------- %
\subsection*{Анализ результатов}
% --------------------------------------------------------------------------- %
\begin{frame}%[allowframebreaks=0.9,t]

	В результате выполнения работы:
	\begin{enumerate}[1)]
		\item расширены функциональные возможности библиотеки comsdk
		\item создана новая архитектура классов, позволяющая в дальнейшем упростить процесс автоматического формирования программного представления графовых моделей;
		\item разработаны структуры данных для программного представления графовых моделей алгоритмов и их элементов;
		\item был разработан алгоритм, осуществляющий выполнение этапов алгоритма в соответствии с его графовой моделью;
	\end{enumerate}

\end{frame}
% --------------------------------------------------------------------------- %
\subsection*{Перспективы разработки}
% --------------------------------------------------------------------------- %
\begin{frame}
	Перспективы развития программного каркаса comsdk включают в себя:
	\begin{itemize}
		\item реализации алгоритма обхода графовой модели с задействованием различных вычислительных ресурсов
		\item интеграцию средства генерации форм ввода\footcite{SokolovPershin2017};
		\item разработку средства визуализации обрабатываемых данных;
		\item разработку средства автоматической документации реализуемых алгоритмов;
	\end{itemize}

\end{frame}

%%%%%%%%%%%%%%%%%%%%%%%%%%%%%%%%%%%%%%%%%%%

%---------------------------------------------
\def\secname{}

\begin{frame}

    \begin{center}
        \LARGE
        Спасибо за внимание!
    \end{center}

\end{frame}
%---------------------------------------------
\end{document}
%---------------------------------------------



