%----------------------------------------------------------
\chapter*{ВВЕДЕНИЕ}\label{chap.introduction}
\addcontentsline{toc}{chapter}{ВВЕДЕНИЕ}
% =========================================================================== %
% ----------------------------- ОСНОВНЫЕ ПУНКТЫ ----------------------------- %
% 1. Описание задач, в которых нужно всякое навороченное математическое ПО
% 2. Примеры наовороченного математического ПО
%   2.1. Почему просто математического ПО не всегда достаточно?
%   2.2. Упомянуть про задачи, у которых одна и та же постановка, но разные 
%         параметры
% 3. Примеры ПО, которое рассчитано на многократное решение задач, автоматизи-
%    рующее их решение (Scientific workflow, hallo?)
%   3.1. Использование графов при описании логики решения в системах научных
%        расчётов
%   3.2. Неудобства в описании данных
%   3.3. Визуальное программирование
% 4. Итог: нужно ПО, где есть какая-то абстракция над обрабатываемыми данными,
%    где они конкретизируются непосредственно в реализациях этапов алгоритма.
% 5. Enter GBSE and comsdk
%    5.1. А чем оно, собсна, так привлекательно?
%    5.2. Сказать про НОВЫХ пользователей (Р А С Ш И Р Я Е М О С Т Ь)
% 6. Сравнение GBSE и DFD
% =========================================================================== %
Современные научно-технические исследования зачастую включают в себя задачи, при решении которых требуется большое количество вычислений, для которых задействуются большие вычислительные мощности. К таким задачам относятся, например, задачи анализа, определения характеристик материалов или технических объектов, моделирования сложных динамических процессов. Как правило, для решения подобных задач применяется или разрабатывается специализированное программное обеспечение (далее -- \glsxtrshort{ПО}).

Среди прочих применяются программные продукты, предоставляющие пользователю формальный язык описания математических выражений и его интерпретатор, выполняющий необходимые вычисления на машине пользователя. К таким системам относятся, например, Mathcad. Также стоит отметить системы специализирующиеся на символьной алгебре, такие, как Maple\cite{CharMaple1983} и Wolfram Mathematica. В настоящее время данные программные комплексы поддерживают решение задач из различных областей математики, включающих в себя теорию графов, теорию множеств и~т.д, предоставляют инструменты визуализации и анализа результатов. Все они позволяют выполнять математическое моделирование, в том числе, сложных технических объектов. При всех их преимуществах необходимость формулировать математические постановки решаемых задач (т.е.~формировать математические модели, составлять системы уравнений и~т.д.) остаётся за пользователем. Зачастую требуется решать множество задач с схожей постановкой, но с различными входными параметрами. Такая необходимость, например, возникает при решении задач оптимизации, где критерием является некоторая характеристика, получаемая в результате решения задачи анализа. Следовательно, целесообразны автоматизированные средства решения типовых задач анализа и моделирования.

Данные средства относятся к специализированному \glsxtrshort{ПО}, а потому при их разработке требуются глубокие познания в предметной области. Кроме того, важно, чтобы создаваемая кодовая база была рассчитана на дальнейшую поддержку, что предъявляет соответствующие требования к структуре исходного кода и документации. Таким образом целесообразно применение некоторых средств, позволяющих организовать разработку программного обеспечения для решения задач моделирования и анализа и повысить его поддерживаемость.

В наши дни популярность приобретает применение т.н. научных систем управления потоком задач (англ.~scientific workflow systems). Они предоставляют средства организации этапов решения вычислительной задачи и управления вычислительными ресурсами. Процесс работы с подобными системами состоит из 4 основных этапов:
\begin{enumerate}[1)]
  \item составление описания операций обработки данных и зависимостей между ними;
  \item распределение процессов обработки данных по вычислительным ресурсам;
  \item выполнение обработки данных;
  \item сбор и анализ результатов и статистики~\cite{DeelmanWorkflow2009}.
\end{enumerate}

Примерами подобных систем могут служить Pegasus\cite{DeelmanPegasus2016}, Kepler\cite{AltintasKepler2004} и pSeven\cite{NazarenkoDFM2015}. Помимо инструментов загрузки пользовательских реализаций этапов решения задачи они, как правило, представляют библиотеку типовых действий и преобразований, таких, как считывание данных и их сохранение в файлы одного из поддерживаемых форматов, операции со строками, работы с базами данных, и~т.д. На рисунке~\ref{fig:intro.keplerScreenshot} изображён пример описания некоторого процесса в системе Kepler.
\begin{figure}[!ht]
  \centering
  \includegraphics[height=0.35\textheight]{figures/screenshot.KeplerWorkflow.jpg}
  \caption{Описание процесса обработки данных в системе Kepler}
  \label{fig:intro.keplerScreenshot}
\end{figure}

Кроме того, для облегчения процесса разработки трудоёмкого ПО существуют т.н. платформы малокодовой разработки (англ.~low-code development platforms, \glsxtrshort{LCPD})\cite{DiRuscio2022}. В них, подобно системам управления потоком задач, логика разрабатываемого программного продукта описывается при помощи некоторого формального языка или с использованием графического редактора. От системы к системе подход к описаниям варьируется. Может применяться структурный подход, описывающий шаги алгоритма, или предметно-ориентированный, при котором описываются взаимодействующие сущности. Некоторые системы позволяют по созданному описанию генерировать готовые компоненты будущего программного продукта. Так платформа Codebots реализует предметно-ориентированный подход и по составленным UML-диаграммам взаимодействующих сущностей позволяет генерировать \glsxtrshort{API}, \glsxtrshort{JSON}-схемы данных и документацию\cite{DiRuscio2022}. Тем не менее, при реализации сложных вычислительных методов целесообразнее использовать структурный подход.

Одной из ключевых особенностей описанных технологических решений является выделение операций обработки данных в отдельные программные модули (функции, подпрограммы, скрипты). Как правило, при созданий описаний алгоритмов в них используется следующий подход. Поскольку известно, что выходные данные одного программного модуля могут являться входными для одного или нескольких других модулей, можно сказать, что между ними формируются зависимости по входным и выходным данным. Тогда возможно составить такой ориентированный граф, описывающий общую логику алгоритма, в котором узлами являются операции обработки данных, а рёбрами -- пути данных. Такой подход получил название ``диаграммы потоков данных'' (англ.~Dataflow Diagram, \glsxtrshort{DFD}). При известных входных и выходных данных каждого модуля становится возможной их независимая разработка\cite{DanilovPar2011}. Таким образом, уменьшается объём работы по написанию исходных кодов, приходящийся на одного исследователя. Это в свою очередь облегчает отладку и написание документации, что положительно сказывается на общем качестве реализуемого ПО.


\texttt{
  !!! --------------------- WARNING ! MISSING PART ---------------------- !!! \newline
  !!! Здесь нужен какой-то переход к тому, зачем может потребоваться вводить абстракцию над обрабатываемыми данными \newline
  !!! -------------------------------------------------------------------- !!! \newline
}

Таким образом, в некоторых случаях может быть целесообразен такой подход к построению описания логики реализуемого решения, что в нём не указываются конкретные обрабатываемые данные. Последовательность выполнения отдельных этапов в таком случае должна задаваться явно. В предпринимательстве и управлении проектами подобный подход широко распространён и реализован в сетевых графиках. Сетевой график представляет собой ориентированный граф, в котором вершины -- это события или состояния проекта, а рёбра -- это работы. В работе~\cite{SokolovPershin2018} рассматривается применение идеи переходов между состояниями при описании логики вычислительных алгоритмов. Описанный подход получил название graph-based software engineering (\glsxtrshort{GBSE}). Кроме того в указанной работе описана реализация GBSE в библиотеке comsdk для языка C++.

Был проведён сравнительный анализ программного каркаса comsdk с одной из реализаций \glsxtrshort{DFD}. В качестве такой реализации был рассмотрен программный комплекс pSeven, разработанный отечественной компанией DATADVANCE. Он направлен в первую очередь на решение конструкторских, оптимизационных задач и, помимо этого, задач анализа данных, что в первом приближении делает его аналогом comsdk по предметному назначению.

В терминах \textsf{pSeven}: графовое описание процесса решения задачи называется \textit{расчетной схемой} (англ.~workflow); узлам орграфа поставлены в соответствие процессы обработки данных (используется термин \textit{блоки}), а рёбра определяют \textit{связи} между блоками и направления передачи данных между процессами~\cite{NazarenkoDFM2015}. При работе с pSeven используются следующие понятия:
\begin{itemize}
  \item \textsf{расчётная схема} -- формальное описание процесса решения некоторой задачи в виде ориентированного графа;
  \item \textsf{блок} -- программный контейнер для некоторого процесса обработки данных, входные и выходные данные для которого задаются через порты (см.~ниже);
  \item \textsf{порт} -- переменная конкретного\footnote{Динамическая типизация не поддерживается.} типа, определённая в блоке и имеющая уникальное имя в его пределах;
  \item \textsf{связь} -- направленное соединение типа ``один к одному'' между выходным и входным портами разных блоков.
\end{itemize}

С учётом данных понятий можно описать используемую методологию диаграмм потов данных следующим образом. Расчётная схема содержит в себе набор процессов обработки данных (блоков), каждый из которых имеет (возможно, пустой) набор именованных входов и выходов (портов). Данные передаются через связи. Для избежания т.н. гонок данных (англ.~data races) множественные связи с одним и тем же входным портом не поддерживаются. Для начала выполнения каждому блоку требуются данные на всех входных портах. Все данные на выходных портах формируются по завершении исполнения блока~\cite{NazarenkoDFM2015}.

Результаты проведённого сравнения представлены в таблице \ref{rndhpcblo.0209}.

\begin{landscape}
  \begin{longtable}{|p{0.025\textwidth}|p{0.2\textwidth}|p{0.5\textwidth}|p{0.5\textwidth}|}
    \caption{Сравнительная таблица}\label{rndhpcblo.0209}                                                                                                                                                                                                                                                                                                                                                                                                                                                                                                                                                                                                                                                                                                                                                                                                                                                                                                                                                                                                       \\
    \hline
    \textbf{№} & \textbf{Признак}                                                                                                                     & \textbf{pSeven}                                                                                                                                                                                                                                                                                                                                                                                                                                                                                                                                                                       & \textbf{GBSE}                                                                                                                                                                                                                                                                                                   \\
    \hline
    1          & Предметное назначение                                                                                                                & Задачи оптимизации, анализ данных                                                                                                                                                                                                                                                                                                                                                                                                                                                                                                                                                     & Задачи автоматизированного проектирования, алгоритмизация сложных вычислительных методов, анализ данных                                                                                                                                                                                                         \\
    \hline
    2          & Принцип формирования графовых моделей                                                                                                & Узлы -- блоки (процессы), рёбра -- связи (направление передачи данных) \cite{NazarenkoDFM2015}.                                                                                                                                                                                                                                                                                                                                                                                                                                                                                       & Узлы -- состояния данных, рёбра -- переходы между состояниями, с указанием функций перехода \cite{SokolovPershin2018}.                                                                                                                                                                                          \\
    \hline
    3          & Формат описания орграфа                                                                                                              & Расчетная схема (в форме орграфа) сохраняется в двоичный файле закрытого формата с расширением \textsf{.p7wf}.                                                                                                                                                                                                                                                                                                                                                                                                                                                                        & Графовая модель (определяет алгоритм проведения комплексных вычислений в форме орграфа) сохраняется в текстовом файле открытого формата, подготовленного на языке \gls{aDOT}\cite{SokolovADOT2020}, являющегося ``сужением'' (частным случаем) известного формата DOT (Graphviz).                               \\
    \hline
    4          & Файловая структура                                                                                                                   & Проект состоит из непосредственно файла проекта, в котором хранятся ссылки на созданные расчётные схемы и локальную базу данных, сами расчётные схемы, файлы с их входными данными, файлы отчётов, где сохраняются выходные данные последних расчётов и результаты их анализа.                                                                                                                                                                                                                                                                                                        & Проект состоит из \textsf{.aDOT} файла с описанием графа, \textsf{.aINI}-файлов с описанием форматов входных данных, библиотек функций-обработчиков, функций-предикатов и функций-селекторов , файлов, куда записываются выходные данные.                                                                       \\
    \hline
    5          & особенности работы с входными и выходными данными графовых моделей                                                                   & Входные данные должны быть указаны при настройках внешних входных портов расчётной схемы. Данные с выходных портов схемы сохраняются в локальной базе данных. Для их записи в файлы для обработки/анализа вне pSeven необходимо воспользоваться специально предназначенными для этого блоками.                                                                                                                                                                                                                                                                                        & Входные данные хранятся в файле в формате \gls{aINI}\cite{SokAINI}, откуда считываются при запуске обхода графа~\cite{SokolovPershin2017}. Для записи выходных/промежуточных данных в файлы или базы данных необходимо добавить соответствующие функции-обработчики. Формат выходных данных не регламентирован. \\
    \hline
    6          & Особенности передачи параметров между узлами графовых моделей                                                                        & Данные между узлами передаются согласно определйнным связям, которые на уровне выполнения создают пространство в памяти для ввода и вывода данных для выполняемых в раздельных процессах блоков. Транзитная передача данных, которые не изменяются в данном блоке, на выход невозможна.                                                                                                                                                                                                                                                                                               & Поскольку узлами графа являются состояния данных, существует возможность задействовать в расчётах только часть данных, оставляя их другую часть неизменной.                                                                                                                                                     \\
    \hline
    7          & Поддержка ветвлений и циклов                                                                                                         & Присутствует. Достигается засчёт специальных управляющих блоков, которые отслеживают выполнение условий: для ветвления используется блок "Условие" (англ. condition), который перенаправляет данные на один из выходных портов в зависимости от выполнения описанного условия (подробнее см. \cite{pSevenDocsConditons2022}); Для реализации циклов в общем случае используются блоки "Цикл" (англ. loop)\cite{pSevenDocsWorkflow2021}, но для некоторых задач существуют специализированные блоки, организующие логику работы цикла (например, блок "Оптимизатор" (англ. optimizer)) & Присутствует по умолчанию                                                                                                                                                                                                                                                                                       \\
    \hline
    8          & Поддержка параллельной обработки данных                                                                                              & Присутствует. Блоки, входящие в состав различных ветвлений схемы могут быть выполнены параллельно, поскольку они не зависят друг от друга по используемым данным.                                                                                                                                                                                                                                                                                                                                                                                                                     & Присутствует. Существует возможность обойти различные ветвления графа одновременно.                                                                                                                                                                                                                             \\
    \hline
    9          & Возможность выбрать из набора однотипных промежуточных результатов расчётов некоторые экземпляры и продолжить расчёт только для них; & Производится на этапе анализа результатов с помощью отчётов, где можно задать фильтрацию выходных данных согласно указанным критерия. В случае, если результаты являются промежуточными, расчётную схему приходится разбивать на части.                                                                                                                                                                                                                                                                                                                                               & Планируется реализовать средство визуализации данных, которое в совокупности с автоматической генерацией форм ввода\cite{SokolovPershin2017} позволят отбирать корректные результаты промежуточных вычислений во время обхода графовой модели.                                                                  \\
    \hline
    10         & Возможность доопределения значений входных данных в процессе обхода графа                                                            & Отсутствует                                                                                                                                                                                                                                                                                                                                                                                                                                                                                                                                                                           & Частично реализована при помощи функций-обработчиков специального типа, создающих формы ввода                                                                                                                                                                                                                   \\
    \hline
  \end{longtable}
\end{landscape}