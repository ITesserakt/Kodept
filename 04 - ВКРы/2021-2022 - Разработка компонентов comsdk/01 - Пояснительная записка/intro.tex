%----------------------------------------------------------
\chapter*{ВВЕДЕНИЕ}\label{chap.introduction}
\addcontentsline{toc}{chapter}{ВВЕДЕНИЕ}
% =========================================================================== %
% ----------------------------- ОСНОВНЫЕ ПУНКТЫ ----------------------------- %
% 1. Описание задач, в которых нужно всякое навороченное математическое ПО
% 2. Примеры наовороченного математического ПО
%   2.1. Почему просто математического ПО не всегда достаточно?
%   2.2. Упомянуть про задачи, у которых одна и та же постановка, но разные 
%         параметры
% 3. Примеры ПО, которое рассчитано на многократное решение задач, автоматизи-
%    рующее их решение (Scientific workflow, hallo?)
%   3.1. Использование графов при описании логики решения в системах научных
%        расчётов
%   3.2. Неудобства в описании данных
%   3.3. Визуальное программирование
% 4. Итог: нужно ПО, где есть какая-то абстракция над обрабатываемыми данными,
%    где они конкретизируются непосредственно в реализациях этапов алгоритма.
% 5. Enter GBSE and comsdk
%    5.1. А чем оно, собсна, так привлекательно?
%    5.2. Сказать про НОВЫХ пользователей (Р А С Ш И Р Я Е М О С Т Ь)
% 6. Сравнение GBSE и DFD
% =========================================================================== %
Современные научно-технические исследования зачастую включают в себя задачи, при решении которых требуется большое количество вычислений, для которых зачастую задействуются большие вычислительные мощности. К таким задачам относятся, например, различные задачи анализа, определения характеристик материалов или технических объектов, моделирования сложных динамических процессов. Как правило, для решения подобных задач применяется или разрабатывается специализированное программное обеспечение (\glsxtrshort{ПО}).

Среди прочих применяются системы, предоставляющие пользователю язык описания математических выражений. К таким системам относятся, например, Abacus\cite{PiessensAbacus1989}, Mathcad. Также стоит отметить системы специализирующиеся на символьной алгебре, такие, как Maple\cite{CharMaple1983} и Wolfram Mathematica. Все эти системы позволяют выполнять математическое моделирование в том числе и сложных технических объектов.При всех преимуществах применения подобных программ при решении сложных вычислительных задач за пользователем остаётся необходимость формулировать их математические постановки (т.е. формировать математические модели, составлять системы уравнений и~т.д.). Зачастую требуется решать множество задач с схожей постановкой, но с различными входными параметрами (как, например, при анализе прочностных характеристик технических объектов). Следовательно, целесообразны автоматизированные средства решения подобных типовых задач.

Данные средства относятся к специализированному \glsxtrshort{ПО}, а потому при их разработке требуются глубокие познания в предметной области. Кроме того, важно, чтобы создаваемая кодовая база была рассчитана на дальнейшую поддержку, что предъявляет соответствующие требования к структуре исходного кода и документации. Таким образом целесообразно применение некоторых средств, позволяющих организовать разработку научно-исследовательского программного обеспечения и повысить его поддерживаемость.

В наши дни популярность приобретает применение научных системы управления потоком задач (англ. scientific workflow systems). Такие системы позвояют автоматизировать процессы решения научно-технических задач, предоставляя средства организации и управления вычислительными процессами~\cite{DeelmanWorkflow2009}. Процесс работы с подобными системами состоит из 4 основных этапов:
\begin{enumerate}[1)]
    \item составление описания операций обработки данных и зависимостей между ними;
    \item распределение процессов обработки данных по вычислительным ресурсам;
    \item выполнение обработки данных;
    \item сбор и анализ результатов и статистики.
\end{enumerate}

Примерами подобных систем могут служить Pegasus\cite{DeelmanPegasus2016}, Kepler\cite{AltintasKepler2004} и pSeven\cite{NazarenkoDFM2015}.

Одной из ключевых особенностей подобного подхода к реализации решений научно-технических задач является выделение операций обработки данных в отдельные программные модули (функции, подпрограммы, скрипты). При известных входных и выходных данных каждого модуля становится возможной их независимая разработка\cite{DanilovPar2011}.

Кроме того, существуют т.н. платформы малокодовой разработки (англ. low-code development platforms, \glsxtrshort{LCPD})\cite{DiRuscio2022}. В них, подобно системам управления потоком задач, логика разрабатываемого программного продукта описывается при помощи некоторого формального языка или с использованием графического редактора. От системы к системе подход к описаниям варьируется. Может применяться структурный подход, описывающий шаги алгоритма, или предметно-ориентированный, при котором описываются взаимодействующие сущности. Некоторые системы позволяют по созданному описанию генерировать готовые компоненты будущего программного продукта. Так платформа Codebots реализует предметно-ориентированный подход и по составленным UML-диаграммам взаимодействующих сущностей позволяет генерировать \glsxtrshort{API}, \glsxtrshort{JSON}-схемы данных и документацию\cite{DiRuscio2022}. Тем не менее, при реализации сложных вычислительных методов целесообразнее использовать структурный подход.

В основном, в описанных системах для описания связей между отдельными шагами алгоритма используются ориентированные графы. Помимо описания связей между вычислительными процессами ориентированные графы также находят применение при планировании деятельности (сетевые графики, граф-схемы). В научно-технической среде большее распространение получили сети Петри, диаграммы потоков данных (\glsxtrshort{DFD}) и диаграммы перехода состояний.