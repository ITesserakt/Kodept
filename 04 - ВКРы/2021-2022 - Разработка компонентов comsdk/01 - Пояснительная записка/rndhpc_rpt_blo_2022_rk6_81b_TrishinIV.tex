%\documentclass[12pt]{report}
\documentclass[utf8, 14pt]{G7-32}  % ГОСТ 7.32-2001
%----------------------------------------------------------
% Значения констант по КП, НИРС, ВКР
%----------------------------------------------------------
\newcommand{\doctype}{presentation} % о научно-исследовательской работе / об опытно-конструкторской работе / об опытно-технологической работе / о патентных исследованиях
\newcommand{\doctypesid}{prs} % prs (Презентация) / vkr (выпускная квалификационная работа) / kp (курсовой проект) / kr (курсовая работа) / nirs (научно-исследовательская работа студента) / nkr (научно-квалификационная работа)
\def\titlepagestyle{detail} % brief (краткая) / detail (подробная)
%----------------------------------------%
\newcommand{\Title}{Разработка механизма вывода типов с использованием системы типов\\Хиндли-Милнера}%Научные основы автоматизированного проектирования композиционных материалов
\newcommand{\SubTitle}{Системы типов в языках программирования} % Методы оптимизации
%----------------------------------------------------------
\newcommand{\eventplace}{факультет \facultyShort, кафедра \department} % Место проведения мероприятия
\newcommand{\eventtype}{название события} % Тип мероприятия (лекция|лабораторная работа|семинар|мозговой штурм)
%\newcommand{\eventplace}{@место проведения@} % Место проведения мероприятия
\newcommand{\eventduration}{5 минут} % Продолжительность мероприятия
%----------------------------------------------------------
\newcommand{\Author}{Никитин В.Л.}
\newcommand{\AuthorFull}{Никитин Владимир Леонидович, студент группы РК6-85Б}
\newcommand{\AuthorEmail}{nikitinvl@student.bmstu.ru}
\newcommand{\ScientificAdviserPosition}{кандидат физико-математических наук}	% Должность научного руководителя
\newcommand{\ScientificAdviser}{Соколов А.П.}	% Научный руководитель
\newcommand{\group}{РК6-85Б}
\newcommand{\Semestr}{весенний семестр} % Например: осенний семестр или весенний семестр
\newcommand{\begdate}{08 февраля 2024} % Дата начала разработки
\newcommand{\Year}{2024}
\newcommand{\country}{Россия}	% Страна, в которой проводится конференция
\newcommand{\city}{Москва}		% Город, в котором проводится конференция
%----------------------------------------%
\newcommand{\depHeadPosition}{Заведующий кафедрой}		% Должность руководителя подразделения
\newcommand{\depHeadName}{А.П.~Карпенко}		% Должность руководителя подразделения
%----------------------------------------%
\newcommand{\presentationtitle}{\Title}
\newcommand{\conferenceperiod}{\country, \city, \begdate\ -- \today}
%----------------------------------------%

% Цель исследования/доклада
\newcommand{\GoalOfResearch}{реализация системы вывода и проверки типов} % с маленькой буквы и без точки на конце

% Объектом исследования называют то, что исследуется в работе. Например, для указанной выше темы объектом может быть популяция медуз, но никак ни модель SIS, ни Южно-Китайское море, ни метод моделирования популяции медуз.
\newcommand{\ObjectOfResearch}{система типов}

% Предмет исследований (уже чем объект, определяется, исходя из задач: формулируется как существительное, как правило, во множественном числе, определяющее "конкретный объект исследований" среди прочих в рамках более общего)
\newcommand{\SubjectOfResearch}{система типов Хиндли-Милнера}

% Основная задача, на решение которой направлена работа
\newcommand{\MainProblemOfResearch}{реализация алгоритма вывода типов на основе выбранной системы типов}

% Выполненные задачи
\newcommand{\SubtasksPerformed}{%
	В результате выполнения работы:
	\begin{inparaenum}[1)]
	\item спроектировано представление абстрактного синтаксического дерева в компиляторе;
	\item реализован семантический анализатор;
	\item показано, что компилятор успешно может вывести тип функции
	\end{inparaenum}}

% Ключевые слова (представляются для обеспечения потенциальной возможности индексации документа)
\newcommand{\keywordsru}{%
	теория типов, языки программирования, компиляторы, фукнциональное программирование, система типов Хиндли-Милнера} % 5-15 слов или выражений на русском языке, для разделения СЛЕДУЕТ ИСПОЛЬЗОВАТЬ ЗАПЯТЫЕ
\newcommand{\keywordsen}{%
	type theory, programming languages, compilers, functional programming, Hindley-Milner type system} % 5-15 слов или выражений на английском языке, для разделения СЛЕДУЕТ ИСПОЛЬЗОВАТЬ ЗАПЯТЫЕ

% Краткая аннотация
\newcommand{\Preface}{
	Работа посвящена реализации механизма вывода типов для языка программирования Kodept.
	Программирование выстроено вокруг глубокой математической теории.
	Благодаря этому появляются возможности для оптимизации, развития и улучшения языков посредством применения математики.
	Одним из важных применений является теория типов, которая помогает программисту в написании кода.
	В последнее время все больше и больше языков почерпывают что-то из этой области.
	Применение мощной системы типов позволяет зачастую снизить количество ошибок, возникающих при разработке.
} % с большой буквы с точкой в конце

%----------------------------------------%
% выходные данные по документу
\newcommand{\DocOutReference}{\Author. \Title\xspace\SubTitle. [Электронный ресурс] --- \City: \Year. --- \total{page} с. URL:~\url{https://\gitlabdomain} (система контроля версий кафедры РК6)}

%----------------------------------------------------------

%----------------------------------------------------------
%общая преамбула для всех лабораторных - настройки общего вида оформления
%----------------------------------------------------------
\usepackage{amsfonts}
\usepackage{amsmath}
%\usepackage{mathabx}
%----------------------------------------------------------
\newcommand{\doclicense}{\includegraphics[width=0.09\textwidth]{doc-spec/by.eps}\xspace}%\ccShareAlike

\usepackage[T2A]{fontenc}
\usepackage[utf8]{inputenc}
\usepackage[russian]{babel} %% это необходимо для включения переносов english
\usepackage{float}
\usepackage{rotating}
\usepackage{multirow}
\usepackage{pdflscape}
\usepackage{bm}
% необходимо для возможности копирования и поиска в готовом PDF
%\usepackage{cmap} 
\usepackage{array}
\usepackage{multicol}
\usepackage{relsize}
\usepackage{booktabs}
% Пакет необходим для поддержки многострочного подчеркивания текста
\usepackage[normalem]{ulem}

%----------------------------------------------------------------
% Сохранение метаданных в PDF об авторе документа
\usepackage{hyperxmp}
\usepackage{hyperref}
\hypersetup{%
    bookmarks=false,        % show bookmarks bar?
    pdftoolbar=true,        % show Acrobat’s toolbar?
    pdfmenubar=true,        % show Acrobat’s menu?
    pdffitwindow=false,     % window fit to page when opened
    pdfstartview={FitH},    % fits the width of the page to the window
    pdftitle={\Title},    	% title
    pdfauthor={\Author},    % author
%		pdfcopyright={Copyright © \Year, \Author. Все права защищены.},
		pdfcopyright={CC BY 4.0, \Year, \Author.},
		pdflicenseurl={http://creativecommons.org/licenses/by/4.0/},
    pdfsubject={\SubjectOfResearch},   % subject of the document
    pdfcreator={\pdftexbanner},   % creator of the document
%		pdfpublisher={Computer-aided design department, Bauman Moscow State Technical University},
		pdfcaptionwriter={Ass. Prof., PhD. Alexandr P. Sokolov},
    pdfproducer={\Author, \group, \Year, Computer-aided design department, Bauman Moscow State Technical University}, % producer of the document
    pdfkeywords={\keywordsru, \keywordsen}, % producer of the document
    pdfnewwindow=true,      % links in new window
    colorlinks=true,
    citecolor=purple,
    linkcolor=red,      % color of internal links (change box color with linkbordercolor)
    urlcolor=green,
    filecolor=black      % color of file links
}
%----------------------------------------------------------------
\usepackage{xspace}
%----------------------------------------------------------
\usepackage[style=long4colheader, translate=babel, section=chapter, toc]{glossaries}
\usepackage[abbreviations, toc=true, xindy, automake]{glossaries-extra}
\setglossarystyle{treenoname}%+
\makeglossaries
%----------------------------------------------------------
% поддержка inparaenum
\usepackage{paralist} 
%----------------------------------------------------------
% нужно для определения окружения description
%\usepackage{enumitem} 
%----------------------------------------------------------------
% Настройки вставки PDF (для вставки, к примеру, направления на защиту, акта об отсутствии заимствования, рецензии)
\usepackage{pdfpages}
\includepdfset{turn=true,scale=0.95,pages=-,pagecommand={\pagestyle{fancy}}}
%----------------------------------------------------------
\usepackage{tikz}
\usetikzlibrary{tikzmark}
\usetikzlibrary{matrix,automata,graphs}
\usetikzlibrary{arrows,positioning,trees}
%----------------------------------------------------------
% необходимо для возможности включать в имена включаемых файлов _
\usepackage[strings]{underscore}
%----------------------------------------------------------
% добавление поддержки команды вывода текста на полях \marginnote
\usepackage{marginnote}
% добавление поддержки команды \color
\usepackage{xcolor}
%--------------------------------------------
% final - удаляет все всплывающие комментарии
\usepackage[author={Alexandr Sokolov},opacity=0.1]{pdfcomment}
%\usepackage[author={Alexandr Sokolov},opacity=0.1,final]{pdfcomment}
\newcommand{\messnote}[1]{\marginnote{\color[rgb]{1,0,0}\Huge\textbf{!}\pdfcomment{#1}}[-1.0cm]}
%----------------------------------------------------------
% Произвольная нумерация списков.
\usepackage{enumerate}
%----------------------------------------------------------
%\raggedbottom
%\textwidth=163mm
%\textheight=220mm
%\oddsidemargin=-0.5pt
%\footskip=30pt
%\topmargin=27pt
%\headheight=12pt
%\headsep=25pt
%\topskip=10pt
%\baselineskip=15pt
%\topmargin=-4mm
%----------------------------------------------------------
\tolerance 1414
\hbadness 1414
\emergencystretch 1.5em
\hfuzz 0.3pt
\widowpenalty=10000
\vfuzz \hfuzz
\raggedbottom
%----------------------------------------------------------
% Настройки колонтитулов
\usepackage{fancyhdr} % Headers and footers
\fancyhf{} % clear all headers and footers - equivalent to %\fancyhead{} and \fancyfoot{}
\renewcommand{\headrulewidth}{0.0pt}
\renewcommand{\footrulewidth}{0.0pt}
\renewcommand{\chaptermark}[1]{\markboth{ \chaptername\ \thechapter }{}} 
%\renewcommand{\chaptermark}[1]{\markboth{ \chaptername\ \thechapter ~ \ #1}{}} 
%\renewcommand{\sectionmark}[1]{\markright{\thesection ~ \ #1}}

%head setting
\fancyhead[C]{\thepage}
\fancyfoot[C]{}
\setlength{\headheight}{17pt}%

\pagestyle{fancy} % All pages have headers and footers
%----------------------------------------%
%Необходимо для того, чтобы при использовании команды \thispagestyle{plain} стиль plain был переопределён на этот
\fancypagestyle{plain}{%
\fancyhf{}% clear all header and footer fields
\renewcommand{\headrulewidth}{0pt}%
\renewcommand{\footrulewidth}{0pt}%
\fancyhead[C]{\thepage}
\fancyfoot[C]{}
}
%----------------------------------------%
%Необходимо для того, чтобы при использовании команды \thispagestyle{tocpage} стиль tocpage был переопределён на этот
\fancypagestyle{tocpage}{%
  \fancyhf{}% Remove header/footer
  \renewcommand{\headrulewidth}{0pt}% Remove header rule
  \renewcommand{\footrulewidth}{0pt}% Remove footer rule (default) 
  \fancyhead[C]{\hfill \thepage \hfill Стр.}% Header
  \fancyfoot[C]{}% Footer
}

%----------------------------------------------------------
% указание 
\setcounter{secnumdepth}{2}
%----------------------------------------------------------
% Пакеты для подсчета количества: страниц, и т.д.
\usepackage{etoolbox}
%----------------------------------------------------------
\usepackage{totcount,assoccnt}
%----------------------------------------------------------

% суперсчетчики всего ! :-)
\regtotcounter{page}

\newtotcounter{ffigure}
\setcounter{ffigure}{0}
\def\oldfigure{} \let\oldfigure=\figure
\def\figure{\stepcounter{ffigure}\oldfigure}

\newtotcounter{ttable}
\setcounter{ttable}{0}
\def\oldtable{} \let\oldtable=\table
\def\table{\stepcounter{ttable}\oldtable}

\newtotcounter{cchapter}
\setcounter{cchapter}{0}
\def\oldchapter{} \let\oldchapter=\chapter
\def\chapter{\stepcounter{cchapter}\oldchapter}

\newtotcounter{eequation}
\setcounter{eequation}{0}
\def\oldequation{} \let\oldequation=\equation
\def\equation{\stepcounter{eequation}\oldequation}

\newtotcounter{bibcnt}
\setcounter{bibcnt}{0}
\def\oldbibitem{} \let\oldbibitem=\bibitem
\def\bibitem{\stepcounter{bibcnt}\oldbibitem}


%\newtotcounter{apxchapters}
%\DeclareAssociatedCounters{chapter}{cchapter,apxchapters}
%
%\preto\appendix{%
  %% save the number of true chapters
  %%\setcounter{truechapters}{\value{chapter}}%
  %% reset the number of chapters
  %\setcounter{apxchapters}{0}%
%}
%----------------------------------------------------------
% необходимо для работы команды \xspace (умный пробел после замены, осуществляемой некоторой командой в тексте)
\usepackage{xspace}
%----------------------------------------------------------
% определение атрибутов сборки Git
%\usepackage[grumpy, maxdepth=6]{gitinfo2}
%\renewcommand{\gitMark}{\textcolor{gray}{[git] \textbullet{} \gitBranch\,@\,\gitAbbrevHash{} \textbullet{} \gitAuthorName (\gitAuthorIsoDate)}}
%----------------------------------------------------------
% необходимо для того, чтобы в окружениях enumerate можно было менять формат нумерации
%\usepackage{enumitem}
%----------------------------------------------------------
%Необходимо для сокращения размера шрифта подписей и сокращения отступов между рисунком и подписью к нему
\usepackage[margin=5pt,font={small, singlespacing}, labelfont={small}, justification=centering, labelsep=period]{caption}
\captionsetup{belowskip=0pt}
%----------------------------------------------------------
%\usepackage[numbers]{natbib}
%\usepackage{bibentry}
%***natbib, bibentry***%
% Следующий код необходим для того, чтобы исправить конфликт между пакетами natbib+bibentry и стилем оформления ссылок согласно российскому ГОСТу cp1251gost705u
%\ifx\undefined\selectlanguageifdefined
%\def\selectlanguageifdefined#1{}\else\fi
%\ifx\undefined\BibEmph
%\def\BibEmph#1{\emph{#1}}\else\fi
%----------------------------------------------------------

% подключение листингов и определение языков
\usepackage{listings}

\lstset
{%
		extendedchars=\true, % включаем не латиницу
		frame=tb, % рамка сверху и снизу
		escapechar=|, % |«выпадаем» в LATEX|
		xleftmargin=0.5cm,
		xrightmargin=0.5cm,
		columns=fullflexible,
%		aboveskip=5pt,
		numbers=left,                    % where to put the line-numbers; possible values are (none, left, right)
		numbersep=4pt,                   % how far the line-numbers are from the code
		showspaces=false,
		showstringspaces=false,
		breakatwhitespace=true,         % sets if automatic breaks should only happen at whitespace
		breaklines=true,                 % sets automatic line breaking
		basicstyle=\color{black}\small\sffamily,%\ttfamily,% \sffamily
		commentstyle=\color{gray}\itshape, % шрифт для комментариев
		stringstyle=\color{orange},
%		stringstyle=\bfseries, % шрифт для строк
		numberstyle=\footnotesize\color{gray},
%		numberstyle=\ttfamily\small\color{gray}, % the style that is used for the line-numbers
		keywordstyle=\color{blue}\bfseries,
%		directivestyle=\color{red},
%		emph={int,char,double,float,unsigned,bool,string},
		emphstyle={\color{blue}\bfseries},
		tabsize=2,
%		morecomment=[l]{//},
%		otherkeywords={=,==,:,&},
		texcl=true,
}

\lstloadlanguages{Python, C++}

%--------------------------------------------
% необходимо для команды \cancelto{0}{x}
\usepackage{cancel}
%----------------------------------------------------------
% необходимо для того, чтобы доопределить спецификатор P, для 
% использования в таблицах при форматировании
\usepackage{array}
\newcolumntype{P}[1]{>{\centering\arraybackslash}p{#1}}
%----------------------------------------%
% необходимо для того, чтобы допускались разрывы страниц внутри align align*
\allowdisplaybreaks
%----------------------------------------%
\makeatletter
\def\dynscriptsize{\check@mathfonts\fontsize{\sf@size}{\z@}\selectfont}
\makeatother
\def\textunderset#1#2{\leavevmode
  \vtop{\offinterlineskip\halign{%
    \hfil##\hfil\cr\strut#2\cr\noalign{\kern-.3ex}
    \hidewidth\dynscriptsize\strut#1\hidewidth\cr}}}

\newcommand\executer[1]{\textunderset{\scriptsize{подпись, дата}}{\signvrule} #1}
%----------------------------------------------------------
% необходимо для поддержки поворотов текста
\usepackage[absolute]{textpos}
\setlength{\TPHorizModule}{30mm}
\setlength{\TPVertModule}{\TPHorizModule}
\textblockorigin{0mm}{25mm} % start everything near the top-left corner
%----------------------------------------------------------
% оформление "теорем"
\usepackage{amsthm}
\usepackage{thmtools}
%----------------------------------------------------------
\newtheoremstyle{theoremstyle}% <name>
{0pt}% <Space above>
{0pt}% <Space below>
{\normalfont}% <Body font>
{0pt}% <Indent amount>
{\bfseries}% <Theorem head font>
{.}% <Punctuation after theorem head>
{.5em}% <Space after theorem headi>
{}% <Theorem head spec (can be left empty, meaning `normal')>
%----------------------------------------------------------
\theoremstyle{theoremstyle}

%\declaretheoremstyle[
  %headfont=\normalfont\bfseries,
%%	numberwithin=section,
  %bodyfont=\normalfont,
  %spaceabove=1em plus 0.75em minus 0.25em,
  %spacebelow=1em plus 0.75em minus 0.25em,
  %qed={$\blacksquare$},
	%headpunct={\newline},
%%  qed={$\square$},
%]{taskstyle}
%
%\declaretheorem[
  %style=taskstyle,
  %title=Задача,
  %refname={задача,задачи},
  %Refname={Задача,Задачи}
%]{task}
%
%\declaretheoremstyle[
  %headfont=\normalfont\bfseries,
	%numberwithin=task,
  %bodyfont=\normalfont,
  %spaceabove=1em plus 0.75em minus 0.25em,
  %spacebelow=1em plus 0.75em minus 0.25em,
	%headpunct={\newline},
%%  qed={$\blacksquare$},
  %qed={$\square$},
%]{variantstyle}
%
%\declaretheorem[
  %style=variantstyle,
  %title=Вариант,
  %refname={вариант,варианты},
  %Refname={Зариант,Варианты}
%]{variant}

%----------------------------------------------------------
%\newtheorem{question}{Вопрос}
%\newtheorem{task}{Задача}
%\newtheorem{solution}{Решение}
\newtheorem{remark}{Замечание}
\newtheorem{dexcription}{Описание}
%%----------------------------------------------------------
%% атрибуты задачи
%\newcommand{\labattributes}[6]{%
%\def\tempempty{}
%\def\tempa{#1}
%\def\tempb{#2}
%\def\tempc{#3}
%\def\tempd{#4}
  %\ifx\tempempty\tempa \def\tempa{ассистент кафедры РК-6, PhD~А.Ю.~Першин}\fi
  %\ifx\tempempty\tempb \def\tempb{Решение и вёрстка:}\fi
  %\ifx\tempempty\tempc \def\tempc{}\fi
  %\ifx\tempempty\tempd \def\tempd{}\else \def\tempd{{\textnormal\copyright}~#4}\fi
%
%\vspace{0.5cm}
%\begin{flushright}
		%\begin{tabular}{p{0.25\textwidth}p{0.7\textwidth}}
		%\hfill Постановка: & \doclicense~\textit{\tempa} \\
		%\hfill \tempb & \doclicense~\textit{#5} \\
		%\hfill \tempc & \textit{\tempd} \\
		%\hfill & \textit{#6}\\
		%\end{tabular}
%\end{flushright}
%}
%----------------------------------------------------------
% Изменяем метод нумерации subsection
%\renewcommand{\thesubsection}{\thesection.\arabic{subsection}}
\renewcommand{\thesubsection}{\arabic{subsection}}
%----------------------------------------------------------


%----------------------------------------------------------
\pdfminorversion=7
%----------------------------------------------------------
% общие вспомогательные определения
%----------------------------------------------------------
\def\argmax{\operatornamewithlimits{argmax}}
%----------------------------------------------------------
% горизонтальная линия для последующего проставления подписи
\newcommand{\signhrule}{\raggedright\baselineskip0.0ex \vrule height 0.5pt width30mm depth0pt}
% место для проставления даты
\newcommand{\datetofill}{<<\uline{\textcolor{white}{\hspace{30pt}}}>>~\uline{\textcolor{white}{\hspace{80pt}}}~\Year~\cyrg.}

% 1 - role	% роль
% 2 - ФИО 	% подпись, дата
\newcommand{\signerline}[3][black]{%
#2 & \textunderset{подпись, дата}{\underline{\textcolor{white}{\hspace{120pt}}}} & & \textunderset{ФИО}{\uline{\textcolor{#1}{#3}}}  %фамилия, и.о.
}
%----------------------------------------------------------
\newcommand{\headerruleseparator}{%
\vrule height 0.6mm width 1.0\textwidth depth0pt
\vspace{-19pt}
\vrule height 0.2mm width 1.0\textwidth depth0pt
}
%----------------------------------------------------------
% 1 -- \depHeadPosition
% 2 -- \department
% 3 -- \depHeadName 
\newcommand{\officialheader}{%
\begin{center}
\UpperFullOrganisationName\newline \FullOrganisationName
\end{center}
\vspace{-20pt}
\headerruleseparator}
%----------------------------------------------------------
% 1 -- \depHeadPosition
% 2 -- \department
% 3 -- \depHeadName 
\newcommand{\signerblock}[3]{%
\parbox[t]{72.0mm}{%
\begin{center}
УТВЕРЖДАЮ\\
\vskip1.0mm
#1 \textunderset{индекс}{\underline{\textit{#2}}}\\%\newline
\vskip1.0mm
\textunderset{}{\signhrule} \quad \textit{#3}\newline
\datetofill
\end{center}
}}
%----------------------------------------------------------
\newcommand{\groupblock}[3]{%
\begin{tabular}{p{0.18\textwidth}p{0.15\textwidth}}
\hfill ФАКУЛЬТЕТ & \underline{#1} \\
\hfill КАФЕДРА & \underline{#2} \\
\hfill ГРУППА & \underline{#3} \\
\end{tabular}}
%-------------------------
\usepackage{ifthen}
\usepackage{calc}
%-------------------------
% #1 - showleft
% #2 - subdocname
% #3 - subdocnamedscra
\newcommand{\officialheaderfull}[3][]{%
\officialheader

\begin{center}
\vspace{-50pt}
\begin{tabular}{P{0.25\textwidth}P{0.3\textwidth}P{0.4\textwidth}}
\ifthenelse{\equal{#1}{showleft}}{\smash{%
		\raisebox{-1.25\height}{%
		\groupblock{\facultyShort}{\departmentShort}{\group}
		}}}{}
& & \signerblock{\depHeadPosition}{\departmentShort}{\depHeadName} \\
\end{tabular}
\end{center}

\begin{center}
\vspace{-15pt}
\large
\MakeUppercase{\textbf{#2}}\\
\textbf{#3}
\end{center}

\noindent\begin{tabular}{p{0.95\textwidth}}
Студент группы: \underline{\group} \\
\AuthorFull \\%[-10pt]
\hline
\multicolumn{1}{P{0.9\textwidth}}{\smaller[2] \vspace{-19pt}(фамилия, имя, отчество)}
\end{tabular}

\noindent%\begin{tabular}{p{0.95\textwidth}}
Тема \doctypec: \expandafter\uline\expandafter{\Title}% \\%[-10pt]
%\hline \\
%\end{tabular}
}
%-------------------------
% #1 - current \doctype
% #2 - destination document
% #3 - text
\newcommand{\myconditionaltext}[3]{%
\ifthenelse{\equal{#1}{kr}\AND\equal{#2}{kr}}{#3}{}%
\ifthenelse{\equal{#1}{kp}\AND\equal{#2}{kp}}{#3}{}%
\ifthenelse{\equal{#1}{vkr}\AND\equal{#2}{vkr}}{#3}{}%
\ifthenelse{\equal{#1}{nirs}\AND\equal{#2}{nirs}}{#3}{}%
\ifthenelse{\equal{#1}{nkr}\AND\equal{#2}{nkr}}{#3}{}%
}
% использование
%\myconditionaltext{\doctypesid}{kp}{XXXXXXX} % вставится только при сборке КП
%-------------------------
% к \doctypeb
\newcommand{\doctypeb}{%
\ifthenelse{\equal{\doctypesid}{vkr}}{выпускной квалификационной работе}{}%
\ifthenelse{\equal{\doctypesid}{kr}}{курсовой работе}{}%
\ifthenelse{\equal{\doctypesid}{kp}}{курсовому проекту}{}%
\ifthenelse{\equal{\doctypesid}{nirs}}{научно-исследовательской работе студента}{}%
\ifthenelse{\equal{\doctypesid}{nkr}}{научно-квалификационной работе}{}%
} 
% на выполнение \doctypec
\newcommand{\doctypec}{%
\ifthenelse{\equal{\doctypesid}{vkr}}{выпускной квалификационной работы}{}%
\ifthenelse{\equal{\doctypesid}{kr}}{курсовой работы}{}%
\ifthenelse{\equal{\doctypesid}{kp}}{курсового проекта}{}%
\ifthenelse{\equal{\doctypesid}{nirs}}{научно-исследовательской работы студента}{}%
\ifthenelse{\equal{\doctypesid}{nkr}}{научно-квалификационной работы}{}%
}
% \doctype (в именительном падеже)
\newcommand{\doctype}{%
\ifthenelse{\equal{\doctypesid}{vkr}}{выпускная квалификационная работа}{}%
\ifthenelse{\equal{\doctypesid}{kr}}{курсовая работа}{}%
\ifthenelse{\equal{\doctypesid}{kp}}{курсовой проект}{}%
\ifthenelse{\equal{\doctypesid}{nirs}}{научно-исследовательская работа студента}{}%
\ifthenelse{\equal{\doctypesid}{nkr}}{научно-квалификационная работа}{}%
} 
% \doctypeshort (сокращение)
\newcommand{\doctypeshort}{%
\ifthenelse{\equal{\doctypesid}{vkr}}{ВКР}{}%
\ifthenelse{\equal{\doctypesid}{kr}}{КР}{}%
\ifthenelse{\equal{\doctypesid}{kp}}{КП}{}%
\ifthenelse{\equal{\doctypesid}{nirs}}{НИРС}{}%
\ifthenelse{\equal{\doctypesid}{nkr}}{НКР}{}%
}
%-------------------------
% атрибуты
\newcommand{\docattributes}[6]{%
\def\tempempty{}
\def\tempa{#1}
\def\tempb{#2}
\def\tempc{#3}
\def\tempd{#4}
  \ifx\tempempty\tempa \def\tempa{\ScientificAdviserPosition, \ScientificAdviser}\fi
  \ifx\tempempty\tempb \def\tempb{Решение и вёрстка:}\fi
  \ifx\tempempty\tempc \def\tempc{}\fi
  \ifx\tempempty\tempd \def\tempd{}\else \def\tempd{{\textnormal\copyright}~#4}\fi

\vspace{0.5cm}
\begin{flushright}
\smaller[1]
		\begin{tabular}{p{0.25\textwidth}p{0.75\textwidth}}
		\hfill Постановка: & \doclicense~\textit{\tempa} \\
		\hfill \tempb & \doclicense~\textit{#5} \\
		\hfill \tempc & \textit{\tempd} \\
		\hfill & \textit{#6}\\
		\end{tabular}
\end{flushright}
}
%----------------------------------------------------------


%----------------------------------------------------------
% база аббревиатур и определений
%----------------------------------------------------------
%Термины и определения по тексту в большинстве случаев выделяются курсивом.
%В настоящем отчете о НИР применяют следующие термины с соответствующими определениями, также используются представленные обозначения и сокращения.


%\newabbreviation[category=inline]{html}{HTML}{hypertext markup language}
%\newabbreviation[category=footer]{shtml}{SHTML}{server-parsed HTML}

%\newglossaryentry{sample2}{name={sample2},
%	symbol={\ensuremath{\mathcal{S}_2}},
%	category=symbol,
%	description={the second sample entry}}

%\newabbreviation
%[prefix={an\space},
%prefixfirst={a~}]
%{svm}{SVM}{support vector machine}

%\newabbreviation
%[category=initialism,description={for example}]
%{eg}{eg}{exempli gratia}
% define the entries:

%\newabbreviation{html}{html}{hypertext markup language}

%\newabbreviation[category=initialism]{eg}{eg}{for example}
%\newabbreviation[category=initialism]{si}{SI}{sample initials}
%\newabbreviation{xml}{XML}{extensible markup language}
%\newabbreviation{css}{CSS}{cascading style sheet}
%\newacronym[description={a device that emits a narrow intense 
%	beam of light}]{laser}{laser}{light amplification by stimulated 
%	emission of radiation}

%\newacronym[description={a form of \gls{laser} generating a beam of
%	microwaves}]{maser}{maser}{microwave amplification by stimulated 
%	emission of radiation}

%\newacronym[description={a system for detecting the location and
%	speed of ships, aircraft, etc, through the use of radio waves}]{radar}{radar}{radio detection and ranging}

%\newacronym[description={portable breathing apparatus for divers}]{scuba}{scuba}{self-contained underwater breathing apparatus}

%%%%% для обычных newglossaryentry по умолчанию category==general.
%%%%% для обычных newabbreviation по умолчанию category==abbreviation.
%%%%% команда для создания своей категории \glscategory{<label>}
%----------------------------------------------------------
\newglossaryentry{slver}{name={Solver}, description={Решатель системы \gls{dcs-gcd}. Регистрируется в таблице \textbf{com.slvrs} БД \gls{gcddb} \gls{dcs-gcd}.}}
\newabbreviation[category=initialism]{ПО}{ПО}{-- программное обеспечение}
\newabbreviation[category=initialism]{API}{API}{-- прикладной программный интерфейс (Application Programming Interface)}
\newabbreviation[category=initialism]{DFD}{DFD}{-- диаграмма потоков данных (Data Flow Diagram)}
\newabbreviation[category=initialism]{GBSE}{GBSE}{-- графоориентированный подход к разработке программного обеспечения (graph based software engineering)}
\newabbreviation[category=initialism]{LCPD}{LCPD}{-- платформы малокодовой разработки (low-code development platforms)}
\newabbreviation[category=initialism]{CASE}{CASE}{-- втоматизированная разработка программного обеспечения (Computer aided software engineering)}
\newabbreviation[category=initialism]{DOT}{DOT}{-- язык описания графов}
\newabbreviation[category=initialism]{JSON}{JSON}{-- файловый формат для хранения структур данных (Javascrtipt Object Notation)}
\newabbreviation[category=initialism]{TO}{ТО}{технический объект, в т.ч. сложный процесс, система}
\newabbreviation[category=initialism]{aINI}{aINI}{-- расширенный формат INI (\href{https://archrk6.bmstu.ru/index.php/f/846701}{описание представлено в \cite{SokAINI}})}
\newabbreviation[category=initialism]{aDOT}{aDOT}{-- расширенный формат DOT (\href{https://archrk6.bmstu.ru/index.php/f/777612}{описание представлено в \cite{SokolovADOT2020}})}


\GlsXtrEnableEntryCounting
{abbreviation}% list of categories to use entry counting
{2}% trigger value

\GlsXtrEnableEntryCounting
{symbol}% list of categories to use entry counting
{2}% trigger value


%----------------------------------------------------------



%----------------------------------------------------------
% сборка документа
\includeonly{%
,referat															% Реферат
,intro																% Введение
,chapters/chap1_task_statement				% Постановка задачи
,chapters/chap0_methods_comparison						% Вычислительный метод (включается в зависимости от задачи)
,chapters/chap3_soft_architecture			% Программная реализация (включается в зависимости от задачи)
,chapters/chap4_soft_testing					% Тестирование и отладка (включается в зависимости от задачи)
,chapters/chap5_comp_experiment				% Вычислительный эксперимент (включается в зависимости от задачи)
,chapters/chap6_results_analysis			% Анализ результатов (включается в зависимости от задачи)
,conclusion														% Заключение
,additionals													% Приложения (включается в зависимости от необходимости)
}
%----------------------------------------------------------
% выключает разворачивание терминов и аббревиатур при первом использовании в том числе, - всегда термины и аббревиатуры будут выводиться кратко 
\glsunsetall
%==========================================================
\begin{document}
% Процедура сборки: 
% 1. Первичная сборка: формирование aux, идентификация ссылок (\ref), цитирований (\cite), использования аббревиатур и определений (\gls)
% pdflatex cpxsln_rpt_YYYY_TaskName_Group_SurnameNS
%
% 2. Собираем глоссарии (работает при установленном Perl, например Strawberry)
% makeglossaries cpxsln_rpt_YYYY_TaskName_Group_SurnameNS
%
% 3. Собираем библиографию
% bibtex cpxsln_rpt_YYYY_TaskName_Group_SurnameNS
%
% 4. Окончательная сборка с учётом всех ссылок, библиографии и глоссария
% pdflatex cpxsln_rpt_YYYY_TaskName_Group_SurnameNS
%
\frontmatter %%% <-- это выключает нумерацию ВСЕГО; здесь начинаются ненумерованные главы типа Исполнители, Обозначения и прочее
%----------------------------------------------------------
% Титульная страница (включается всегда, поэтому командой input)
%-------------------------
\thispagestyle{empty}

\vspace*{-\baselineskip}
\vspace*{-\headheight}
\vspace*{-\headsep}
\vspace*{-2pt}

\begin{center}

\begin{textblock}{1}(0,0)
\rotatebox{90}{\textcolor{gray!20.}{МГТУ им. Н.Э.Баумана, кафедра <<Системы автоматизированного проектирования>> (РК-6), шаблон RPT (размещение sa2tml)}}
\end{textblock}

{\centering%
\begin{tabular}{P{0.15\textwidth}P{0.75\textwidth}}
\smash{%
		\raisebox{-0.7\height}{%
		\includegraphics[width=0.15\textwidth]{doc-spec/bmstu.pdf}
		}}
 & \smaller[1] \UpperFullOrganisationName\newline \FullOrganisationName \\
\end{tabular}}

\headerruleseparator

\vspace{-40pt}
\begin{flushleft}
\begin{tabular}{P{0.15\textwidth}P{0.75\textwidth}}
\multicolumn{1}{p{0.15\textwidth}}{} & \multicolumn{1}{p{0.85\textwidth}}{} \\
\multicolumn{1}{p{0.15\textwidth}}{ФАКУЛЬТЕТ}	&	\multicolumn{1}{p{0.85\textwidth}}{\underline{\faculty}}\\[5pt]
\multicolumn{1}{p{0.15\textwidth}}{КАФЕДРА}	&	\multicolumn{1}{p{0.85\textwidth}}{\underline{\department}}	\\
\end{tabular}
\end{flushleft}

\vspace{2.5cm}

\begin{center}
\Large
\MakeUppercase{Расчётно-пояснительная записка}

\vspace{0.35cm}

% Почему-то не работает
%К~\expandafter\uppercase\expandafter{\doctypeb}
к\xspace\doctypeb

\vspace{0.4cm}

\myconditionaltext{\doctypesid}{kp}{%
	\SubTitle
}

%\vspace{0.35cm}

{\smaller[1]
на тему

<<\Title>>}
\end{center}

\vspace{3.0cm}

\large

\begin{tabular}{p{0.45\textwidth}P{0.25\textwidth}P{0.25\textwidth}} 
\signerline{Студент \textunderset{группа}{\underline{\group}}}{\Author} \\[10pt]
\signerline{Руководитель \doctypeshort}{\ScientificAdviser} \\[10pt]
\signerline{Консультант}{\ConsultantA} \\[10pt]
\signerline[white]{Консультант}{\ConsultantB} \\[10pt]
\signerline{Нормоконтролёр}{\Normocontroller} \\
\end{tabular}

\vspace{4.5cm}

\City, \Year

\end{center}
%-------------------------





%----------------------------------------------------------
% !!! ВНИМАНИЕ! Документы ниже нужно включить перед печатью всего документа !!! %
% Задание (включается для КП и ВКР)
% Вставится только при сборке КП
\myconditionaltext{\doctypesid}{kp}{%
    %-------------------------
\newpage
%-------------------------
\officialheaderfull[]{ЗАДАНИЕ}{на выполнение \doctypec}
%-------------------------

\noindent Источник тематики (кафедра, предприятие, НИР): \underline{\TitleSource}

\myconditionaltext{\doctypesid}{vkr}{%
\noindent Тема \doctypec\xspace утверждена распоряжением по факультету \facultyShort~№~\underline{\textcolor{white}{XXXX}} от \datetofill
}

\myconditionaltext{\doctypesid}{kp}{%
\noindent Тема \doctypec\xspace утверждена на заседании кафедры \department, Протокол~№~\underline{\textcolor{white}{XXXX}} от \datetofill
}

\noindent \textbf{Техническое задание}

\noindent \textbf{Часть 1.} \textit{Аналитический обзор литературы.\\
\uline{В рамках аналитического обзора литературы необходимо проанализировать актуальность исследований в области оптимизации процесса реализации вычислительных методов, найти существующие программные инструменты, позволяющие оптимизировать процесс разработки. Должны быть определены перспективы использования графоориентированного подхода для реализации вычислительных методов.}}

\noindent \textbf{Часть 2.} \textit{Математическая постановка задачи, разработка архитектуры программной реализации, программная реализация.\\
\uline{Необходимо описать архитектуру разрабатываемого редактора графов, разработать web-ориетированного приложение, позволяющее создавать графовые модели вычислительных методов.}}

\noindent \textbf{Часть 3.} \textit{Проведение вычислительных экспериментов, отладка и тестирование.\\
\uline{В рамках тестирования необходимо представить ряд примеров, показывающих реализованные в редакторе функциональные возможности.}}

\vspace{1cm}

\noindent \textbf{Оформление \doctypec:}

\noindent Расчетно-пояснительная записка на \total{page} листах формата А4.

\noindent Перечень графического (иллюстративного) материала (чертежи, плакаты, слайды и т.п.):

\noindent\begin{tabular}{|p{0.95\textwidth}|}
\hline
\textit{количество: \total{ffigure}~рис., \total{ttable}~табл., \total{bibcnt}~источн.} \\
\hline
\textit{[здесь следует ввести количество чертежей, плакатов]} \\
\hline
	\\
\hline
	\\
\hline
	\\
\hline
\end{tabular}

\noindent Дата выдачи задания \TaskStatementDate\\

\noindent \begin{tabular}{p{0.55\textwidth}>{\raggedleft}p{0.2\textwidth}P{0.2\textwidth}} 
\signerline{\textbf{Студент}}{\Author} \\[5pt]
\signerline{\textbf{Руководитель \doctypec}}{\ScientificAdviser} \\
\end{tabular}

\vspace{10pt}
\noindent {\smaller[1] Примечание: Задание оформляется в двух экземплярах: один выдается студенту, второй хранится на кафедре.}

}
% Вставится только при сборке ВКР
\myconditionaltext{\doctypesid}{vkr}{%
    %-------------------------
\newpage
%-------------------------
\officialheaderfull[]{ЗАДАНИЕ}{на выполнение \doctypec}
%-------------------------

\noindent Источник тематики (кафедра, предприятие, НИР): \underline{\TitleSource}

\myconditionaltext{\doctypesid}{vkr}{%
\noindent Тема \doctypec\xspace утверждена распоряжением по факультету \facultyShort~№~\underline{\textcolor{white}{XXXX}} от \datetofill
}

\myconditionaltext{\doctypesid}{kp}{%
\noindent Тема \doctypec\xspace утверждена на заседании кафедры \department, Протокол~№~\underline{\textcolor{white}{XXXX}} от \datetofill
}

\noindent \textbf{Техническое задание}

\noindent \textbf{Часть 1.} \textit{Аналитический обзор литературы.\\
\uline{В рамках аналитического обзора литературы необходимо проанализировать актуальность исследований в области оптимизации процесса реализации вычислительных методов, найти существующие программные инструменты, позволяющие оптимизировать процесс разработки. Должны быть определены перспективы использования графоориентированного подхода для реализации вычислительных методов.}}

\noindent \textbf{Часть 2.} \textit{Математическая постановка задачи, разработка архитектуры программной реализации, программная реализация.\\
\uline{Необходимо описать архитектуру разрабатываемого редактора графов, разработать web-ориетированного приложение, позволяющее создавать графовые модели вычислительных методов.}}

\noindent \textbf{Часть 3.} \textit{Проведение вычислительных экспериментов, отладка и тестирование.\\
\uline{В рамках тестирования необходимо представить ряд примеров, показывающих реализованные в редакторе функциональные возможности.}}

\vspace{1cm}

\noindent \textbf{Оформление \doctypec:}

\noindent Расчетно-пояснительная записка на \total{page} листах формата А4.

\noindent Перечень графического (иллюстративного) материала (чертежи, плакаты, слайды и т.п.):

\noindent\begin{tabular}{|p{0.95\textwidth}|}
\hline
\textit{количество: \total{ffigure}~рис., \total{ttable}~табл., \total{bibcnt}~источн.} \\
\hline
\textit{[здесь следует ввести количество чертежей, плакатов]} \\
\hline
	\\
\hline
	\\
\hline
	\\
\hline
\end{tabular}

\noindent Дата выдачи задания \TaskStatementDate\\

\noindent \begin{tabular}{p{0.55\textwidth}>{\raggedleft}p{0.2\textwidth}P{0.2\textwidth}} 
\signerline{\textbf{Студент}}{\Author} \\[5pt]
\signerline{\textbf{Руководитель \doctypec}}{\ScientificAdviser} \\
\end{tabular}

\vspace{10pt}
\noindent {\smaller[1] Примечание: Задание оформляется в двух экземплярах: один выдается студенту, второй хранится на кафедре.}

}
%----------------------------------------------------------
% Календарный план, только для ВКР (для НИРС и КП этот документ не включается) 
\myconditionaltext{\doctypesid}{vkr}{%
    \newpage
%-------------------------
\officialheaderfull[showleft]{КАЛЕНДАРНЫЙ ПЛАН}{выполнения \doctypec}
%-------------------------

{\smaller[1]
\noindent\begin{longtable}{|P{0.03\textwidth}|p{0.35\textwidth}|>{\smaller[1]}P{0.08\textwidth}|>{\smaller[1]\itshape}P{0.08\textwidth}|>{\smaller[1]}P{0.15\textwidth}|>{\smaller[1]}P{0.15\textwidth}|}
\hline
\textbf{№ п/п} & \textbf{Наименование этапов \doctypec} &	\multicolumn{2}{|P{0.16\textwidth}|}{\textbf{Сроки выполнения этапов}} & \multicolumn{2}{|P{0.3\textwidth}|}{\textbf{Отметка о выполнении}} \\
\cline{3-6}
	&  & \textbf{план} & \textbf{факт} & \textbf{Должность} & \textbf{ФИО, подпись} \endhead
\hline
1. & Задание на выполнение работы. Формулировка проблемы, цели и задач работы &  &  & Руководитель \doctypeshort & \ScientificAdviser \\
\hline
2. & 1 часть &  &  & Руководитель \doctypeshort & \ScientificAdviser \\
\hline
3. & Утверждение окончательных формулировок решаемой проблемы, цели работы и перечня задач &  &  & \depHeadPosition & \depHeadName \\
\hline
4. & 2 часть &  &  & Руководитель \doctypeshort & \ScientificAdviser \\
\hline
5. & 3 часть &  &  & Руководитель \doctypeshort & \ScientificAdviser \\
\hline
6. & 1-я редакция работы &  &  & Руководитель \doctypeshort & \ScientificAdviser \\
\hline
7. & Подготовка доклада и презентации &  &  & & \\
\hline
8. & Отзыв руководителя &  &  & Руководитель \doctypeshort & \ScientificAdviser \\
\hline
9. & Нормоконтроль &  &  & Нормоконтролёр & С.В.~Грошев \\
\hline
10. & Внешняя рецензия &  &  &  & \\
\hline
11. & Защита работы на ГЭК &  &  &  & \\
\hline
\end{longtable}}

{\smaller[1]
\noindent\begin{tabular}{ll}
	\hspace{-20pt}\textit{Студент} \textunderset{подпись, дата}{\underline{\textcolor{white}{\hspace{80pt}}}} \textunderset{ФИО}{\underline{\Author}} &
	\textit{Руководитель работы} \!\textunderset{подпись, дата}{\underline{\textcolor{white}{\hspace{80pt}}}} \textunderset{ФИО}{\underline{\ScientificAdviser}} \\
\end{tabular}}

}
%----------------------------------------------------------
% Направление на защиту (включается только для ВКР, для НИРС и КП не нужно)
\myconditionaltext{\doctypesid}{vkr}{%
    %----------------------------------------------------------------
% В этот документ вставится заполненный и подписанный бланк официального направления на защиту КП/ВКР
% Для КП, НИРС этот документ не включается.
% Предварительно документ следует подготовить в MS Word, подписать, преобразовать в PDF и далее разместить в нужном каталоге
\newpage
{\catcode`\_=11
\includepdf{doc-additional/cpxsln_vkr_20YY_ShortTitle_group_SurnameNF_referral_for_defense.pdf}
}
%----------------------------------------------------------------

}
%----------------------------------------------------------

% !!! ВНИМАНИЕ! Строчку ниже нужно закомментировать перед печатью всего документа !!! %
% |
% V
%\setcounter{page}{7}

% Реферат
%----------------------------------------------------------
\chapter*{РЕФЕРАТ}
%----------------------------------------------------------
\doctype\xspace: \total{page}~с., \total{ffigure}~рис., \total{ttable}~табл., \total{bibcnt}~источн.%, \textbf{apxchapters} прил.

\vspace{3mm}

%Ключевые слова:
\MakeUppercase{\keywordsru}.

\Preface

\textbf{Тип работы}: \doctype.

\textbf{Тема работы}: \textit{<<\Title>>}.

\textbf{Объект исследования}: \ObjectOfResearch.

\textbf{Основная задача, на решение которой направлена работа}: \MainProblemOfResearch.

\textbf{Цели работы}: \GoalOfResearch

\SubtasksPerformed

% При оформлении согласно ГОСТ 7.32-2001 
%(все освещать следует в этом же порядке - разделы не обязательны)
%объект исследования или разработки
%цель работы
%метод или методологию проведения работы
%результаты работы
%основные конструктивные, технологические и технико-эксплуатационные характеристики
%степень внедрения
%рекомендации по внедрению или итоги внедрения результатов НИР
%область применения
%экономическую эффективновность или значимость
%прогнозные предположения о развитии объекта исследования.


%----------------------------------------------------------
% Сокращения и определения
%\newpage
%\pagestyle{fancy}
%\printglossary[type=\acronymtype, title={СОКРАЩЕНИЯ}, nopostdot=false, nonumberlist]
%\thispagestyle{plain}
%\printglossary[type=main, title={ОПРЕДЕЛЕНИЯ}, nopostdot=true]
%----------------------------------------------------------
% Содержание
% Глубина содержания должна быть не более, чем глава (chapter), раздел (section) и подраздел (subsection) 
\setcounter{tocdepth}{3}
% Добавление в оглавление сверху Стр.
%\makeatletter
%\addtocontents{toc}{\string\pagestyle{TOC}}
%\addtocontents{toc}{\string\thispagestyle{fancy}}
%\addtocontents{toc}{\hfill Стр.\par}
%\def\ps@TOC{%
%\def\@oddhead{\hfill \thepage \hfill Стр.} % нечетные хедеры
%\let\@oddfoot\@empty % нечетные футеры
%\def\@evenhead{\hfill \thepage \hfill Стр.} % четные хедеры
%\let\@evenfoot\@empty % четные футеры
%}
%\makeatother
%----------------------------------------------------------
\renewcommand{\contentsname}{\MakeUppercase{Содержание}}
\newpage
%\pagestyle{tocpage}
\tableofcontents
%----------------------------------------------------------
% Введение
{%
    \def\thesection{В.\arabic{section}}
    \def\thefigure{В.\arabic{figure}}
    \def\thetable{В.\arabic{table}}
    %----------------------------------------------------------
\chapter*{ВВЕДЕНИЕ}\label{chap.introduction}
\addcontentsline{toc}{chapter}{ВВЕДЕНИЕ}

% --------------------------------------------------------
% Определения
\newglossaryentry{GraphOrientedApproach}{
    name={Графоориентированный подход},
    description={подход к организации процесса решения сложной вычислительной задачи, состоящего из нескольких процессов обработки данных, подразумевающий представление данного решения в виде графа}
    }

%----------------------------------------------------------

Процесс работы со сложными САПР и системами инженерного анализа для пользователя обычно сопряжен с требованием глубоких знаний предметной области решаемых задач. Обычно многие вычислительные задачи, которые требуется решать в процессе проектирования сложных технических объектов, предполагают процедуры пре- и пост- процессинга, и кроме того процедуры обработки данных, каждая из которых может включать необходимость запуска множества специальных функций, назначение и принципы работы которых пользователь вынужден предварительно изучить, пользуясь документацией по системе. В связи с разнообразностью вычислительных задач и часто изменяющимися требованиями соответствующее программное обеспечение претерпевает изменения, что также должно отражаться в документации \cite{SokolovPershin2020}.
\begin{statement}
    Одним из возможных путей к упрощению процесса применения и изучения наукоемкого ПО служит исключение необходимости его изучения за счет автоматизации самопроцедуры его использования. Такая автоматизация предполагает: а) формализацию метода организации вычислительных процессов в автоматизированной системе; б) определение классов прикладных задач, поддерживаемых в системе, и сопоставление с каждым классом формального описания метода решения; в) ограничение доступа к функциональным возможностям системы, обеспечивающим решение отдельных подзадач.\cite{SokolovGolubev2021}
\end{statement}
\begin{statement}
    Многие известные методики предполагают организацию вычислительных процессов в графовой форме. Например, применяют диаграммы потоков данных (DFD), граф-схемы, конечные автоматы, диаграммы перехода состояний. Такое описание позволяет выделять структурные единицы приложения в виде функций или подпрограмм и связывать их между собой в определенной последовательности.
\end{statement}
Более подробное описание применяемых методик можно найти в \cite{SokolovGolubev2021}. В данном разделе внимание будет сосредоточено на сравнении нескольких конкретных уже реализованных продуктов для решения различных задач проектирования, где вычислительные процессы организованы в графовой форме.
К сравнению были выбраны и рекомендованы следующие программные комплексы:
\begin{enumerate}
    \item Pradis - разработка отечественной компании "Ладуга"
    \item pSeven - разработка отечественной компаниии DATADVANCE
    \item GBSE - разработка группы преподавателей и студентов МГТУ им. Баумана
\end{enumerate}
Для сравнения были выделены следующие группы признаков:
\begin{itemize}
    \item Общие признаки;
    \item Признаки, относящиеся к топологии создаваемых графов, описывающих процессы обработки данных;
    \item Признаки, относящиеся к обходу данных графов.
\end{itemize}
К первой группе относятся такие признаки, как спектр задач, особенности работы с входными и выходными данными, файловая структура проектов. Ко второй группе относятся формат описания графов, подход к их формированию, возможность включения одного графа в состав другого, особенности передачи параметров между узлами, наличие поддержки ветвлений и циклов. К третьей группе признаков относятся поддержка параллельной обработки данных, поддержка распределённого выполнения, особенности ввода дополнительных данных Особенности ввода дополнительных данных и взаимодействия с пользователем в процессе обработки данных, особенности отбора корректных результатов расчета вручную, возможности доопределять значения входных данных в процессе обхода графа.

Программный комплекс Pradis, разработанный отечественной компанией "Ладуга", был рекомедован к обзору и сравнению, однако после проведённого обзора официальной документации\cite{PradisGeneral2007}\cite{PradisMethods2007}, не было получено достаточного представления об использовании графооринтированного подхода в данном копмлексе, поэтому было принято решение исключить его из дальнейшего рассмотрения.

Что касается программного комплекса pSeven, разработанном компанией DATADVANCE, используется методология диаграмм потоков данных, т.е. топология графа, описывающего процесс решения некоторой задачи проектирования, определяется только зависимостями между входными и выходными данными каждого отдельного процесса обработки данных, входящиего в решение. \cite{Nazarenko2015} В реализованном в pSeven подходе вводятся следующие понятия:
\begin{itemize}
    \item \emph{Расчётная схема (workflow)} - формальное описание процесса решения некоторой задачи в виде ориентированного графа;
    \item \emph{Блок} - программный контейнер для некоторого процесса обработки данных, входные и выходные данные для которого задаются через порты;
    \item \emph{Порт} - переменная определённого типа, имеющая определённое имя, привязанная к блоку;
    \item \emph{Связь} - направленное соединение типа "один к одному" между входным и выходным портами разных блоков;
\end{itemize}

С учётом данных понятий можно описать методологию диаграмм потов данных следующим образом. Расчётная схема содержит в себе набор процессов обработки данных (блоков), каждый из которых имеет (возможно, пустой) набор именованных входов и выходов (портов). Данные передаются через связи. Для избежания гонок данных множественные связи с одним и тем же входным портом не поддерживаются. Для начала выполнения каждому блоку требуются данные на всех входных портах. Все данные на выходных портах формируются по завершении исполнения блока.\cite{Nazarenko2015}

Все порты, которые не привязаны к другим блокам, автоматически становятся внешними входами и выходами для всей расчётной схемы. Для начала обхода расчётной схемы должен быть предоставлен набор входных данных и указаны внешние выходные порты, значения которых обязательно должны быть вычислены в результате обхода. Он производится в несколько этапов: сперва отслеживаются пути от необязательных выходных портов к входным, все встреченные на пути блоки помечаются, как неактуальные и не будут выполнены в дальнейшем; затем отслеживаются пути от обязательных выходных портов к входным и все встреченные на пути блоки помечаются, как обязательные к исполнению. Наконец обязательные к исполнению блоки запускаются, начиная с тех, которые подключены к внешним входам расчётной схемы, а неактуальные игнорируются. Обход прекращается, когда не остаётся необходимых для выполнения блоков. \cite{Nazarenko2015}

Результаты проведённого сравнения были оформлены в общую таблицу, приведённую ниже.
\noindent\begin{longtable}{|p{3.5cm}|p{6.5cm}|p{6.5cm}|}
    \caption{Сравнительная таблица \label{thetable}} \\
    \hline
    \textbf{Признак} & \textbf{pSeven} & \textbf{GBSE} \\
    \hline
    Cпектр задач & Задачи оптимизации, анализ данных & Задачи автоматизированного проектирования, анализ данных \\
    \hline
    Подход к формированию графа & Согласно описанному в \cite{Nazarenko2015} подходу, узлами графа являются блоки, рёбрами - связи, по которым передаются данные. & Узлами графа являются состояния данных, рёбрами - переходы между состояниями, к которым привязываются функции-обработчики. \cite{SokolovPershin2018} \\
    \hline
    Формат описания графа & Сформированное описание сохраняеся в двоичном файле закрытого формата с расширением \textsf{.p7wf} & Описание графа и функций-обработчиков сохраняется в текстовом файле специального формата \textsf{.aDOT}, являющегося расширением формата DOT\cite{SokolovPershin2018} \\
    \hline
    Файловая структура проекта & Проект состоит из непосредственно файла проекта, в котором хранятся ссылки на созданные расчётные схемы и базу данных, сами расчётные схемы, файлы с их входными данными, файлы отчётов, где сохраняются выходные данные последних расчётов и результаты их анализа. & Проект состоит из \textsf{.aDOT} файла с описанием графа, \textsf{.aINI}-файлов с описанием входных данных, библиотеки функций-обработчиков, файлов, куда записываются выходные данные. \\
    \hline
    Особенности работы с входными и выходными данными & Входные данные должны быть указаны при настройках внешних входных портов расчётной схемы. Данные с выходных портов схемы сохраняются в локальной базе данных. Для их записи в файлы для обработки/анализа вне pSeven необходимо воспользоваться специально предназначенными для этого блоками. & Входные данные хранятся в файле с расширением .aINI, откуда считываются при запуске обхода графа\cite{SokolovPershin2017}. Для записи выходных/промежуточных данных в файлы или базы данных необходимо добавить соответствующие функции-обработчики. \\
    \hline
    Особенности передачи параметров между узлами & Данные между узлами передаются через связи, которые на уровне выполнения создают пространство в памяти для ввода и вывода данных для выполняемых в раздельных процессах блоков.  Транзитная передача данных, которые не изменяются в данном блоке, на выход невозможна. & Поскольку узлами графа являются состояния данных, существует возможность задействовать в расчётах только часть данных, оставляя их другую часть без изменений \\
    \hline
    Поддержка ветвлений и циклов & Присутствует. Достигается засчёт специальных управляющих блоков, которые отслеживают выполнение условий & Присутствует по умолчанию\\
    \hline
    Поддержка параллельной обработки данных & Присутствует. Блоки, входящие в состав различных ветвлений схемы могут быть выполнены параллельно, поскольку они не зависят друг от друга по используемым данным. & Присутствует. Существует возможность обойти различные ветвления графа одновременно.\\
    \hline
    Особенности отбора корректных результатов расчёта вручную & Производится на этапе анализа результатов с помощью отчётов, где можно задать фильтрацию выходных данных по указанным параметрам. В случае, если результаты являются промежуточными, расчётную схему приходится разбивать на части. & Планируется реализовать средство визуализации данных, которое вкупе с автоматической генерацией форм ввода позволят отбирать корректные результаты промежуточных вычислений во время обхода одного цельного графа. \\
    \hline
    Возможность доопределения значений входных данных в процессе обхода графа & Отсутствует & Реализована при помощи функций-обработчиков, создающих формы ввода \\
    \hline
\end{longtable}
%Во введении должны быть представлены: введение в проблему, описание объекта исследований, обзор научно-технических источников\footnote{Следует изучать источники следующих типов в следующем порядке по убыванию приоритетности: \textbf{научные статьи}, патенты, электронные источники, книги (общее количество не менее 15)} по направлению поставленной задачи, примеры существующих аналогичных научно-технических решений.

%\textbf{Целью} обзора научно-технических источников является \textbf{обоснование актуальности} решения поставленной задачи.

%Обоснование актуальности предполагает проведение обзора литературы. Обзор литературы рекомендуется осуществлять, используя инструкцию\footnote{Инструкция о проведении обзора литературы: \url{https://archrk6.bmstu.ru/index.php/f/2597}}.

%В состав материалов проводимого обзора литературы должны включаться выводы/заключения, ставшие результатом анализа соответствующих источников, на которые при этом обязательно следует делать ссылки (например, так). Источник, при этом, следует включать в список литературы в последний раздел настоящего документа (в настоящем документе список источников формируется автоматически с помощью компилятора \textsf{BibTeX} на основании файла \textsf{bibliography.bib} и ссылок по тексту).

%В результате анализа всех источников должно стать возможным сделать вывод об обоснованности работ в направлении поставленной задачи.

%\begin{remark}
%Отметим, что в процессе подготовке текста возникает необходимость вводить аббревиатуры и использовать специальные термины, которые для документов большого объёма, выносятся в отдельные разделы ``Сокращения'' и ``Определения''. При использовании \LaTeX\xspace нет необходимости формировать этим разделы специально, -- рекомендуется использовать т.н. глоссарии. Создаётся файл \textsf{abbreviations.tex}, вносятся в него все необходимые термины и аббревиатуры и далее в любом месте текста ссылаются на них с использованием команды {\verb_\gls{SID}_} с одновременным появлением соответствующей расшифровки термина в соответствующем разделе (например, вызов команды {\verb_\gls{IND}_} приведёт к формированию \gls{IND}).
%\end{remark}


%\textbf{В последнем абзаце} введения следует указывать цель работы в целом.

%\underline{Обязательность представления:} раздел обязателен. 

% \underline{Объём:} как правило, не должен быть больше 5-7 страниц.

%----------------------------------------------------------

}
%----------------------------------------------------------
\mainmatter %% это включает нумерацию глав и секций в документе ниже
%----------------------------------------------------------
%=============================================================
% ---------------------------------------------------
\chapter{Обзор средств взаимодействия пользователя в графоориентированных системах} \label{chap1.comparison}
Среди вычислительных задач, с которыми сталкиваются современные исследователи и разработчики наукоёмкого программного обеспечения, можно выделить те, в которых в результате проведения расчётов получается несколько различных результатов, из которых требуется выбрать наиболее подходящий на основе каких-то критериев или провести ту или иную операцию в зависимости от полученных результатов. В наши дни наблюдается тенденция к автоматизации подобного процесса на основе различных алгоритмов анализа и принятия решений, некоторые из которых разрабатываются специально под конкретную задачу, а некоторые, более универсальные, адаптируются под неё, как, например, описано в \cite{KatalOpt2020}. Тем не менее, остаётся широкий спектр задач, где разработка подобных алгоритмов не ведётся ввиду слишком узкой направленности или отсутствии технической возможности автоматизировать принятие решений (как правило, в исследовательских задачах). В таких случаях за него отвечает лицо, принимающее решение (\glsxtrshort{ЛПР}). При разработке универсального программного комплекса, позволяющего решать различные задачи проектирования и оптимизации было бы полезно включить возможность \glsxtrshort{ЛПР} взаимодействовать с промежуточными результатами вычислений. Помимо прочего, подобная необходимость возникает, когда:
\begin{enumerate}
    \item нет формально определённых критериев отбора, на основе которых его можно было бы автоматизировать;
    \item критериев анализа результатов слишком много для того, чтобы реализовать автоматизированную процедуру для его проведения в пределах исследовательской работы;
\end{enumerate}

Поскольку в данной работе рассматривается, в первую очередь, система, реализующая графоориентированный подход к решению сложных вычислительных задач, то целесообразно рассмотреть подходы к организации взаимодействия пользователя с процессом решения. На основании изложенного выше были выделены следующие сценарии взаимодействия с пользователем в данной системе:
\begin{itemize}
    \item Введение дополнительных данных, которые требуются на дальнейших этапах расчётов, но которые не были получены автоматически до этого;
    \item Выбор конкретных данных из некоторого однородного набора для его сужения;
    \item Выбор дальнейшей логики выполнения расчётов на основании полученных на текущем этапе результатов.
\end{itemize}
Помимо этого, для эффективной работы с подобной системой исследователю необходим графический пользовательский интерфейс, в котором модель организации вычислений может быть представлена визуально. Большие перспективы в автоматизации процесса решения сложных задач перед исследователем открывает возможность прямого взаимодействия с вычислительной моделью: остановка вычислительного процесса на определенном этапе, изучение обрабатываемых данных, просмотр истории изменения обрабатываемых данных, возврат к определенному этапу вычислений, ввод
дополнительных параметров на определенной
стадии вычислений и т. д.\cite{SokolovCADCMInteraction2021}
Кроме того, целесообразно рассмотреть основные понятия, вводимые в данной системе.
\begin{itemize}
    \item \emph{Состояние данных} - некоторый набор данных, в котором они хранятся тройками вида "тип - имя - значение". В GBSE реализован в виде специального класса с названием \textsf{Anymap}.
    \item \emph{Функция-обработчик} - функция, которая вызывается при переходе из одного состояния данных в другое. Фактически данная функция каким-то образом модифицирует объект состояния данных.
    \item \emph{Функция-предикат} - функция, связанная с тем же переходом, что и некоторая функция-обработчик, проверяющая соответствие входных данных тому формату, в котором они ожидаются на входе обработчика.
\end{itemize}
На концептуальном уровне абстракции в рассматриваемой системе получение каких-то данных или решений от пользователя может быть реализовано, как и любой другой процесс модификации данных, через соответствующие функции-обработчики и предикаты. Рассмотрим возможные подходы к реализации сценариев взаимодействия пользователя, описанных выше.

Для введения дополнительных данных в GBSE реализован специальный инструмент автоматической генерации графических программных интерфейсов (\glsxtrshort{GUIen}) с формами ввода необходимых данных\cite{SokolovPershin2017}, однако этот инструмент ещё не связан с основным графоориентированным программным каркасом. Средства, реализующие два других сценария на момент написания данной работы находятся в разработке. Для предоставления пользователю возможности сделать выбор относительно дальнейшей обработки данных необходимо разработать следующие средства:
\begin{enumerate}
    \item Средство визуализации текущего состояния данных
    \item Средство содержательной интерпретации текущего состояния данных
\end{enumerate}
Кроме того, для реализации рассматриваемых сценариев будет необходимо доработать средство генерации форм ввода и реализовать в нём дополнительную категорию форм, направленных не на внесение новых данных, а на выбор, в том числе и множественный, из представленных вариантов, отображённых, в том числе, и с помощью средства визуализации.
%----------------------------------------------------------
\include{chapters/chap2_comparison}
%----------------------------------------------------------
%----------------------------------------------------------
%----------------------------------------------------------
%----------------------------------------------------------
%=============================================================

%----------------------------------------------------------
\backmatter %% Здесь заканчивается нумерованная часть документа и начинаются Заключение и Список использованных источников
%----------------------------------------------------------
%----------------------------------------------------------
\chapter*{ЗАКЛЮЧЕНИЕ}\label{chap_conclusion}
\addcontentsline{toc}{chapter}{ЗАКЛЮЧЕНИЕ}
%----------------------------------------------------------

В результате выполнения данной работы были собраны сведения о необходимых для реализации средствах взаимодействия пользователя в системе автматизированного решения исследовательских задач GBSE, что даёт направление для дальнейшей разработки и программной реализации данных средств. Кроме того, были рассмотрены некоторые существующие на рынке аналоги GBSE, в частности, продукт pSeven, и была проведена сравнительная характеристика данных программных комплексов с учётом возможностей взаимодействия пользователя с процессом решения задач.

%----------------------------------------------------------

%----------------------------------------------------------
% Список литературы
\bibliographystyle{utf8gost705u}
\addcontentsline{toc}{chapter}{СПИСОК ИСПОЛЬЗОВАННЫХ ИСТОЧНИКОВ}
\bibliography{bibliography}
%----------------------------------------------------------
\subsubsection*{Выходные данные}
%----------------------------------------------------------
\textit{\DocOutReference}
%----------------------------------------------------------
% Атрибуты задачи
% \docattributes{}{}{}{}{студент группы \group, \Author}{\Year, \Semestr}
%----------------------------------------------------------
% Акт об отсутствии заимствования (включается только для ВКР, для НИРС и КП не нужно)
% Вставится только при сборке ВКР
\myconditionaltext{\doctypesid}{vkr}{%
    %----------------------------------------------------------------
% В этот документ следует вставить акт об отсутствии заимствования
% Для КП, НИРС этот документ не включается.
% Предварительно документ следует подготовить в MS Word, подписать, преобразовать в PDF и далее разместить в нужном каталоге
\newpage
{\catcode`\_=11
\includepdf{doc-additional/cpxsln_vkr_20YY_ShortTitle_group_SurnameNF_plagiarism.pdf}
}
%----------------------------------------------------------------

}
%----------------------------------------------------------
%приложения
%
% листы A1
% Акт об отсутствии заимствований
% Рецензии
%

\appendix

% команды далее необходимы для того, чтобы нумерация элементов текста в приложениях была корректной.
\renewcommand{\theequation}{\thechapter.\arabic{equation}}
\renewcommand{\thefigure}{\thechapter.\arabic{figure}}
\renewcommand{\thetable}{\thechapter.\arabic{table}}
% -------
\renewcommand{\appendixname}{ПРИЛОЖЕНИЯ}
\def\chaptername{ПРИЛОЖЕНИЕ}
\def\thechapter{\Asbuk{chapter}\unskip}
\renewcommand{\thesection}{\thechapter.\arabic{section}\unskip}
%%%%%%%%%%%%%%%%%%%%%%%%%%%%%%%%%%%%%%%%%%%%%%%%%%%%%%%%%%%%%%%%%%%%%%%%
\addcontentsline{toc}{chapter}{ПРИЛОЖЕНИЯ}
%%%%%%%%%%%%%%%%%%%%%%%%%%%%%%%%%%%%%%%%%%%%%%%%%%%%%%%%%%%%%%%%%%%%%%%%
% ПРИЛОЖЕНИЕ
\fancyhead[C]{\thepage \\ \textbf{\leftmark}}
\fancyfoot[C]{}
%----------------------------------------------------------------
\includepdfset{turn=true,scale=0.85,linktodoc=true,pages=-,pagecommand={\pagestyle{fancy}}}
%----------------------------------------------------------------
\chapter{Листы A1}\label{apx_a1}

{\catcode`\_=11
\newpage
\includepdf[pagecommand=\label{appx_A1_list1},pages=-]{appendices/appx_A1_list1.pdf}
}

%{\catcode`\_=11
%\newpage
%\includepdf[pagecommand=\label{appx_A1_list1},pages=-]{appendices/appx_A1_list1.pdf}
%}
%----------------------------------------------------------------
\chapter{Акты и рецензии}\label{apx_acts_reviews}

{\catcode`\_=11
\newpage
\includepdf[pagecommand=\label{appx_act_plagiat},pages=-]{appendices/appx_act_plagiatpdf}
}

{\catcode`\_=11
\newpage
\includepdf[pagecommand=\label{appx_review_1},pages=-]{appendices/appx_review_1.pdf}
}

%%%%%%%%%%%%%%%%%%%%%%%%%%%%%%%%%%%%%%%%%%%%%%%%%%%%%%%%%%%%%%%%%%%%%%%%




%----------------------------------------------------------
% метка нужна для отслеживания общего числа страниц в документе
\label{lastpage}
%----------------------------------------------------------
\end{document}
%==========================================================



