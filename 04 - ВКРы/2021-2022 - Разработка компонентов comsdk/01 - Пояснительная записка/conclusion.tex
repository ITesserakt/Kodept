%----------------------------------------------------------
\chapter*{ЗАКЛЮЧЕНИЕ}\label{chap_conclusion}
\addcontentsline{toc}{chapter}{ЗАКЛЮЧЕНИЕ}
%----------------------------------------------------------
Таким образом, в ходе выполнения данной работы были выполнены следующие задачи:
\begin{enumerate}[1)]
    \item в ходе аналитического обзора литературы и сравненительного анализа разработки с аналогичной обоснована её актуальность;
    \item обозначены теоретические основы подхода \glsxtrshort{GBSE} к построению описаний алгоритмов;
    \item сформулированы требования к алгоритму интерпретации описаний, построенных по методологии GBSE;
    \item спроектированы структуры данных, отвечающие за программное представление описаний алгоритмов и их элементов в программном каркасе comsdk;
    \item спроектированные структуры данных были реализованы на языке С++.
\end{enumerate}

В результате разработки:
\begin{enumerate}[1)]
    \item была повышена безопасность существующих элементов программного каркаса comsdk;
    \item программная реализация графовых моделей была приближена к их концептуальному определению в методологии GBSE;
    \item были разработаны интерфейсы, позволяющие дальнейшее расширение функциональных возможностей программного каркаса;
    \item были расширены возможности документирования реализуемых при помощи программного каркаса comsdk алгоритмов.
\end{enumerate}

Перспективы дальнейшей разработки включают в себя, помимо прочего, следующие:
\begin{itemize}
    \item повышение надёжности и быстродействия разработанных средств;
    \item реализация алгоритма параллельного обхода для различных вариантов задействованных вычислительных ресурсов;
    \item разработка и интеграция средств взаимодействия с пользователем;
    \item интеграция разработанных программных средств в распределённую вычислительную систему GCD.
\end{itemize}
%----------------------------------------------------------
