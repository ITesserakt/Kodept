%----------------------------------------%
% общие определения
\newcommand{\UpperFullOrganisationName}{Министерство науки и высшего образования Российской Федерации}
\newcommand{\ShortOrganisationName}{МГТУ~им.~Н.Э.~Баумана}
\newcommand{\FullOrganisationName}{федеральное государственное бюджетное образовательное\newline учреждение высшего профессионального образования\newline <<Московский государственный технический университет имени Н.Э.~Баумана\newline (национальный исследовательский университет)>> (\ShortOrganisationName)}
\newcommand{\OrganisationAddress}{105005, Россия, Москва, ул.~2-ая Бауманская, д.~5, стр.~1}
%----------------------------------------%
\newcommand{\gitlabdomain}{sa2systems.ru:88}
%----------------------------------------------------------
\newcommand{\doctypesid}{vkr} % vkr (выпускная квалификационная работа) / kp (курсовой проект) / kr (курсовая работа) / nirs (научно-исследовательская работа студента) / nkr (научно-квалификационная работа)

% Тема должна быть сформулирована так, чтобы рассказать, о чем работа, но сделать это так, чтобы у читателя возникло желание читать аннота-цию. При формулировке темы не следует стараться рассказать о работе всё. Пример корректной темы: "Математическое моделирование процесса размножения медуз в Южно-Китайском море". Пример некорректной темы: "Применение модели SIS для моделирования процесса размножения медуз в Южно-Китайском море с использованием метода Рунге-Кутты и многопроцессорных вычислительных систем".
\newcommand{\Title}{Разработка компонентов графоориентированного программного каркаса для реализации сложных вычислительных методов}%{}
\newcommand{\TitleSource}{НИР кафедры} % кафедра, предприятие, НИР, НИР кафедры, заказ организации

\newcommand{\SubTitle}{по дисциплине <<Модели и методы анализа проектных решений>>} % Методы оптимизации
\newcommand{\faculty}{<<Робототехника и комплексная автоматизация>>}
\newcommand{\facultyShort}{РК}
\newcommand{\department}{<<Системы автоматизированного проектирования (РК-6)>>}
\newcommand{\departmentShort}{РК-6}

\newcommand{\Author}{Тришин~И.В.}
\newcommand{\AuthorFull}{Тришин~Илья~Вадимович}
\newcommand{\ScientificAdviserPosition}{канд.~физ.-мат.~наук}	% Должность научного руководителя
\newcommand{\ScientificAdviser}{Соколов~А.П.}	% Научный руководитель
\newcommand{\ConsultantA}{Першин~А.Ю.}				% Консультант 1
\newcommand{\group}{РК6-81Б}
\newcommand{\Semestr}{весенний семестр} % Например: осенний семестр или весенний семестр
\newcommand{\Normocontroller}{Грошев~C.В.}		% Нормоконтролёр
\newcommand{\BeginYear}{2022}
\newcommand{\Year}{2022}
\newcommand{\Country}{Россия}
\newcommand{\City}{Москва}
\newcommand{\TaskStatementDate}{<<\underline{\textit{08}}>> \underline{февраля} \Year~г.} %Дата выдачи задания 

\newcommand{\depHeadPosition}{Заведующий кафедрой}		% Должность руководителя подразделения
\newcommand{\depHeadName}{А.П.~Карпенко}		% Должность руководителя подразделения

% Цель выполнения 
\newcommand{\GoalOfResearch}{создание новых программных средств для описания и представления графовых моделей сложных вычислительных методов и их обхода} % с маленькой буквы и без точки на конце

% Объектом исследования называют то, что исследуется в работе. Напри-мер, для указанной выше темы объектом может быть популяция медуз, но никак ни модель SIS, ни Южно-Китайское море, ни метод моделирования популяции медуз. 
\newcommand{\ObjectOfResearch}{методы описания бизнес-логики в системах автоматизированной разработки программного обеспечения}

% Предмет исследований (уже чем объект, определяется, исходя из задач: формулируется как существительное, как правило, во множественном числе, определяющее "конкретный объект исследований" среди прочих в рамках более общего)
\newcommand{\SubjectOfResearch}{@Предмет исследований@}

% Основная задача, на решение которой направлена работа
\newcommand{\MainProblemOfResearch}{@Основная задача, на решение которой направлена работа@}

% Выполненные задачи
\newcommand{\SubtasksPerformed}{%
	В результате выполнения работы:
	\begin{inparaenum}[1)]
		\item в ходе аналитического обзора литературы и сравненительного анализа разработки с аналогичной обоснована её актуальность;
		\item обозначены теоретические основы подхода \glsxtrshort{GBSE} к построению описаний алгоритмов;
		\item сформулированы требования к алгоритму интерпретации описаний, построенных по методологии GBSE;
		\item спроектированы структуры данных, отвечающие за программное представление описаний алгоритмов и их элементов в программном каркасе comsdk;
		\item спроектированные структуры данных были реализованы на языке С++.
	\end{inparaenum}}

% Ключевые слова (представляются для обеспечения потенциальной возможности индексации документа)
\newcommand{\keywordsru}{%
	CASE-системы, распределённые вычисления, графоориентированный подход, сложные вычислительные методы, описание бизнес-логики} % 5-15 слов или выражений на русском языке, для разделения СЛЕДУЕТ ИСПОЛЬЗОВАТЬ ЗАПЯТЫЕ
\newcommand{\keywordsen}{%
	CASE systems, distributed computing, graph-based software engineering, computational mathematics, algorithm specification} % 5-15 слов или выражений на английском языке, для разделения СЛЕДУЕТ ИСПОЛЬЗОВАТЬ ЗАПЯТЫЕ

% Краткая аннотация
\newcommand{\Preface}{Данная работа посвящена разработке программного инструментария, позволяющего описывать и реализовать логику решения различных задач, требующих большого количества трудоёмких вычислений. При описании применяется т.н. графоориентированный подход, который позволяет пользователю задавать действия алгоритма или вычислительного метода в виде переходов между состояниями данных. Формируемое описание затем интерпретируется и выполняется с примененим стандартных или пользовательских реализаций каждого из переходов.

	Реализованные программные средства позволяют структурировать и ускорить разработку наукоёмкого программного обеспечения, применяемого при анализе больших объёмов данных и научно-технических исследованиях.} % с большой буквы с точкой в конце

%----------------------------------------%
% выходные данные по документу
\newcommand{\DocOutReference}{\Author. \Title\xspace\SubTitle. [Электронный ресурс] --- \City: \Year. --- \total{page} с. URL:~\url{https://\gitlabdomain} (система контроля версий кафедры РК6)}

%----------------------------------------------------------

