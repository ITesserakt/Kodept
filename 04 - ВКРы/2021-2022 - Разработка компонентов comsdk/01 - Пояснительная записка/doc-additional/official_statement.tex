%-------------------------
\newpage
%-------------------------
\officialheaderfull[]{ЗАДАНИЕ}{на выполнение \doctypec}
%-------------------------

\noindent Источник тематики (кафедра, предприятие, НИР): \underline{\TitleSource}

\myconditionaltext{\doctypesid}{vkr}{%
	\noindent Тема \doctypec\xspace утверждена распоряжением по факультету \facultyShort~№~\uline{\textcolor{white}{\hspace{40pt}}} от \datetofill
}

\myconditionaltext{\doctypesid}{kp}{%
	\noindent Тема \doctypec\xspace утверждена на заседании кафедры \department, Протокол~№~\uline{\textcolor{white}{\hspace{40pt}}} от \datetofill
}

\noindent \textbf{Техническое задание}

\noindent \textbf{Часть 1.} \textit{Аналитический обзор литературы.\\
	\uline{В рамках аналитического обзора должны быть рассмотрены различные подходы, направленные на упрощение реализации сложных вычислительных методов. Должно быть проведено сравнение разрабатываемой системы с некоторой аналогичной ей (на усмотрение студента)}}

\noindent \textbf{Часть 2.} \textit{Разработка архитектуры программной реализации, программная реализация.\\
	\uline{Должны быть спроектированы программные средства для описания и выполнения обхода графовых моделей сложных вычислительных методов, созданных по методологии GBSE}}

\noindent \textbf{Часть 3.} \textit{Проведение вычислительных экспериментов, тестирование.\\
	\uline{Более подробная формулировка задания. Должна быть представлена некоторая конкретизация: какие вычислительные эксперименты требовалось реализовать, какие тесты требовалось провести для проверки работоспособности разработанных программных решений. Формулировка задания должна включать некоторую конкретику, например: какими средствами требовалось пользоваться для проведения расчетов и/или вычислительных эксперименто. Например: <<Вычислительные эксперименты должны быть проведены с использованием разработанного в рамках ВКР программного обеспечения>>.}}

%\vspace{0.3cm}

\noindent \textbf{Оформление \doctypec:}

\noindent Расчетно-пояснительная записка на \total{page} листах формата А4.

\noindent Перечень графического (иллюстративного) материала (чертежи, плакаты, слайды и т.п.):

\noindent\begin{tabular}{|p{0.95\textwidth}|}
	\hline
	\textit{количество: \total{ffigure}~рис., \total{ttable}~табл., \total{bibcnt}~источн.} \\
	\hline
	\textit{[здесь следует ввести количество чертежей, плакатов]}                           \\
	\hline
	\\
	\hline
\end{tabular}

\noindent Дата выдачи задания \TaskStatementDate\\

\vspace{-15pt}
\noindent \begin{tabular}{p{0.5\textwidth}>{\raggedleft}p{0.2\textwidth}p{0.01\textwidth}P{0.2\textwidth}}
	\signerline{\textbf{Студент}}{\Author}                           \\[5pt]
	\signerline{\textbf{Руководитель \doctypec}}{\ScientificAdviser} \\
\end{tabular}

\vspace{2pt}
\noindent {\smaller[1] Примечание: Задание оформляется в двух экземплярах: один выдается студенту, второй хранится на кафедре.}
