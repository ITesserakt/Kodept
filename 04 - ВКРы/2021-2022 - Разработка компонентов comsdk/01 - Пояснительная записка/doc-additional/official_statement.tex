%-------------------------
\newpage
%-------------------------
\officialheaderfull[]{ЗАДАНИЕ}{на выполнение \doctypec}
%-------------------------

\noindent Источник тематики (кафедра, предприятие, НИР): \underline{\TitleSource}

\myconditionaltext{\doctypesid}{vkr}{%
	\noindent Тема \doctypec\xspace утверждена распоряжением по факультету \facultyShort~№~\uline{\textcolor{white}{\hspace{40pt}}} от \datetofill
}

\myconditionaltext{\doctypesid}{kp}{%
	\noindent Тема \doctypec\xspace утверждена на заседании кафедры \department, Протокол~№~\uline{\textcolor{white}{\hspace{40pt}}} от \datetofill
}

\noindent \textbf{Техническое задание}

\noindent \textbf{Часть 1.} \textit{Аналитический обзор литературы.\\
	\uline{В рамках аналитического обзора должны быть рассмотрены различные методы упрощения реализации сложных вычислительных методов. Должны быть изучены теоретические основы графоориентированного подхода. Должно быть проведено сравнение текущей версии разразрабатываемой системы с некоторой аналогичной ей (на усмотрение студента).}}

\noindent \textbf{Часть 2.} \textit{Разработка архитектуры программной реализации.\\
	\uline{Должны быть спроектированы программные средства для описания и выполнения обхода графовых моделей сложных вычислительных методов, созданных по графооритентированной методологии}}

\noindent \textbf{Часть 3.} \textit{Программная реализация, тестирование.\\
	\uline{Спроектированные программные средства должны быть реализованы на языке С++ в рамках программного каркаса comsdk.}}

%\vspace{0.3cm}

\noindent \textbf{Оформление \doctypec:}

\noindent Расчетно-пояснительная записка на \total{page} листах формата А4.

\noindent Перечень графического (иллюстративного) материала (чертежи, плакаты, слайды и т.п.):

\noindent\begin{tabular}{|p{0.95\textwidth}|}
	\hline
	\textit{количество: \total{ffigure}~рис., 1~табл., \total{bibcnt}~источн., 5 графических листов} \\
	\hline
	\\
	\hline
	\\
	\hline
\end{tabular}

\noindent Дата выдачи задания \TaskStatementDate\\

\noindent В соответствии с учебным планом выпускную квалификационную работу выполнить в полном объёме в срок до <<\underline{\textit{09}}>>~\underline{июня}~\Year~г.

\vspace{30pt}

\noindent \begin{tabular}{p{0.5\textwidth}>{\raggedleft}p{0.2\textwidth}p{0.01\textwidth}P{0.2\textwidth}}
	\signerline{\textbf{Студент}}{\Author}                           \\[5pt]
	\signerline{\textbf{Руководитель \doctypec}}{\ScientificAdviser} \\
\end{tabular}

\vspace{10pt}

\noindent {\smaller[1] Примечание: Задание оформляется в двух экземплярах: один выдается студенту, второй хранится на кафедре.}
