\section*{Элементы абстрактного синтаксического дерева}

\begin{figure}
    \centering
    \input{figures/.generated/ast_nodes}
    \caption{UML-диаграмма }
    \label{fig:ast_nodes}
\end{figure}

%\begin{lstlisting}[label=lst:nodes, caption={Имена всех вершин, составляющих абстрактное синтаксическое дерево}, language=C]
%    File, // root element
%    Module, // module {name} {rest}
%    Struct, // struct {name}({params}) {rest}
%    Enum, // enum {name} {rest}
%    TypedParameter, // {name}: {type}
%    UntypedParameter, // {name}
%    Variable, // val {name}: {type}
%    InitializedVar, // val {name}: {type} = {expr}
%    BodiedFunction, // fun {name}({params}) => {expr}
%    ExpressionBlock, // {  {expr1}; {expr2}; ... }
%    Application, // {expr}({expr})
%    Lambda, // \textbackslash {binds} => {expr}
%    Reference, // {name}
%    Access, // {expr}.{expr}
%    Number, // number literal
%    Char, // char literal
%    String, // string literal
%    Tuple, // ({expr1}, {expr2}, ...)
%    If, // if {expr} => {expr} {другие ветки}
%    Elif, // elif {expr} => {expr}
%    Else, // else {expr}
%    Binary, // binary operator: +, -, *, /, \%, \textasciicircum
%    Unary, // unary operator: -, +, !, \texttildelow
%    AbstractFunction, // abstract fun {name}({params}): {type}
%    ProdType, // ({type1}, {type2}, ...)
%\end{lstlisting}