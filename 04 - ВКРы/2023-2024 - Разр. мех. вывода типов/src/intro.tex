%----------------------------------------------------------
\chapter*{ВВЕДЕНИЕ}\label{ch:introduction}
\addcontentsline{toc}{chapter}{ВВЕДЕНИЕ}

%-------------------------------------------------------

Языки программирования, использующие функциональную парадигму программирования, названы функциональными.
В отличие от императивных языков, которые трактуют программу как последовательность инструкций, большинству функциональных языков присуща декларативная парадигма.
В ней программа представляет собой процесс вычисления функций в математическом стиле.
Функциональные языки программирования имеют много преимуществ~\cite{WhyFunctional},
и применяются в таких областях, как обработка больших данных, математическое моделирование и прочих~\cite{ScalaBigData}.

Актуальна проблема непростого освоения функциональных языков программирования, в том числе из-за использования непривычного и местами сложного синтаксиса~\cite{HaskellBook}.
Для сравнения на листингах~\ref{lst:haskell_syntax},~\ref{lst:c_syntax} приведены примеры кода функции факториала числа на функциональном и императивном языках.
В качестве попытки решить эту проблему, появилась идея создать собственный язык программирования, который имел бы привычный Си-подобный синтаксис, но использовал идеи из функциональных языков.

\begin{lstlisting}[label={lst:haskell_syntax}, language=Haskell, caption={Пример функции факториала числа в языке Haskell}]
factorial :: Num a => a => a
factorial n = take n . product [1..]
\end{lstlisting}

\begin{lstlisting}[label={lst:c_syntax}, caption={Пример функции факториала числа в языке C}, language=C]
int factorial(int n) {
    if (n < 1) return 1;
    int product = 1;
    for (int i = 2; i <= n; i++ {
        product *= i;
    }
    return product;
}
\end{lstlisting}

В процессе обучения на кафедре был разработан язык программирования Kodépt, вместе с этим велась разработка прототипа компилятора для него.
На момент написания работы в компиляторе были разработаны модули для чтения и разбора файлов на языке Kodept, проведена работа по составлению внутренней архитектуры программы.

Компилятор~--- программа, позволяющая из текста на исходном языке, получить выходной файл, понятный исполняющей машине.
При этом выделяют трансляторы~--- программы, преобразующие тексты на одном языке программирования в другой, и интерпретаторы, которые сразу выполняют исходный код программы без предобработки.

Рассмотрим проблематику текущего состояния языка.
Использование языков программирования без системы типов затруднительно~\cite{Typing}.
Разработчику необходимо вручную следить за правильностью использования конструкций, а компилятор (интерпретатор) не может оптимизировать код.
Введение системы типов в язык программирования Kodept является естественным продолжением развития языка.
Система типов является математическим описанием того, как работают взаимоотношения между типами данных в программе.
Механизм вывода типов является непосредственной реализацией системы типов в компиляторе (интерпретаторе).

\textbf{Целью} работы является совершенствование языка программирования Kodept с помощью введения системы типов с последующей реализацией механизма вывода типов в прототипе компилятора языка Kodept.

Для достижения поставленной цели определены следующие задачи:

\begin{enumerate}[1)]
    \item провести сравнительный анализ систем типов в некоторых современных языках программирования,
    \item выбрать систему типов для языка программирования Kodept,
    \item описать алгоритм для вывода типов на основе выбранной системы типов,
    \item расширить прототип компилятора языка Kodept с помощью реализации механизма вывода типов,
    \item провести тестирование для проверки правильности реализации.
\end{enumerate}

\todo{
    Улучшить качество новых языков программирования, и применить ее на примере Kodept.
    Актуальность: статическая типизация в языках программирования важна.
    Проблема: уменьшение багов.
}

%----------------------------------------------------------
