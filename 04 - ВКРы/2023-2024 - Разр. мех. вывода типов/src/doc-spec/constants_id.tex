%----------------------------------------%
% общие определения
\newcommand{\UpperFullOrganisationName}{\textbf{Министерство науки и высшего образования Российской Федерации}}
\newcommand{\ShortOrganisationName}{МГТУ~им.~Н.Э.~Баумана}
\newcommand{\FullOrganisationName}{\textbf{Федеральное государственное бюджетное образовательное учреждение\newline высшего образования\newline <<Московский государственный технический университет имени Н.Э.~Баумана\newline (национальный исследовательский университет)>>\newline (\ShortOrganisationName)}}
\newcommand{\OrganisationAddress}{105005, Россия, Москва, ул.~2-ая Бауманская, д.~5, стр.~1}
%----------------------------------------%
\newcommand{\gitlabdomain}{sa2systems.ru:88}
%----------------------------------------------------------
\newcommand{\doctypesid}{vkr} % vkr (выпускная квалификационная работа) / kp (курсовой проект) / kr (курсовая работа) / nirs (научно-исследовательская работа студента) / nkr (научно-квалификационная работа)

% Тема должна быть сформулирована так, чтобы рассказать, о чем работа, но сделать это так, чтобы у читателя возникло желание читать аннота-цию. При формулировке темы не следует стараться рассказать о работе всё. Пример корректной темы: "Математическое моделирование процесса размножения медуз в Южно-Китайском море". Пример некорректной темы: "Применение модели SIS для моделирования процесса размножения медуз в Южно-Китайском море с использованием метода Рунге-Кутты и многопроцессорных вычислительных систем".
\newcommand{\Title}{Разработка механизма вывода типов с использованием системы типов Хиндли-Милнера}%{}
\newcommand{\TitleSource}{кафедра} % кафедра, предприятие, НИР, НИР кафедры, заказ организации

\newcommand{\SubTitle}{по дисциплине <<Модели и методы анализа проектных решений>>} % Методы оптимизации
\newcommand{\faculty}{<<Робототехника и комплексная автоматизация>>}
\newcommand{\facultyShort}{РК}
\newcommand{\department}{<<Системы автоматизированного проектирования (РК-6)>>}
\newcommand{\departmentShort}{РК-6}

\newcommand{\Author}{Никитин В.Л.}
\newcommand{\AuthorFull}{Никитин Владимир Леонидович}
\newcommand{\ScientificAdviserPosition}{кандидат физико-математических наук}    % Должность научного руководителя
\newcommand{\ScientificAdviser}{Соколов А.П.}    % Научный руководитель
%\newcommand{\ConsultantA}{Соколов А.П.}                % Консультант 1
% \newcommand{\ConsultantB}{@Фамилия~И.О.@}				% Консультант 2
\newcommand{\Normocontroller}{Грошев~С.В.}        % Нормоконтролёр
\newcommand{\group}{РК6-85Б}
\newcommand{\Semestr}{весенний семестр} % Например: осенний семестр или весенний семестр
\newcommand{\BeginYear}{2023}
\newcommand{\Year}{2024}
\newcommand{\Country}{Россия}
\newcommand{\City}{Москва}
\newcommand{\TaskStatementDate}{<<\underline{\textit{08}}>> \underline{февраля} \Year~г.} %Дата выдачи задания

\newcommand{\depHeadPosition}{Заведующий кафедрой}        % Должность руководителя подразделения
\newcommand{\depHeadName}{А.П.~Карпенко}        % Должность руководителя подразделения

% Цель выполнения 
\newcommand{\GoalOfResearch}{развитие языка программирования Kodept посредством введения системы типов} % с маленькой буквы и без точки на конце

% Объектом исследования называют то, что исследуется в работе. Например, для указанной выше темы объектом может быть популяция медуз, но никак ни модель SIS, ни Южно-Китайское море, ни метод моделирования популяции медуз. 
\newcommand{\ObjectOfResearch}{система типов}

% Предмет исследований (уже чем объект, определяется, исходя из задач: формулируется как существительное, как правило, во множественном числе, определяющее "конкретный объект исследований" среди прочих в рамках более общего)
\newcommand{\SubjectOfResearch}{система типов Хиндли-Милнера}

% Основная задача, на решение которой направлена работа
\newcommand{\MainProblemOfResearch}{реализация механизма вывода типов на основе выбранной системы типов в прототипе компилятора языка программирования Kodept}

% Выполненные задачи
\newcommand{\SubtasksPerformed}{%
    В результате выполнения работы:
    \begin{inparaenum}[1)]
        \item проведён сравнительный анализ некоторых систем типов;
        \item рассмотрена система типов, которая будет применена в языке Kodept;
        \item в прототип компилятора языка Kodept добавлен механизм вывода типов на основе рассмотренной системы типов,
        \item показано, что прототип компилятора Kodept успешно выводит типы для исходного кода.
    \end{inparaenum}}

% Ключевые слова (представляются для обеспечения потенциальной возможности индексации документа)
\newcommand{\keywordsru}{%
    теория типов, языки программирования, компиляторы, фукнциональное программирование, система типов Хиндли-Милнера} % 5-15 слов или выражений на русском языке, для разделения СЛЕДУЕТ ИСПОЛЬЗОВАТЬ ЗАПЯТЫЕ
\newcommand{\keywordsen}{%
    type theory, programming languages, compilers, functional programming, Hindley-Milner type system} % 5-15 слов или выражений на английском языке, для разделения СЛЕДУЕТ ИСПОЛЬЗОВАТЬ ЗАПЯТЫЕ

% Краткая аннотация
\newcommand{\Preface}{
	Работа посвящена реализации механизма вывода типов для языка программирования Kodept.
	Программирование выстроено вокруг глубокой математической теории.
	Благодаря этому появляются возможности для оптимизации, развития и улучшения языков посредством применения математики.
	Одним из важных применений является теория типов, которая помогает программисту в написании кода.
	Применение мощной системы типов позволяет зачастую снизить количество ошибок, возникающих при разработке.
} % с большой буквы с точкой в конце

%----------------------------------------%
% выходные данные по документу
\newcommand{\DocOutReference}{\Author. \Title\xspace\SubTitle. [Электронный ресурс] --- \City: \Year. --- \total{page} с. URL:~\url{https://\gitlabdomain} (система контроля версий кафедры РК6)}

%----------------------------------------------------------

