%-------------------------
\newpage
%-------------------------
\officialheaderfull[]{ЗАДАНИЕ}{на выполнение \doctypec}
%-------------------------

\noindent Источник тематики (кафедра, предприятие, НИР): \underline{\TitleSource}

\noindent При выполнении ВКР:

\vspace{0.2cm}

{\small
    \noindent\begin{tabular}{|p{0.817\textwidth}|p{0.133\textwidth}|}
              \hline
              \centering Используются / Не используются                                                                                             & Да / Нет \\\hline
              1)~Литературные источники и документы, имеющие гриф секретности                                                                       & Нет      \\\hline
              2)~Литературные источники и документы, имеющие пометку «Для служебного пользования», иных пометок, запрещающих открытое опубликование & Нет \\\hline
              3)~Служебные материалы других организаций                                                                                             & Нет      \\\hline
              4)~Результаты НИР (ОКР), выполняемой в МГТУ им. Н.Э.Баумана                                                                           & Нет      \\\hline
              5)~Материалы по незавершенным исследованиям или материалы по завершенным исследованиям, но ещё не опубликованные в открытой печати & Нет \\\hline
\end{tabular}}

\vspace{0.2cm}

\myconditionaltext{\doctypesid}{vkr}{%
    \noindent Тема \doctypec\xspace утверждена распоряжением по факультету \facultyShort~№~\uline{\textcolor{white}{\hspace{40pt}}} от \datetofill
}

\myconditionaltext{\doctypesid}{kp}{%
    \noindent Тема \doctypec\xspace утверждена на заседании кафедры \department, Протокол~№~\uline{\textcolor{white}{\hspace{40pt}}} от \datetofill
}

\noindent \textbf{Техническое задание}

\noindent \textbf{Часть 1.} \textit{Анализ актуальности.\\
\uline{Должен быть выполнен анализ существующих система систем типов в современных языках программирования. Также должна быть обоснована актуальность разработки в целом.}}

\noindent \textbf{Часть 2.} \textit{Математическая постановка задачи, разработка архитектуры программной реализации, программная реализация.\\
\uline{Должна быть описана система типов и реализована в языке программирования Kodept в качестве механизма вывода типов}}

\noindent \textbf{Часть 3.} \textit{Проведение тестирования.\\
\uline{Должно быть проведено тестирование разработанной в ходе работы над курсовым проектом программы.}}

%\vspace{0.3cm}

\noindent \textbf{Оформление \doctypec:}

\noindent Расчетно-пояснительная записка на \total{page} листах формата А4.

\noindent Перечень графического (иллюстративного) материала (чертежи, плакаты, слайды и т.п.):

\noindent\begin{tabular}{|p{0.95\textwidth}|}
             \hline
             \textit{количество: \total{ffigure}~рис., \total{ttable}~табл., \total{bibcnt}~источн.} \\
             \hline
\end{tabular}

\noindent Дата выдачи задания \TaskStatementDate\\

\myconditionaltext{\doctypesid}{vkr}{%
    \noindent В соответствии с учебным планом выпускную квалификационную работу выполнить в полном объёме в срок до \datetofill}

%\vspace{-15pt}
\noindent \begin{tabular}{p{0.5\textwidth}>{\raggedleft}p{0.2\textwidth}p{0.01\textwidth}P{0.2\textwidth}}
              \signerline{\textbf{Студент}}{\Author} \\[5pt]
              \signerline{\textbf{Руководитель ВКР}}{\ScientificAdviser} \\
\end{tabular}

\vspace{2pt}
\noindent {\smaller[1] Примечание. Задание оформляется в двух экземплярах: один выдается студенту, второй хранится на кафедре.}
