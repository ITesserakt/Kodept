%----------------------------------------------------------
\chapter*{ЗАКЛЮЧЕНИЕ}
\label{ch:conclusion}
\addcontentsline{toc}{chapter}{ЗАКЛЮЧЕНИЕ}
%----------------------------------------------------------

В результате выполнения выпускной квалификационной работы был реализован механизм вывода типов для языка программирования Kodept.
Для этого была сформирована система типов на основе системы типов Хиндли-Милнера, а также алгоритм для вывода типов $\mathcal{W}_c$.
Показано, что программная реализация в виде прототипа компилятора может выводить типы в предоставляемых исходных файлах на языке Kodept.

В ходе работы рассмотрена архитектура компилятора, а также методы, применяемые при работе с абстрактным синтаксическим деревом.
Реализованы программные модули, отвечающие как за сам вывод типов, так и за смежные с ним части компилятора.
Успешно решены все поставленные задачи, а именно:
\begin{enumerate}[1)]
    \item проанализированы некоторые системы типов в современных языках программирования,
    \item реализована модификация системы типов Хиндли-Милнера,
    \item рассмотрен алгоритм вывода типов согласно выбранной системы типов,
    \item реализован модуль в компиляторе языка Kodept для семантического анализа,
    \item выполнено тестирование компилятора на правильность вывода типов.
\end{enumerate}

Дальнейшие планы по работе над языком и компилятором Kodept:
\begin{itemize}
    \item дальнейшая доработка и усложнение системы типов,
    \item расширение семантического анализатора,
    \item добавление поддержки многопоточного выполнения компиляции,
    \item реализация следующих стадий компиляции, таких как генерация промежуточного представления и генерация машинного кода.
\end{itemize}

%----------------------------------------------------------
