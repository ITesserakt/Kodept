%----------------------------------------------------------
\newcommand{\doctype}{presentation} % о научно-исследовательской работе / об опытно-конструкторской работе / об опытно-технологической работе / о патентных исследованиях
\newcommand{\doctypesid}{prs} % prs (Презентация) / vkr (выпускная квалификационная работа) / kp (курсовой проект) / kr (курсовая работа) / nirs (научно-исследовательская работа студента) / nkr (научно-квалификационная работа)
\def\titlepagestyle{detail} % brief (краткая) / detail (подробная)
%----------------------------------------%
\newcommand{\Title}{Разработка механизма вывода типов с использованием системы типов\\Хиндли-Милнера}%Научные основы автоматизированного проектирования композиционных материалов
\newcommand{\SubTitle}{Системы типов в языках программирования} % Методы оптимизации
%----------------------------------------------------------
\newcommand{\eventplace}{факультет \facultyShort, кафедра \department} % Место проведения мероприятия
\newcommand{\eventtype}{название события} % Тип мероприятия (лекция|лабораторная работа|семинар|мозговой штурм)
%\newcommand{\eventplace}{@место проведения@} % Место проведения мероприятия
\newcommand{\eventduration}{5 минут} % Продолжительность мероприятия
%----------------------------------------------------------
\newcommand{\Author}{Никитин В.Л.}
\newcommand{\AuthorFull}{Никитин Владимир Леонидович, студент группы РК6-85Б}
\newcommand{\AuthorEmail}{nikitinvl@student.bmstu.ru}
\newcommand{\ScientificAdviserPosition}{кандидат физико-математических наук}	% Должность научного руководителя
\newcommand{\ScientificAdviser}{Соколов А.П.}	% Научный руководитель
\newcommand{\group}{РК6-85Б}
\newcommand{\Semestr}{весенний семестр} % Например: осенний семестр или весенний семестр
\newcommand{\begdate}{08 февраля 2024} % Дата начала разработки
\newcommand{\Year}{2024}
\newcommand{\country}{Россия}	% Страна, в которой проводится конференция
\newcommand{\city}{Москва}		% Город, в котором проводится конференция
%----------------------------------------%
\newcommand{\depHeadPosition}{Заведующий кафедрой}		% Должность руководителя подразделения
\newcommand{\depHeadName}{А.П.~Карпенко}		% Должность руководителя подразделения
%----------------------------------------%
\newcommand{\presentationtitle}{\Title}
\newcommand{\conferenceperiod}{\country, \city, \begdate\ -- \today}
%----------------------------------------%

% Цель исследования/доклада
\newcommand{\GoalOfResearch}{реализация системы вывода и проверки типов} % с маленькой буквы и без точки на конце

% Объектом исследования называют то, что исследуется в работе. Например, для указанной выше темы объектом может быть популяция медуз, но никак ни модель SIS, ни Южно-Китайское море, ни метод моделирования популяции медуз.
\newcommand{\ObjectOfResearch}{система типов}

% Предмет исследований (уже чем объект, определяется, исходя из задач: формулируется как существительное, как правило, во множественном числе, определяющее "конкретный объект исследований" среди прочих в рамках более общего)
\newcommand{\SubjectOfResearch}{система типов Хиндли-Милнера}

% Основная задача, на решение которой направлена работа
\newcommand{\MainProblemOfResearch}{реализация алгоритма вывода типов на основе выбранной системы типов}

% Выполненные задачи
\newcommand{\SubtasksPerformed}{%
	В результате выполнения работы:
	\begin{inparaenum}[1)]
	\item спроектировано представление абстрактного синтаксического дерева в компиляторе;
	\item реализован семантический анализатор;
	\item показано, что компилятор успешно может вывести тип функции
	\end{inparaenum}}

% Ключевые слова (представляются для обеспечения потенциальной возможности индексации документа)
\newcommand{\keywordsru}{%
	теория типов, языки программирования, компиляторы, фукнциональное программирование, система типов Хиндли-Милнера} % 5-15 слов или выражений на русском языке, для разделения СЛЕДУЕТ ИСПОЛЬЗОВАТЬ ЗАПЯТЫЕ
\newcommand{\keywordsen}{%
	type theory, programming languages, compilers, functional programming, Hindley-Milner type system} % 5-15 слов или выражений на английском языке, для разделения СЛЕДУЕТ ИСПОЛЬЗОВАТЬ ЗАПЯТЫЕ

% Краткая аннотация
\newcommand{\Preface}{
	Работа посвящена реализации механизма вывода типов для языка программирования Kodept.
	Программирование выстроено вокруг глубокой математической теории.
	Благодаря этому появляются возможности для оптимизации, развития и улучшения языков посредством применения математики.
	Одним из важных применений является теория типов, которая помогает программисту в написании кода.
	В последнее время все больше и больше языков почерпывают что-то из этой области.
	Применение мощной системы типов позволяет зачастую снизить количество ошибок, возникающих при разработке.
} % с большой буквы с точкой в конце

%----------------------------------------%
% выходные данные по документу
\newcommand{\DocOutReference}{\Author. \Title\xspace\SubTitle. [Электронный ресурс] --- \City: \Year. --- \total{page} с. URL:~\url{https://\gitlabdomain} (система контроля версий кафедры РК6)}

%----------------------------------------------------------
