%%%%%%%%%%%%%%%%%%%%%%%%%%%%%%%%%%%%%%%%%%%
\subsection{Концептуальная постановка задачи}
%%%%%%%%%%%%%%%%%%%%%%%%%%%%%%%%%%%%%%%%%%%
\begin{frame}%[allowframebreaks=0.9,t]

\begin{block}{Объект исследований}
	Система типов.
\end{block}

\begin{block}{Цель исследования}
	Целью курсового проекта является реализация механизма вывода и проверки типов для языка программирования Kodept в качестве его дальнейшего развития.
\end{block}

\begin{block}{Задачи исследования}
\begin{enumerate}
	\item спроектировать представления AST в компиляторе,
	\item реализовать анализатор областей видимости,
	\item написать алгоритм для вывода типов.
\end{enumerate}
\end{block}

%\note{skdjfsk jdsdk jslkfd jsldkf jslkdfjslkdjf slkfdjslkdfjsl kdjfsl kdjf}

\end{frame}
%%%%%%%%%%%%%%%%%%%%%%%%%%%%%%%%%%%%%%%%%%%
\subsection{Математическая постановка задачи. Теория типов.}
%%%%%%%%%%%%%%%%%%%%%%%%%%%%%%%%%%%%%%%%%%%
\begin{frame}%[allowframebreaks=0.9,t]

	\begin{definition}
		Терм $x$ - чаще всего элемент языка программирования, будь то переменная, константа, вызов функции и др.
		Например, в Haskell, термами будут: лямбда-функция \lstinline{\x -> x + 1}, определение переменной \lstinline{let x = 1 in ()} и т.д.
	\end{definition}

	\begin{definition}
		Типом $A$ обозначается метка, приписываемая объектам.
		Обычно каждому терму соответствует определенный тип - $x: A$.
		Типы позволяют строго говорить о возможных действиях над объектом, а также формализовать взаимоотношения между ними.
	\end{definition}

	Система типов определяет правила взаимодействия между типами и термами, используя суждения:

	\[
		\frac{\Gamma \vdash t: T_1, \Delta \vdash T_1 = T_2}{\Gamma, \Delta \vdash t: T_2}
	\]

\end{frame}
%%%%%%%%%%%%%%%%%%%%%%%%%%%%%%%%%%%%%%%%%%%
\subsection{Математическая постановка задачи. Классификация систем типов.}
%%%%%%%%%%%%%%%%%%%%%%%%%%%%%%%%%%%%%%%%%%%
\begin{frame}%[allowframebreaks=0.9,t]

	\begin{columns}[t]
		\begin{column}{0.5\textwidth}
			Системы типов в современных языках программирования можно разделить:
			\begin{enumerate}[1)]
				\item по времени проверки соответствия типам: статическая и динамическая,
				\item по поддержке неявных конверсий: сильная (англ. strong) и слабая,
				\item по необходимости вручную типизировать выражение: явная и неявная.
			\end{enumerate}
		\end{column}
		\begin{column}{0.5\textwidth}
			Система типов Хиндли-Милнера популярна и хорошо изучена.
			Согласно классификации она статическая, сильная и неявная.

			К ее термам относятся:

			\begin{flalign*}
				a, b, c &::=  \\
				& x && (\text{переменная}) \\
				& \lambda x. a && (\text{лямбда-функция}) \\
				& a(b) && (\text{вызов функции}) \\
				& let ~ a = b ~ in ~ c && (\text{объявление переменной}) \\
				& 1, 2, 3, \ldots && (\text{целочисленный литерал}) \\
				& 1.1, 1.2, 10.0, \ldots && (\text{вещественный литерал}) \\
				& (a, b) && (\text{объединение})
			\end{flalign*}
		\end{column}
	\end{columns}

\end{frame}
