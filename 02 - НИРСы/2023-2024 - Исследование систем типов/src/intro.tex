%----------------------------------------------------------
\chapter*{ВВЕДЕНИЕ}
\label{ch:introduction}
\addcontentsline{toc}{chapter}{ВВЕДЕНИЕ}

%----------------------------------------------------------

В современном программировании становится все более важным сколько будет потрачено времени на создание того или иного продукта.
В это число входит как время, потраченное непосредственно на создание приложения, так и время, потраченное на его поддержку.
Поэтому инструменты, применяемые программистом в повседневной работе, должны всячески помочь ему в этом.

Пожалуй, самым главным таким инструментом является компилятор.
Разработчики компиляторов прикладывают большие усилия, чтобы язык программирования отвечал требованиям надежности и скорости.
При создании инструмента такого рода важно правильно выбирать и проектировать каждую часть.
Одной из основных таких частей является то, как в языке программирования взаимодействуют друг с другом типы.

В высокоуровневых языках программирования типы окружают разработчика повсюду.
Чем более развитая система типов, там больше можно выразить, используя ее, а значит, если она надежна и подкреплена математической основой, то в программе станет меньше ошибок.
Кроме того, в таком случае программы можно будет применять в качестве доказательств для различных теорий \cite{AutoProvement}.
Сейчас такое уже применяет компания Intel при проектировании новых алгоритмов умножения или деления.

Актуальна проблема высокого порога входа в некоторые функциональные языки программирования, например Haskell и др.
Он завышен, так как в них применяются сложные математические теории, повлиявшие и на синтаксис языка, и на всю его идеологию в целом.
Несмотря на это, они активно применяются в промышленной разработке.
В рамках данной работы, а также курсового проекта разрабатывается язык программирования Kodept, целью которого является развитие идей функциональных языков при сохранении простоты синтаксиса C-подобных языков, а также изучение всей цепочки создания языка программирования в целом.

\textbf{Целью} научно-исследовательской работы является выбор системы типов, которая будет реализована в языке Kodept.

В \textbf{задачи} входит:
\begin{inparaenum}[1)]
    \item составление классификации систем типов в языках программирования,
    \item проведение анализа на основе выбранной классификации.
\end{inparaenum}

%----------------------------------------------------------
