%----------------------------------------------------------
\chapter{Теория типов}
\label{ch:type_theory}
%----------------------------------------------------------

В разделе представлена информация о специальном разделе математики - теории типов \cite{TypeTheoryBook}.
Освящены важные понятия - \textit{терм}, \textit{тип}, \textit{суждение} и \textit{система типов}.

Терм $x$ - чаще всего элемент языка программирования, будь то переменная, константа, вызов функции и др.
Например, в Haskell, термами будут: лямбда-функция \lstinline{\x -> x + 1}, определение переменной \lstinline{let x = "Hello" in ()} и т.д.
Как можно заметить, термы могут включать в себя другие термы.

Типом $A$ обозначается метка, приписываемая объектам - этот объект принадлежит к типу (классу) яблоко.
Обычно каждому терму соответствует определенный тип - $x: A$.
Типы позволяют строго говорить о возможных действиях над объектом, а также формализовать взаимоотношения между ними.

Система типов же, определяет правила взаимодействия между типами и термами.
В программировании это понятие равноценно понятию типизация.

С помощью суждений можно создавать логические конструкции и \textit{правила вывода}.
Именно благодаря этому теория типов активно применяется в компиляторах в фазе статического анализа программы, как для вывода, так и для проверки соответствия типов.
Более того, согласно изоморфизму Карри-Ховарда \cite{TypeTheoryArticle} (таблица \ref{tab:curry-hovard-iso}), программы могут быть использованы для доказательства логических высказываний.
Такие доказательства называют автоматическими, и они широко применяется среди таких языков, как \textit{Agda}, \textit{Coq}, \textit{Idris}.

\begin{table}[h]
    \centering
    \caption{Изоморфизм Карри-Ховарда}
    \label{tab:curry-hovard-iso}
    \begin{tabular}{|c|c|}
        \hline
        \textbf{Логическое высказывание} & \textbf{Язык программирования} \\\hline
        Высказывание, $F$, $Q$           & Тип, $A$, $B$                  \\\hline
        Доказательство высказывания $F$  & $x: A$                         \\\hline
        Высказывание доказуемо           & Тип $A$ обитаем                \\\hline
        $F \implies Q$                   & Функция, $A \to B$             \\\hline
        $F \wedge Q$                     & Тип-произведение, $A \times B$ \\\hline
        $F \vee Q$                       & Тип-сумма, $A + B$             \\\hline
        Истина                           & Единичный тип, $\top$          \\\hline
        Ложь                             & Пустой тип, $\bot$             \\\hline
        $\neg F$                         & $A \to \bot$                   \\\hline
    \end{tabular}
\end{table}

Тип $T$ обитаем (англ. inhabitat), если выполняется следующее: $\exists t: \Gamma \vdash t: T$

Наборы суждений образуют предположения (англ. \textit{assumptions}), которые образуют контекст $\Gamma$.
Правила вывода записываются следующим образом, например правило подстановки:

\begin{equation}
    \label{eq:judgement_substitution}
    \frac{\Gamma \vdash t: T_1, \Delta \vdash T_1 = T_2}{\Gamma, \Delta \vdash t: T_2}
\end{equation}

Выражение \ref{eq:judgement_substitution} можно трактовать следующим образом: если в контексте $\Gamma$ терм $t$ имеет тип $T_1$, а в контексте $\Delta$ тип $T_1$ равен типу $T_2$, то можно судить, что при наличии обоих контекстов, терм $t$ имеет тип $T_2$.

Как уже было отмечено ранее, теория типов может быть в той или иной мере применяться в языках программирования.
Правильный выбор системы типов для создаваемого языка программирования Kodept является важным решением.

%----------------------------------------------------------

