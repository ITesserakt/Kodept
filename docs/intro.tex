%----------------------------------------------------------
\chapter*{ВВЕДЕНИЕ}\label{ch:introduction}
\addcontentsline{toc}{chapter}{ВВЕДЕНИЕ}

%-------------------------------------------------------
Повышение удобства и эффективности разработки \gls{SOFTWARE} является важной задачей в сфере информатики.
Сюда входит как проектирование различных библиотек и фреймворков, так и создание пакетов и средств разработки, в том числе и языков программирования.

Говоря о языках программирования, их можно разделить на две группы:
\begin{itemize}
    \item[со статической типизацией] - языки, в которых типы переменных известны во время компиляции
    \item[с динамической типизацией] - языки, где проверка соответствия типов происходит по время работы программы
\end{itemize}

У обоих подходов есть как плюсы, так и минусы, однако подход со статической типизацией дает возможность разработчику заранее увидеть возможные ошибки, поэтому при проектировании больших программ удобнее использовать именно его.

Чаще всего у компилятора есть достаточно информации и для \textbf{вывода типов}.
Этот механизм позволяет разработчику не указывать большую часть информации о типах в процессе написания программы.

Для вывода типов и их проверки существует большое количество алгоритмов, описанных и формализированных посредством \textbf{теории типов}.


\textbf{В последнем абзаце} введения следует указывать цель работы в целом.

%----------------------------------------------------------
