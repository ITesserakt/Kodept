%----------------------------------------------------------
\chapter*{ВВЕДЕНИЕ}\label{chap.introduction}
\addcontentsline{toc}{chapter}{ВВЕДЕНИЕ}

%----------------------------------------------------------

Во введении должны быть представлены: введение в проблему, описание объекта исследований, обзор научно-технических источников\footnote{Следует изучать источники следующих типов в следующем порядке по убыванию приоритетности: \textbf{научные статьи}, патенты, электронные источники, книги (общее количество не менее 15)} по направлению поставленной задачи, примеры существующих аналогичных научно-технических решений.

\textbf{Целью} обзора научно-технических источников является \textbf{обоснование актуальности} решения поставленной задачи.

Обоснование актуальности предполагает проведение обзора литературы. Обзор литературы рекомендуется осуществлять, используя инструкцию\footnote{Инструкция о проведении обзора литературы: \url{https://archrk6.bmstu.ru/index.php/f/2597}}.

В состав материалов проводимого обзора литературы должны включаться выводы/заключения, ставшие результатом анализа соответствующих источников, на которые при этом обязательно следует делать ссылки (например, так \cite{FloatingPointData2021}). Источник, при этом, следует включать в список литературы в последний раздел настоящего документа (в настоящем документе список источников формируется автоматически с помощью компилятора \textsf{BibTeX} на основании файла \textsf{bibliography.bib} и ссылок по тексту).

В результате анализа всех источников должно стать возможным сделать вывод об обоснованности работ в направлении поставленной задачи.

\begin{remark}
Отметим, что в процессе подготовке текста возникает необходимость вводить аббревиатуры и использовать специальные термины, которые для документов большого объёма, выносятся в отдельные разделы ``Сокращения'' и ``Определения''. При использовании \LaTeX\xspace нет необходимости формировать этим разделы специально, -- рекомендуется использовать т.н. глоссарии. Создаётся файл \textsf{abbreviations.tex}, вносятся в него все необходимые термины и аббревиатуры и далее в любом месте текста ссылаются на них с использованием команды {\verb_\gls{SID}_} с одновременным появлением соответствующей расшифровки термина в соответствующем разделе (например, вызов команды {\verb_\gls{IND}_} приведёт к формированию \gls{IND}).
\end{remark}


\textbf{В последнем абзаце} введения следует указывать цель работы в целом.

\underline{Обязательность представления:} раздел обязателен. 

\underline{Объём:} как правило, не должен быть больше 5-7 страниц.

%----------------------------------------------------------
