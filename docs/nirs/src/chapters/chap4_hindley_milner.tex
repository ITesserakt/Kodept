%----------------------------------------------------------
\chapter{Система типов Хиндли-Милнера}
\label{ch:chap4_hindley_milner}
%----------------------------------------------------------

Исходя из классификации выше, было принято решение использовать систему типов Хиндли-Милнера.
Фактически она является подобием системы $\lambda 2$ и является, пожалуй, самой популярной.
Среди прочих ее особенностей, важно отметить то, что она способна вывести наиболее общий тип выражения, основываясь на аннотациях типов программиста и окружающем контексте.
Здесь приведена её небольшая модификация с добавлением типов-объединений и примитивных типов.

Наиболее классическим алгоритмом в этой области является так называемый алгоритм $W$ \cite{UrbanN2009}.

Для определения системы типов необходимо 3 составляющие: набор термов, набор типов и набор суждений.

\textbf{Термы:}
\begin{flalign*}
    a, b, c &\Coloneqq  \\
    & x && (\text{переменная}) \\
    & \lambda x. a && (\text{лямбда-функция}) \\
    & a(b) && (\text{применение аргумента к функции}) \\
    & let ~ a = b ~ in ~ c && (\text{объявление переменной}) \\
    & 1, 2, 3, \ldots && (\text{целочисленный литерал}) \\
    & 1.1, 1.2, 10.0, \ldots && (\text{вещественный литерал}) \\
    & (a, b) && (\text{объединение})
\end{flalign*}

\textbf{Типы:}
\begin{flalign*}
    \iota &\Coloneqq && (\text{примитивный тип}) \\
    & Integral && (\text{целочистенный}) \\
    & Floating && (\text{вещественный}) \\
    \tau, \sigma &\Coloneqq && (\text{мономорфный тип}) \\
    & \iota \\
    & T && (\text{переменная типа}) \\
    & \tau \to \sigma && (\text{функциональный тип}) \\
    & (\tau, \sigma) && (\text{тип-объединение}) \\
    & \Lambda && (\text{пользовательский тип}) \\
    \alpha &\Coloneqq && (\text{полиморфный тип}) \\
    & \tau \\
    & \forall a. \alpha && (\text{параметрический тип})
\end{flalign*}

Полиморфные типы - основное отличие $\lambda \to$ от $\lambda 2$.
Благодаря им имеется возможность определять более обобщенные функции.
Самый простой пример - функция $id$ - имеет следующий тип: $id: \forall a. a \to a$.
Таким образом ее можно вызвать и с аргументом-числом, и с аргументом-функцией.

Алгоритм $W$ работает, исходя из набора суждений:

\begin{equation}
    \label{eq:taut}
    \tag{TAUT}
    \frac{}{\Gamma \vdash x: \sigma}
\end{equation}

\begin{equation}
    \label{eq:inst}
    \tag{INST}
    \frac{\Gamma \vdash x: \sigma, \sigma' < \sigma}{\Gamma \vdash x: \sigma'}
\end{equation}

Запись $\sigma' < \sigma$ означает, что тип $\sigma'$ более конкретный, чем $\sigma$.

\begin{equation}
    \label{eq:gen}
    \tag{GEN}
    \frac{\Gamma \vdash x: \sigma, a \notin free(\Gamma)}{\Gamma \vdash x: \forall a. \sigma}
\end{equation}

\begin{equation}
    \label{eq:comb}
    \tag{COMB}
    \frac{\Gamma \vdash f: \tau \to \tau', x: \tau}{\Gamma \vdash f(x): \tau'}
\end{equation}

\begin{equation}
    \label{eq:abs}
    \tag{ABS}
    \frac{\Gamma \cup x: \tau \vdash y: \tau'}{\Gamma \vdash \lambda x. y : \tau \to \tau'}
\end{equation}

\begin{equation}
    \label{eq:let}
    \tag{LET}
    \frac{\Gamma \vdash x: \sigma, \Gamma \cup y: \sigma \vdash z: \tau}{\Gamma \vdash (let ~ y = x ~ in ~ z): \tau}
\end{equation}

\begin{equation}
    \label{eq:tuple}
    \tag{TUPLE}
    \frac{\Gamma \vdash x_1: \tau_1, x_2: \tau_2, x_3: \tau_3, \ldots}{\Gamma \vdash (x_1, x_2, x_3, \ldots): (\tau_1, \tau_2, \tau_3, \ldots)}
\end{equation}

Исходя из этих суждений, алгоритм $W$ составляет так называемое дерево вывода.
Если дерево построить удалось, то написанная программа считается верной.

Разберём небольшой пример: пусть имеется следующий контекст: $\Gamma = \left\{ fst: \forall a, b. a \to b \to a \right\}$.
Удостоверимся, что следующее выражение $let ~snd = \lambda x, y. fst(y)(x) ~in ~snd$, имеет тип $\forall a, b. a \to b \to b$.

\[
    \frac{
        \frac{
            \Gamma \cup x: \alpha \cup y: \beta \vdash \frac{
                \Gamma \vdash \frac{
                    \Gamma \vdash fst: \beta \to \alpha \to \beta, y: \beta
                }{
                    \Gamma \vdash fst(y): \alpha \to \beta
                }: \alpha \to \beta, x: \alpha
            }{
                \Gamma \vdash fst(y)(x): \beta
            }
        }{
            \Gamma \vdash \lambda x, y. fst(y)(x): \alpha \to \beta \to \beta
        }, \Gamma \cup y: \forall a, b. a \to b \to b \vdash z: \forall a, b. a \to b \to b
    }{
        \Gamma \vdash (let ~y = x ~in~ z): \forall a, b. a \to b \to b
    }
\]

Однако не всегда можно легко вывести тип выражения.
Например, выражение $\lambda x. x(x)$ имеет бесконечный рекурсивный тип, который нельзя выразить в $\lambda 2$.

%----------------------------------------------------------

