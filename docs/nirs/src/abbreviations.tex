%----------------------------------------------------------
%Термины и определения по тексту в большинстве случаев выделяются курсивом.
%В настоящем отчете о НИР применяют следующие термины с соответствующими определениями, также используются представленные обозначения и сокращения.


%\newabbreviation[category=inline]{html}{HTML}{hypertext markup language}
%\newabbreviation[category=footer]{shtml}{SHTML}{server-parsed HTML}

%\newglossaryentry{sample2}{name={sample2},
%	symbol={\ensuremath{\mathcal{S}_2}},
%	category=symbol,
%	description={the second sample entry}}

%\newabbreviation
%[prefix={an\space},
%prefixfirst={a~}]
%{svm}{SVM}{support vector machine}

%\newabbreviation
%[category=initialism,description={for example}]
%{eg}{eg}{exempli gratia}
% define the entries:

%\newabbreviation{html}{html}{hypertext markup language}

%\newabbreviation[category=initialism]{eg}{eg}{for example}
%\newabbreviation[category=initialism]{si}{SI}{sample initials}
%\newabbreviation{xml}{XML}{extensible markup language}
%\newabbreviation{css}{CSS}{cascading style sheet}
%\newacronym[description={a device that emits a narrow intense 
%	beam of light}]{laser}{laser}{light amplification by stimulated 
%	emission of radiation}

%\newacronym[description={a form of \gls{laser} generating a beam of
%	microwaves}]{maser}{maser}{microwave amplification by stimulated 
%	emission of radiation}

%\newacronym[description={a system for detecting the location and
%	speed of ships, aircraft, etc, through the use of radio waves}]{radar}{radar}{radio detection and ranging}

%\newacronym[description={portable breathing apparatus for divers}]{scuba}{scuba}{self-contained underwater breathing apparatus}

%%%%% для обычных newglossaryentry по умолчанию category==general.
%%%%% для обычных newabbreviation по умолчанию category==abbreviation.
%%%%% команда для создания своей категории \glscategory{<label>}
%----------------------------------------------------------
\newglossaryentry{slver}{name={Solver}, description={Решатель системы \gls{dcs-gcd}. Регистрируется в таблице \textbf{com.slvrs} БД \gls{gcddb} \gls{dcs-gcd}.}}

\newabbreviation[category=initialism]{atype}{ATYPE}{Предметный тип функции в системе РВС GCD.}
\newabbreviation[category=initialism]{IND}{ИД}{исходные данные}

\newabbreviation[category=initialism]{dacm}{ДАКМ}{дисперсно-армированный композиционный материал}
\newabbreviation[category=initialism]{DFD}{DFD}{диаграмма потоков данных (Data Flow Diagram)}
\newabbreviation[category=initialism]{DOT}{DOT}{язык описания графов}

\newabbreviation[category=initialism]{OOP}{ООП}{Объектно-ориентированное программирование}
\newabbreviation[category=initialism]{JVM}{JVM}{Виртуальная машина Java}

\newglossaryentry{LC}{
    name={лямбда-исчисление},
    description={формальная система, предназначенная для анализа вычислимости}
}

\newglossaryentry{TLC}{
    name={типизированное лямбда-исчисление},
    description={\gls{LC}, где лямбда-термам приписаны типы}
}

\newglossaryentry{TS}{
    name={система типов},
    description={набор правил в языке программирования, по которым выражениям назначаются типы}
}

\newglossaryentry{Kodept}{
    name={Kodept},
    description={экспериментальный язык программирования, разрабатываемый в рамках обучения на кафедре}
}

\GlsXtrEnableEntryCounting
{abbreviation}% list of categories to use entry counting
{2}% trigger value

\GlsXtrEnableEntryCounting
{symbol}% list of categories to use entry counting
{2}% trigger value


%----------------------------------------------------------


