%----------------------------------------------------------
\chapter*{ВВЕДЕНИЕ}
\label{ch:introduction}
\addcontentsline{toc}{chapter}{ВВЕДЕНИЕ}

%----------------------------------------------------------

В современном программировании становится все более важным сколько будет потрачено времени на создание того или иного продукта.
В это число входит как время, потраченное непосредственно на создание приложения, так и время, потраченное на его поддержку.
Поэтому инструменты, применяемые программистом в повседневной работе должны всячески помогуть ему в этом.

Пожалуй, самым главным таким инструментом является компилятор.
Разработчики компиляторов прикладывают большие усилия, чтобы язык программирования отвечал требованиям надежности и скорости.
При создании инструмента такого рода важно правильно выбирать и проектировать каждую часть.
Одной из основных таких частей является то, как в языке программирования взаимодействуют друг с другом типы, иначе говоря, какая у него \gls{TS}.

В высокоуровневых языках программирования типы окружают разработчика повсюду.
Чем более развитая система типов, там больше можно выразить, используя ее, а значит, если она надежна и подкреплена математической основой, то в программе станет меньше ошибок.
Кроме того, в таком случае программы можно будет применять в качестве доказательств для различных теорий.
Сейчас такое уже применяет компания Intel при проектировании новых алгоритмов умножения или деления.

%----------------------------------------------------------
