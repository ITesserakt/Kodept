%----------------------------------------%
% Направления R&D 
%----------------------------------------%
\newcommand{\rndfem}{Конечно-элементный анализ конструкций и гетерогенных сред}
\newcommand{\rndhpc}{Разработка систем инженерного анализа и ресурсоемкого ПО}
\newcommand{\rndgrn}{Механика силовых сетей и гранулированных сред}
\newcommand{\rndcse}{Автоматизация научных исследований}
\newcommand{\rndflu}{Динамика жидкости}
\newcommand{\rndcmp}{Разработка и анализ численных методов}
\newcommand{\rndbgg}{Обработка больших массивов геоданных}
\newcommand{\rndall}{Прикладные научно-исследовательские работы}
\makeatletter
\newcommand{\rnd@sln@}{@Направление исследований(следует выбрать конкретное в файле \textsf{research-ids.tex})@}
\makeatother
%----------------------------------------%
% Проекты в рамках выбранного направления (представлены примеры)
%----------------------------------------%
\newcommand{\rndchminv}{Применение методов интервальной арифметики для построения препроцессоров решателей обратных задач химической кинетики}
\newcommand{\rndmedcan}{Автоматизация диагностики прогрессирования отдельных видов онкологических процессов}
\newcommand{\rndhpcdbg}{Разработка графоориентированного дебаггера}
\newcommand{\rndhpcedt}{Разработка web-ориентированного редактора графовых моделей}
\newcommand{\rndhpcpar}{Реализация поддержки различных стратегий распараллеливания в рамках графоориентированного программного каркаса}
\newcommand{\rndhpcblo}{Реализация графоориентированной технологии определения бизнес-логики работы пользователя в системе}
\newcommand{\rndchmsen}{Численный анализ чувствительности в обратных задачах химической кинетики}
\newcommand{\rndcseclo}{Открытые репозитории научно-образовательной документации}
\newcommand{\rndcsedoc}{Динамическое документирование научно-образовательной деятельности}
\newcommand{\rndfemsth}{Стохастическая гомогенизация}
\newcommand{\rndfemcnt}{Численный анализ упруго-прочностных характеристик керамоматричных наномодифицированных УНТ КМ}
\makeatletter
\newcommand{\@rndproject@}{@Строковый идентификатор проекта (следует выбрать конкретный в файле \textsf{research-ids.tex})@}
\makeatother
%----------------------------------------------------------
\newcommand{\rndfield}{rndhpc} % Следует выбрать среди одного из доступных направлений выше
% команда, возвращающая проект по умолчанию
\newcommand{\rndproject}{rndhpcdbg} % Следует выбрать или добавить команду среди одного из доступных проектов выше
\newcommand{\Faculty}{<<Робототехники и комплексной автоматизации>>}
\newcommand{\Department}{<<Системы автоматизированного проектирования (РК-6)>>}
\newcommand{\ScientificAdviser}{Соколов~А.П., Першин А.Ю.} % Научные руководители по направлению
\newcommand{\Consultants}{@Фамилия~И.О.@} % Консультанты по направлению
\newcommand{\AllAuthors}{Крехтунова Д., Ершов В., Муха В., Тришин И.} % В случае, если текущий автор не первый вносящий заметки, то следует добавить свои ФИО после уже представленных.
\newcommand{\BeginYear}{2021} % заполняется при создании первой заметки, далее остаётся постоянным
\newcommand{\Year}{2021}
\newcommand{\Country}{Россия}
\newcommand{\City}{Москва}
\newcommand{\doctype}{researchnotes}
%----------------------------------------------------------
% методическое пособие
\newcommand{\RNDCredits}{Работа (документирование) над научным направлением начата 20~сентября~2021~г.}
%----------------------------------------------------------

