%----------------------------------------------------------
% общие определения
\newcommand{\UpperFullOrganisationName}{Министерство науки и высшего образования Российской Федерации}
\newcommand{\ShortOrganisationName}{МГТУ~им.~Н.Э.~Баумана}
\newcommand{\FullOrganisationName}{федеральное государственное бюджетное образовательное учреждение высшего профессионального образования\newline <<Московский государственный технический университет имени Н.Э.~Баумана (национальный исследовательский университет)>> (\ShortOrganisationName)}
\newcommand{\OrganisationAddress}{105005, Россия, Москва, ул.~2-ая Бауманская, д.~5, стр.~1}

%----------------------------------------%
% Направления R&D 
%----------------------------------------%
\newcommand{\rndfem}{Конечно-элементный анализ конструкций и гетерогенных сред}
\newcommand{\rndhpc}{Разработка систем инженерного анализа и ресурсоемкого ПО}
\newcommand{\rndgrn}{Механика силовых сетей и гранулированных сред}
\newcommand{\rndcse}{Автоматизация научных исследований}
\newcommand{\rndflu}{Динамика жидкости}
\newcommand{\rndcmp}{Разработка и анализ численных методов}
\newcommand{\rndbgg}{Обработка больших массивов геоданных}
\newcommand{\rndall}{Прикладные научно-исследовательские работы}
\makeatletter
\newcommand{\rnd@sln@}{@Направление исследований(следует выбрать конкретное в файле \textsf{research-ids.tex})@}
\makeatother
%----------------------------------------%
% Проекты в рамках выбранного направления (представлены примеры)
%----------------------------------------%
\newcommand{\rndhpcdsl}{Применение предметно-ориентированных языков для автоматизации разработки подсистем ввода и вывода данных}
\newcommand{\rndhpcrpc}{Методы удалённого запуска приложений}
\newcommand{\rndchminv}{Применение методов интервальной арифметики для построения препроцессоров решателей обратных задач химической кинетики}
\newcommand{\rndmedcan}{Автоматизация диагностики прогрессирования отдельных видов онкологических процессов}
\newcommand{\rndhpcdbg}{Разработка графоориентированного дебаггера}
\newcommand{\rndhpcedt}{Разработка web-ориентированного редактора графовых моделей}
\newcommand{\rndhpcpar}{Реализация поддержки различных стратегий распараллеливания в рамках графоориентированного программного каркаса}
\newcommand{\rndhpcthr}{Теоретические основы графоориентированного программного каркаса}
%\newcommand{\rndhpcblo}{Реализация графоориентированной технологии определения бизнес-логики работы пользователя в системе}
\newcommand{\rndhpcblo}{Графоориентированная методология разработки средств взаимодействия пользователя в системах автоматизированного проектирования и инженерного анализа}
\newcommand{\rndchmsen}{Численный анализ чувствительности в обратных задачах химической кинетики}
\newcommand{\rndcseclo}{Открытые репозитории научно-образовательной документации}
\newcommand{\rndcsedoc}{Динамическое документирование научно-образовательной деятельности}
\newcommand{\rndfemsth}{Стохастическая гомогенизация}
\newcommand{\rndfemcnt}{Численный анализ упруго-прочностных характеристик керамоматричных наномодифицированных УНТ КМ}
%\makeatletter
%\newcommand{\@rndproject@}{@Строковый идентификатор проекта (следует выбрать конкретный в файле \textsf{research-ids.tex})@}
%\makeatother
%----------------------------------------------------------
\newcommand{\rndfield}{rndhpc} % Следует выбрать среди одного из доступных направлений выше
% команда, возвращающая проект по умолчанию
\newcommand{\rndproject}{rndhpcdbg} % Следует выбрать или добавить команду среди одного из доступных проектов выше
\newcommand{\Faculty}{<<Робототехники и комплексной автоматизации>>}
\newcommand{\Department}{<<Системы автоматизированного проектирования (РК-6)>>}
\newcommand{\ScientificAdviser}{Соколов~А.П.}%, Першин А.Ю.} % Научные руководители по направлению
\newcommand{\Consultants}{@Фамилия~И.О.@} % Консультанты по направлению
\newcommand{\AllAuthors}{Крехтунова~Д., Ершов~В., Муха~В., Тришин~И., Василян~А.Р., Журавлев~Н.В.} % В случае, если текущий автор не первый вносящий заметки, то следует добавить свои ФИО после уже представленных.
\newcommand{\BeginYear}{2021} % заполняется при создании первой заметки, далее остаётся постоянным
\newcommand{\Year}{2023}
\newcommand{\Country}{Россия}
\newcommand{\City}{Москва}
\newcommand{\doctype}{researchnotes}
%----------------------------------------------------------
% методическое пособие
\newcommand{\RNDCredits}{Работа (документирование) над научным направлением начата 20~сентября~2021~г.}
%----------------------------------------%
\makeatletter
\newcommand{\rndfielddscr}[1]{%
%\csname #1\endcsname (\expandafter\@gobble\string#1)} % Следует выбрать среди одного из доступных направлений
\expandafter\csname #1\endcsname\xspace (#1)}
\makeatother
%----------------------------------------%
\makeatletter
\newcommand{\rndprojectdscr}[1]{%
\expandafter\csname #1\endcsname\xspace}
\makeatother
%----------------------------------------------------------
\newcommand{\TargetAudience}{%
Документ разработан для оценки результативности проведения научных исследований по направлению <<\expandafter\csname\rndfield\endcsname>>\xspace в рамках реализации курсовых работ, курсовых проектов, выпускных квалификационных работ бакалавров и магистров, а также диссертационных исследований аспирантов кафедры <<Системы автоматизированного проектирования>> (РК6) МГТУ~им.~Н.Э.~Баумана.}
%----------------------------------------------------------
\newcommand{\PrefaceIntro}{%
Документ содержит краткие материалы, формируемые обучающимися и исследователями в процессе их работ по одному научному направлению.}
%----------------------------------------------------------
\newcommand{\DocOutReference}{\textbf{\AllAuthors}. \textbf{\rndfielddscr{\rndfield}}: \DocumentType.~/ Под редакцией Соколова~А.П. [Электронный ресурс] --- \City: \Year. --- \total{page} с. URL:~\url{https://arch.rk6.bmstu.ru} (облачный сервис кафедры РК6)}
%----------------------------------------------------------
\newcommand{\Preface}{\PrefaceIntro \TargetAudience \newline \DocOutReference}
%----------------------------------------------------------
% Предмет исследований (уже чем объект, определяется, исходя из задач: формулируется как существительное, как правило, во множественном числе, определяющее "конкретный объект исследований" среди прочих в рамках более общего)
\newcommand{\SubjectOfResearch}{Исследования в области технологий, применяемых для разработки ресурсоемкого программного обеспечения инженерного анализа}
%----------------------------------------------------------
% Ключевые слова (представляются для обеспечения потенциальной возможности индексации документа)
\newcommand{\keywordsru}{%
	@keywordsru@} % 5-15 слов или выражений на русском языке, для разделения СЛЕДУЕТ ИСПОЛЬЗОВАТЬ ЗАПЯТЫЕ
\newcommand{\keywordsen}{%
	@keywordsen@} % 5-15 слов или выражений на английском языке, для разделения СЛЕДУЕТ ИСПОЛЬЗОВАТЬ ЗАПЯТЫЕ
%----------------------------------------------------------
\newcommand{\DocumentType}{Научно-исследовательские заметки} % тип документа
\newcommand{\SubTitle}{по направлению <<\rndfielddscr{\rndfield}>>}
\newcommand{\Title}{\DocumentType~\SubTitle}
%----------------------------------------------------------

