%----------------------------------------------------------
%Термины и определения по тексту в большинстве случаев выделяются курсивом.
\newglossaryentry{studentaccess}{name={student_access}, description={логин: \textbf{rk6_student} пароль: \textbf{2afeu33f}}}
%----------------------------------------------------------
\newabbreviation[category=initialism]{PO}{ПО}{программное обеспечение}
\newabbreviation[category=initialism]{CAE}{ИПО}{инженерное программное обеспечение}
\newabbreviation[category=initialism]{NIR}{НИР}{научно-исследовательская работа}
\newabbreviation[category=initialism]{NIRS}{НИРС}{научно-исследовательская работа студента}
\newabbreviation[category=initialism]{NID}{НИД}{научно-исследовательская деятельность}
\newabbreviation[category=initialism]{OKR}{ОКР}{опытно-конструкторская работа}
\newabbreviation[category=initialism]{RID}{РИД}{результ исследовательской деятельности}
\newabbreviation[category=initialism]{db}{БД}{база данных}

\newabbreviation[category=initialism]{LW}{ЛР}{лабораторная работа}
\newabbreviation[category=initialism]{CW}{КР}{курсовая работа}
\newabbreviation[category=initialism]{CP}{КП}{курсовой проект}
\newabbreviation[category=initialism]{VKR}{ВКР}{выпускная квалификационная работа}
\newabbreviation[category=initialism]{TO}{ТО}{технический объект, в т.ч. сложный процесс, система}

\newabbreviation[category=initialism]{aINI}{aINI}{Расширенный формат INI (\href{https://archrk6.bmstu.ru/index.php/f/846701}{описание представлено в \cite{SokAINI}})}

\newabbreviation[category=initialism]{aDOT}{aDOT}{Расширенный формат DOT (\href{https://archrk6.bmstu.ru/index.php/f/777612}{описание представлено в \cite{SokADOT}})}

\newabbreviation[category=initialism]{gbse}{ГПИ}{\href{https://archrk6.bmstu.ru/index.php/f/824891}{графо-ориентированная программная инженерия} (англ., graph-based software engineering (GBSE)), ориентированная для создания программных реализаций \gls{ccm} (патент на изобретение RU 2681408 \cite{patentRU2681408})}

\newabbreviation[category=initialism]{ccm}{СВМ}{сложный вычислительный метод}

%----------------------------------------------------------

