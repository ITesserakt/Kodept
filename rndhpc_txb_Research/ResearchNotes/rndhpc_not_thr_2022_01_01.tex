%----------------------------------------------------------
\def\notedate{2022.01.01}
\def\currentauthor{Соколов А.П. (РК-6)}
%----------------------------------------------------------
\notestatement{rndhpcthr}{Отличие сетевых моделей от сетевых графиков}
%----------------------------------------------------------

\mycitation{Сетевые модели отличаются от сетевых графиков тем, что в их вершинах могут реализовываться сложные логические и вероятностные функции, а также тем, что в них допускаются контуры}{Малинин~Л.И.\footnotemark,~1970, \cite{ErmilovLI1970}.} 

В работе \cite{ErmilovLI1970} проф.~Л.И.~Малинин использует термин \textit{контуры}, что в современной литературе по теории графов часто называют \textit{циклами}.

В работах \cite{NechiporenkoVI1968, NechiporenkoVI1977} д.т.н.~В.И.~Нечипоренко представляет обобщёный подход к графовому описанию сложных процессов и систем.

\footnotetext{Профессор Малинин~Л.И. использовал псевдоним и публиковался как Ермилов~Л.И.}
%----------------------------------------------------------
% Атрибуты задачи
\noteattributes{}
%----------------------------------------------------------