%----------------------------------------------------------
\def\notedate{2021.11.06}
\def\currentauthor{Тришин И.В. (РК6)}
%----------------------------------------------------------
\notestatement{rndcsedoc}{Особенности графоориентированного подхода в pSeven (DATADVANCE)}
\> pSeven – это платформа для анализа данных, оптимизации и создания аппроксимационных моделей, 
дополняющая средства проектирования и инженерного анализа. pSeven позволяет интегрировать в единой программной среде различные
инженерные приложения, алгоритмы многодисциплинарной оптимизации и инструменты анализа данных для упрощения принятия 
конструкторских решений.\cite{DatadvanceOffWebsite2021}
\newline
\textbf{Основы подхода}
\> В pSeven графоориентированный подход используется несколько иначе, нежели чем в разрабатываемом проекте.
Любой проект pSeven содержит определённый рабочий процесс (workflow), в котором размещаются отдельные блоки, связанные с получением,
обработкой и анализом данных. Они и являются узлами графа. Рёбрами же являются так называемые связи, которые определяют потоки данных.
Каждый блок имеет входные и выходные порты - наборы переменных, которыми блок может через связи обмениваться с другими блоками. \cite{pSevenDocs2021}
Сам же рабочий процесс хранится в бинарном файле с расширением \texttt{.p7wf}. Предположительно, для хранения описаний рабочих процессов
в pSeven используется специально разработанный бинарный формат.
\newline
\textbf{Параллельное исполнение}
\> При исполнении рабочего процесса pSeven запускает каждый блок в отдельном процессе на уровне операционной системы. \cite{pSevenDocs2021}
Это позволяет избежать конфликтов по данным и, кроме того, добиться одновременного исполнения тех блоков, которые находятся 
в независимых потоках данных.
\newline
\textbf{Поддержка циклов}
\> В pSeven циклы поддерживаются засчёт специальных блоков, которые относятся к категории "управляющих циклами" (cycle drivers).
К ним относятся, например, широко применяемые блоки Оптимизаторов, которые, как предполагает название, используются при решении
различных задач оптимизации. Они позволяют задавать условия окончания цикла, как, например, максимальное число итераций или требуемая
точность. Если речь идёт о задачах анализа данных, существует отдельный блок, обозначающий входную точку цикла (loop). Такой цикл
принимает на вход список наборов аргументов, каждый из которых будет обработан на соответствующем шаге цикла, и выдаёт список наборов
выходных значений, которые сохраняются в базе данных проекта. 
