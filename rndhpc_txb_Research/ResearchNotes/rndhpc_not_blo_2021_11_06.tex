%----------------------------------------------------------
\def\notedate{2021.11.06}
\def\currentauthor{Тришин И.В. (РК6)}%
%----------------------------------------------------------
\notestatement{rndhpcblo}{Графовое описание процессов обработки данных в pSeven (DATADVANCE)} 

\textsf{pSeven} -- это платформа для анализа данных, оптимизации и создания аппроксимационных моделей, дополняющая средства проектирования и инженерного анализа.
\textsf{pSeven} позволяет интегрировать в единой программной среде различные инженерные приложения, алгоритмы многодисциплинарной оптимизации и инструменты анализа данных для упрощения принятия конструкторских решений~\cite{DatadvanceOffWebsite2021}.

%--------------------------------
\subsubsection{Принципы функционирования}

На концептуальном уровне в \textsf{pSeven} вводятся следующие понятия.
\begin{itemize}
    \item Проект -- набор файлов, используемых в \textsf{pSeven} для описания решений одной или нескольких задач и хранения результатов их решения.
    \item Расчётная схема (workflow) -- формальное описание процесса решения некоторой задачи в виде ориентированного графа, узлами которого являются блоки, а рёбрами - связи. Такое описание хранится в бинарном файле с расширением \textsf{.p7wf}, использующем некоторый специализированный формат хранения подобного рода описаний.
    \item Блок -- функциональный элемент расчётной схемы, отвечающий за обработку входных данных и формирование выходных данных~\cite{pSevenDocsWorkflow2021}.
    \item Порт -- переменная определённого типа, описанная в блоке и имеющая в нём уникальное имя, значение которой может быть передано в другие блоки или получено от них через связи.
    \item Связь (link) -- одностороннее соединение между двумя портами, обеспечивающее передачу данных от одного к другому.
\end{itemize}

Проекты в \textsf{pSeven} имеют единую базу данных, куда записываются все результаты запусков расчётных схем и откуда берутся данные для их последующей презентации пользователю и их анализа. Для определения переменных, значения которых должны быть записаны в неё записаны, предусмотрены специализированные порты для самих расчётных схем, с которыми связываются те блоки, результаты выполнения которых интересуют пользователя.

Связи служат для маршрутизации данных. С их помощью осуществляется и взаимодействие между блоками и, кроме того, определяется очерёдность их запуска. В момент добавления связи в пакете \textsf{pSeven} выполняется проверка портов на совместимость. Они считаются совместимыми, если тип данных источника можно преобразовать к типу данных адресата~\cite{pSevenDocsWorkflow2021}.

%--------------------------------
\subsubsection{Поддержка циклов и ветвлений}

Расчётная схема может включать расчётные циклы. Для их создания применяются специализированные блоки, имеющие функциональную возможность управления запуском других блоков в теле цикла и принятия решения о его прекращении. Такие блоки называются управляющими блоками циклов. Одним из характерных примеров является \textit{оптимизационный цикл}, управление которым осуществляется из блока \textsf{Optimizer}~\cite{pSevenDocsWorkflow2021}, позволяющего, например, настроить максимальное число итераций или требуемую точность, как условия окончания.

Если речь идёт о задачах анализа данных, существует отдельный блок, обозначающий входную точку цикла (\textsf{Loop}). Такой цикл принимает на вход список наборов аргументов, каждый из которых будет обработан на соответствующем шаге цикла, и выдаёт список наборов выходных значений, которые сохраняются в базе данных проекта.

Кроме того, существует возможность включения в расчётную схему условного и безусловного ветвления. Первое достигается за счёт создания связей для подключения одного и того же порта вывода к различным портам ввода. В этом случае по каждой связи передаётся копия выходных данных, полученных у источника~\cite{pSevenDocsWorkflow2021}. Условные ветвления создаются с помощью специального блока \textsf{Condition}, который по определённому условию передаёт входные данные одному из подключенных блоков. Их целесообразно использовать для устранения ошибок в работе блока, отбраковки некорректных входных данных и других аналогичных целей~\cite{pSevenDocsWorkflow2021}.

%--------------------------------
\subsubsection{Особенности выполнения расчётных схем}

При выполнении расчётных схем каждый блок запускается в отдельном процессе на уровне операционной системы. Как сказано ранее, начало выполнения блока определяется его связями с другими. Любой блок будет ожидать завершения работы другого блока только в том случае, если ему необходимо получить от него входные данные. Это означает, что два блока, не имеющих связей друг с другом, входящие в состав различных ветвлений расчётной схемы, могут запускаться параллельно, поскольку они не зависят друг от друга по используемым данным~\cite{pSevenDocsWorkflow2021}.

Кроме того, при обработке больших выборок данных может потребоваться обрабатывать по несколько наборов данных одновременно в независимых потоках исполнения. Помимо прочего этой цели служит блок \textsf{Composite}, который является контейнером для нескольких блоков. В его настройках можно включить опцию параллельного исполнения, указать, с какого порта данные будут обрабатываться параллельно, и указать максимальное число потоков. При запуске расчётной схемы пакет \textsf{pSeven} создаёт несколько виртуальных экземпляров блока \textsf{Composite} и автоматически распределяет входные наборы данных между ними. Как только в одном из таких виртуальных блоков завершается расчёт, он получает из выборки следующий набор для обработки~\cite{pSevenDocsParallel2021}.

%----------------------------------------------------------
% Атрибуты задачи
\noteattributes{}
%----------------------------------------------------------
