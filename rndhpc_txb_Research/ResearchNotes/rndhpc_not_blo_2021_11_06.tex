%----------------------------------------------------------
\def\notedate{2021.11.06}
\def\currentauthor{Тришин И.В. (РК6)}%
%----------------------------------------------------------
\notestatement{rndhpcblo}{Особенности применения графового описания процессов оброботки данных в pSeven (DATADVANCE)}

\textsf{pSeven} -- это платформа для анализа данных, оптимизации и создания аппроксимационных моделей, дополняющая средства проектирования и инженерного анализа. 

\textsf{pSeven} позволяет интегрировать в единой программной среде различные инженерные приложения, алгоритмы многодисциплинарной оптимизации и инструменты анализа данных для упрощения принятия конструкторских решений\cite{DatadvanceOffWebsite2021}.

%-------------------------
\subsubsection{Принципы функционирования}

В \textsf{pSeven} графоориентированный подход используется несколько иначе, нежели чем в разрабатываемом проекте.\messnote{Так писать нельзя, т.к. о нашем подходе Вы ничего не сказали. Нужно сказать о нашем, а лишь потом говорить о том, что что-то используется иначе.}

\messnote{Следует ввести понятия: проект, рабочий процесс, блок(нужно ли?)}

Любой проект \textsf{pSeven} содержит определённый рабочий процесс (workflow), в котором размещаются отдельные блоки, связанные с получением,
обработкой и анализом данных. Они и являются узлами графа.\messnote{Следует писать подробнее и строже.} Рёбрами же являются так называемые связи, которые определяют потоки данных.\messnote{Следует ввести понятие: поток данных. Что это такое в контексте pSeven?}

Каждый блок имеет входные и выходные ``порты'' -- наборы переменных, с помощью которых осуществляется связь блока с другими блоками и обмен информацией\cite{pSevenDocs2021}.\messnote{Не хватает информации о том: определяются ли типы данных для каждого из портов?}

Сам же рабочий процесс хранится в бинарном файле с расширением \textsf{.p7wf}. Предположительно, для хранения описаний рабочих процессов
в \textsf{pSeven} используется специально разработанный бинарный формат.

%-------------------------
\subsubsection{Возможности параллельного исполнения}

При исполнении рабочего процесса \textsf{pSeven} запускает каждый блок в отдельном процессе на уровне операционной системы\cite{pSevenDocs2021}. 
Это позволяет избежать конфликтов по данным и, кроме того, добиться одновременного исполнения тех блоков, которые находятся 
в независимых потоках данных.\messnote{Непонятно! Что значит ``блоки находятся в независимых потоках данных?'', возможно ``потоках выполнения''? Какие именно блоки могут выполняться параллельно? Это разве не зависит от топологии графа?}

%-------------------------
\subsubsection{Поддержка циклов}

В \textsf{pSeven} циклы поддерживаются за счёт применения специальных блоков,\messnote{Понятие блока не было введено, поэтому непонятно...} которые относятся к категории ``управляющих циклами'' (cycle drivers).\messnote{также требует пояснения! есть ли еще какие-то категории?} К ``управляющих циклами'' блокам относятся, например, широко применяемые блоки ``Оптимизаторов'', которые, как предполагает название, используются при решении различных задач оптимизации. Такие блоки обеспечивают возможности определения условий окончания цикла, как, например, максимальное число итераций или требуемая точность.\messnote{Следует дополнить описание: какие еще атрибуты могут быть связаны с блоками и с блоками ``управляющих циклами''?} 

Если речь идёт о задачах анализа данных, существует отдельный блок, обозначающий входную точку цикла (loop). Такой цикл принимает на вход список наборов аргументов, каждый из которых будет обработан на соответствующем шаге цикла, и выдаёт список наборов выходных значений, которые сохраняются в базе данных проекта.

%----------------------------------------------------------
% Атрибуты задачи
\noteattributes{}
%----------------------------------------------------------
