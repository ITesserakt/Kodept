%----------------------------------------------------------
\def\notedate{2021.09.19}
\def\currentauthor{Соколов А.П. (РК6)}
%----------------------------------------------------------
\notestatement{rndcsedoc}{Содержание научно-исследовательской заметки}

Заметка размещается в рамках \LaTeX-подраздела (\textsf{\textbackslash subsection}).

В состав заметки следует включать:
\begin{itemize}
	\item заметку следует создавать с помощью вспомогательной команды\newline \textsf{\textbackslash notestatement\{@prjsid@\}\{@NoteTitle@\}};
	\item атрибуты заметки (дата, автор, идентификатор исследовательского проекта, тема заметки) следует заполнять явно, без введения дополнительных макроподстановок;
	\item \textbf{рекомендуется} в состав заметки включать: рисунки; схемы; графические результаты расчетов; формулы; математические постановки задач, представляемые исключительно в математически строгом виде;
	\item при включении в состав заметки утверждения следует добавлять сноску с выходными данными источника (при этом следует добавлять соответствующий источник в файл библиографии \textsf{bibliography.bib});
	\item все сопроводительные документы по текущей заметке следует размещать в каталоге, имеющем такое же имя, как имя файла заметки (рис.~\ref{fig.txb.research.notes.filesystem});
	\item объём одной заметки: не более 2-3 страницы.
\end{itemize}

\begin{figure}[!ht]
	\centering
\tikzstyle{every node}=[draw=black,thick,anchor=west]
\tikzstyle{file}=[draw=black,anchor=west]
\tikzstyle{textt} = [draw=none,fill=none, text centered, color = red, font=\scriptsize]
\tikzstyle{folder}=[draw=black,fill=yellow!20]
\tikzstyle{optional}=[dashed,fill=gray!40]
\begin{tikzpicture}[%
  grow via three points={one child at (0.5,-0.7) and
  two children at (0.5,-0.7) and (0.5,-1.4)},
  edge from parent path={(\tikzparentnode.south) |- (\tikzchildnode.west)}]
  \node (root) {@cpxsln@::@student_branch_sid@}
    child { node [folder] (rt) {rndsln_txb_Research}
      child { node [folder] (spec) {doc-spec}
				child { node [file] (preamble) {preamble_common.tex}}
				child { node [file] (ids) {rndrnt_rnt_Research-id.tex}}
				child { node [file] (title) {title_common.tex}}
			}
			child [missing] {}
			child [missing] {}
			child [missing] {}
			child { node [folder] (notesdir) {ResearchNotes}
				child { node [folder] (notedir) {rnd@sln@_not_@prj@_@year@_@month@_@id@}
					child { node [optional, file] (add) {...}}
				}
				child [missing] {}
				child { node [file] (notefile) {rnd@sln@_not_@prj@_@year@_@month@_@id@.tex}}
				child { node [optional, file] (add) {...}}
			}
			child [missing] {}
			child [missing] {}
			child [missing] {}
			child [missing] {}
			child { node [file] (main) {rndsln_txb_Research.tex}}
			child { node [file] (notesfile) {rndsln_txb_Research-notes.tex}}
			child { node [file] (bib) {bibliography.bib}}
		};
	\node [textt, right of = preamble, xshift = 3.2cm] (dpreamble) {$\leftarrow$ общая преамбула};
	\node [textt, right of = ids, xshift = 3.7cm] (dids) {$\leftarrow$ определения констант};
	\node [textt, right of = title, xshift = 3.0cm] (dtitle) {$\leftarrow$ титульная страница};
	\node [textt, right of = main, xshift = 3.7cm] (dmain) {$\leftarrow$ основной файл сборника};
	\node [textt, right of = notesdir, xshift = 2.6cm] (dnotesdir) {$\leftarrow$ каталог с заметками};
	\node [textt, right of = notesfile, xshift = 4.0cm] (dnotesfile) {$\leftarrow$ подключение заметок};
	\node [textt, right of = bib, xshift = 3.0cm] (dbib) {$\leftarrow$ источники литературы};
\end{tikzpicture}
\caption{Структура файловой системы исходников сборника исследовательских заметок}
	\label{fig.txb.research.notes.filesystem}
\end{figure}

%----------------------------------------------------------
% Атрибуты задачи
\noteattributes{}
%----------------------------------------------------------

