%----------------------------------------------------------
\def\notedate{2022.10.13}
\def\currentauthor{Василян А.Р. (РК6-73Б)}
%----------------------------------------------------------
\notestatement{rndhpcdsl}{Построение пользовательского интерфейса с использованием интерактивного машинного обучения}

%---------------------------------------------------------
\subsubsection{Построение пользовательского интерфейса}
	Функционал приложения доступный пользователю следует разделить на отдельные функции, которые пользователь может использовать посредством взаимодействия с интерфейсом или последовательностью таких действий. Применение каждой такой функции необходимо для достижения конкретного результата, который может быть достигнут различными путями. В процессе использования приложения пользователь определяет такие пути по совокупности критериев. К таким критериям относятся:

\begin{itemize}
	\item личный опыт; 
	\item размер и расположение элементов управления;
	\item планирование процесса взаимодействия при выполнении задания;
	\item рекомендации, заложенные в ПО; 
	\item интерактивные подсказки.
\end{itemize}

\subsubsection{Построение пользовательского интерфейса с использованием интерактивного машинного обучения}
	На первом этапе производится сбор входных данных. В качестве таких данных будут выступать частота, последовательность, достигаемый результат и время между применениями рассматриваемых функций. Исследованием в данной области занимается человеко-компьютерное взаимодействие, где в настоящее время основную роль играет машинное обучение.
	На основании собранных данных проводится обучение, целью которого является сократить путь для достижения конкретного результата, сокращение количества шагов и затрачиваемого времени для выполнения идентичных задач. Для обучения используется алгоритм дерева градиентного повышения (ДГП). Одной из его возможностей является визуализация процесса обучения. Строится аддитивная модель $F(x)$ состоящую из $M$ деревьев решений $h_m(x)$, с входной функцией $x$, формула  \ref{Fx}.

\begin{equation}
  \label{Fx}
   F(x)=\sum_{m=1}^M y_m h_m(x)
\end{equation}

Каждое дерево решений $h_m(x)$ является слабым учеником и может оперировать со смешанными типами данных. Строится модель с прямой поэтапной модой по формуле \ref{Fmx1}.

\begin{equation}
  \label{Fmx1}
   F_m(x)=F_{m-1}(x)+y_m h_m(x)
\end{equation}

	Для уменьшения потерь функции $L$ на каждом шагу выбирается дерево решений $h_m(x)$, фиксируя предыдущий ансамбль деревьев, формула \ref{Fmx2}.

\begin{equation}
  \label{Fmx2}
   F_m(x)=\operatorname{argmin}_h \sum_{i=1}^n L\left(u_i, F_{m-1}\left(x_i\right)-h\left(x_i\right)\right)
\end{equation}

	Для минимизации потерь используется алгоритм по формуле \ref{Fmx3}.

\begin{equation}
  \label{Fmx3}
   F_m(x)=F_{m-1}(x)+y_m \sum_{i=1}^n \nabla L\left(u_i, F_{m-1}\left(x_i\right)\right)
\end{equation}

	Алгоритм обучения выбирает лучший порог для каждого. Использование матрицы Гессиана и весов позволяет вычислять прирост информации, вызванный применением каждой функции и правила принятия решения для узла.
	Обучение может производиться на любом приложении с графическим пользовательским интерфейсом, имеющем длинные цепочки выполнения действий, например, пакет офисных приложений. По результатам обучения строится последовательность действий для достижения необходимого результата. При её построении учитывается время поиска элемента интерфейса, его доступность (подмножество действий необходимых к выполнению для получения доступа к данному элементу), соответствие описания и ожидаемого результат (использование других элементов, требующих большего числа шагов для достижения результата). На основании полученных результатов вносятся корректировки в существующий интерфейс, после чего обучение продолжается.
%----------------------------------------------------------
% Атрибуты задачи
\noteattributes{}
%----------------------------------------------------------

