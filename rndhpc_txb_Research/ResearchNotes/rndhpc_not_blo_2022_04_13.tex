%----------------------------------------------------------
\def\notedate{2022.04.13}
\def\currentauthor{Тришин И.В. (РК6-81Б)}
%----------------------------------------------------------
\notestatement{rndhpcblo}{Дополнительный обзор литературы и наброски для введения}
Методы решения задач, возникающие в процессе современных научно-технических исследований, зачастую предполагают выполнение большого количества операций обработки данных. Каждой такой операции требуются входные данные. По завершении выполнения операции получаются выходные данные. При этом выходные данные одной операции могут являться входными для одной или нескольких других операций. Между ними формируются зависимости по входным и выходным данным. Для учёта этих зависимостей возникает необходимость правильным образом организовать выполнение операций в пределах отдельно взятого метода и, в частности, при разработке программного обеспечения (ПО), которое реализовало бы данный метод. 

В наши дни популярность приобретает применение научных систем организации рабочего процесса (англ. scientific workflow systems). Такие системы позвояют автоматизировать процессы решения научно-технических задач, предоставляя средства организации и управления вычислительными процессами~\cite{DeelmanWorkflow2009}. Процесс работы с подобными системами состоит из 4 основных этапов:
\begin{enumerate}
    \item составление описания операций обработки данных и зависимостей между ними;
    \item распределение процессов обработки данных по вычислительным ресурсам;
    \item выполнение обработки данных;
    \item сбор и анализ результатов и статистики.
\end{enumerate}

Одной из ключевых особенностей подобного подхода к реализации методов решения научно-технических задач является выделение операций обработки данных в отдельные программные модули (функции, подпрограммы). При известных входных и выходных данных каждого модуля становится возможной их независимая разработка\cite{DanilovPar2011}. Это позволяет распределить их разработку между членами команды исследователей. Вследствие этого уменьшается объём работы по написанию исходных кодов, приходящийся на одного исследователя. Это в свою очередь облегчает отладку и написание документации, что положительно сказывается на общем качестве реализуемого ПО. Кроме того, 