%----------------------------------------------------------
\def\notedate{2021.12.19}
\def\currentauthor{Крехтунова Д.Д. (РК6-73Б), Соколов А.П. (РК6)}
%----------------------------------------------------------
\notestatement{rndhpcdbg}{Концептуальная постановка задачи}

%---------------------------------------------------------
Требуется реализовать web-ориентированный программный инструмент (далее \textit{GBSE-отладчик}), обеспечивающий проведение отладки графовых реализации некоторых сложных вычислительных методов. GBSE-отладчик должен обеспечивать отслеживание текущих значений каждого отдельного элемента данных в состояниях, соответствующих узлам связанной графовой модели.

\textbf{Цель разработки.} Создать программный инструментарий (web-ориентированный), который бы позволил визуализировать значения отдельных элементов общих данных\cite{SokPersh2018GBSE}, остановив обработку на произвольном, выбираемом(ых) заранее, состоянии(ях) данных.  

\textbf{Назначение.} Реализация задачи разработки GBSE-отладчика позволит отслеживать текущие значения отдельных элементов данных в произвольном состоянии данных (узле) при проведении текущего расчета, что упростит процесс разработки графоориентированных решателей. 

%\textbf{Объект разработки}: графовая модель вычислительного метода, представленная ориентированными графами с формализуемым назначением узлов и рёбер.

\textbf{Поставленные задачи (частичный перечень).}
\begin{enumerate}
	\item Провести обзор литературы по темам:
	\begin{enumerate}
		\item ``Автоматические методы отладки наукоемкого кода'' (автоматизированные и автоматические методы отладки, приме- няемые при реализации сложных вычислительных методов (СВМ), отладка наукоёмкого кода, science code debugging, graph based programming);
		\item ``Методы визуализации ассоциативных массивов'' (web-инструменты для древовидного представления ассоциативных массивов, визуализация ассоциативных массивов).
	\end{enumerate}
	\item Определить перечень поддерживаемых типов элементов данных СВМ, предложить и реализовать методы визуализации для каждого из них.
	\item Разработать функцию на языке \textsf{Python}, позволяющую определять тип данных, хранящихся в каждом отдельном элементе ассоциативного массива (хранение данных осуществляется в объекте типа \textsf{dict}). 
	\item С использованием программного каркаса Django разработать программное обеспечение для визуализации объектов типа \textit{dict} в древовидном виде.
\end{enumerate}


%----------------------------------------------------------
% Атрибуты задачи
\noteattributes{}
%----------------------------------------------------------

