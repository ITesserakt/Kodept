%----------------------------------------------------------
\def\notedate{2021.12.19}
\def\currentauthor{Крехтунова Д.Д. (РК6-73Б)}
%----------------------------------------------------------
\notestatement{rndhpcdbg}{Постановка задачи}

%---------------------------------------------------------
\subsubsection{Концептуальная постановка задачи}\

Разработка web-ориентированного отладчика для отслеживания текущих значений элементов данных в узлах графовой модели реализации вычислительного метода.

Реализация отладчика поможет отслеживать текущие значения отдельных элементов данных в узле графа, что упростит процесс разработки графоориентированных решателей. 
\newline

\textbf{Объект разработки}: графовая модель вычислительного метода, представленная ориентированными графами с формализуемым назначением узлов и рёбер.

\textbf{Цель разработки}: Создание программного web-инструментария для визуализации данных в произвольном состоянии данных (узле) при проведении текущего расчета.

\textbf{Задачи} курсового проекта:
\begin{enumerate}
	\item Провести обзор литературы по теме: «Автоматические методы отладки наукоемкого кода».
	\item Определить возможные типы элементов данных графовой модели и реализовать методы визуализации для каждого из них.
	\item Разработать функцию, позволяющую определить тип данных, хранящихся в ассоциативном массиве (хранение данных осуществляется в объекте типа dict). 
	\item Визуализировать данные в виде древовидной структуры.
\end{enumerate}


%----------------------------------------------------------
% Атрибуты задачи
\noteattributes{}
%----------------------------------------------------------

