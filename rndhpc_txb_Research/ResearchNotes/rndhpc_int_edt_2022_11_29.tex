\def\notedate{2022.11.29}
\def\currentauthor{Журавлев Н. (РК6-72Б)}
\notestatement{rndhpcedt}{Выбор алгоритмов для решения задачи укладки ориентированного графа и поиска в нём циклов}
\subsubsection{Выбор алгоритма для укладки графа}
Для размещания графов существует множество алгоритмов. Они делятся на 3 типа:
\begin{itemize}
\item Force-Directed and Energy-Based - методы этой категории используют симуляцию физических сил. Вершины представляются как заряженные частицы, которые отталкивают друг друга, а рёбра - как упругие струны, которые стягивают смежные вершины. Движение вершин в такой системе моделируется, пока не установится устойчивое состояние. В Energy-Based подходах пытаются описать потенциальную энергию системы и найти положение вершин, которое будет соответствовать минимуму.
\item Dimension Reduction - граф можно задавать матрицей смежности. Такое представление позволяет использовать универсальные методы снижения размерности.
\item Feature-Based Layout - вершины представляются, как реальные объекты и располагаются отталкиваясь от их свойств.
\end{itemize}
Так как последние два предполагают уменьшение размерности графа, следовательно больше всего подходят Force-Directed and Energy-Based. Всего их существует 3 вида:
\begin{itemize}
\item Force-Directed - вершины представляются как заряженные частицы, которые отталкивают друг друга с помощью физической силы, а рёбра — как упругие струны, которые стягивают смежные вершины. 
\item Multidimensional scaling - вершины являются силой, а рёбра уже являются пружинами.
\item Energy-Based - в этом подходе пытаются описать потенциальную энергию системы и найти положение вершин, которое будет соответствовать минимуму.
\end{itemize}

Для данного случая не имеет значения к какой категории принадлежит, поэтому надо рассмотреть самые часто используемые, а именно: Fruchterman-Reingold, Алгоритм Идеса, Kamada Kawaii, Force Atlas 2, OpenOrd, Гамма – алгоритм

Чтобы выбрать нужный для данного случая алгоритм необходимо выделить свойства результирующего графа, по котором будет произведён отбор.
\begin{itemize}
\item Граф является ориентированным
\item Начало графа должно находится в противоположной стороне от конца
\end{itemize}

В алгоритме Fruchterman-Reingold используется пружинная физическая модель, где вершины определяются как тела системы, а ребра как пружины. Силы могут действовать только на вершины, вес пружин при этом не учитывается. Данные алгоритм не подходит, так как не предусмотрен для ориентированных графов

Алгоритм Kamada Kawaii похож на алгоритм Fruchterman-Reingold, но выбирается вершина, на которую действует максимальная сила, затем остальные вершины фиксируются, энергия системы минимизируется. Данный метод не подходит, так как имеет самую высокое время работы из выбранных - O$(V^3)$.

Force Atlas 2 представляет граф в виде металлических колец, связанных между собой пружинами. Деформированные пружины приводят систему в движение, она колеблется и в конце концов принимает устойчивое положение. Основная его задача - визуализация графов, в которых имеются подмножества с высокой степенью взаимодействия, таким свойством целевой граф не обладает.

OpenOrd - это алгоритм, специально предназначенный для очень больших сетей, который работает на очень высокой скорости при средней степени точности. Это хороший компромисс для больших сетей, но часто он нежелателен для небольших графов где потеря точности может быть значительной по сравнению с другими подходами к компоновке. Все вершины изначально помещаются в начало координат, а затем проводятся итерации оптимизации. Итерации контролируются через алгоритм имитации отжига. Предназначен для визуализации больших графов, следовательно не подходит для целевого графа, так как он является графом маленького или среднего размера.

Гамма - алгоритм основная идея которого заключается в том, что выделяются сегменты, затем в минимальном выделяется цепь, которая укладывается в любую грань, вмещающую данный сегмент. Не подходит, так как в целевом графе может не оказаться циклов.

Алгоритм Идеса (Magnetic-Spring Algorithm) рёбра назначаются магнитной пружиной и затем при воздействии магнитных полей, вершины перемешаются. Лучше всего для целевого графа.

Исходя из сравнений выбранных алгоритмов следует, что лучше всего подходит Алгоритм Идеса.

\subsubsection{Выбор алгоритма для поиск циклов в графе}
Необходимо найти циклы в имеющемся графе для этого существуют множество алгоритмов, выделим основные: поиск в глубину (Depth-First Search, DFS), поиск в ширину (Breadth-First Search, BFS), алгоритмы "Черепахи и зайца".

Единственным требованием к алгоритму для решения задачи является быстродействие, следовательно выбор алгоритма происходит исходя из этой характеристики.

Алгоритмы Черепахи и зайца - группа алгоритмов, которые основы на перемещении два указателя, работа алгоритма заканчивается, когда два указателя встретятся. Однако работает за O$(n^3)$ времени.

Основными алгоритмами обхода графа являются поиск в ширину  и поиск в глубину. Основное различие между DFS и BFS состоит в том, что DFS проходит путь от начальной вершины до конечной, а BFS двигается вперед уровень за уровнем. Из этого следует, что алгоритмы применяются для решения разных задач. BFS используется для более эффективного нахождения кратчайшего пути в графе, определения связанных компонент в графе, а также обнаружения двудольного графа. DFS применяется для проверки графа на ацикличность или для решения задачи поиска циклов в графе.

Исходя из выше сказанного становится ясно, что лучше всего для данной задачи подходит алгоритм DFS.