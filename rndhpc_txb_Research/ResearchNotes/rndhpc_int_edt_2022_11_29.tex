\def\notedate{2022.11.29}
\def\currentauthor{Журавлев Н.В. (РК6-72Б)}
\notestatement{rndhpcedt}{Выбор алгоритмов для поиска циклов в ориентированном графе}

\begin{task}
Найти циклы в ориентированном графе.
\end{task}

Для нахождения циклов в ориентированных графах существуют множество алгоритмов, выделим основные: поиск в глубину (Depth-First Search, DFS), поиск в ширину (Breadth-First Search, BFS), алгоритмы "Черепахи и зайца".

Единственным требованием к алгоритму для решения задачи является быстродействие, следовательно выбор алгоритма происходит исходя из этой характеристики.

Алгоритмы Черепахи и зайца - группа алгоритмов, которые основаны на перемещении двух указателе. Работа алгоритмов заканчивается, когда два указателя встретятся \cite{alg-search}. Однако они работают за $O(n^3)$ времени \cite{alg-floyd}, что намного больше, чем DFS и BFS, которые работают за $O(n)$.

Основными алгоритмами обхода графа являются поиск в ширину и поиск в глубину. Основное различие между DFS и BFS состоит в том, что DFS проходит путь от начальной вершины до конечной, а BFS двигается вперед уровень за уровнем. Из этого следует, что алгоритмы применяются для решения разных задач. BFS используется для более эффективного нахождения кратчайшего пути в графе, определения связанных компонент в графе, а также обнаружения двудольного графа. DFS применяется для проверки графа на ацикличность или для решения задачи поиска циклов в графе.

Исходя из написанного выше можно сделать вывод, что лучше всего для решаемой задачи подходит алгоритм DFS.
%----------------------------------------------------------
% Атрибуты задачи
\noteattributes{}
%---------------------------------------------------------- 