%----------------------------------------------------------
\def\notedate{2022.10.24}
\def\currentauthor{Василян А.Р. (РК6-73Б)}
%----------------------------------------------------------
\notestatement{rndhpcgui}{Теоретическое ознакомление с Django и Docker.}

%---------------------------------------------------------
\subsubsection{Django}
	Django – это высокоуровневый Python веб-фреймворк для бэкенда, который позволяет быстро создавать безопасные и поддерживаемые веб-сайты. Фреймворк — это программное обеспечение, облегчающее разработку и объединение разных компонентов большого программного проекта.
	Главная цель фреймворка Django – позволить разработчикам вместо того, чтобы снова и снова писать одни и те же части кода, сосредоточиться на тех частях своего приложения, которые являются новыми и уникальными для их проекта.
	Достоинства Django:
\begin{enumerate}
	\item Масштабируемый.  Django использует компонентную архитектуру, то есть каждая её часть независима от других и, следовательно, может быть заменена или изменена, если это необходимо. Чёткое разделение частей означает, что Django может масштабироваться при увеличении трафика, путём добавления оборудования на любом уровне;
	\item Разносторонний. Django может быть использован для создания практически любого типа веб-сайтов;
	\item Безопасный. Django помогает разработчикам избежать многих распространённых ошибок безопасности, предоставляя фреймворк, разработанный чтобы «делать правильные вещи» для автоматической защиты сайта. Например, Django предоставляет безопасный способ управления учётными записями пользователей и паролями, избегая распространённых ошибок, таких как размещение информации о сеансе в файлы cookie, где она уязвима или непосредственное хранение паролей вместо хэша пароля;
	\item Переносным. Django написан на Python, который работает на многих платформах;
	\item Удобным в сопровождении. Код Django написан с использованием принципов и шаблонов проектирования, которые поощряют создание поддерживаемого и повторно используемого кода.
\end{enumerate}

\subsubsection{Docker}
	Docker — программное обеспечение с открытым исходным кодом, применяемое для разработки, тестирования, доставки и запуска веб-приложений в средах с поддержкой контейнеризации. Он нужен для более эффективного использование системы и ресурсов, быстрого развертывания готовых программных продуктов, а также для их масштабирования и переноса в другие среды с гарантированным сохранением стабильной работы.
	Основной принцип работы Docker — контейнеризация приложений. Этот тип виртуализации позволяет упаковывать программное обеспечение по изолированным средам — контейнерам. Каждый из этих виртуальных блоков содержит все нужные элементы для работы приложения. Это дает возможность одновременного запуска большого количества контейнеров на одном хосте.
	Преимущества использования Docker:
\begin{enumerate}
	\item Минимальное потребление ресурсов — контейнеры не виртуализируют всю операционную систему (ОС), а используют ядро хоста и изолируют программу на уровне процесса;
	\item Скоростное развертывание — вспомогательные компоненты можно не устанавливать, а использовать уже готовые docker-образы (шаблоны);
	\item Удобное скрытие процессов — для каждого контейнера можно использовать разные методы обработки данных, скрывая фоновые процессы;
	\item Работа с небезопасным кодом — технология изоляции контейнеров позволяет запускать любой код без вреда для ОС;
	\item Простое масштабирование — любой проект можно расширить, внедрив новые контейнеры;
	\item Удобный запуск — приложение, находящееся внутри контейнера, можно запустить на любом docker-хосте;
	\item Оптимизация файловой системы — образ состоит из слоев, которые позволяют очень эффективно использовать файловую систему.
\end{enumerate}
	Определения Docker:
\begin{enumerate}
	\item Docker-образ (Docker-image) — файл, включающий зависимости, сведения, конфигурацию для дальнейшего развертывания и инициализации контейнера;
	\item Docker-контейнер (Docker-container) — это легкий, автономный исполняемый пакет программного обеспечения, который включает в себя все необходимое для запуска приложения: код, среду выполнения, системные инструменты, системные библиотеки и настройки;
	\item Docker-файл (Docker-file) — описание правил по сборке образа, в котором первая строка указывает на базовый образ. Последующие команды выполняют копирование файлов и установку программ для создания определенной среды для разработки;
	\item Docker-клиент (Docker-client / CLI) — интерфейс взаимодействия пользователя с Docker-демоном. Клиент и Демон — важнейшие компоненты "движка" Докера (Docker Engine). Клиент Docker может взаимодействовать с несколькими демонами;
	\item Docker-демон (Docker-daemon) — сервер контейнеров, входящий в состав программных средств Docker. Демон управляет Docker-объектами (сети, хранилища, образы и контейнеры). Демон также может связываться с другими демонами для управления сервисами Docker;
	\item Том (Volume) — эмуляция файловой системы для осуществления операций чтения и записи. Она создается автоматически с контейнером, поскольку некоторые приложения осуществляют сохранение данных;
	\item Реестр (Docker-registry) — зарезервированный сервер, используемый для хранения docker-образов;
	\item Docker-хаб (Docker-hub) или хранилище данных — репозиторий, предназначенный для хранения образов с различным программным обеспечением. Наличие готовых элементов влияет на скорость разработки;
	\item Docker-хост (Docker-host) — машинная среда для запуска контейнеров с программным обеспечением;
	\item Docker-сети (Docker-networks) — применяются для организации сетевого интерфейса между приложениями, развернутыми в контейнерах.
\end{enumerate}
%----------------------------------------------------------
% Атрибуты задачи
\noteattributes{}
%----------------------------------------------------------

