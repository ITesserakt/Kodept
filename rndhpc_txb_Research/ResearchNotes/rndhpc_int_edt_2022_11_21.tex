\def\notedate{2022.11.21}
\def\currentauthor{Журавлев Н.В. (РК6-72Б)}
\notestatement{rndhpcedt}{Выбор алгоритмов для решения задачи укладки ориентированного графа}
Задача: выбрать алгоритм для укладки ориентированного графа

Для размещения графов известны алгоритмы следующих типов\cite{alg-graph}:
\begin{itemize}
\item Force-Directed - вершины представляются как заряженные частицы, которые отталкивают друг друга с помощью физической силы, а рёбра — как упругие струны, которые стягивают смежные вершины
\item Multidimensional scaling - вершины являются силой, а рёбра уже являются пружинами
\item Energy-Based - в этом подходе пытаются описать потенциальную энергию системы и найти положение вершин, которое будет соответствовать минимуму
\end{itemize}

Для решени поставленной задачи не имеет значения к какой категории принадлежит исследуемый алгоритм, поэтому необходимо рассмотреть самые часто упоминаемые, а именно: Fruchterman-Reingold, Алгоритм Идеса, Kamada Kawaii, Force Atlas 2, OpenOrd, Гамма – алгоритм.

Чтобы выбрать нужный для данного случая алгоритм необходимо выделить свойства результирующего графа, по котором будет произведён отбор:
\begin{itemize}
\item Граф является ориентированным
\item Конец графа должен находится в противоположной стороне от начала
\item Максимально допустимый граф является среднего размера
\item В графе могут не иметься циклы
\end{itemize}

В алгоритме Fruchterman-Reingold используется пружинная физическая модель, где вершины определяются как тела системы, а ребра как пружины. Силы могут действовать только на вершины, вес пружин при этом не учитывается. Данные алгоритм не подходит, так как он не предусмотрен для ориентированных графов \cite{alg-fruchterman}.

Алгоритм Kamada Kawaii похож на алгоритм Fruchterman-Reingold, но выбирается вершина, на которую действует максимальная сила, затем остальные вершины фиксируются, энергия системы минимизируется. Данный метод не подходит, так как имеет самую высокое время работы из рассматриваемых - O$(V^3)$ \cite{alg-kamada-kawai}.

Force Atlas 2 представляет граф в виде металлических колец, связанных между собой пружинами. Деформированные пружины приводят систему в движение, она колеблется и в конце концов принимает устойчивое положение. Основная его задача - визуализация графов, в которых имеются подмножества с высокой степенью взаимодействия, таким свойством целевой граф не обладает \cite{alg-force-atlas-2}.

OpenOrd - это алгоритм, специально предназначенный для очень больших сетей, который работает на очень высокой скорости при средней степени точности. Это хороший компромисс для больших сетей, но часто он нежелателен для небольших графов где потеря точности может быть значительной по сравнению с другими подходами к компоновке. Все вершины изначально помещаются в начало координат, а затем проводятся итерации оптимизации. Итерации контролируются через алгоритм имитации отжига. Предназначен для визуализации больших графов \cite{alg-open-ord}, следовательно не подходит для целевого графа, так как он является графом маленького или среднего размера.

Гамма - алгоритм основная идея которого заключается в том, что выделяются сегменты, затем в минимальном выделяется цепь, которая укладывается в любую грань, вмещающую данный сегмент \cite{alg-gamma}. Не подходит, так как в целевом графе может не оказаться циклов.

Алгоритм Идеса (Magnetic-Spring Algorithm) рёбра назначаются магнитной пружиной, и затем при воздействии магнитных полей вершины перемешаются \cite{alg-eades}. Лучше всего для целевого графа.

Исходя из сравнений выбранных алгоритмов следует, что лучше всего подходит Алгоритм Идеса.

\noteattributes{}