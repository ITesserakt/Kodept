\def\notedate{2022.12.23}
\def\currentauthor{Журавлев Н.В. (РК6-72Б)}
\notestatement{rndhpcedt}{Постановка задач}

В разработанном А.П.~Соколовым и А.Ю.~Першиным графоориентированном программном каркасе, для описания графовых моделей был разработан формат aDOT, который является расширением формата описания графов DOT.
Для визуализации графов, описанных с использованием формата DOT, используются специальные программы визуализации. Формат aDOT расширяет формат DOT с помощью дополнительных атрибутов и определений, которые описывают функции-предикаты, функции-обработчики и функции-перехода в целом.
Из этого вытекает, что для построения графовых моделей с использованием формата aDOT необходим графический редактор.

Разрабатываемый графический редактор должен:
\begin{itemize}
\item  Предоставлять возможность создавать ориентированный граф с нуля (добавление/удаление ребра/узла)
\item Предоставлять возможность нахождение циклов в графе
\item Предоставлять возможность экспортировать созданный граф в формат aDOT
\item Предоставлять возможность загрузить граф из формата aDOT
\end{itemize}

Так же должен удовлетворять следующим требованиям:
\begin{itemize}
\item Иметь возможность экспорта программного кода на языке программирования Python
\item Являться плагином для comwpc
\end{itemize}

Из описания функция редактора можно выделить следующие задачи:
\begin{enumerate}
\item Определить алгоритм укладки графа
\item Определить алгоритм поиска циклов
\item Определить с помощью какого инструмента будет происходить визуализация графа
\item Выбрать наиболее безопасный способ экспорта и выполнения кода на Python
\end{enumerate}

На более прикладном уровне нужно будет решить следующие задачи:
\begin{enumerate}
\item Изучить и программно реализовать выбранный алгоритм укладки графа
\item Изучить и программно реализовать выбранный поиска циклов
\item Изучить выбранный инструмент визуализации
\item Разработать механизм реализующий связь информации из aDOT файла с выбранным инструментом визуализации
\item Изучить и программно реализовать выбранный способ экспорта и выполнения кода на Python
\item Преобразовать разработанный редактор в плагин для comwpc
\end{enumerate}
\noteattributes{} 