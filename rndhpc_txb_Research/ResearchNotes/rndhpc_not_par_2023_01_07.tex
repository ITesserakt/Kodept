%----------------------------------------------------------
\def\notedate{2023.01.07}
\def\currentauthor{Тришин И.В. (РК6-11М)}
%----------------------------------------------------------
\notestatement{rndhpcpar}{Задача параллельного обхода ориентированного графа с циклами}
%----------------------------------------------------------
\subsubsection{Введение в предметную область}

На сегодняшний день существует класс прикладного программного обеспечения (\glsxtrshort{PO}), направленного на автоматизацию подготовки и проведения вычислительных экспериментов. К такому ПО, среди прочего, относятся так называемые научные системы управления потоком задач (англ. scientific workflow systems). Данные системы предполагают описание вычислительного эксперимента в виде набора вычислительных задач, которые должны быть решены, и информации о последовательности их выполнения. Наиболее распространены системы, где такой информацией служат зависимости по данным между задачами. Под зависимостью по данным понимается ситуация, когда выходные данные решения одной задачи являются входными для решения другой.

По списку задач и зависимостей между ними автоматически определяется очерёдность их решения, и система автоматически осуществляет решение задач в установленной последовательности. Такой подход даёт следующие преимущества.
\begin{enumerate}
	\item Становится возможным разделение обязанности по разработке программных реализаций решений отдельных задач между разработчиками.
	\item Становится возможным переиспользовать разработанные программные решений вычислительных задач в других вычислительных экспериментах.
	\item Становится возможным создание базы программных реализаций решений типовых задач.
\end{enumerate}

Кроме того, в зависимости от реализации системы, для проведения того или иного вычислительного эксперимента могут быть задействованы различные вычислительные мощности: от нескольких потоков процессора на одном персональном компьютере до тысяч узлов на вычислительном кластере. Для задач, легко поддающихся распараллеливанию и масштабированию, научные системы управления потоком задач дают универсальный интерфейс взаимодействия с различными вычислительными ресурсами, например при помощи процедуры <<отображения>>\cite{DeelmanWorkflow2009}, что даёт возможность проводить вычислительные эксперименты на машинах с самыми различными архитектурами.

В основе работы научных систем управления потоком задач лежит представление списка задач и зависимостей между ними в виде ориентированного графа. Некоторые системы, такие как DAGMan\cite{DAGMan2023} и VisTrails\cite{VTDoc2016}, поддерживают только ациклические графы. Другие, как, например, pSeven\cite{pSevenDocsConditons2022}, имеют специализированные встроенные средства для обработки циклов в ориентированном графе. Для выполнения задач в порядке, определяемым ориентированным графом, необходимо произвести его обход. Как правило, для повышения производительности в научных системах управления потоком задач предусмотрено одновременное выполнение задач, не имеющих между собой зависимостей по данным. Таким образом, возникает дополнительное требование выполнять обход графа в параллельном режиме там, где это возможно.

\subsubsection{Применение стандартных алгоритмов на графах}
В задаче параллельного обхода ориентированного графа могут быть применены некоторые стандартные алгоритмы на графах.

Так для алгоритма обхода может быть важна проверка на наличие в входном графе циклов, поскольку в зависимости от её результата для обхода могут быть выбраны, например, алгоритмы предназначенные только для ациклических графов. Для поиска циклов применяется так называемый поиск в глубину. Его псевдокод приведён в листинге~\ref{lst:dfs}
\begin{algorithm}[H]
	\caption{Поиск в глубину}\label{lst:dfs}
	\begin{algorithmic}[1]
		\Procedure{DFS_Visit}{u}
		\State $color[u]$ := GRAY
		\State time := time + 1
		\State $d[u]$ := time \Comment{Устанавливаем момент входа в вершину}
		\For{each $v \in Adj[u]$}
		\If{$color[v]$ = WHITE}
		\State DFS_Visit(v)
		\EndIf
		\EndFor
		\State $color[u]$ = BLACK
		\State time := time + 1
		\State $f[u]$ := time \Comment{Устанавливаем момент выхода из вершины}
		\EndProcedure
	\end{algorithmic}
\end{algorithm}
Доказано\cite{Cormen2005}, что в ориентированном графе нет циклов, если при поиске в глубину в нём не обнаружено ни одного обратного ребра, т.е. по завершении поиска в глубину нет такой пары вершин $u$ и $v$, что $d[u]<d[v]$ и $f[u]>f[v]$, где $d[u]$ -- временная метка входа в вершину, а $f[u]$ -- временная метка выхода из неё.

Для определения очерёдности посещения вершин в графе может быть применена топологическая сортировка. Цель топологической сортировки заключается в том, чтобы упорядочить вершины ациклического ориентированного графа таким образом, чтобы каждая вершина была посещена до того, как будут обработаны все вершины, на которые она указывает~\cite{Sedge2002}.

%----------------------------------------------------------
% Атрибуты задачи
\noteattributes{}
%---------------------------------------------------------- 