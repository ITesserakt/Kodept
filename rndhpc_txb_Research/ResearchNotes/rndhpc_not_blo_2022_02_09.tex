\def\notedate{2022.02.09}
\def\currentauthor{Тришин И.В. (РК6)}%
%----------------------------------------------------------
\notestatement{rndhpcblo}{Сравнительная характеристика pSeven и GBSE}

К сравнению с GBSE были выбраны и рекомендованы следующие программные комплексы:
\begin{enumerate}
    \item Pradis - разработка отечественной компании "Ладуга"
    \item pSeven - разработка отечественной компаниии DATADVANCE
\end{enumerate}

\subsubsection{Выделение признаков для сравнения}
При выделении сравнительных признаков необходимо было, чтобы они охватывали достаточно широкую область сведений о программном продукте. Среди прочих должны были быть выделены признаки, относящиеся как к общей структуре программного комплекса, так и к особенностям реализации в нём графоориентированного подхода и, кроме того, к особенностям взаимодействия с пользователем при решении задач, требующих действий со стороны пользователя.

На основании данных требований были выделены следующие признаки для сравнения:
\begin{enumerate}
    \item спектр задач;
    \item подход к формированию графовой модели;
    \item формат описания графовой модели;
    \item особенности работы с входными и выходными данными;
    \item особенности передачи данных между узлами графовой модели;
    \item поддержка ветвлений и циклов в топологии графа;
    \item поддержка параллельной обработки данных;
    \item особенности отбора результатов расчёта вручную;
    \item возможность доопределять входные данные непосредственно во время обхода графовой модели.
\end{enumerate}

\subsubsection{Описание сравниваемых объектов}
\textbf{Pradis}

Программный комплекс Pradis, разработанный отечественной компанией <<Ладуга>>, предназначен для анализа динамических процессов в механических системах и системах другой физической природы. Предметом решения являются  нелинейные нестационарные задачи. Расчет проводится в функции времени в исходных координатах. Анализ статических задач обеспечивается как частный случай динамического расчета. Круг задач, которые могут быть решены с помощью PRADIS, достаточно широк. Принципиально возможен анализ любых технических объектов, модели поведения которых представимы системой дифференциальных уравнений (ДУ). Практические возможности по анализу конкретных задач определяются текущим составом библиотек комплекса, прежде всего библиотеки моделей элементов.\cite{PradisGeneral2007}. Данный копмплекс был рекомедован к обзору и сравнению, однако после проведённого обзора официальной документации\cite{PradisMethods2007}, не было получено достаточного представления об использовании графоориентированного подхода в данном копмлексе, поэтому было принято решение исключить его из дальнейшего рассмотрения.

\textbf{pSeven}

В программном комплексе pSeven, разработанном компанией DATADVANCE, используется методология диаграмм потоков данных, т.е. топология графа, описывающего процесс решения некоторой задачи проектирования, определяется только зависимостями между входными и выходными данными каждого отдельного процесса их обработки, входящиего в решение. \cite{Nazarenko2015} В реализованном в pSeven подходе вводятся следующие понятия:
\begin{itemize}
    \item \emph{Расчётная схема (workflow)} - формальное описание процесса решения некоторой задачи в виде ориентированного графа;
    \item \emph{Блок} - программный контейнер для некоторого процесса обработки данных, входные и выходные данные для которого задаются через порты;
    \item \emph{Порт} - переменная определённого типа, имеющая определённое имя, привязанная к блоку;
    \item \emph{Связь} - направленное соединение типа "один к одному" между входным и выходным портами разных блоков;
\end{itemize}

С учётом данных понятий можно описать методологию диаграмм потов данных следующим образом. Расчётная схема содержит в себе набор процессов обработки данных (блоков), каждый из которых имеет (возможно, пустой) набор именованных входов и выходов (портов). Данные передаются через связи. Для избежания гонок данных множественные связи с одним и тем же входным портом не поддерживаются. Для начала выполнения каждому блоку требуются данные на всех входных портах. Все данные на выходных портах формируются по завершении исполнения блока.\cite{Nazarenko2015}

Все порты, которые не привязаны к другим блокам, автоматически становятся внешними входами и выходами для всей расчётной схемы. Для начала обхода расчётной схемы должен быть предоставлен набор входных данных и указаны внешние выходные порты, значения которых обязательно должны быть вычислены в результате обхода. Он производится в несколько этапов: сперва отслеживаются пути от необязательных выходных портов к входным, все встреченные на пути блоки помечаются, как неактуальные и не будут выполнены в дальнейшем; затем отслеживаются пути от обязательных выходных портов к входным и все встреченные на пути блоки помечаются, как обязательные к исполнению. Наконец обязательные к исполнению блоки запускаются, начиная с тех, которые подключены к внешним входам расчётной схемы, а неактуальные игнорируются. Обход прекращается, когда не остаётся необходимых для выполнения блоков. \cite{Nazarenko2015}

\newpage
\subsubsection{Сравнительная таблица}
Результаты проведённого сравнения были оформлены в общую таблицу, приведённую ниже.
\noindent\begin{longtable}{|p{3.5cm}|p{6.625cm}|p{6.625cm}|}
    \caption{Сравнительная таблица \label{thetable}} \\
    \hline
    \textbf{Признак} & \textbf{pSeven} & \textbf{GBSE} \\
    \hline
    Cпектр задач & Задачи оптимизации, анализ данных & Задачи автоматизированного проектирования, анализ данных \\
    \hline
    Подход к формированию графа & Согласно описанному в \cite{Nazarenko2015} подходу, узлами графа являются блоки, рёбрами - связи, по которым передаются данные. & Узлами графа являются состояния данных, рёбрами - переходы между состояниями, к которым привязываются функции-обработчики. \cite{SokolovPershin2018} \\
    \hline
    Формат описания графа & Сформированное описание сохраняеся в двоичном файле закрытого формата с расширением \textsf{.p7wf} & Описание графа и функций-обработчиков сохраняется в текстовом файле специального формата \textsf{.aDOT}, являющегося расширением формата DOT\cite{SokolovPershin2018} \\
    \hline
    Файловая структура проекта & Проект состоит из непосредственно файла проекта, в котором хранятся ссылки на созданные расчётные схемы и базу данных, сами расчётные схемы, файлы с их входными данными, файлы отчётов, где сохраняются выходные данные последних расчётов и результаты их анализа. & Проект состоит из \textsf{.aDOT} файла с описанием графа, \textsf{.aINI}-файлов с описанием входных данных, библиотеки функций-обработчиков, файлов, куда записываются выходные данные. \\
    \hline
    Особенности работы с входными и выходными данными & Входные данные должны быть указаны при настройках внешних входных портов расчётной схемы. Данные с выходных портов схемы сохраняются в локальной базе данных. Для их записи в файлы для обработки/анализа вне pSeven необходимо воспользоваться специально предназначенными для этого блоками. & Входные данные хранятся в файле с расширением .aINI, откуда считываются при запуске обхода графа\cite{SokolovPershin2017}. Для записи выходных/промежуточных данных в файлы или базы данных необходимо добавить соответствующие функции-обработчики. \\
    \hline
    Особенности передачи параметров между узлами & Данные между узлами передаются через связи, которые на уровне выполнения создают пространство в памяти для ввода и вывода данных для выполняемых в раздельных процессах блоков.  Транзитная передача данных, которые не изменяются в данном блоке, на выход невозможна. & Поскольку узлами графа являются состояния данных, существует возможность задействовать в расчётах только часть данных, оставляя их другую часть без изменений \\
    \hline
    Поддержка ветвлений и циклов & Присутствует. Достигается засчёт специальных управляющих блоков, которые отслеживают выполнение условий & Присутствует по умолчанию\\
    \hline
    Поддержка параллельной обработки данных & Присутствует. Блоки, входящие в состав различных ветвлений схемы могут быть выполнены параллельно, поскольку они не зависят друг от друга по используемым данным. & Присутствует. Существует возможность обойти различные ветвления графа одновременно.\\
    \hline
    Особенности отбора корректных результатов расчёта вручную & Производится на этапе анализа результатов с помощью отчётов, где можно задать фильтрацию выходных данных по указанным параметрам. В случае, если результаты являются промежуточными, расчётную схему приходится разбивать на части. & Планируется реализовать средство визуализации данных, которое вкупе с автоматической генерацией форм ввода позволят отбирать корректные результаты промежуточных вычислений во время обхода одного цельного графа. \\
    \hline
    Возможность доопределения значений входных данных в процессе обхода графа & Отсутствует & Реализована при помощи функций-обработчиков, создающих формы ввода \\
    \hline
\end{longtable}
%----------------------------------------------------------