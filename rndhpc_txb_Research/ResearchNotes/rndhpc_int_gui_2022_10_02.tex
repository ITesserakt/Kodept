%----------------------------------------------------------
\def\notedate{2022.10.02}
\def\currentauthor{Василян А.Р. (РК6-73Б)}
%----------------------------------------------------------
\notestatement{rndhpcgui}{Разработка программного обеспечения генерации кода на основе шаблонов при создании систем инженерного анализа)}

%---------------------------------------------------------
	
Была изучена статья <<Разработка программного обеспечения генерации кода на основе шаблонов при создании систем инженерного анализа>>. Из статьи были учтены особенности использования параметров шаблонов векторного типа, архитектура программнго инструментария и выводы:
\begin{itemize}
	\item Представляется целесообразным при разработке программного обеспечения инженерного анализа применение <<гибридных>> подходов, а именно: для разработки общесистемных стандартных функциональных возможностей следует использовать генераторы кода на основе шаблонов, тогда как при разработке программных реализаций алгоритмически сложных вычислительных процедур следует использовать генерацию кода на основе графовых представлений алгоритмов.
	\item Применение средств поддержки процессов разработки и средств генерации кода позволяет систематизировать процесс разработки многофункциональных программных комплексов.
	\item Актуальными и востребованными на практике являются программные механизмы предоставления удалённого доступа к разработанному программному инструментарию и библиотеке шаблонов посредством web-приложения. Решение такой задачи позволит сформировать основу средств автоматизации процессов разработки программного обеспечения, документирования, безбумажного документооборота, а также позволит предоставить доступ к созданному инструментарию широкому кругу пользователей.
\end{itemize}

%----------------------------------------------------------
% Атрибуты задачи
\noteattributes{}
%----------------------------------------------------------

