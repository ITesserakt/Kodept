%----------------------------------------------------------
\def\notedate{2021.09.19}
\def\currentauthor{Тришин~И.В.}
%----------------------------------------------------------
\notestatement{rndhpcblo}{Требования к возможностям обхода графовых моделей в GBSE}
Методология программного каркаса GBSE подразумевает параллельное исполнение рёбер графа, выходящих из одной вершины. На рисунке \ref{fig:parallelExample} после выполнения рёбер $F_{12}$~и~$F_{13}$ будет получено два независимых состояния данных $S_2$ и $S_3$ соответственно. Далее возникает задача правильным образом перевести данные из этих состояний в общее состояние $S_4$.
\begin{figure}[!ht]
    \centering
    \includegraphics[scale=0.4]{ResearchNotes/rndhpc_not_blo_2022_03_23/example.parallel.png}
    \caption{Пример графовой модели, требующей параллельного исполнения}
    \label{fig:parallelExample}
\end{figure}

Данный подход значительно увеличивает эффективность использования ресурсов вычислительной системы и ускоряет процесс решения, однако добавляет большое количество второстепенных задач, которые нужно решить при разработке. Так в примере на рисунке \ref{fig:parallelExample} рёбра $F_{12}$~и~$F_{13}$ выполнялись параллельно, а значит полученные в результате их выполнения данные существуют в общем случае в двух отдельных потоках исполнения (возможно даже на двух разных вычислительных машинах). Возникает задача сбора этих данных из разных потоков выполнения. В момент разветвления графа должно происходить разделение обрабатываемых данных, чтобы каждая ветвь работала со своим экземпляром. Помимо этого алгоритм обхода графовой модели должен корректно отрабатывать слияние ветвей графа.

\begin{figure}[!ht]
    \centering
    \includegraphics[scale=0.4]{ResearchNotes/rndhpc_not_blo_2022_03_23/example.parallel_then_linear.png}
    \caption{Пример графовой модели с совмещением ветвей}
    \label{fig:parallelThenLinearExample}
\end{figure}

На рисунке \ref{fig:parallelThenLinearExample} ветви $S_1 \rightarrow S_2 \rightarrow S_4$ и $S_1 \rightarrow S_3 \rightarrow S_4$ выполняются в разных потоках исполнения, но ребро $F_{45}$ должно быть выполнено только в одном потоке (если обратного не предполагает функция перехода этого ребра). Таким образом, требуется некоторая управляющая структура, которая бы занималась запуском и сворачиванием различных потоков выполнения.

Кроме того, разрабатываемая архитектура должна поддерживать несколько вариантов параллельного исполнения. Среди прочих желательна поддержка для:
\begin{itemize}
    \item поочерёдного выполнения (в первую очередь для отладки);
    \item многопроцессного выполнения;;
    \item многопоточного выполнения;
    \item выполнения на удалённых узлах (через SSH-соединение).
\end{itemize}

Соответственно, целесообразно спроектировать обозначенную выше управляющую структуру таким образом, чтобы, имея один и тот же интерфейс для всех способов параллельного или распределённого исполнения, разработчик мог создавать различные реализации этой структуры для каждого отдельного способа. 

%----------------------------------------------------------
% Атрибуты задачи
\noteattributes{}
%----------------------------------------------------------

