%----------------------------------------------------------
\def\notedate{2022.05.17}
\def\currentauthor{Тришин И.В.}
%----------------------------------------------------------
\notestatement{rndhpcblo}{Современные форматы описания иерархических структур данных}

При выполнении курсового проекта по дисциплине ``Технологии Интернет'' была поставлена задача разработать инструмент описания структуры состояния данных вычислительных методов~\cite{SokolovPershin2018}. Данный инструмент планируется в дальнейшем встроить как компонент в средство визуализации состояний данных. Разработка данного средства направлена на повышение прозрачности и наглядности взаимодействия с программным инструментарием GBSE.

Кроме того, разработка инструмента описания состояний данных направлена на поддержание идеи документирования алгоритмов и вычислительных методов, реализуемых по методологии GBSE.

\subsubsection{Описание структуры состояния данных}

Т.н. <<состояние данных>> представляет собой набор поименованных переменных фиксированного типа, характерных для конкретного этапа вычислительного метода или алгоритма. Тип переменных может быть как скалярным, так и векторным. Частным случаем переменной векторного типа является переменная, содержащая в себе ассоциативный массив с строковыми ключами. При этом в общем случае элементы данного массива могут иметь разные типы. Достоиством использования таких ассоциативных массивов является возможность группировки данных. Помимо этого, поскольку каждый элемент ассоциативного массива обладает ключом, типом и значением, что повторяет общую структуру элемента состояния данных, возникает возможность организовать состояния данных в виде иерархических структур.

Таким образом, для описания состояний данных требуется формат, который бы поддерживал гетерогенные (т.е. разнотипные) иерархические структуры данных.