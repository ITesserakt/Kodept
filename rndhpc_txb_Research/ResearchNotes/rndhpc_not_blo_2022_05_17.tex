%----------------------------------------------------------
\def\notedate{2022.05.17}
\def\currentauthor{Тришин И.В.}
%----------------------------------------------------------
\notestatement{rndhpcblo}{Современные форматы описания иерархических структур данных}

При выполнении курсового проекта по дисциплине ``Технологии Интернет'' была поставлена задача разработать инструмент описания структуры состояния данных вычислительных методов~\cite{SokolovPershin2018}. Данный инструмент планируется в дальнейшем встроить как компонент в средство визуализации состояний данных. Разработка данного средства направлена на повышение прозрачности и наглядности взаимодействия с программным инструментарием GBSE.

Кроме того, разработка инструмента описания состояний данных направлена на поддержание идеи документирования алгоритмов и вычислительных методов, реализуемых по методологии GBSE.

\subsubsection{Описание структуры состояния данных}

Т.н. <<состояние данных>> представляет собой набор поименованных переменных фиксированного типа, характерных для конкретного этапа вычислительного метода или алгоритма. Тип переменных может быть как скалярным, так и векторным. Частным случаем переменной векторного типа является переменная, содержащая в себе ассоциативный массив с строковыми ключами. При этом в общем случае элементы данного массива могут иметь разные типы. Достоиством использования таких ассоциативных массивов является возможность группировки данных. Помимо этого, поскольку каждый элемент ассоциативного массива обладает ключом, типом и значением, что повторяет общую структуру элемента состояния данных, возникает возможность организовать состояния данных в виде иерархических структур.

Таким образом, для описания состояний данных требуется формат, который бы поддерживал гетерогенные (т.е. разнотипные) иерархические структуры данных.

\subsubsection{Рассмотренные форматы}

Одними из первых рассмотренных были форматы для хранения научных данных HDF4 и HDF5~\cite{HDFOffCite}. Данные бинарные форматы позволяют хранить большие объёмы гетерогенной информации и поддерживают иерархическое представление данных. В нём используется понятие набора данных (англ. dataset), которые объединяются в группы (англ. group). Кроме того, формат HDF5 считается <<самодокументирующимся>>, поскольку каждый его элемент -- набор данных или их группа -- имеет возможность хранить метаданные, служащие для описания содержимого элемента. Существует официальный API данного формата для языка С++ с открытым исходным кодом. Одним из гланвых недостатков HDF5 является необходимость дополнительного ПО для просмотра и редактирования данных в этом формате, поскольку он является бинарным.

Альтернативой бинарным форматам описания данных являются текстовые. Среди них были рассмотрены форматы XML (Extensible Markup Language) и JSON (Javascript Object Notation). Главным преимуществом формата XML является его ориентированность на древовидные структуры данных и лёгкость лексико-синтаксического разбора файлов этого формата. Среди недостатков стоит выделить потребность в сравнительно большом количестве вспомогательных синтаксических конструкций, необходимых для структурирования (тегов, атрибутов). Они затрудняют восприятие чистых данных и увеличивают итоговый объём файла. 

Формат JSON, так же, как и XML рассчитан на иерархические структуры данных, но является не столь синтаксически нагруженным, что облегчает восприятие информации человеком~\cite{JSONvsXML}. Кроме того, крайне важным преимуществом JSON является его поддержка по умолчанию языком программирования Javascript, который используется при разработке веб-приложений. При этом JSON также обладает рядом недостатков. Среди них сниженная, по сравнению с XML надёжность, отсутствие встроенных средств валидации и отсутствие поддержки пространств имён, что снижает его расширяемость.