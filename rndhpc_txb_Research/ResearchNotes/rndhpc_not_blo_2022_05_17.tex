%----------------------------------------------------------
\def\notedate{2022.05.17}
\def\currentauthor{Тришин И.В., Соколов А.П.}
%----------------------------------------------------------
\notestatement{rndhpcblo}{Современные форматы описания иерархических структур данных}

При выполнении курсового проекта по дисциплине \flqq Технологии Интернет\frqq\xspace была поставлена задача разработать программный инструмент описания\pdfcomment{Скорее речь об инструменте визуализации...} структуры состояния данных вычислительных методов~\cite{SokolovPershin2018}. Данный инструмент планируется в дальнейшем встроить как компонент в средство визуализации состояний данных. Разработка данного средства направлена на повышение прозрачности и наглядности взаимодействия с программным инструментарием \gls{gbse}.

Кроме того, разработка инструмента описания состояний данных направлена на поддержание идеи документирования алгоритмов и вычислительных методов, реализуемых по методологии GBSE.

%-------------------
\subsubsection{Описание структуры состояния данных}

Т.н. <<состояние данных>>\cite{SokolovPershin2018} представляет собой множество именованных переменных фиксированного типа, характерное для конкретного этапа вычислительного метода или алгоритма. Данные в соответствующем состоянии, как правило, удобно хранить в виде ассоциативного массива. Тип отдельной переменной может быть как скалярным (целое, логическое, вещественное с плавающей запятой и пр.), так и сложным <<векторным>> (структурой, классом, массивом и пр.). Примером сложного <<векторного>> типа является, в свою очередь, ассоциативный массив со строковыми ключами, при этом конкретная переменная этого типа будет хранить, как правило, адрес этого массива. В общем случае элементы данного массива могут иметь разные типы. 

%Достоиством использования таких ассоциативных массивов является возможность группировки данных.
В рассматриваемом случае возникает возможность организации хранения состояния данных в виде иерархических структур.

%Помимо этого, поскольку каждый элемент ассоциативного массива обладает ключом, типом и значением, что повторяет общую структуру элемента состояния данных, возникает возможность организовать состояния данных в виде иерархических структур.

Таким образом, для описания состояний данных требуется формат, который бы поддерживал гетерогенные (т.е. разнотипные) иерархические структуры данных.

%-------------------
\subsubsection{Анализ известных форматов}

Одними из первых рассмотренных были форматы для хранения научных данных HDF4 и HDF5~\cite{HDFOffCite}. Данные бинарные форматы позволяют хранить большие объёмы гетерогенной информации и поддерживают иерархическое представление данных. В нём используется понятие набора данных (англ. dataset), которые объединяются в группы (англ. group). Кроме того, формат HDF5 считается <<самодокументирующимся>>, поскольку каждый его элемент -- набор данных или их группа -- имеет возможность хранить метаданные, служащие для описания содержимого элемента. Существует официальный API данного формата для языка С++ с открытым исходным кодом. Одним из гланвых недостатков HDF5 является необходимость дополнительного ПО для просмотра и редактирования данных в этом формате, поскольку он является бинарным.

Альтернативой бинарным форматам описания данных являются текстовые. Среди них были рассмотрены форматы XML (Extensible Markup Language) и JSON (Javascript Object Notation). Главным преимуществом формата XML является его ориентированность на древовидные структуры данных и лёгкость лексико-синтаксического разбора файлов этого формата. Среди недостатков стоит выделить потребность в сравнительно большом количестве вспомогательных синтаксических конструкций, необходимых для структурирования (тегов, атрибутов). Они затрудняют восприятие чистых данных и увеличивают итоговый объём файла. 

Формат JSON, так же, как и XML рассчитан на иерархические структуры данных, но является не столь синтаксически нагруженным, что облегчает восприятие информации человеком~\cite{JSONvsXML}. Кроме того, крайне важным преимуществом JSON является его поддержка по-умолчанию средствами языкы программирования Javascript, который используется при разработке веб-приложений. При этом JSON также обладает рядом недостатков. Среди них сниженная, по сравнению с XML надёжность, отсутствие встроенных средств валидации и отсутствие поддержки пространств имён, что снижает его расширяемость.


%----------------------------------------------------------
% Атрибуты задачи
\noteattributes{}
%----------------------------------------------------------
