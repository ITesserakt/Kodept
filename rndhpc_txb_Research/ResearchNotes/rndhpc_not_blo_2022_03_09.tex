\def\notedate{2022.03.09}
\def\currentauthor{Тришин И.В. (РК6)}

\notestatement{rndhpcblo}{Нынешнее представление графовых моделей в библиотеке comsdk}
На рисунке~\ref{fig:oldGraphStructure} представлена UML-диаграмма классов, связанных с представлением в comsdk ориентированного графа, описывающего организацию вычислительных процессов.

\begin{figure}[H]
    \includegraphics[width=\textwidth]{ResearchNotes/rndhpc_not_blo_2022_03_09/structure.png}
    \caption{Текущая структура классов, связанная с графовыми моделями в comsdk}\label{fig:oldGraphStructure}
\end{figure}

В существующей структуре классов можно выделить следующие недостатки:
\begin{enumerate}
    \item Отсутствует класс графа, который давал бы удобный интерфейс графовым моделям.
    \item Отсутствует контейнер, который бы инкапсулировал все узлы, относящиеся к конкретной графовой модели.
    \item Индекс узла графа задаётся пользователем при инициализации, что не гарантирует его уникальности.
    \item Отсутствует контейнер, который бы инкапсулировал все рёбра, относящиеся к конкретной графовой модели.
    \item Отсутствует объект, который бы описывал связи между узлами и рёбрами; вместо этого эти связи прописаны в самих узлах и рёбрах, что затрудняет операции с графовой моделью (преобразования и проч.).
    \item Функции-предикаты привязываются к узлам, а не к рёбрам, что не соответствует новым требованиям.
    \item В текущей версии задачей функций-предикатов фактически является отбор рёбер, которые должны быть выполнены, а не проверка соответствия данных в узле определённому формату.
\end{enumerate}