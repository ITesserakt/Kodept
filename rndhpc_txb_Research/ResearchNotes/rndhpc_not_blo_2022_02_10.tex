\def\notedate{2022.02.09}
\def\currentauthor{Тришин И.В. (РК6)}
%----------------------------------------------------------
\notestatement{rndhpcblo}{Требования к представлению графовых моделей в comsdk}
\subsubsection{Введение}
На сегодняшний день существуют две реализации графоориентированного программного каркаса GBSE~\cite{SokolovPershin2018}. Обе они представляют собой подключаемые объектно-ориентированные библиотеки, включающие в себя API для создания состояний СВМ (узлов графовой модели) и их объединения с помощью функций-предикатов и функций-обработчиков~\cite{SokolovPershin2018}. Хронологически первой из них является реализованная на языке С++ библиотека comsdk. Более поздней и потому более актуальной в некоторых аспектах является библиотека на языке Python pycomsdk. Поскольку их разработка ведётся не параллельно, между ними существует большое количество алгоритмических и архитектурных отличий. Среди прочих особо выделяется подход к работе с языком описания графовых моделей aDot и, как следствие, программное представление этих моделей в описанных выше API.

\subsubsection{Описание синтаксиса языка aDot}
При описании графовых моделей в GBSE вводятся следующие понятия:
\begin{itemize}
    \item \textit{Состояние данных} - некоторый строго определённый набор именованных переменных фиксированного типа, характерных для решаемой задачи;
    \item \textit{Морфизм} - некоторое отображение одного состояния данных в другое;
    \item \textit{Функция-предикат} - функция, определяющая соответствие подаваемого ей на вход набора данных тому виду, который требуется для выполнения отображения;
    \item \textit{Функция-обработчик} - функция, отвечающая за преобразование данных из одного состояния в другое;
    \item \textit{Функция-селектор} - функция, отвечающая в процессе обхода графовой модели за выбор тех рёбер, которые необходимо выполнить на следующем шаге в соответствии с некоторым условием.
\end{itemize}

Самая актуальная версия представленного формата содержится в \cite{SokolovADOT2020}.

В текущей версии comsdk поддерживается только уже устаревшая версия формата aDot, в связи с чем актуальна потребность в обновлении модуля этой библиотеки, содержащего в себе работу с графовыми моделями, для поддержки новейшей версии этого формата.

\subsubsection{Список требований}
В результате анализа текущего синтаксиса языка aDot были сформулированы следующие требования к представлению графовых моделей в новой версии библиотеки comsdk:
\begin{enumerate}[label=\arabic*)]
    \item Каждое ребро графа должно иметь возможность привязать к нему до трёх морфизмов - препроцессор, обработчик и постпроцессор
    \item Каждый морфизм должен содержать в себе функцию-предикат и функцию-обработчик;
    \item Каждый узел графа должен хранить данные в том состоянии, которое он описывает;
    \item Каждый узел графа должен хранить данные о стратегии выполнения рёбер, исходящих из него (поочерёдное выполнение, выполнение в отдельных потоках, выполнение в отдельных процессах, выполнение на удалённом узле через SSH-подключение);
\end{enumerate}

Кроме того, были сформулированы требования к формированию данных обновлённых моделей:
\begin{enumerate}
    \item При формировании графовой модели все узлы $N_i$, которые обозначены в aDot файле с её описанием, как подграфы, должны заменяться загружаемыми из соответсвующих aDot-файлов графовыми моделями $G_i$, при этом рёбра, входящие в $N_i$ должны быть подключены к начальному узлу $G_i$, а исходящие из $N_i$ - к конечному узлу $G_i$;
\end{enumerate}